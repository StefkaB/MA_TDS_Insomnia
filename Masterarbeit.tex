% Dokumentenklasse Report mit Koma-Script für deutschen Sprachgebrauch
\documentclass[a4paper,12pt]{scrreprt}


% ============= Packages =============

% deutsche Silbentrennung (neue Rechtschreibung)
\usepackage[ngerman]{babel}
% Ausgabefonts
\usepackage[T1]{fontenc}
% Eingabekodierung (Umlaute)
\usepackage[utf8]{inputenc}
% Latin Modern (Standardschrift mit Serifen)
\usepackage{lmodern}
% microtype unterstützt bessere Lesbarkeit
\usepackage{microtype}
% Zeilenabstand (kann im jederzeit Dokument geändert werden)
\usepackage[]{setspace}
\onehalfspacing
% Bilder
\usepackage{graphicx}
% zur Positionierung der Bilder (anderenfalls platziert Latex sie beliebig)
% Abbildungen in Rahmen setzen
\usepackage[export]{adjustbox}
\usepackage{float}
% zur Erstellung von Abstract und Kurzfassung
\usepackage{abstract}
% zur Manipulation von Kopf- und Fußzeilen
\usepackage{fancyhdr}
% Tabellen
\usepackage{array}
% für Begrenzungen in Tabellen
\usepackage{tabularx}
% zusätzliche mathematische Schriftzeichen
\usepackage{amssymb}
\usepackage{amsmath}
% Anpassung der Abbildungs- und Tabellenbeschriftungen (10 pt bei 12 pt scrreprt und fetter Bezeichnung)
\usepackage[font=footnotesize,labelfont=bf, justification=centering, format = plain]{caption}
% für mehrseitige Tabellen
\usepackage{longtable}
% Booktabs für Tabellen ohne vertikale Linien
\usepackage{booktabs}
% Abkürzungsverzeichhnig mit im Dokument verwendeten Abkürzungen und Seitenzahl
\usepackage[printonlyused, withpage]{acronym}
% griechische Buchstaben nicht kursiv
\usepackage{upgreek}
% Sonder-/Formelzeichen fett mit Befehl \bm
\usepackage{bm}
% Zähler auf Null zurücksetzen, z.B. bei Formeln eines neuen Kapitels
\usepackage{chngcntr}
% Rücksetzpunkte werden aufgehoben für Abbildungen, Tabellen und Fußnoten
%\counterwithout{figure}{chapter}
%\counterwithout{table}{chapter}
\counterwithout{footnote}{chapter}
%\counterwithout{equation}{chapter}
% Erhöhung der Vertiefungsebene auf 5
\setcounter{secnumdepth}{3}
% Darstellung aller 5 Vertiefungsebenen im Inhaltsverzeichnis
\setcounter{tocdepth}{3}
% Einrücktiefe der ersten Zeile eines Absatzes nach einem Absatz festlegen
\setlength{\parindent}{0pt}
% besondere Trennungen
\hyphenation{Frig-yes}
\usepackage{hyperref}
% Zeilenumbruch in URLs nach einem Bindestrich erlauben
% \usepackage[hyphens]{url}
\usepackage{url}
% URL wird wie normaler Text dargestellt
\renewcommand\UrlFont{\textnormal}
% Seitenränder (geometry immer zuletzt)
\usepackage[left= 3 cm,right = 2.5 cm, bottom = 2.5 cm]{geometry}

% ============= Literaturverzeichnis mit BibLaTeX =============

\usepackage[backend=biber, style=alphabetic, natbib=true, hyperref=true, isbn = false, url = false, doi = false, maxparens = 5, maxbibnames = 50, maxcitenames = 1, useprefix = true, block = space]{biblatex}
\addbibresource{ma_literatur.bib} %% Einbinden der bib-Datei

% ============= Kopf- und Fußzeile =============

% globale Veränderung des Seitenlayouts
% "`fancy"' erlaubt die Verwendung der im Paket fancyhdr definierten Befehle zur Erstellung eigener Kopf- und Fußzeilen
\pagestyle{fancy}
% Ausgabe des aktuellen Kapitels links in Kopfzeile
\lhead{\slshape \MakeUppercase{\leftmark}}
\chead{}
\rhead{}
% Ausgabe der aktuellen Seitenzahl zentral in Fußzeile
\lfoot{}
\cfoot{\thepage}
\rfoot{}
% Linie in Kopfzeile
\renewcommand{\headrulewidth}{0.4pt}
% keine Linie in Fußzeile
\renewcommand{\footrulewidth}{0pt}

% ============= Autor und Titel =============

\title{Masterarbeit}
\author{Stefanie Breuer}



% ============= Start Document =============
\begin{document}

% ============= Deckblatt =============
% ohne Kopfzeile und ohne Seitennummerierung
\pagestyle{empty}
\newgeometry{left= 3 cm,right = 2.5 cm, bottom = 1.5 cm, top = 2 cm}
\begin{center}
\begin{tabular}{p{\textwidth}}

\begin{center}
\includegraphics[scale=1.5]{img/HTW_Logo_quer_rgb.jpg}
\end{center}

\\

\begin{center}
Hochschule für Technik und Wirtschaft Berlin\\
Fachbereich 4 - Informatik, Kommunikation und Wirtschaft\\
Studiengang Angewandte Informatik
\end{center}

\\

\begin{center}
\LARGE{\textsc{Abschlussarbeit\\}}
\large{zur Erlangung des akademischen Grades\\
"Master of Science" (M. Sc.)}
\end{center}

\\

\begin{center}
\LARGE{Analyse physiologischer Netzwerke anhand des \\
Time Delay Stability Verfahrens bei Insomniepatienten}
\end{center}

\\

\begin{center}
\large{\textsc{Stefanie Breuer}} \\
\small{Matrikelnummer 535098\\
s0535098@htw-berlin.de}
\end{center}

\\

\begin{center}
\small{03. April 2017}
\end{center}

\\

\begin{tabbing}
    \hspace*{-1cm}\=\hspace*{9cm}\= \kill
    \>\underline{Erstgutachterin:} \> \underline{Zweitgutachter:} \\
    \> Prof. Dr. Dagmar Krefting \> Prof. Dr. Peter Hufnagl \\
    \> HTW Berlin \> HTW Berlin \\
    \> Campus Wilhelminenhof, Fachbereich 4 \> Campus Wilhelminenhof, Fachbereich 4 \\
    \> Wilhelminenhofstraße 75A, 12459 Berlin \> Wilhelminenhofstraße 74A, 12459 Berlin \\
  \end{tabbing}

\end{tabular}
\end{center}
\restoregeometry

\pagestyle{fancy}
\pagenumbering{Roman}
\lhead{}
\renewcommand{\headrulewidth}{0pt}

% ============= Abstract =============

\newpage
%\pagestyle{empty}
\newgeometry{left= 3 cm,right = 2.5 cm, bottom = 2.5 cm, top = 6 cm}

% der Befehl section* unterdrückt die Auflistung im Inhaltsverzeichnis
\section*{\Huge{Kurzfassung}}

deutscher Text

\section*{\Huge{Abstract}}

english text

\restoregeometry

% ============= Inhaltsverzeichnis =============

\newpage
%\pagestyle{empty}
\tableofcontents

% ============= Abkürzungsverzeichnis =============

\newpage
%\setcounter{page}{1}
\newgeometry{left= 3 cm,right = 2.5 cm, bottom = 2.5 cm, top = 6 cm}

% in Inhaltsverzeichnis aufnehmen
\addcontentsline{toc}{chapter}{Abkürzungsverzeichnis}
\sectionmark{Abkürzungsverzeichnis}
\section*{\Huge{Abkürzungsverzeichnis}}
% keine Punkte im Inhaltsverzeichnis
%\renewcommand*{\bflabel}[1]{{\textsf{#1}\dotfill}}
\

\begin{acronym}[ABCDEFGHIKL]
\acro{AASM}{American Academy of Sleep Medicine}
\acro{BMI}{Body Mass Index}
\acro{CRAN}{Comprehensive R Archive Network}
\acro{EDF}{European Data Format}
\acro{EEG}{Elektroenzephalogramm}
\acro{EKG}{Elektrokardiogramm}
\acro{EMG}{Elektromyogramm}
\acro{EOG}{Elektrookulogramm}
\acro{FFT}{Fast Furier-Transformation}
\acro{ICSD-2}{International Classification of Sleep Disorders}
\acro{NREM}{Non-REM}
\acro{PSG}{Polysomnographie}
\acro{REM}{Rapid Eye Movement}
\acro{SR}{Sampling Rate}
\acro{TDS}{Time Delay Stability}
\end{acronym}

\restoregeometry

% ============= Tabellenverzeichnis =============

\newpage
\addcontentsline{toc}{chapter}{Tabellenverzeichnis}
%\setcounter{page}{2}
\listoftables

% ============= Abbildungsverzeichnis =============

\newpage
\addcontentsline{toc}{chapter}{Abbildungsverzeichnis}
%\setcounter{page}{3}
\listoffigures

% ============= Text =============

\newpage
% mit Kopf- und Fußzeilen gemäß Paket fancyhdr
\pagestyle{fancy}
\pagenumbering{arabic}
% aktuelles Kapitel in Kopfzeile
\lhead{\slshape \MakeUppercase{\leftmark}}
\renewcommand{\headrulewidth}{0.4pt}

\newpage
\chapter{Einleitung}

\section{Motivation}

Gegenwärtig existieren diverse Netzwerke, welche das Merkmal von miteinander über Kanten verbundenen Knoten erfüllen, wie beispielsweise Straßennetze, Stromnetze oder das menschliche Gehirn. Der Ursprung der Netzwerkforschung liegt jedoch in der Sozialwissenschaft. Der Soziologe Georg Simmel sowie dessen Schüler Leopold von Wiese beschäftigten sich erstmals mit Beziehungen und Netzwerken anhand von sozialen Gefügen. Simmel bezeichnet im Jahr 1908 die Soziologie als "`Geometrie sozialer Beziehungen"' \parencite{simmel_soziologie.:_2013}. Von Wiese beschreibt im Jahr 1933 die formale Soziologie als "`ein scheinbar undurchdringliches Netz von Linien [...], die von Punkten (Menschen) ausgehen"' \parencite{wiese_system_1966}. Im darauffolgenden Jahr 1934 erklärt der Soziologe Jacob Moreno, dass die Verbindungen zwischen Knoten innerhalb eines Netzwerks nicht gleichmäßig seien \parencite{moreno_grundlagen_2014}, so dass zentrale Knoten mit vielen von ihnen abgehenden Verbindungen sowie zahlreiche Knoten mit wenigen abgehenden Verbindungen existieren. \parencite{lenzen_alles_2016, kneer_handbuch_2009}\\

Unabhängig von diesen Erkenntnissen veröffentlicht der ungarische Schriftsteller Frigyes Karinthy im Jahr 1929 seine Erzählung "`Kettenglieder"' (Chain-Links), in der er als eine Art Gesellschaftsspiel die These aufstellt, dass jeder Mensch auf der Erde mit jedem anderen Menschen über maximal fünf Bekanntschaften verbunden sei \parencite{karinthy_chain-links_1929}. Diese These greift der amerikanische Psychologe Stanley Milgram im Jahr 1967 auf. Mit einem Experiment, bei dem Briefe durch Probanden über deren Bekanntschaften an zufällig ausgewählte Personen großer Unternehmen oder Parteien zugestellt werden sollen, kann er Karinthys These belegen \parencite{milgram_small-world-problem_1967}. Jedoch ergibt das Experiment, dass durchschnittlich nicht maximal fünf, sondern sechs Verbindungen benötigt werden. Milgram prägt damit den Begriff "`Kleine-Welt-Phänomen"'. \parencite{lenzen_alles_2016}

\newpage

In den folgenden Jahren wird das Kleine-Welt-Phänomen als Netzwerk vielfach untersucht und sogar auf das menschliche Gehirn angewendet. Ein elementares Ziel der Netzwerkforschung stellt bis heute die Erforschung des menschlichen Gehirns dar, welches mit rund $10^{14}$ Verbindungen zwischen den etwa $10^{11}$ Neuronen bekannt ist als das komplexeste natürliche Netzwerk \parencite{ertel_grundkurs_2013}. Während beispielsweise die Funktion verschiedener Hirnareale oder die Kommunikation zwischen Neuronen mittels biochemischen Transmittern ergründet sind, stellt das Verständnis über Wissens- und Persönlichkeitsspeicherung im Gehirn jedoch noch eine enorme Herausforderung dar. Mit Hilfe der Netzwerktheorie ist es möglich, hochkomplexe Zusammenhänge zu untersuchen und besser zu verstehen. Daher dient die Netzwerktheorie nicht nur konventionellen Netzwerken, wie Straßen- oder Stromnetzen, sondern auch solchen, deren Wechselwirkungen noch nicht gänzlich ergründet sind. \parencite{lenzen_alles_2016}\\

Grundlage dieser Arbeit ist ein innovativer Netzwerkansatz - das Verfahren der \acl{TDS} (\acs{TDS}), welches verschiedene physiologische Systeme des menschlichen Körpers im Schlaf als Netzwerk betrachtet. Auf dieser Grundlage kann die \acs{TDS}-Analyse das Zusammenwirken der einzelnen Systeme im nicht wachen Zustand abbilden. In der Schlafmedizin spielen die Signalaufzeichnungen der physiologischen Systeme eine entscheidende Rolle, um Schlafstadien zu klassifizieren und schlafbezogene Krankheiten zu diagnostizieren. Die exakten Wechselwirkungen zwischen den Systemen und deren Einfluss auf den Schlaf sind jedoch bislang noch weitgehend unerforscht, nicht zuletzt aufgrund der offenen Fragen in der Gehirnforschung. Dieses neue Verfahren der \acs{TDS} eröffnet daher bislang ungenutzte Möglichkeiten der Untersuchung von Zusammenhängen einzelner Körperfunktionen im Schlaf und kann ebenfalls Ansätze für ein profunderes Verständnis des menschlichen Gehirns liefern. Insbesondere in der Schlafmedizin erweitert die \acs{TDS}-Analyse die bisherigen Methoden der Diagnostik, Behandlung und Patientenüberwachung. \parencite{bashan_network_2012, penzel_schlafstorungen_2005}

\section{Aufgabenstellung und Zielsetzung}

Es sollen mit Hilfe der \acs{TDS}-Analyse die Zusammenhänge zwischen Körperfunktionen im Schlaf untersucht werden. Zu diesem Zweck werden multivariate Biosignalaufzeichnungen aus dem Schlaflabor der Charité Berlin verwendet. Nachdem Untersuchungen der Signalaufzeichnungen gesunder Probanden anhand der \acs{TDS}-Analyse bereits eindeutige Ergebnisse erzielten \parencite{bashan_network_2012} und das Verfahren als robust gegenüber Artefakten bezeichnet werden kann \parencite{breuer_netzwerktopologie_2016}, sind nunmehr Untersuchungen in Hinblick auf verschiedene Krankheitsbilder in der Schlafmedizin von Interesse. 

\newpage

Zu diesem Zweck soll der Fokus dieser Arbeit auf die Anwendung des \acs{TDS}-Verfahrens auf Signalaufzeichnungen von Insomniepatienten gelegt werden, welche krankheitsbedingt an Ein- und Durchschlafstörungen leiden \parencite{mayer_s3-leitlinie_2009}. Hierbei werden die Schlafstadien Wachzustand, Leicht- und Tiefschlaf sowie \acl{REM} Schlaf (\acs{REM}) näher betrachtet. Darüber hinaus sollen Untersuchungen zur Geschlechts- und Altersabhängigkeit angestellt werden.\\

Ziel ist es, die Ergebnisse dieser Arbeit mit den bisherigen Erkenntnissen aus den Untersuchungen gesunder Probanden zu vergleichen und ggf. Hinweise für krankheitsbedingte Netzwerkmerkmale zu extrahieren. Darüber hinaus sollen konkrete Aussagen über die Geschlechts- und Altersabhängigkeit der \acs{TDS} bei Insomniepatienten getroffen werden. 

\section{Gliederung der Arbeit}

Nach Klärung der Motivation, Aufgabenstellung und Zielsetzung wird im Folgenden zunächst auf die medizinischen und methodischen Grundlagen eingegangen, welche die Basis für die durchzuführenden Untersuchungen bilden. Anschließend wird die Thematik in den aktuellen Forschungsstand eingebunden. Daran anschließend werden die Daten sowie die verwendeten Untersuchungsumgebungen und -methoden vorgestellt. Des Weiteren wird das Vorgehen der Untersuchungen beschrieben. Nach Ausführung der Ergebnisse bilden Zusammenfassung, Fazit und Ausblick den Abschluss dieser Arbeit. 

\newpage
\chapter{Grundlagen}


\section{Medizinische Grundlagen}\label{medgrundlagen} 


\subsection{Schlafmedizin}\label{schlafmedizin} 

Früher wurde Schlaf als ein Zustand definiert, welcher durch die Abwesenheit von Wachheit geprägt ist. Aufgrund der durch die Elektrophysiologie errungenen Erkenntnisse, welche auf den elektrochemischen Prozessen des Zentralnervensystems beruhen, konnten dem Schlaf mit der Zeit jedoch spezifische Merkmale zugewiesen werden. Vor allem das durch Hans Berger im Jahr 1929 entwickelte \acl{EEG} (\acs{EEG}) zur Messung von Gehirnströmen bildete die Grundlage der frühen Schlafmedizin, so dass beispielsweise die Schlafregulation, verschiedene Schlafphasen oder auch damit verbundene Veränderungen im menschlichen Organismus definiert werden konnten. \parencite{ebner_eeg_2006, penzel_schlafstorungen_2005}\\

Der Schlaf wird demnach zum einen durch die vom Hypothalamus gesteuerte "`Innere Uhr"' des Menschen reguliert und zum anderen durch den "`Hell-Dunkel-Rhythmus"', der die Innere Uhr direkt beeinflusst. Während bei Tageslicht der wachheitsfördernde Neurotransmitter Serotonin freigesetzt wird, wird aus ihm bei zunehmender Dunkelheit das schlaffördernde Hormon Melatonin gebildet \parencite{steinberg_schlafmedizin_2010}. Darüber hinaus ist bekannt, dass Veränderungen im Muskeltonus, in Atmung, Kreislauf und Verdauung direkt an den Schlaf-Wach-Rhythmus sowie an die einzelnen Schlafphasen gekoppelt sind. Andere biologische Prozesse, wie die Körperkerntemperatur, sind fest an die Innere Uhr und den 24-Stunden-Rhythmus gebunden, unabhängig von Schlaf oder Wachzustand. \parencite{penzel_schlafstorungen_2005, rasche_update_2003}\\

Obwohl Methoden zur Ableitung physiologischer Funktionen bereits in der ersten Hälfte des 20. Jahrhunderts entwickelt worden sind, wurde die Schlafmedizin erst nach dem 2. Weltkrieg und hauptsächlich ab den 1980er Jahren verstärkt untersucht. Heutzutage existieren in Deutschland mehrere Hundert akkreditierte Schlaflabore, die Relevanz der Schlafmedizin muss jedoch weiterhin durch Forschung, Lehre und Entwicklung verstärkt vertieft werden.\parencite{penzel_schlafstorungen_2005}\\

Der Stellenwert der Schlafmedizin zeigt sich zum einen darin, dass die Erholungsfunktion des Schlafs bislang noch nicht hinreichend erklärt werden kann. Zum anderen leiden Untersuchungen zufolge etwa 15 \% bis 35 \% der deutschen Bevölkerung an Ein- und Durchschlafstörungen. Frauen sind hiervon stärker betroffen als Männer und die Verbreitung steigt mit dem Alter an (Abb. \ref{fig:ein-durchschlafstörung}). Eine Schlafstörung wird oft erst durch eine Beeinträchtigung der Tagesbefindlichkeit deutlich. Schwerwiegende Schlafstörungen, die nicht auf mangelnde Schlafhygiene\footnote{mangelnde Schlafhygiene: beispielsweise zu ausgedehnter Mittagsschlaf oder unregelmäßige Aufsteh- und Zubettgehzeiten} zurückzuführen sind, beeinflussen die Erholungsfunktion des Schlafes und können sich maßgeblich auf die Lebensqualität auswirken, z. B. durch Befindensstörungen oder Leistungseinschränkungen. So sind beispielsweise 30 \% der Autounfälle in Deutschland auf Müdigkeit am Steuer zurückzuführen. \parencite{mayer_s3-leitlinie_2009, happe_schlafmedizin_2009, penzel_schlafstorungen_2005}

\begin{figure}[H]
	\centering
	\includegraphics[width = \textwidth]{img/Ein-Durchschlafstorung.png}
	\caption[Prävalenz von Ein- und Durchschlafstörungen]{Prävalenz von Ein- und Durchschlafstörungen:\\a) Prävalenz von Einschlafstörungen 2013 nach Alter und Geschlecht (unter Verwendung von \parencite{schlack_haufigkeit_2013});\\b) Prävalenz von Durchschlafstörungen 2013 nach Alter und Geschlecht(unter Verwendung von  \parencite{schlack_haufigkeit_2013})}
	\label{fig:ein-durchschlafstörung}
\end{figure}

\subsection{Polysomnographie}\label{psg} 

Um Schlafstörungen und den Einfluss einzelner physioligischer Systeme auf den Schlaf zu untersuchen, werden \acl{PSG}n (\acs{PSG}) erstellt. Diese stellen Langzeitbiosignalaufzeichnungen dar, im Rahmen derer verschiedene Körperaktivitäten elektrophysiologisch über Nacht abgeleitet und digital aufgezeichnet werden (Abb. \ref{fig:beispiel-psg}). Die Aufnahmen erfolgen zumeist über zwei Nächte im Schlaflabor. Dies hat den Hintergrund, dass der Patient in der ersten Nacht aufgrund der ungewohnten Umgebung sowie der Verkabelung oft unruhiger schläft und dadurch die Aufzeichnungen falsch gedeutet werden können. Dieses Phänomen ist auch als "`Erste-Nacht-Effekt"' bekannt und konnte in aktuellen Untersuchungen auf eine erhöhte Alarmbereitschaft der linken Hemisphäre während des Tiefschlafs zurückgeführt werden \parencite{tamaki_night_2016}. \\

\begin{figure}[H]
	\centering
	\includegraphics[width = \textwidth]{img/Beispiel-PSG.png}
	\caption[Beispielhafte \acs{PSG}]{Beispielhafte Aufzeichnung von Körperfunktionen in einer \acs{PSG} und Andeutung der Elektroden- und Sensorenpositionen am Körper \parencite{penzel_schlafstorungen_2005}}
	\label{fig:beispiel-psg}
\end{figure}

Die \acl{AASM} (\acs{AASM}) regt die Verwendung standardisierter Regeln für die Auswertung von Schlaf an. Hierzu zählen Empfehlungen über die zu verwendenden Ableitungen, die entsprechenden Ableitungsmethoden sowie Abtastfrequenzen \parencite{iber_aasm_2007}. Im Folgenden werden diese abzuleitenden Signale vorgestellt. Die Ausführungen beziehen sich auf den Schlaf eines jungen, normal schlafenden Erwachsenen.\\

%---------------
%------EEG------
%---------------
\textbf{\acs{EEG}}

Das \acs{EEG} misst Potenzialschwankungn in der Großhirnrinde. Gemäß den Kriterien der \acs{AASM} werden die Elektroden nach dem 10-20-System frontal, zentral und okzipital mit einem Abstand von 10 \% oder 20 \% des Kopfes auf beiden Hemisphären angebracht (Abb. \ref{fig:10-20jasper}) \parencite{iber_aasm_2007}. Die Signale der linken Hemispähre werden über die rechte Mastoid\footnote{Mastoid: Warzenfortsatz hinter der Gehörgangswand}-Elektrode und umgekehrt abgeleitet (A1 und A2 in Abb. \ref{fig:10-20jasper}). In der Praxis werden anstelle der frontalen Elektroden F3 und F4 häufig auch die vorgelagerten frontopolaren Elektrodenpositionen Fp1 und Fp2 verwendet, welche sich direkt an der Stirn befinden. Die Signaldynamik zeigt jedoch keine signifikanten Unterschiede zu den frontalen Positionen \parencite{dorffner_effects_2015}. \acs{EEG}-Signale weisen eine stetige Wellenform und je nach Ableitungsposition und Schlafphase Frequenzen zwischen 0.5 Hz und 35 Hz sowie Amplituden im Bereich von 5 $\upmu$V bis 200 $\upmu$V auf \parencite{lee-chiong_sleep_2008}.

\begin{figure}[H]
	\centering
	\includegraphics[scale = 0.7]{img/10-20jasper.png}
	\caption[Elektrodenpositionen des \acs{EEG}]{Elektrodenpositionen für die Ableitung des \acs{EEG} nach dem 10-20-System nach Jasper (1958); die frontalen, zentralen und okzipitalen Positionen werden von der \acs{AASM} empfohlen \parencite{kemp_edf+:_????}}
	\label{fig:10-20jasper}
\end{figure}

Darüber hinaus wird das \acs{EEG} in fünf verschiedene Frequenzbänder eingeteilt. Das Frequenzband der Delta-Wellen liegt im Bereich 0.5 Hz bis 3 Hz. Theta-Wellen weisen eine Frequenz von 4 Hz bis 7 Hz auf. Eine etwas höhere Frequenz ist mit 8 Hz bis 13~Hz bei den Alpha-Wellen zu verzeichnen. Sigma-Wellen liegen im Frequenzbereich von 14 Hz bis 16 Hz. Beta-Wellen weisen dagegen die höchste Frequenz von 17 Hz bis 35 Hz auf. \parencite{lee-chiong_sleep_2008, steinberg_schlafmedizin_2010}\\

%---------------
%------EOG------
%---------------
\textbf{EOG}

Mit Hilfe des \acl{EOG}s (\acs{EOG}) werden die Potenzialschwankungen der Augen zwischen Hornhaut und Retina aufgezeichnet, um Augenrollen, Blinzeln oder Rapid-Eye-Movements im Schlaf zu erkennen. Die Signale werden wie das \acs{EEG} gegen eine Referenzelektrode am Ohr abgeleitet. Die \acs{AASM} gibt die Anbringung der Elektroden an den äußeren Augenwinkeln gemäß Abb. \ref{fig:eog} vor. Eine Augenbewegung in Richtung einer Elektrode verursacht einen positiven Ausschlag, in entgegengesetzter Richtung einen negativen. Bei synchroner binokularer Augenbewegung zeichnet die eine Elektrode demnach einen positiven und die anderen einen negativen Ausschlag auf, so dass sich die Signale ohne Einfluss von Störfaktoren gegenseitig annähernd aufheben (Abb. \ref{fig:beispiel-psg}). Dieses als normal geltende Verhalten wird als "`außer Phase"' bezeichnet. Artefakte, die beispielsweise durch hohe \acs{EEG}-Ausschläge auftreten können, verursachen Ausschläge in nur einem \acs{EOG}-Kanal. Dieses Verhalten wird als "`in Phase"' bezeichnet. \parencite{iber_aasm_2007, lee-chiong_sleep_2008}

\begin{figure}[H]
	\centering
	\includegraphics[scale = 0.6]{img/eog.png}
	\caption[Elektrodenpositionen des \acs{EOG}]{Ableitung des \acs{EOG} an den äußeren Augenwinkeln (links parallel und rechts diagonal) gemäß \acs{AASM} \parencite{iber_aasm_2007}}
	\label{fig:eog}
\end{figure}

%---------------
%------EMG------
%---------------
\textbf{EMG}

Mittels \acl{EMG} (\acs{EMG}) werden Muskelaktivitäten oberhalb des Unterkiefers im Kinn (Abb. \ref{fig:emgchin}) sowie in den vorderen Schienbeinen gemessen. Das \acs{EMG} dient neben dem \acs{EEG} sowie dem \acs{EOG} der Einteilung der Schlafphasen sowie der Diagnose bewegungsbezogener Schlafkrankheiten. Minimale Muskelaktivität von bis zu 15 $\upmu$V ist während eines Großteils des Schlafs messbar, die Amplitudenwerte eindeutiger Bewegungen liegen jedoch mindestens 8 $\upmu$V höher als in Phasen geringer Muskelaktivität. Der Frequenzbereich von Muskelkontraktionen liegt zwischen 0.5 Hz und 3 Hz. \parencite{iber_aasm_2007, leroux_handbuch_2009, lee-chiong_sleep_2008}

\begin{figure}[H]
	\centering
	\includegraphics[scale = 0.6]{img/EMGchin.jpg}
	\caption[Elektrodenpositionen des \acs{EMG}]{Ableitung des \acs{EMG} am Kinn gemäß \acs{AASM} \parencite{leroux_handbuch_2009}}
	\label{fig:emgchin}
\end{figure}

%---------------
%------EKG------
%---------------
\textbf{EKG}

Mittels \acl{EKG} (\acs{EKG}) werden die Spannungspotenziale der Herzaktivität gemessen. Die Extremitätenmessung erfolgt entweder an beiden Armen oder an einem Arm und einem Bein. Die \acs{AASM} empfiehlt die Platzierung der Elektroden am rechten Arm und linken Bein und zusätzlich eine Brustwandableitung (Abb. \ref{fig:ekg}). Je nach Elektrodenposition äußert sich die Aufzeichnung des Herzschlags jedoch unterschiedlich (Abb. \ref{fig:qrs_extr_brustw}b)/ \ref{fig:qrs_extr_brustw}c)). \parencite{bamberger_1x1_2015, iber_aasm_2007}

\begin{figure}[H]
	\centering
	\includegraphics[scale = 0.5]{img/ekg.png}
	\caption[Elektrodenpositionen des \acs{EKG}]{Brustwandableitung (links) und Extremitätenableitung (rechts) gemäß \acs{AASM} \parencite{leroux_handbuch_2009}}
	\label{fig:ekg}
\end{figure}

Das Herz eines gesunden, jungen Erwachsenen schlägt etwa 60- bis 80-mal pro Minute. Bei einem langsameren Herzschlag liegt eine Bradykardie vor, bei einer Frequenz von mehr als 100 Schlägen pro Minute eine Tachykardie. Die Frequenz eines gesunden Herzschlags liegt demnach zwischen 1 Hz und 1,6 Hz. Das \acs{EKG} weist eine typische Zackenform auf, bestehend aus einer P- und T-Welle sowie dem QRS-Komplex, welche die Ausbreitung einer Herzerregung darstellen (Abb. \ref{fig:beispiel-psg} und \ref{fig:qrs_extr_brustw}a)). Die P-Welle entspricht der Erregungsausbreitung in den Herzvorhöfen, der QRS-Komplex stellt hingegen die Erregungsausbreitung in den Herzkammern dar. Die Erregungsrückbildung in den Herzkammern wird durch die T-Welle abgebildet. \parencite{bamberger_1x1_2015, oresick_kania_ekg_2016}\\

\begin{figure}[H]
	\centering
	\includegraphics[width = \textwidth]{img/qrs_extr_brustw.png}
	\caption[Ableitung des \acs{EKG}]{Ableitung des \acs{EKG}:\\a) P-Welle, QRS-Komplex und T-Welle als typische Ausprägung des Herzschlags im \acs{EKG} \parencite{oresick_kania_ekg_2016};\\b) Extremitätenableitung mit unterschiedlichen Elektrodenpositionen;\\c) Brustwandableitung mit unterschiedlichen Elektrodenpositionen;\\ein Kästchen in b) und c) entspricht 500 $\upmu$V \parencite{bamberger_1x1_2015}}
	\label{fig:qrs_extr_brustw}
\end{figure}

%---------------
%----Atmung-----
%---------------
\textbf{Atmung}

Gemäß den Kriterien der \acs{AASM} wird die Aufzeichnung von Atemfluss sowie Atemanstrengung empfohlen. Die Messung des Atemflusses soll die Unterschiede zwischen In- und Exspiration aufzeigen, um festzustellen, ob im Schlaf atmungsbezogene Störungen auftreten. Hierzu werden Temperatur- oder Drucksensoren empfohlen, welche die nasale oder oronasale\footnote{oronasal: Mund und Nase betreffend} Atmung aufzeichnen. Die Messung der Atemanstrengung verfolgt das gleiche Ziel, erfolgt gemäß den Empfehlungen der \acs{AASM} jedoch entweder mittels Ösophagusmanometrie\footnote{Ösophagusmanometrie: Druckmessung in der Speiseröhre} in der Speiseröhre oder äußerlich mittels um Brust und Bauch gelegte Gurte. Bei beiden Methoden werden die durch die Atmung hervorgerufenen Druckunterschiede gemessen. Eine gesunde Atmung erzeugt im Signal eine gleichmäßige Wellenform (Abb. \ref{fig:beispiel-psg}), wobei die Atemfrequenz eines ruhenden Erwachsenen mit etwa zwölf bis 18 Atemzügen pro Minute bei 0,2 Hz bis 0,3 Hz liegt. \parencite{iber_aasm_2007, lee-chiong_sleep_2008, leroux_handbuch_2009, steffel_lunge_2014}\\

%---------------
%----Weitere----
%---------------
\textbf{Weitere Signale}

Die \acs{AASM} empfiehlt darüber hinaus die Ableitung weiterer Signale. Hierzu zählen die Sauerstoffsättigung, Körperlage sowie Schnarchgeräusche. Da die in dieser Arbeit verwendete \acs{TDS} Analyse lediglich auf den zuvor beschriebenen Signalen beruht, wird auf die weiteren Signale nicht näher eingegangen. In der Praxis werden darüber hinaus je nach Krankheitsverdacht und Untersuchung häufig noch andere Signale abgeleitet, beispielsweise weitere \acs{EEG}s, so dass \acs{PSG}s zumeist 15 bis 40 Signale enthalten. \\

Die Qualität der Signale innerhalb einer \acs{PSG} ist nicht in jedem Schlaflabor und bei jedem Patienten identisch, sondern hängt vielmehr von den Aufnahmegeräten, Ableitungsmethoden (Abb. \ref{fig:qrs_extr_brustw}b) und \ref{fig:qrs_extr_brustw}c)) und Störfaktoren ab. Um die digitale Analyse der Daten zu vereinfachen, wurde das \acl{EDF} (\acs{EDF}) im Jahr 1987 speziell für die Schlafmedizin entwickelt und im Jahr 2001 durch das EDF+ erweitert \parencite{kemp_european_????}. Dieses Datenformat definiert Standards zur Speicherung und zum Datenaustausch von Biosignalen. Es regelt die Aufteilung der aufgezeichneten Daten in ein Header Record sowie mehrere Data Records entsprechend der Anzahl der in der \acs{PSG} enthaltenen Signale. Im Header Record werden in 256 Bytes allgemeine Informationen über den Patienten sowie technische Aufnahmeeigenschaften gespeichert (Tab.~\ref{tab:edf_header}). Hierzu zählen insbesondere die Patienten-ID, Startzeitpunkt der Aufnahme, Dauer und Anzahl der Data Records sowie die Anzahl der enthaltenen Signale. In den Data Records werden jeweils in den ersten 256 Bytes allgemeine Informationen über jedes aufgezeichnete Signal gespeichert, wie beispielsweise die Signalbezeichnung (Label), die physikalische und digitale Auflösung sowie die Anzahl der aufgenommenen Signalwerte (Samples) (Tab. \ref{tab:edf_data}). Die einzelnen Signale selbst werden anschließend zum zugehörigen Data Record gespeichert. Darüber hinaus werden im Rahmen dieses Format-Standards bestimmte Labels für die Signalbezeichnungen vorgegeben. \parencite{kemp_european_2003}

\begin{table}[H]
\begin{small}
\centering
\begin{tabular}{|l|c|}
\hline
\multicolumn{1}{|c|}{\textbf{Header Record}}     & \textbf{Beispieldaten}                                                                                   \\ \hline
8 ascii : version of this data format (0)        & 0                                                                                                        \\ \hline
80 ascii : local patient identification          & \begin{tabular}[c]{@{}c@{}}MCH-0234567 F 02-MAY-1951\\ Haagse\_Harry\end{tabular}                        \\ \hline
80 ascii : local recording identification.       & \begin{tabular}[c]{@{}c@{}}Startdate 02-MAR-2002 EMG561 BK/\\ JOP Sony. MNC R Median Nerve.\end{tabular} \\ \hline
8 ascii : startdate of recording (dd.mm.yy)      & 17.04.01                                                                                                 \\ \hline
8 ascii : starttime of recording (hh.mm.ss).     & 11.25.00                                                                                                 \\ \hline
8 ascii : number of bytes in header record       & 768                                                                                                      \\ \hline
44 ascii : reserved                              & EDF+D                                                                                                    \\ \hline
8 ascii : number of data records (-1 if unknown) & 1200                                                                                                        \\ \hline
8 ascii : duration of a data record, in seconds  & 1.00                                                                                                    \\ \hline
4 ascii : number of signals (ns) in data record  & 2                                                                                                        \\ \hline
\end{tabular}
\caption[Header Record einer EDF-Datei]{Aufteilung des Header Records mit Beispiel-Headerdaten (unter Verwendung von \parencite{kemp_edf+_????})}
\label{tab:edf_header}
\end{small}
\end{table}

\begin{table}[H]
\begin{small}
\centering
\begin{tabular}{|l|c|c|}
\hline
\multicolumn{1}{|c|}{\textbf{Data Record}}                                                              & \textbf{Signal 1} & \textbf{Signal 2} \\ \hline
ns * 16 ascii : ns * label                                                                              & EEG C4-A1         & ECG               \\ \hline
\begin{tabular}[c]{@{}l@{}}ns * 80 ascii : ns * transducer type \\ (e.g. AgAgCl electrode)\end{tabular} & AgAgCl electrodes &                   \\ \hline
ns * 8 ascii : ns * physical dimension (e.g. uV)                                                        & uV                & uV                \\ \hline
ns * 8 ascii : ns * physical minimum (e.g. -500 or 34)                                                  & -300              & -1000             \\ \hline
ns * 8 ascii : ns * physical maximum (e.g. 500 or 40)                                                   & 300               & 1000              \\ \hline
ns * 8 ascii : ns * digital minimum (e.g. -2048)                                                        & -32768            & -32768            \\ \hline
ns * 8 ascii : ns * digital maximum (e.g. 2047)                                                         & 32767             & 32767             \\ \hline
\begin{tabular}[c]{@{}l@{}}ns * 80 ascii : ns * prefiltering \\ (e.g. HP:0.1Hz LP:75Hz)\end{tabular}    & HP:0.5Hz LP:50Hz  &                   \\ \hline
ns * 8 ascii : ns * nr of samples in each data record                                                   & 256               & 256               \\ \hline
ns * 32 ascii : ns * reserved                                                                           &                   &                   \\ \hline
\end{tabular}
\caption[Data Record einer EDF-Datei]{Aufteilung des Data Records mit Beispiel-Signaldaten (unter Verwendung von \parencite{kemp_edf+_????, kemp_edf+:_????})}
\label{tab:edf_data}
\end{small}
\end{table}

Dieser Standard ist zwar allgemeingültig, jedoch wird er nicht in jedem Schlaflabor verwendet. Viele Aufnahmegeräte unterstützen lediglich ein proprietäres Datenformat, wodurch die Vergleichbarkeit von \acs{PSG}s zunächst beeinträchtigt wird. Darüber hinaus werden die Labels manuell gesetzt. Um umfangreiche Studien mit Aufzeichnungen aus verschiedenen Schlaflaboren durchführen zu können, ist meist eine Konvertierung in \acs{EDF} und eine Standardisierung der Labels erforderlich. 

\subsection{Insomnie}\label{insomnie} 

Zur Klassifikation verschiedener schlafbezogener Krankheiten dient der internationale Katalog über Schlafstörungen "`\acl{ICSD-2}"' (\acs{ICSD-2}), welcher die bislang etwa 80 bekannten primären Schlafstörungen nach empirischen und pragmatischen Merkmalen in acht Gruppen einteilt. Diese gliedern sich in Insomnien, schlafbezogene Atmungsstörungen, Hypersomnien\footnote{Hypersomnie: Tagesschläfrigkeit}, Schlaf-Wach-Rhythmus-Störungen, Parasomnien\footnote{Parasomnie: Störung beim Erwachen (Arousal) oder bei Schlafstadienwechsel}, schlafbezonene Bewegungsstörungen sowie andere Schlafstörungen. In dieser Arbeit relevant sind die Insomnien, welche im Folgenden näher beschrieben werden.\parencite{happe_schlafmedizin_2009, american_academy_of_sleep_medicine_international_2005}\\

Gemäß \acs{ICSD-2} werden als Symptome einer Insomnie Ein- und Durchschlafstörungen oder nicht erholsamer Schlaf über mindestens einen Monat und mindestens dreimal pro Woche bezeichnet. Symptomatisch sind hierbei eine reduzierte Gesamtschlafzeit, eine verlängerte Einschlaflatenz und eine erhöhte Anzahl an Aufwachmomenten (Arousals). Zusätzlich muss sich die Schlafeffizienz negativ auf die Tagesbefindlichkeit oder tägliche Atktivitäten auswirken, beispielsweise in Form von Konzentrations- oder Motivationsschwierigkeiten, Tagesschläfrigkeit, sozialen oder beruflichen Einschränkungen oder auch Befindensbeeinträchtigungen. Gemäß \acs{ICSD-2} erfolgt eine Unterteilung in zehn unterschiedliche, nach Ursachen und Symptomen klassifizierte, Insomnietypen (primäre\footnote{primäre Insomnien: idiopathische, psychophysiologische oder paradoxe Insomnie (vgl. Anhang Tab. \ref{tab:idiopathische_insomnie}, \ref{tab:psycho_insomnie}, \ref{tab:paradoxe_insomnie}}, infolge äußerer Einflüsse\footnote{Insomnien infolge äußerer Einflüsse: hervorgerufen durch z. B. Hitze, Kälte, Lärm, Vibration, Gebrauch von Genussmitteln und Pharmaka sowie andere verhaltensabhängige Faktoren (vgl. Anhang Tab. \ref{tab:akute_insomnie}, \ref{tab:schlafhygiene_insomnie})} oder symptomatische Insomnien\footnote{symptomatische Insomnien: bei vorbestehenden körperlichen oder psychiatrischen Erkrankungen (vgl. Anhang Tab. \ref{tab:insomnie_psychisch}, \ref{tab:korperliche_insomnie})}, vgl. Anhang Tab. \ref{tab:allgemeine_insomnie} bis \ref{tab:schlafhygiene_insomnie}). In der Regel wird die Insomnie anhand einer umfangreichen, auf Fragebögen und Schlaftagebüchern basierenden Anamnese diagnostiziert. In komplizierteren oder chronischen Fällen, bei therapierefraktärer\footnote{therapierefraktär: nicht auf die Therapie anschlagend} Insomnie oder zur Abklärung gegenüber anderen Schlafstörungen kann eine \acs{PSG} jedoch indiziert sein. \parencite{happe_schlafmedizin_2009, american_academy_of_sleep_medicine_international_2005, mayer_s3-leitlinie_2009}\\

Eine Studie zur Gesundheit Erwachsener in Deutschland aus dem Jahr 2013 ergab, dass 5,7 \% der Befragten (7,7 \% der Frauen und 3,8 \% der Männer) an eindeutigen Symptomen einer Insomnie leiden (Abb. \ref{fig:praevalenz}). Hierbei ist zu beobachten, dass das Auftreten einer Insomnie im Alter von 40 bis 59 Jahren bei Frauen und Männern am höchsten ist und Frauen stärker betroffen sind (Abb. \ref{fig:praevalenz}). Die Prävalenz der Insomnie in Deutschland von 5,7 \% ist vergleichbar mit anderen Ländern.\footnote{In Spanien wurde eine Prävalenzrate von 6,4 \% ermittelt, in Italien von 7 \%. Deutlich höhere Prävalenzraten zeigen jedoch beispielsweise Kanada mit 9,5 \% und Großbritannien mit 22 \%.} Eine epidemiologische Studie zeigt hingegen eine weltweite Prävalenz der Insomnie von durchschnittlich 10 \% bis 30 \%\parencite{ohayon_epidemiological_2011}. Eine Untersuchung im Schlaflabor ist bei etwa 1 \% der Gesamtbevölkerung erforderlich \parencite{penzel_schlafstorungen_2005}.
\parencite{schlack_haufigkeit_2013, robert_koch_institut_gesundheit_2015}

\begin{figure}[H]
	\centering
	\includegraphics[scale = 0.5]{img/Praevalenz.png}
	\caption[Prävalenz von Insomnie]{Prävalenz von Insomnie in Deutschland im Jahr 2011 (unter Verwendung von \parencite{schlack_haufigkeit_2013})}
	\label{fig:praevalenz}
\end{figure}

Eine weitere Untersuchung aus dem Jahr 2009 an 1476 Erwerbstätigen im Alter zwischen 35 und 65 Jahren ergab, dass wiederholtes Aufwachen (37,5 \%) und zu kurzer Nachtschlaf (32,6 \%) als häufigste subjektiv wahrgenommenen Symptome der Insomnie berichtet werden (Abb.~\ref{fig:symptome}). Nicht erholsamer Schlaf, Nicht-Durchschlafen-Können, sehr frühes Erwachen oder spätes Einschlafen werden von durchschnittlich rund 26~\% der Studienteilnehmer angegeben. Am seltensten (7,5 \%) wurde die Angst, nicht einschlafen zu können, als Symptom genannt. \parencite{dak_forschung_gesundheitsreport_2010}

\begin{figure}[H]
	\centering
	\includegraphics[scale = 0.5]{img/Symptome.png}
	\caption[Symptome der Insomnie]{Symptome einer Insomnie bei Erwerbstätigen in Deutschland im Jahr 2009 (unter Verwendung von \parencite{dak_forschung_gesundheitsreport_2010})}
	\label{fig:symptome}
\end{figure}

\subsection{Schlafphasen}\label{schlafphasen} 

Der Schlaf lässt sich anhand seiner Merkmale in den Biosignalen in unterschiedliche Schlafphasen einteilen. Nach Rechtschaffen und Kales erfolgt die Klassifizierung in insgesamt sechs Stadien (Wachzustand, \acs{REM}-Schlaf sowie vier \acl{NREM} (\acs{NREM}) Schlafphasen\footnote{\acs{NREM}1 und \acs{NREM}2 gelten als Leichtschlaf, \acs{NREM}3 und \acs{NREM}4 als Tiefschlaf.}) \parencite{rechtschaffen_manual_1968}. Seit dem Jahr 2007 empfiehlt die \acs{AASM} jedoch eine Einteilung in Wachzustand, \acs{REM}, \acs{NREM}1, \acs{NREM}2 sowie \acs{NREM}3, wobei \acs{NREM}3 die zwei Tiefschlafphasen nach Rechtsschaffen und Kales zusammenfasst. Die Schlafphasen werden während der Nacht durchschnittlich fünfmal zyklisch durchlaufen, wobei ein Zyklus etwa 90 Minuten andauert und typischerweise eine Sequenz den Übergang von Leichtschlaf zum Tiefschlaf und anschließendem REM-Schlaf darstellt. Sämtliche durchlaufenen Zyklen und Schlafphasen werden durch das Hypnogramm erfasst (Abb. \ref{fig:hypnogram}). Dieses wird entweder automatisch vom Aufnahmesystem generiert oder manuell eingeteilt. Computerbasierte Einteilungen der Schlafphasen erzielen hierbei eine Übereinstimmung mit der menschlichen Klassifizierung von etwa 80 \% \parencite{rasche_update_2003, penzel_computer_2000}. \parencite{happe_schlafmedizin_2009, iber_aasm_2007}

\begin{figure}[H]
	\centering
	\includegraphics[width = \textwidth]{img/hypnogram.png}
	\caption[Hypnogram von gesundem Schlaf]{Hypnogramm eines 32-jährigen gesunden Schläfers \parencite{happe_schlafmedizin_2009}:\\"`Awake"' entspricht dem Wachzustand;\\"`Stage REM"' entspricht dem \acs{REM}-Schlaf;\\"`Stage 1"' bis "`Stage 4"' entsprechen den \acs{NREM}-Schlafphasen gemäß \parencite{rechtschaffen_manual_1968}}
	\label{fig:hypnogram}
\end{figure}

Die Schlafphasen werden anhand der Ausprägungen und Frequenzen in den \acs{EEG}-, \acs{EOG}- sowie \acs{EMG}-Signalen klassifiziert. Im Wachzustand (Tab. \ref{tab:wach}) herrschen im \acs{EEG} Sigma- und Beta-Wellen vor, bei geschlossenen Augen vorwiegend Alpha-Wellen. Im \acs{EOG} und \acs{EMG} können Bewegungen und erhöhte Frequenzen beobachtet werden. Der Leichtschlaf (\acs{NREM}1 und \acs{NREM}2; Tab. \ref{tab:leichtschlaf}) wird klassifiziert durch niedrigamplitudige, gemischte Aktivität sowie Schlafspindeln\footnote{Schlafspindeln: Stoß von schnellen Wellen (12 Hz bis 14 Hz) mit einer Dauer von über 0.5 Sekunden (vgl. Anhang Abb. \ref{fig:spindel_k-komplex}) \parencite{lee-chiong_sleep_2008}} und K-Komplexe\footnote{K-Komplex: spitzer negativer Ausschlag mit anschließender positiver Welle mit einer Dauer von über 0.5 Sekunden (vgl. Anhang Abb. \ref{fig:spindel_k-komplex}) \parencite{lee-chiong_sleep_2008}} im \acs{EEG}. Im Stadium \acs{NREM}1 treten langsame, rollende Augenbewegungen auf. Geringe Muskelaktivität ist im Stadium \acs{NREM}2 messbar. Der Tiefschlaf (Tab. \ref{tab:tiefschlaf}) ist geprägt durch vorwiegend langsame, hochamplitudige Delta-Wellen im \acs{EEG}, keine Augenbewegungen sowie stark herabgesetzte Muskelaktivität im Kinn. Der \acs{REM}-Schlaf (Tab. \ref{tab:rem}) hingegen ist charakterisiert durch niederamplitudige, gemischte \acs{EEG}-Aktivität, schnelle Augenbewegungen sowie keinen oder lediglich sehr geringen Muskeltonus. Da im Rahmen des \acs{TDS} Verfahrens die Herz- und Atemaktivität ebenfalls als Netzwerksysteme betrachtet werden, sind diese in den Tab. \ref{tab:wach} bis \ref{tab:rem} mit erfasst. Diese beiden Systeme zeigen eine Frequenzreduzierung von Wachzustand zum Tiefschlaf. \parencite{lee-chiong_sleep_2008, steinberg_schlafmedizin_2010, rasche_update_2003, ebner_eeg_2006}

\begin{table}[H]
\begin{small}
\centering
\begin{tabular}{|l|p{13cm}|}
\hline
\textbf{Signal} & \textbf{Klassifizierungsmerkmale für den Wachzustand} \\
\hline
\acs{EEG} & bei geöffneten Augen Sigma- und Beta-Wellen; bei geschlossenen Augen in Entspannung Alpha-Wellen (am stärksten im okzipitalen Bereich)\\
\hline
\acs{EOG} & Augenbewegungen, Zwinkern und erhöhte Frequenzen möglich\\
\hline
\acs{EMG} & Muskelaktivitäten und erhöhte Frequenzen möglich\\
\hline
\acs{EKG} & Herzfrequenz höher als in allen anderen Schlafphasen\\
\hline
Atmung & unregelmäßige Atemfrequenz möglich\\
\hline
\end{tabular}
\caption[Klassifizierungsmerkmale im Wachzustand]{Klassifizierungsmerkmale in den Biosignalen im Wachzustand (unter Verwendung von \parencite{lee-chiong_sleep_2008, steinberg_schlafmedizin_2010, rasche_update_2003, ebner_eeg_2006})}
\label{tab:wach}
\end{small}
\end{table}

\begin{table}[h]
\begin{small}
\centering
\begin{tabular}{|l|p{13cm}|}
\hline
\textbf{Signal} & \textbf{Klassifizierungsmerkmale für den Leichtschlaf} \\
\hline
\acs{EEG} & hauptsächlich Theta-Wellen und bis zu 20 \% Delta-Wellen; Schlafspindeln möglich (12 Hz bis 14 Hz, zentral am stärksten); K-Komplexe möglich (zentral und zentral-parietal am stärksten)\\
\hline
\acs{EOG} & langsame, rollende Augenbewegungen im \acs{NREM}1 keine Augenbewegungen im \acs{NREM}2\\
\hline
\acs{EMG} & geringe Muskelaktivität im Kinn möglich\\
\hline
\acs{EKG} & Herzfrequenz niedriger als im Wachzustand\\
\hline
Atmung & regelmäßige Atemfrequenz; geringer als im Wachzustand oder gleichbleibend\\
\hline
\end{tabular}
\caption[Klassifizierungsmerkmale im Leichtschlaf]{Klassifizierungsmerkmale in den Biosignalen im Leichtschlaf (unter Verwendung von \parencite{lee-chiong_sleep_2008, steinberg_schlafmedizin_2010, rasche_update_2003, ebner_eeg_2006});}
\label{tab:leichtschlaf}
\end{small}
\end{table}

\begin{table}[h]
\begin{small}
\centering
\begin{tabular}{|l|p{13cm}|}
\hline
\textbf{Signal} & \textbf{Klassifizierungsmerkmale für den Tiefschlaf} \\
\hline
\acs{EEG} & hauptsächlich Delta-Wellen; Schlafspindeln möglich\\
\hline
\acs{EOG} & keine Augenbewegungen\\
\hline
\acs{EMG} & sehr geringe Muskelaktivität im Kinn möglich\\
\hline
\acs{EKG} & Herzfrequenz niedriger als im Wachzustand\\
\hline
Atmung & regelmäßige Atemfrequenz; geringer als im Wachzustand oder gleichbleibend\\
\hline
\end{tabular}
\caption[Klassifizierungsmerkmale im Tiefschlaf]{Klassifizierungsmerkmale in den Biosignalen im Tiefschlaf (unter Verwendung von \parencite{lee-chiong_sleep_2008, steinberg_schlafmedizin_2010, rasche_update_2003, ebner_eeg_2006})}
\label{tab:tiefschlaf}
\end{small}
\end{table}

\begin{table}[H]
\begin{small}
\centering
\begin{tabular}{|l|p{13cm}|}
\hline
\textbf{Signal} & \textbf{Klassifizierungsmerkmale für den \acs{REM}-Schlaf} \\
\hline
\acs{EEG} & hauptsächlich Theta-Wellen (zentral und temporal am stärksten)\\
\hline
\acs{EOG} & Stöße von schnellen Außer-Phase-Bewegungen der Augen (1 Hz bis 4 Hz)\\
\hline
\acs{EMG} & keine Muskelaktivität; Stöße sehr geringer Muskelaktivität im Kinn möglich\\
\hline
\acs{EKG} & Herzfrequenz niedriger als im Wachzustand und höher als in den \acs{NREM}-Schlafphasen\\
\hline
Atmung & Anstieg oder Schwankungen in der Atemfrequenz möglich\\
\hline
\end{tabular}
\caption[Klassifizierungsmerkmale im \acs{REM}-Schlaf]{Klassifizierungsmerkmale in den Biosignalen im \acs{REM}-Schlaf (unter Verwendung von \parencite{lee-chiong_sleep_2008, steinberg_schlafmedizin_2010, rasche_update_2003, ebner_eeg_2006})}
\label{tab:rem}
\end{small}
\end{table}

Der Anteil des Wachzustands am Gesamtschlaf bei jungen, gesunden Menschen liegt bei maximal 5 \%. Mit zunehmendem Alter können auch die Wachanteile ansteigen, so dass bei Menschen in hohem Alter ein Anteil des Wachzustandes von 20 \% ebenfalls als erwartungskonform gilt. Der Leichtschlaf liegt in jedem Alter bei etwa 45 \% bis 55~\%. Der Tiefschlaf macht einen Anteil von 5 \% bis 20 \% aus, wobei mit zunehmendem Alter eine fallende Tendenz beobachtbar ist. Auch der prozentuale Anteil des \acs{REM}-Schlafs reduziert sich mit dem Alter auf durchschnittlich etwa 15 \%, wobei junge, gesunde Menschen ihren Schlaf zu 20 \% bis 25 \% im \acs{REM}-Schlaf verbringen (Tab. \ref{tab:anteile_schlafphasen}). In der Regel erfolgt der Großteil des Tiefschlafs in der ersten Nachthälfte. In der zweiten Nachthälfte hingegen überwiegen bei gesunden Schläfern Leichtschlaf (\acs{NREM}1 und \acs{NREM}2) und \acs{REM}-Schlaf (Abb. \ref{fig:hypnogram}). \parencite{lee-chiong_sleep_2008, steinberg_schlafmedizin_2010, danker-hopfe_percentile_2005}

\begin{table}[H]
\begin{small}
\centering
\begin{tabular}{|l|c|}
\hline
\textbf{Schlafphase} & \textbf{Prozentualer Anteil} \\
\hline
Wachzustand & 1 \% bis 20 \% \\
\hline
Leichtschlaf (\acs{NREM}1 und \acs{NREM}2) & 45 \% bis 55 \%\\
\hline
Tiefschlaf (\acs{NREM}3) & 5 \% bis 20 \%\\
\hline
\acs{REM}-Schlaf & 15 \% bis 25 \%\\
\hline
\end{tabular}
\caption[Prozentuale Schlafphasenverteilung]{Prozentualer Anteil der Schlafphasen am Gesamtschlaf bei gesunden Menschen aller Altersklassen \parencite{lee-chiong_sleep_2008, steinberg_schlafmedizin_2010, danker-hopfe_percentile_2005}}
\label{tab:anteile_schlafphasen}
\end{small}
\end{table}

Die Schlafphasenverteilung kann bei Insomniepatienten aufgrund reduzierter Gesamtschlafzeit, verlängerter Einschlaflatenz oder vermehrter Aufwachmomente von der Norm abweichen. Tiefschlaf und/oder \acs{REM}-Schlaf können beispielsweise reduziert sein oder vollständig ausbleiben, während Wachanteile und Leichtschlaf deutlich erhöht sein können. Auch die Schlafzyklen können aperiodisch oder gänzlich gestört sein, wie das beispielhafte Hypnogramm in Abb. \ref{fig:hypnogram_insomnia} zeigt. \parencite{happe_schlafmedizin_2009}

\begin{figure}[H]
	\centering
	\includegraphics[width = \textwidth]{img/hypnogram_insomnia.png}
	\caption[Hypnogramm von gestörtem Schlaf]{Hypnogramm einer 47-jahrigen Patientin mit Insomnie durch körperliche Erkrankung: deutlich fragmentiertes Schlafprofil mit verzögertem Einschlafen, Fehlen von Tiefschlaf und reduziertem Gesamtschlaf  \parencite{happe_schlafmedizin_2009}}
	\label{fig:hypnogram_insomnia}
\end{figure}

\section{Stand der Forschung}\label{stand} 

Das vegetative (autonome) Nervensystem ist maßgeblich an der Schlafregulation beteiligt und regelt vor allem in den \acs{NREM}-Schlafphasen unter anderem Herzschlag, Blutdruck sowie die Atmung \parencite{steinberg_schlafmedizin_2010}. Wie die einzelnen Systeme interagieren und welchen Einflüssen sie unterliegen, ist ein wesentlicher Forschungsschwerpunkt der Schlafmedizin. Es ist bereits bekannt, dass einzelne Systeme in den unterschiedlichen Schlafphasen spezifische Merkmale aufweisen. Beispielsweise sinken Herz- und Atemfrequenz vom Wachzustand zum Tiefschlaf kontinuierlich, während sie im REM-Schlaf wieder ansteigen (vgl. Tab. \ref{tab:wach} bis \ref{tab:rem} unter Punkt \ref{schlafphasen}). \parencite{lee-chiong_sleep_2008, rasche_update_2003, penzel_schlafstorungen_2005}\\

Univariate Untersuchungen können verschiedene signalspezifische und schlafphasenabhängige Eigenschaften einzelner Systeme spezifizieren. Mit Hilfe der Trendbereinigenden Fluktuationsanalyse (Detrended Fluctuation Analysis, DFA) zur Quantifizierung von Langzeitkorrelationen in Zeitreihen kann beispielsweise gezeigt werden, dass die Herzfrequenz im \acs{REM}-Schlaf eine ähnlich starke Langzeitkorrelation aufweist wie im Wachzustand. Die Fluktuationen in der Herzfrequenz sind in diesen beiden Schlafphasen demnach gering. In den \acs{NREM}-Schlafphasen hingegen ist die Korrelation schwach und die Fluktuation demnach erhöht. Dies wird bislang darauf zurückgeführt, dass die durch das vegetative Nervensystem gesteuerten Systeme im Leicht- und Tiefschlaf weitgehend autonom arbeiten. Auch für die Atmung konnte eine starke Langzeitkorrelation während des \acs{REM}-Schlafs herausgestellt werden, während in den \acs{NREM}-Schlafphasen keine signifikante Korrelation existiert. \parencite{penzel_cardiovascular_2007}\\

Bivariate Untersuchungen hingegen werden angestellt, um Korrelationen zwischen zwei Signalen zu erkennen. Die kardiorespiratorischen Zusammenhänge (Korrelationen zwischen Herzschlag und Atmung) im Schlaf sind bereits intensiv untersucht. Auf Grundlage der Phasensynchronisation sowie der Erstellung von Synchrogrammen kann beispielsweise gezeigt werden, dass Herzschlag und Atmung im \acs{REM}-Schlaf kaum und in den \acs{NREM}-Schlafphasen stark synchronisiert sind. Auch dies wird bislang auf das vegetative Nervensystem sowie die unterschiedliche Hirnaktivität in den einzelnen Schlafphasen zurückgeführt. Dieser Zusammenhang kann unabhängig von Geschlecht und Alter beobachtet werden. Da insbesondere der Tiefschlaf durch niedrigen Energieverbrauch und eine Erholungsfunktion des Körpers charakterisiert ist, wird angenommen, dass eine Synchronisierung zwischen Herzschlag und Atmung im \acs{NREM}-Schlaf energiesparsam ist und die Erholungsfunktion fördert, so dass auch hier eine Ähnlichkeit zwischen Leicht- und Tiefschlaf sowie zwischen Wachzustand und \acs{REM}-Schlaf manifestiert worden ist. \parencite{penzel_cardiovascular_2007, hamann_automated_2009, bartsch_experimental_2007}\\

Auf Basis uni- oder bivariater Signaluntersuchungen einzelner Systeme können jedoch weder Interaktionen mit anderen physiologischen Systemen noch mögliche störende Einflüsse auf den Schlafprozess oder die Schlafregulation hinreichend analysiert werden. An diesem Punkt setzt unter anderem das erstmals im Jahr 2012 von Bashan et al. vorgestellte Verfahren der \acs{TDS} an, mit dem Untersuchungen multidimensionaler Biosignalaufzeichnungen angestellt werden können. Hierbei werden die einzelnen physiologischen Systeme, deren Biosignale durch die \acs{PSG} erfasst werden, als Knoten eines zusammenhängenden Netzwerks betrachtet und hinsichtlich Netzwerktopologie und Vernetzungsstärke untersucht (vgl. Punkt \ref{TDS}). Die grundlegenden Errungenschaften dieser Methode liegen in der Möglichkeit, Zusammenhänge zwischen den Systemen darzustellen. Auf Basis eines relativ kleinen Datensatzes aus der SIESTA-Studie (36 \acs{PSG}s) \parencite{klosch_siesta_2001} und unter Verwendung von \acs{EKG}, nasaler Atmung, \acs{EOG}, fünf requenzbänder eines \acs{EEG}, Kinn- und Bein-\acs{EMG} konnte gezeigt werden, dass sich Netztopologie und Verbindungsstärke in Abhängigkeit von der jeweiligen Schlafphase ändern. Insbesondere wird deutlich, dass sich die \acs{NREM}-Schlafphasen unter den Gesichtspunkten der \acs{TDS} stark voneinander unterscheiden. Dies widerspricht massiv den bisherigen Erkenntnissen uni- und bivariater Biosignaluntersuchungen. Der Leichtschlaf ähnelt vielmehr dem Wachzustand, wobei beide von einem hohen Verbindungsgrad sowie einer großen Verbindungsstärke geprägt sind. Der Tiefschlaf hingegen weist stark reduzierte Verbindungsstärken und einen schwachen Verbindugsgrad auf, während im \acs{REM}-Schlaf Ergebnisse zwischen denen des Leicht- und Tiefschlafs zu beobachten sind (Abb. \ref{fig:TDS}). \parencite{bashan_network_2012}

\begin{figure}[H]
	\centering
	\includegraphics[width = \textwidth]{img/TDS.png}
	\caption[Vernetzungsstärke und Netztopologie im \acs{TDS} Verfahren]{Darstellung der Vernetzungsstärke (oben) und der Netztopologie (unten) im Wachzustand, Leicht-, Tief- und \acs{REM}-Schlaf nach Anwendung der \acs{TDS} und gemittelt über den gesamten Untersuchungsdatensatz (36 junge, gesunde Probanden) \parencite{bashan_network_2012}:\\HR = Herzschlag; Resp = Atmung; Chin = Kinn-\acs{EMG}; Leg = Bein-\acs{EMG}; $\delta$, $\theta$, $\alpha$, $\sigma$, $\beta$ = Frequenzbänder eines \acs{EEG}; diese Systeme sind in Zeilen und Spalten der Matrizen in gleicher Reihenfolge abgetragen;\\Wachzustand und Leichtschlaf weisen hohe Vernetzungsstärke und nahezu vollständig verknüpftes Netz auf; \acs{REM}-Schlaf und Tiefschlaf zeigen mäßig bis stark reduzierte Vernetzungsstärke und Netztopologie\\}
	\label{fig:TDS}
\end{figure}

Auch unter Verwendung des gesamten Kontrolldatensatzes der SIESTA-Studie (385 \acs{PSG}s) sowie unter Einbeziehung weiterer Systeme (\acs{EKG}, nasale Atmung, Atemanstrengung in Brust und Bauch, Kinn- und Bein-\acs{EMG}, zwei \acs{EOG}s sowie jeweils fünf Frequenzbänder von insgesamt sechs \acs{EEG}s) konnten diese Ergebnisse reproduziert werden. Die Untersuchungen zeigen insbesondere, dass verschiedene Systeme des Gehirns (frontopolare, zentrale und okzipitale Ableitungen) in sämtlichen Schlafphasen miteinander verbunden sind. Eine Studie über die Robustheit der \acs{TDS} Analyse zeigt darüber hinaus, dass Artefakte im \acs{NREM}- und \acs{REM}-Schlaf keine signifikanten und im Wachzustand lediglich geringe Auswirkungen auf die Ergebnisse haben, was hauptsächlich auf Bewegungen im Wachzustand zurückgeführt werden kann. \parencite{bartsch_network_2015, breuer_netzwerktopologie_2016}\\

Abhängigkeiten zwischen der Vernetzungsstärke und dem Alter gesunder Probanden können insbesondere in den \acs{NREM}-Schlafphasen beobachtet werden. Der Vergleich zwischen den Altersgruppen 20 bis 25 Jahre und 78 bis 95 Jahre zeigt im Tiefschlaf massive Unterschiede. Insbesondere im Gehirn findet eine Reduzierung der Vernetzungsstärke statt. Der Leichtschlaf zeigt eine ähnliche Ausprägung, während diese im Wachzustand weniger signifikant ist. Im \acs{REM}-Schlaf existieren keine signifikanten Veränderungen in Abhängigkeit vom Alter. Obwohl eine Abhängigkeit zwischen der Vernetzungsstärke und dem Geschlecht weniger stark ausgeprägt scheint, können Unterschiede zwischen Männern und Frauen im Leichtschlaf (N2) und \acs{REM}-Schlaf beobachtet werden. Hierbei sind bei den Frauen im Gehirn stärkere Verbindungen unabhängig vom Alter erkennbar. Da gemäß Krefting et al. Alter und Geschlecht mögliche Störfaktoren des physiologischen Netzwerks bei schlafgestörten Patienten darstellen können, sollen diese Eigenschaften im Rahmen dieser Arbeit bei Insomniepatienten untersucht werden. \parencite{krefting_altersabhangigkeit_2016, krefting_age_2017}

%zeitgleich
%mono-/univariat: Spektralanalysen, DFA (für Korrelationen und Fluktiationen), phase rectified signal averaging
%bivariat: Kreuzkorrelation,  transfer function analysis, phase synchronization analysis, bivariate phase rectified signal averaging (phasengleichgerichtete Signalmittelung)
%
%SCT:
%established methods: cross correlation, mutual information, and cross recurrence analysis
%"method of symbolic coupling traces SCT is used to analyze and quantify time-delayed coupling"
%"these methods (cross correlation, mutual information, and cross recurrence analysis) are more sensitive to nonstationarities, nonlinearities,
%and noise."
%
%bivariat: Synchronisation zwischen Resp und EKG existiert im DS bei Probanden; bei Apnoe nicht erkennbar (mit Synchrogram)
%basiert alles auf Korrelation ohne Time Delay

\newpage
\chapter{Methodik}

\section{Time Delay Stability}\label{TDS}

Das Verfahren der \acs{TDS} stellt eine bivariate, lineare Methode zur Untersuchung multivariater Biosignalaufzeichnungen und zur Bestimmung von Korrelationen zweier Biosignale innerhalb eines Netzwerks dar. Die mittels \acs{PSG} aufgezeichneten Signale stellen hierbei die einzelnen Knoten des Netzwerks dar und repräsentieren jeweils eigenständige physiologische Systeme, die innerhalb des Netzwerks interagieren. Auf diese Weise können Aussagen über die Netztopologie sowie die Verbindungsstärke jeweils zweier Systeme getroffen werden. Diese Merkmale werden in Abhängigkeit der jeweiligen Schlafstadien untersucht. Als Schlafstadien gelten der Wachzustand, \acs{REM}-Schlaf und Tiefschlaf (\acs{NREM}3 und \acs{NREM}4 zusammengefasst) gemäß \acs{AASM} sowie der Leichtschlaf in Form des Schlafstadiums \acs{NREM}2. Das Schlafstadium \acs{NREM}1 wird hierbei außer Acht gelassen, da dieses als Übergang zwischen Wachzustand und Leichtschlaf (\acs{NREM}2) gilt. \parencite{bashan_network_2012, iber_aasm_2007}\\

Die Bestimmung einer Korrelation zwischen zwei Systemen erfolgt auf Basis der Kreuzkorrelation, jedoch werden zur Berechnung der Korrelationsstärke nicht die Kreuzkorrelationskoeffizienten verwendet, sondern die Zeit, über welche die Werte der maximalen Koeffizienten konstant sind. Im Folgenden wird beschrieben, in welcher Weise die zu untersuchenden Signale vorverarbeitet werden müssen und wie die Korrelationen anhand der \acs{TDS}-Analyse anschließend ermittelt werden. Auf die Implementierung dieses Verfahrens wird in Kapitel \ref{Implementierung} eingegangen.

%TODO:
%Bartsch et al. verwendet anderes edge handling (anschauen)! aber hier TDS Methode von Dagmar verwenden, damit die Ergebnisse vergleichbar sind

\subsection{Datenvorverarbeitung}\label{datenvorverarbeitung}

Da die Biosignale zum einen sehr unterschiedliche Merkmale aufweisen (vgl. Abb. \ref{fig:beispiel-psg}), zum anderen mit uneinheitlichen Abtastraten aufgezeichnet werden können und darüber hinaus die Kreuzkorrelation maßgeblich von den Signalwerten (Amplituden) abhängt, muss zunächst eine Vergleichbarkeit der Daten herbeigeführt werden. Zu diesem Zweck werden Zeitreihen extrahiert, welche im Ergebnis homogene Merkmale aufweisen.

%\textbf{\acs{EEG}}
\paragraph{\acs{EEG}}
Auf jedes zu untersuchende \acs{EEG}-Signal wird zunächst eine Spektralanalyse mit einer Fensterung von 2 s und einer Überlappung von 1 s ausgeführt, um das Signal vom Zeit- in den Frequenzbereich zu überführen. Anschließend werden die charakteristischen Frequenzbänder (vgl. Kapitel \ref{psg}) extrahiert. Um die einzelnen Frequenzbänder möglichst eindeutig voneinander trennen zu können, werden gemäß Bashan et al. die Übergänge auf 0.5 Hz genau definiert (Tab. \ref{tab:frequenzen}). Es resultieren demnach für ein \acs{EEG}-Signal insgesamt fünf Zeitreihen mit einer Abtastfrequenz von 1 Hz. \parencite{bashan_network_2012}\\


\begin{table}[H] 
\centering
\begin{small}
\begin{tabular}{lc}
\toprule
\multicolumn{1}{l}{\textbf{Frequenzband}} & \multicolumn{1}{c}{\textbf{Frequenz in Hz}}\\  
\midrule
Delta-Wellen & 0.5 Hz - 3.5 Hz\\
Theta-Wellen & 4 Hz - 7.5 Hz\\
Alpha-Wellen & 8 Hz - 11.5 Hz\\
Sigma-Wellen & 12 Hz - 15.5 Hz\\
Beta-Wellen & 16 Hz - 19,5 Hz\\
\bottomrule
\end{tabular}
\caption[Frequenzbänder des \acs{EEG}]{Frequenzbänder des \acs{EEG} gemäß \parencite{bashan_network_2012}}
\label{tab:frequenzen}
\end{small}
\end{table}


\paragraph{\acs{EOG} und \acs{EMG}}
Aus \acs{EOG}- und \acs{EMG}-Signalen wird innerhalb von 2 s langen Fenstern mit einer Überlappung von 1 s die Varianz ermittelt und als Zeitreihe gespeichert.\parencite{bashan_network_2012}\\

\paragraph{\acs{EKG} und Atmung}
Zu den Atemsignalen zählen nasale Atmung sowie die Atemanstrengung in Brust und Bauch. Zur Extraktion homogener Zeitreihen aus \acs{EKG} und Atmung werden Herz- und Atemfrequenz berechnet und auf 1 Hz neu gesampelt. \parencite{bashan_network_2012}\\

Sämtliche extrahierte Zeitreihen weisen demnach eine identische Auflösung von 1 Hz mit gleicher Länge $N$ auf. Bei der Untersuchung der Korrelation zweier Zeitreihen $x$ und $y$ werden diese anschließend in $N_L$ sich überlappende Segmente $v$ eingeteilt. Jedes Segment $v$ weist eine Länge von 60 s bei einer Überlappung von 30 s auf (Formel~\ref{eq:zeitreihen}). Sodann erfolgt eine Normalisierung der Zeitreihen, so dass der Mittelwert 0 und die Standardabweichung 1 betragen. Auf diese Weise sind sämtliche Merkmale der ursprünglichen Signale homogenisiert und zwecks Ermittlung von Korrelationen vorverarbeitet.\parencite{bashan_network_2012}

\begin{equation}
L = 60 s \nonumber
\end{equation}
\begin{equation}
overlap = \frac{L}{2} \nonumber
\end{equation}
\begin{equation}
N_L = (2\frac{N}{L})-1
	\label{eq:zeitreihen}
\end{equation}

\subsection{Ermittlung von Korrelationen}\label{korrelationen}

Über zwei Zeitreihen $x$ und $y$ wird die Kreuzkorrelation durchgeführt, indem die eine Zeitreihe $x$ segmentweise und innerhalb jedes Segments $v$ sekundenweise über die zweite Zeitreihe $y$ geschoben wird. Für jeden Zeitpunkt der Verschiebung $T$ wird eine Zwischensumme der Kreuzkorrelation berechnet, wobei die entsprechenden Werte beider Zeitreihen miteinander multipliziert und anschließend alle Produkte addiert werden. Demnach entsteht für jeden Zeitpunkt der Verschiebung $T$ ein Eintrag in den Ergebnisvektor $xcorr$: \parencite{bashan_network_2012}

\begin{equation}
xcorr(x_v, y_v)(T) = \sum \limits_{i=1}^L x_v(i) \cdot y_v(T+i).
	\label{eq:xcorr}
\end{equation}

Der maximale Absolutwert innerhalb des Vektors $xcorr$ stellt in seinem Wert die höchstmögliche Ähnlichkeit der Zeitreihen und in seiner Position die Verschiebung $T$ zum Zeitpunkt dieser maximalen Korrelation dar. Für jedes Segment $v$ wird sodann die Verschiebung $T_0$ (Time Delay) zum maximalen Absolutwert aus dem Kreuzkorrelationsvektor $xcorr$ bestimmt. Als stabile Verbindungen gelten mindestens vier zusammenhängende Segmente $v$ (5 $*$ 30 s), deren Verschiebung $T_0$ annähernd gleich ($\pm$ 1~s) bleibt (Abb. \ref{fig:TimeDelay}). Je länger diese zusammenhängenden Epochen andauern, desto stabiler ist die Verbindung zwischen den beiden untersuchten Systemen. Anschließend wird der Prozentsatz der TDS im Verhältnis zur gesamten Zeitreihe berechnet. Je höher dieser Prozentanteil ausfällt, desto größer ist die Verbindungsstärke. Hierbei gilt ein Schwellenwert (Significance Threshold) von 7 \%, ab dem zwei Systeme tatsächlich als verbunden gelten. \parencite{bashan_network_2012}

\begin{figure}[H]
	\centering
	\includegraphics[scale = 0.5]{img/Time_Delay.png}
	\caption[Zeitverschiebung (Time Delay) und stabile Verbindungen im \acs{TDS}-Verfahren]{Zeitverschiebung (Time Delay) und stabile Verbindungen im \acs{TDS}-Verfahren innerhalb eines physiologischen Netzwerks (unter Verwendung von \parencite{bashan_network_2012}):\\jeder Punkt steht für den Time Delay $T$ einer 30 s Epoche; fünf zusammenhängende 30 s Epochen mit stabilem Time Delay $T$ entsprechen vier zusammenhängenden Segmenten $v$;\\gelbe Punkte = stabile Verbindung zwischen Kinn-\acs{EMG} und Alpha-Wellen des \acs{EEG};\\rote Punkte = stabile Verbindung zwischen \acs{EKG} und \acs{EOG}; blaue Punkte = keine stabile Verbindung}
	\label{fig:TimeDelay}
\end{figure}

Auf diese Weise werden sämtliche Zeitreihen des Netzwerks miteinander verglichen. Im Ergebnis der \acs{TDS}-Analyse entsteht für jedes untersuchte Schlafstadium eine quadratische Matrix, welche die prozentualen Verbindungsstärken zwischen den Systemen farbig abbildet. Beispielhaft sind die Ergebnismatrizen für gesunde Probanden aus der Untersuchung von Krefting et al. \parencite{krefting_altersabhangigkeit_2016} (388 \acs{PSG}s aus der SIESTA-Studie) für jedes untersuchte Schlafstadium in der Abb. \ref{fig:TDS_Matrizen} dargestellt. Deutlich erkennbar ist hierbei die Ähnlichkeit zwischen Wachzustand und Leichtschlaf sowie zwischen Tief- und \acs{REM}-Schlaf. In den Zeilen und Spalten der Matrizen sind sämtliche untersuchten Systeme in gleicher Reihenfolge abgetragen (Herz, nasale Atmung, Atemanstrengung in Brust und Bauch, Kinn- und Beinbewegung, Augen sowie die jeweils fünf Frequenzbänder der sechs \acs{EEG}s). Dementsprechend stellt das obere rechte Dreieck die Spiegelung des unteren linken Dreiecks dar. Die Diagonale bildet die Autokorrelation der Systeme ab.\\

Das Verfahren der TDS stellt demnach ein Werkzeug zur Untersuchung der physiologischen Zusammenhänge im Schlaf dar. Hierbei ist jedoch nicht der Wert des Kreuzkorrelationskoeffizienten ausschlaggebend, sondern die Dauer der Korrelation. Auf diese Weise können nicht nur zeitgleiche, sondern auch zeitversetzte Ähnlichkeiten zweier Signale erkannt werden. \parencite{bashan_network_2012}

\begin{figure}[H]
	\centering
	\includegraphics[width = \textwidth]{img/TDS_W_LS_DS_REM.png}
	\caption[Ergebnis-Matrizen der \acs{TDS}]{Gemittelte Ergebnis-Matrizen der \acs{TDS}-Analyse für den Wachzustand, Leichtschlaf (oben), Tiefschlaf und \acs{REM}-Schlaf (unten) über die Kontrollgruppe der SIESTA-Studie (388 \acs{PSG}s):\\Verbindungsstärken sämtlicher Signalkopplungen in Prozent entsprechend der Colorbar; die abfallende Diagonale stellt die Autokorrelation der Systeme dar;\\$HR$ = Herz, $BR_{air}$ = nasaler Atemfluss, $BR_c$ = Atemanstrengung in der Brust, $BR_a$ = Atemanstrengung im Bauch, $Chin$ = Kinnbewegung, $Leg$ = Beinbewegung, $Eye1$ und $Eye2$ = Augen, $\delta$, $\theta$, $\alpha$, $\sigma$, $\beta$ = Frequenzbänder der frontopolaren ($Fp1$, $Fp2$), zentralen ($C3$, $C4$) und okzipitalen ($O1$, $O2$) Elektroden}
	\label{fig:TDS_Matrizen}
\end{figure}

\newpage

\section{Mittelwertberechnung}

In dieser Arbeit werden verschiedene Methoden der Mittelwertberechnung verwendet. Zur Mittelung der Verbindungsstärken innerhalb der Ergebnis-Matrizen aller Patienten, werden die Ergebnis-Matrizen der Patienten für jedes Schlafstadium übereinandergelegt, addiert und anschließend durch die Summe der Patienten dividiert gemäß

\begin{equation}
smean(x,y,s) = \frac{\sum \limits_{i=1}^N l_{tds}(x,y,s)_i}{N},
	\label{eq:smean}
\end{equation}

wobei $l_{tds}$ die prozentualen Verbindungsstärken beschreibt, $x$ und $y$ die Zeitreihen, $s$ das jeweilige Schlafstadium und $N$ die Anzahl der Patienten. Das Ergebnis \textit{smean} stellt demnach wiederum eine Matrix mit den gleichen Dimensionen wie die einzelnen Ergebnis-Matrizen der Patienten dar. Bashan et al. definieren die Mittelwertberechnung über die Ergebnis-Matrizen der Patienten durch Aneinanderreihung der Zeitreihen und Division durch die Anzahhl der Epochen des entsprechenden Schlafstadiums als 

\begin{equation}
gmean(x,y,s) = \frac{\sum \limits_{i=1}^N tds_{x,y}(v(s))_i}{\sum \limits_{i=1}^N v_{si}}
	\label{eq:gmean}
\end{equation}

mit den Zeitreihen $x$ bzw. $y$, der Anzahl der Patienten $N$, der Anzahl der Epochen pro Schlafstadium $v_s$ sowie den Schlafstadien $s$. Ein Vergleich beider Verfahren von Krefting et al. zeigt keine signifikanten Unterschiede in den Ergebnissen beider Verfahren, so dass in dieser Arbeit lediglich die Mittelwertberechnung von \textit{smean} verwendet wird. \parencite{bashan_network_2012, krefting_age_2017}\\

Darüber hinaus wird zur Berechnung des globalen Mittelwerts die mittlere Verbindungsstärke über eine Ergebnis-Matrix berechnet gemäß

\begin{equation}
nmean(i,s) = \frac{\sum \limits_{x=1}^N tds_{x,y}(v(s))_i}{\sum \limits_{i=1}^N v_{si}}
	\label{eq:nmean}
\end{equation}

mit den Zeilen $x$ und Spalten $y$ der Ergebnis-Matrix, der jeweiligen prozentualen Verbindungsstärker zweier Systeme $l_{tds}$ sowie $N_n$ als Anzahl der Netzwerkknoten bzw. Systeme. \textit{nmean} beschreibt demnach die Gesamtverbindungsstärke eines Patienten pro Schlafstadium, wobei diese über das obere rechte Dreieck und ohne die Autokorrelation der Systeme berechnet wird. \parencite{krefting_age_2017}

\section{Lineare Regression}

Das Ziel der linearen Regression ist die Vorhersage einer Zielgröße anhand bekannter Variablen. Diese Vorhersage beruht zum einen auf unabhängigen Werten $x_i$ und zum anderen auf von $x_i$ abhängigen Werten $y_i$. $x$ und $y$ stellen demnach Messreihen dar, wobei $y$ von $x$ abhängt. Ein einfaches Beispiel stellt die Schuhgröße in Abhängigkeit zur Körpergröße dar. Die Wertpaare lassen sich als Punktwolke grafisch darstellen. Die lineare Regression stellt hierbei ein statistisches Werkzeug zur Bestimmung einer idealen Regressionslinie durch die Punktwolke gemäß

\begin{equation}
y = a + b \cdot x + \epsilon
	\label{eq:linreg}
\end{equation}

dar mit $y$ als abhängige Zielgröße, der unabhängigen Variablen $x$, der Konstanten $a$ als Schnittpunkt mit der y-Achse, dem Regressionskoeffizienten $b$ sowie einem Fehlerwert $\epsilon$. Der Regressionskoeffizient wird anhand der einzelnen Werte in den Messreihen sowie deren Mittelwerte gemäß

\begin{equation}
b = \frac{\sum \limits_{i=1}^n(x_i-\overline{x}) \cdot (y_i-\overline{y})}{\sum \limits_{i=1}^n(x_i-\overline{x})^2} 
	\label{eq:regcoef}
\end{equation}

berechnet, wobei $\overline{x}$ und $\overline{y}$ die jeweiligen Mittelwerte der Messreihen darstellen. Die Konstante $a$ wird anschließend anhand von 

\begin{equation}
a = \overline{y} - b \cdot \overline{x}
	\label{eq:konstante}
\end{equation}

berechnet. Mit Hilfe von $a$ und $b$ kann demnach für jedes $x$ eine Vorhersage für den Zielwert $y$ getroffen werden. Da der Zielwert meist Schwankungen unterliegt, definiert $\epsilon$ den Fehlerwert. \parencite{frank_einfach_2006}\\

In einem linearen Modell wird zunächst eine Null-Hypothese aufgestellt 

\begin{equation}
H_0: b = 0.
	\label{eq:null-hypo}
\end{equation}

In dem Beispiel der Körper- und Schuhgröße würde dies bedeuten, dass die Körpergröße keinen Einfluss auf die Schuhgröße hat. Der p-Wert gibt darüber hinaus die Signifikanz der Korrelation an. Je näher der p-Wert dem Wert 0 ist, desto höher ist die Signifikanz. Ein p-Wert, der größer als 0,05 ist beschreibt keine signifikante Korrelation der Variablen.  

\section{Korrelationsverfahren}

Korrelationsverfahren dienen zur Messung des Zusammenhangs zweier Variablen. Mit Hilfe des Korrelationskoeffizienten nach Pearson wird der lineare Zusammenhang zweier Messreihen $x$ und $y$ ermittelt durch

\begin{equation}
r =  \frac{\sum \limits_{i=1}^n (x_i-\overline{x}) \cdot (y_i-\overline{y})}{\sqrt{\sum \limits_{i=1}^n (x_i-\overline{x})^2} \cdot \sqrt{\sum \limits_{i=1}^n (y_i-\overline{y})^2}}.
	\label{eq:pearson}
\end{equation}

Der Korrelationskoeffizient nach Spearman, auch Rangkorrelation genannt, wird nicht anhand der Messwerte, sondern anhand ihrer Ränge berechnet. Hierzu wird für jeden Wert beider Messreihen ein Rang vergeben in der Form, dass beispielsweise die höchste Körpergröße den Rang 1 erhält und die geringste den letzten Rang innerhalb der Messreihe. Für die Messreihe der Schuhgröße erfolgt die Rangvergabe nach identischem Muster, so dass die größte Schuhgröße den Rang 1 und die kleinste den letzten Rang der Messreihe erhält. Bei der grafischen Darstellung dieser Ränge geht der Zusammenhang der vorherigen Messwertpaare demnach verloren. Berechnet wird die Rangkorrelation in gleicher Weise wie der Korrelationskoeffizient nach Pearson, jedoch werden die Messwerte $x_i$ und $y_i$ durch ihre Ränge ersetzt, so dass 

\begin{equation}
r_{Sp} =  \frac{\sum \limits_{i=1}^n (Rang(x_i)-\overline{Rang(x)}) \cdot (Rang(y_i)-\overline{Rang(y)})}{\sqrt{\sum \limits_{i=1}^n (Rang(x_i)-\overline{Rang(x)})^2} \cdot \sqrt{\sum \limits_{i=1}^n (Rang(y_i)-\overline{Rang(y)})^2}}
	\label{eq:spearman}
\end{equation}

gilt. Demnach wird mit Hilfe des Korrelationskoeffizienten nach Spearman nicht der lineare, sondern der monotone Zusammenhang zwischen zwei Variablen ermittelt. \parencite{frank_einfach_2006}\\

Für beide Verfahren gilt, dass der Korrelationskoeffizient $r$ bzw. $r_{Sp}$ stets zwischen -1 und 1 liegt. Bei einem Korrelationskoeffizienten $\approx$ 0 besteht keine Korrelation. Sind $r$ bzw. $r_{Sp}$ kleiner als der Wert 0, existiert eine negative Korrelation, so dass steigende $x$-Werte abfallende $y$-Werte bewirken. Bei einem Korrelationskoeffizienten, der größer als der Wert 0 ist, besteht hingegen eine positive Korrelation, so dass steigende $x$-Werte wachsende $y$-Werte hervorrufen. In einem linearen Modell wird wiederum ein p-Wert erstellt, welcher die Signifikanz der Korrelation zwischen den Variablen repräsentiert. Ein p-Wert, der kleiner als der Wert 0,05 ist, gilt als signifikante Korrelation. \parencite{frank_einfach_2006}\\

Voraussetzung für die Anwendung des Korrelationsverfahrens nach Pearson ist eine Normalverteilung der Messreihen. Um diese Voraussetzung zu prüfen, wird der Shapiro-Wilk-Test verwendet. Hierbei wird die Null-Hypothese aufgestellt, dass eine Stichprobe~$x$ normalverteilt ist. Insbeosndere bei kleinen Stichproben mit weniger als 50 Messwerten zeichnet sich der Shapiro-Wilk-Test durch eine hohe Teststärke aus. Auf Grundlage geschätzter Varianzen dieser Stichprobe wird mit dem Shapiro-Wilk-Test die Teststatistik

\begin{equation}
W =  \frac{b^2}{(n - 1)s^2}
	\label{eq:shapiro}
\end{equation}

berechnet. Die Teststatistik $W$ stellt dabei eine Kennzahl dar, welche das Verhältnis zwischen geschätzter ($b^2$) und tatsächlicher Varianz ($s^2$) der Stichprobe angibt. Zunächst werden die Stichprobenwerte aufsteigend sortiert, so dass $x_1 < x_2 < ... < x_n$ gilt. Anschließend wird die geschätzte Varianz berechnet durch

\begin{equation}
b =  \sum \limits_{i=1}^k a_{n-i+1} \cdot (x_{n-i+1} - x_i), \text{ mit} \begin{cases}
     k = \frac{n}{2} & \text{für gerades n}\\
     k = \frac{n-1}{2} & \text{für ungerades n}
   \end{cases}
	\label{eq:var1}
\end{equation}

und die tatsächliche Stichprobenvarianz durch 

\begin{equation}
s^2 = \frac{\sum \limits_{i=1}^n (x_i - \overline{x})^2}{n - 1}.
	\label{eq:var1}
\end{equation}

Hierbei entspricht $n$ der Menge der Stichprobenwerte und $a$ einem Signifikanzniveau (meistens 5~\%). Die Teststatistik $W$ wird anschließend mit einem kritischen Wert $W_{kritisch}$ verglichen, welcher der Shapiro-Wilk-Tabelle entnommen werden kann. Ist $W > W_{kritisch}$, wird die Null-Hypothese bestätigt und eine Normalverteilung der Grundgesamtheit wird angenommen. Darüber hinaus gibt der p-Wert die Wahrscheinlichkeit an, dass die Stichprobe aus einer normalverteilten Grundgesamtheit stammt. Ist der p-Wert größer als das Signifikanzniveau $a$, so kann die Null-Hypothese ebenfalls bestätigt werden. \parencite{frank_einfach_2006, shapiro_analysis_1965}

\section{Matlab}

Matlab ist eine in den 1970er Jahren in Fortran entwickelte und von The Mathworks vertriebene kommerzielle Software. Sie dient der Analyse und Berechnung numerischer Daten und Problemstellungen sowie der Visualisierung der Ergebnisse. Berechnungen basieren in Matlab (von MATrix LABoratory) auf Matrizen. Aus diesem Grund ist die Nutzung von Matlab im mathematischen, technischen und wissenschaftlichen Kontext stark verbreitet. Insbesondere die Signalverarbeitung kann mit Matlab geeignet umgesetzt werden.\footnote{\url{https://de.mathworks.com/products/matlab.html} und \url{https://de.mathworks.com/solutions.html?s_tid=gn_sol} (Stand: 23.02.2017)}\\

Matlab ähnelt in seiner Syntax zwar der Programmiersprache C, stellt jedoch eine proprietäre Programmier- bzw. Skriptsprache dar. Anwendungen können entweder in Skripten geschrieben oder als atomare Funktionen entwickelt werden, was eine Erstellung der in Matlab charakteristischen Toolboxes unterstützt. Matlab ist darüber hinaus eine Sprache mit dynamischer Typisierung. Dies erhöht die Lesbarkeit des Codes und die Produktivität des Entwicklers. Funktionen sind hierdurch einerseits flexibler, da sie oft weniger Code enthalten als in einer stark typisierten Sprache. Andererseits müssen Funktionen nicht überladen werden, um für unterschiedliche Datentypen zu gelten. Variablen müssen darüber hinaus nicht mit expliziten Datentypen initialisiert werden. Als interpretierte Programmiersprache (oder Skriptsprache) bietet Matlab ferner den Vorteil, dass Code schnell getestet und angewendet werden kann - ohne den zeitaufwendigen Zwischenschritt der Kompilierung. Dies ermöglicht das Testen von Skripten oder Skriptausschnitten mit beliebigen Startpunkten der Ausführung sowie das Testen atomarer Funktionen, ohne eine Main-Klasse ausführen zu müssen, wie z. B. in Java. Eine Erstellung eigenständig lauffähiger Programme ist jedoch durch Erweiterung der Grundfunktionen von Matlab ebenfalls möglich. \parencite{stein_programmieren_2012}\\

Nachteil von Matlab ist zum einen die Speicherverwaltung, da der Standarddatentyp für numerische Variablen \texttt{double} mit einem Speicherplatz von 64 bit ist. Eine starke Typisierung könnte hingegen das Speichermanagement optimieren. Zum anderen stellt Matlab keine kompilierte Sprache dar, so dass in der Grundfunktionalität die Erzeugung lauffähiger Anwendungsprogramme nicht möglich ist. Darüber hinaus dient die statische Codeanalyse lediglich der Überprüfung des Datenflusses, es entfällt jedoch die Überprüfung des Programmcodes zur Übersetzungszeit, so dass Fehler erst bei Ausführung der Funktion erkannt werden.\\

In dieser Arbeit soll für die Anwendung der \acs{TDS}-Analyse Matlab verwendet werden, da die verwendeten Analyseverfahren mathematisch basiert sind und Matlab einerseits als funktionale Sprache für numerische Berechnungen besonders geeignet ist und andererseits Berechnungen einfacher als in kompilierten Sprachen zu implementieren sind. Da im Rahmen dieser Arbeit kein eigenständiges System entwickelt wird, sondern Analyseverfahren angewendet werden sollen, ist Matlab als interpretierte Sprache mit dynamischer Typisierung optimal geeignet, um schnell Zwischenergebnisse zur erzeugen und diese zu verifizieren, auszuwerten und grafisch darzustellen. Aufgrund der Aufgabenstellung sowie des relativ kleinen Untersuchungsdatensatzes (Kapitel \ref{analyse}) kann auch der Nachteil einer weniger effizienten Speicherverwaltung vernachlässigt werden. Da Matlab darüber hinaus unter anderem für die Signalverarbeitung ausgelegt ist, soll die Anwendung des \acs{TDS}-Verfahrens in Matlab umgesetzt werden. Hierfür wird  Matlab in der Version 2015b verwendet.

%benutzt, weil: Flexibilität, Erweiterbarkeit (Toolboxes von Mathworks), und Workspace möglich, numerische Stabilität (kein Datenverlust oder Rundungsfehler), Effizienz und Optimierung der Berechnungen (Beispiel rekursive Fakultät)
%
%überschreiben: bei Vererbung, Änderung des Methodenkörpers
%überladen: Änderung des Methodenheaders (Parameteranzahl oder -datentyp) = Methodensignatur


\section{R}

R ist als Teil des GNU Projekts eine Open-Source-Programmierumgebung zur Anwendung statistischer Verfahren sowie zur grafischen Darstellung von Daten und steht seit dem Jahr 1995 unter der GNU General Public License. Die Sprache R wurde im Jahr 1992 von Robert Gentleman und Ross Ihaka in C und Fortran sowie später in R selbst in Anlehnung an die Programmiersprache S entwickelt und stellt eine interpretierte Multiparadigmensprache mit dynamischer Typisierung dar. Dies bezieht sich darauf, dass R zum einen objektorientiert und zum anderen funktional ist. Sämtliche Daten (z. B. Variablen, Funktionen, Ausdrücke etc.) in R stellen Objekte dar, wohingegen sämtliche Aktionen Funktionsaufrufe repräsentieren. Eine R-Distribution enthält einige Standard-Pakete, welche Basisfunktionen für statistische Berechnungen und grafische Darstellungen realisieren, und kann simpel über das \acl{CRAN} (\acs{CRAN}) erweitert werden.\footnote{https://www.r-project.org/about.html (Stand: 18.03.2017)} Das Paket \texttt{R.matlab} beispielsweise ermöglicht das Lesen und Schreiben von Matlab-Code (.m-Dateien) und gespeicherten Variablen (.mat-Dateien) sowie das Starten von Matlab und die Ausführung von Matlab-Code im Hintergrund aus R heraus. Im \acs{CRAN} können Nutzer darüber hinaus selbst entwickelte Pakete zur Verfügung stellen. \parencite{adler_r_2012}\\

Jedes Objekt in R besitzt einen bestimmten Datentypen und gehört einer Klasse an. Funktionen sind beispielsweise Objekte der Klasse \texttt{function}. R basiert auf Vektoren. Die grundlegenden Objekte in R stellen demnach Zeichen- und numerische Vektoren und darüber hinaus Matrizen, Arrays und Data Frames dar. Vektoren sind hierbei eindimensional, Matrizen zwei- und Arrays mehrdimensional. Data Frames können hingegen als Tabellen verstanden werden. Der Standard-Datentyp in R ist wie in Matlab \texttt{double}. Es können ebenso Funktionen und Skripte erstellt werden. Darüber hinaus kann in R das Paket \texttt{R.compiler} zur Erstellung lauffähiger Programme geladen werden.\\

Als interpretierte Sprache mit dynamischer Typisierung bietet R die gleichen Vorteile wie Matlab. Ein Nachteil liegt darin, dass R die Daten vor der Weiterverarbeitung zunächst in den RAM lädt. Dies kann bei besonders großen Datensätzen zu Laufzeitproblemen führen. Gegenüber Matlab bietet R jedoch den Vorteil, dass es speziell für statistische Berechnungen ausgelegt ist mit einer besonderen Stärke in der grafischen Darstellung von Daten. In Matlab hingegen sind in der Grundausführung lediglich statistische Standardfunktionen implementiert, welche durch eine Statistik-Toolbox erweitert werden können. R ist vor allem im wissenschaftlichen Umfeld weit verbreitet. Dies zeigt das IEEE-Ranking\footnote{http://spectrum.ieee.org/static/interactive-the-top-programming-languages-2016 (Stand: 19.03.2017)} über die Top-Prgrammiersprachen im Jahr 2016, in dem R mit einem Ranking von 87,9 (von 100) Platz 5 belegt, während Matlab mit einem Ranking von 68,5 lediglich Platz 10 erreicht. Aus diesen Gründen wird für die statistische Analyse der \acs{TDS}-Ergebnisse R verwendet. Darüber hinaus wird R Studio als Open-Source-IDE für R genutzt, da es zum einen eine umfangreichere grafische Oberfläche von R implementiert mit R-Konsole, Workspace, Editor und History, während R selbst lediglich die Konsole umfasst. Zum anderen bietet R Studio u. a. Syntax-Highlighting, Ausführung aus dem Editor heraus sowie Debugging-Funktionen.\footnote{https://www.rstudio.com/products/RStudio/ (Stand: 18.03.2017)} In dieser Arbeit wird R in der Version 3.3.2 und R Studio in der Version 1.0.136 verwendet.

\section{Vorgehen}

Um das \acs{TDS}-Verfahren auf die hiesigen Daten der Insomniepatienten anwenden zu können, muss im ersten Schritt eine Datenanalyse erfolgen. Insbesondere die Heterogenität der Voraussetzungen der Patienten durch verschiedene Begleiterkrankungen und ggf. unterschiedliche Arten der Insomnie mit differenter Ausprägung der Schlafstadien macht Überlegungen notwendig, inwiefern eine Vergleichbarkeit der Daten untereinander sowie mit den Ergebnissen von Krefting et al. \parencite{krefting_age_2017} herbeigeführt werden kann. Aus diesem Grund werden Kriterien definiert, anhand derer ggf. Gruppen mit ähnlichen Merkmalen gebildet und Daten bei Nichterfüllung dieser Kriterien von den weiteren Untersuchungen ausgeschlossen werden. Im nächsten Schritt werden die Matlab-Funktionen des \acs{TDS}-Pakets um Codeblöcke zur Anwendung auf den hiesigen Datensatz der Charit\'{e} Berlin erweitert. Dies ist notwendig, da sich der hiesige Datensatz in Anzahl und Sortierung der Signale sowie in den Hypnogrammen von dem von Krefting et al. untersuchten Datensatz unterscheidet. Anschließend wird das \acs{TDS}-Verfahren auf Daten der jeweiligen Gruppen ausgeführt. Um Gruppenergebnisse zusammenzufassen und auszuwerten, werden die Mittelwerte \textit{smean} mit Matlab berechnet und die Ergebnisse gegenübergestellt sowie mit den gemittelten \acs{TDS}-Matrizen der SIESTA-Daten verglichen.\\

Im nächsten Schritt werden mit Hilfe von R statistische Verfahren angewendet, um die Ergebnisse mit denen von Krefting et al. bei gesunden Probanden zu vergleichen. Zu diesem Zweck werden die Subnetzwerke gemäß Krefting et al. untersucht und für diese die globalen Verbindungsstärken \textit{nmean} und die Anzahl der Verbindungen gegenübergestellt. Darüber hinaus sollen die Mittelwerte \textit{smean} der weiblichen sowie der männlichen Insomniepatienten durch Subtraktion der \acs{TDS}-Matrizen miteinander verglichen werden (+Signifikanzlevel). Ferner erfolgt ein Vergleich der jüngsten und ältesten Patienten des Datensatzes sowie eine Gegenüberstellung der globalen Verbindungsstärke \textit{nmean} und dem Alter jedes Patienten. Eine lineare Regression sowie die Anwendung der Korrelationsverfahren (+Signifikanzlevel) soll Aufschluss über den Zusammenhang beider Parameter geben.\\

Anhand dieser Untersuchungen sollen \acs{TDS}-Merkmale für Insomniepatienten abgeleitet werden. Darüber hinaus soll ein Leitfaden für die Vorauswahl der Daten sowie für die Anwendung des \acs{TDS}-Verfahrens auf Datensätze der Charit\'{e} Berlin entwickelt werden. Ferner werden Alters- und Geschlechtsabhängigkeiten des \acs{TDS}-Verfahrens bei Insomniepatienten beschrieben. Die Ergebnisse werden in Hinblick auf die Beschaffenheit der Daten sowie im Vergleich zu den Ergebnissen gesunder Probanden diskutiert. Sämtliche Plots von \acs{TDS}-Matrizen werden mit Matlab erstellt. Die übrigen Plots werden mit Hilfe von R erzeugt.


%da lediglich die Anwendung verschiedener Verfahren programmiert werden soll, jedoch kein Programm zur häufigen Anwendung, ist keine Kompilierung notwendig und Matlab vor allem aufgrund des schnellen Testens geeignet, um die hiesige Problemstellung zu lösen. 
%hauptsächlich werden Skripte (in dieser Arbeit zur einmaligen Ausführung) erstellt

\newpage
\chapter{Analyse}\label{analyse}

\section{Datenanalyse}\label{datenanalyse}

Die Datenanalyse erfolgt auf Grundlage der zur Verfügung stehenden Dateien des Untersuchungsdatensatzes. Auf die Implementierung der entsprechenden Analysealgorithmen wird in Kapitel \ref{Datenvorauswahl} eingegangen.

\paragraph{Untersuchungsdatensatz} Für diese Arbeit steht ein Untersuchungsdatensatz von insgesamt 64 Patienten der Charit\'{e} Berlin mit der Diagnose Insomnie zur Verfügung. Von jedem Patienten liegt die \acs{PSG} der ersten Nacht vor. Da bei Insomnie der umgekehrte Erste-Nacht-Effekt gilt, so dass die Patienten zumeist erst nach der zweiten Nacht von einer stärker reduzierten Schlafqualität berichten \parencite{happe_schlafmedizin_2009}, kann das Fehlen weiterer \acs{PSG}s vernachlässigt werden. Welche Art der Insomnie bei den Patienten diagnostiziert worden ist, ist nicht bekannt. Sämtliche Aufzeichnungen stammen aus den Jahren 2008 und 2009. Zu jedem Patienten liegt eine \acs{PSG} als \acs{EDF}-Datei vor. Die Signalbezeichnungen in den \acs{PSG}-Aufzeichnungen sind bezogen auf die für die \acs{TDS}-Analyse relevanten Signale einheitlich und folgen einer identischen Sortierung (Tab. \ref{tab:labels}). Frontale oder frontopolare \acs{EEG}-Signale sind nicht vorhanden.

\begin{table}[H]
\centering 
\begin{tabularx}{0.76\textwidth}{cXcX}
\toprule
\multicolumn{1}{c}{\textbf{Kanal}} & \multicolumn{1}{l}{\textbf{Signalbezeichnung}} & \multicolumn{1}{l}{\textbf{Kanal}} & \multicolumn{1}{l}{\textbf{Signalbezeichnung}}\\
\midrule 
1 & - & 9 & EMG chin\\
2 & EEG C4-A1 & 10 & EMG RAT\\
3 & EEG O2-A1 & 11 & EMG LAT\\
4 & - & 12 & ECG\\
5 & EEG C3-A2 & 13 & Resp nasal\\
6 & EEG O1-A2 & 14 & Resp chest\\
7 & EOG E1-A2 & 15 & Resp abdomen\\
8 & EOG E2-A2 & ab 16 & weitere Signale\\
\bottomrule
\end{tabularx}
\caption[Einheitliche Signalbezeichnungen und -sortierung]{Signalbezeichnungen und -sortierung der für die \acs{TDS}-Analyse relevanten Signale des Untersuchungsdatensatzes}
\label{tab:labels}
\end{table}

\paragraph{Patienteninformationen} Vorerkrankungen und Begleitmedikationen sind für jeden Patienten in einer Excel-Datei dokumentiert. In dieser Datei sind für jeden Patienten ebenso Geburtstag, Alter, Geschlecht, Größe, Gewicht, \acl{BMI} (\acs{BMI}) sowie Datum der ersten Aufzeichnung aufgeführt. Besonders häufige Vorerkrankungen sind Blinddarm- bzw. Mandelentzündungen, welche bei 17 bzw. 20 Patienten eine Entfernung des Organs zur Folge hatten. Ebenso auffällig sind diagnostizierte Kurz- bzw. Weitsichtigkeit oder Grüner Star bei insgesamt 19 Patienten. Bei acht weiblichen Patienten kommen Hormone zur Milderung von Wechseljahressymptomen zum Einsatz, teilweise in Verbindung mit Schlafmitteln oder Antidepressiva. Bei zwölf weiblichen Patienten liegen darüber hinaus gynäkologische Vorerkrankungen, wie Myome\footnote{Myom: gutartiger Tumor in der Gebärmutter}, Eierstockzysten oder eine Entfernung der Eierstöcke, vor. Fünf Patienten aus dem Untersuchungsdatensatz leiden an Atemwegserkrankungen, wie beispielsweise Asthma, und nehmen atemwegserweiternde Medikamente ein. Bei etwa einem Viertel (15) der Patienten wurden Bluthochdruck, Herzrhythmusstörungen oder verengte Blutgefäße diagnostiziert, so dass zwölf von ihnen blutgefäßerweiternde Medikamente einnehmen. Sechs Patienten leiden an Schilddrüsendysfunktionen oder -knoten, vier dieser Patienten nehmen Schilddrüsenhormone ein. Bei insgesamt acht Patienten sind Schlafmittel als Begleitmedikation dokumentiert, teilweise in Abhängigkeit von anderen Erkrankungen als die Insomnie, wie beispielsweise Wechseljahresbeschwerden.

\paragraph{Ereignisse} Zu jeder \acs{PSG} existiert eine Ereignis-Datei im TXT-Format. Hierin werden Patienten-ID, Datum der Aufzeichnung, mögliche Ereignisse, wie beispielsweise das Abfallen einer Elektrode oder Augenbewegungen, Name des Auswerters und Datum der Auswertung dokumentiert. Darauffolgend werden tabellarisch das jeweils klassifizierte Schlafstadium, Zeitpunkt der Epoche, Ereignis und dessen Dauer sowie in einigen Fällen auch die Körperlage zu jedem Zeitpunkt aufgelistet. Als Ereignis wurden bei fast allen Patienten Schlafapnoen vermerkt. Diese sind in der Excel-Datei über die Begleiterkrankungen jedoch nicht aufgeführt. Die Ereignis-Dateien müssen aus diesem Grund nach Apnoen durchsucht werden. Treten mehr als fünf Apnoe-Ereignisse pro Stunde bei einem Patienten auf, so ist diesem Patienten das Krankheitsbild von Schlafapnoen zuzuordnen. 
Da die \acs{TDS}-Analyse als robust gegenüber Artefakten gilt \parencite{breuer_netzwerktopologie_2016}, kann auf eine Bereinigung der Daten verzichtet werden.

\paragraph{Erstellung von Hypnogrammen} Die Ereignis-Dateien weisen bei allen Patienten einen anderen Startzeitpunkt der Aufzeichnung auf als in dem Header der zugehörigen \acs{EDF}-Datei (Abb.~\ref{fig:zeitpunkte}). Dies führt dazu, dass die Anzahl der Epochen in den \acs{EDF}-Dateien größer ist als in den Ereignis-Dateien. Ein aus der Ereignis-Datei extrahiertes Hypnogramm würde demnach nicht den Verlauf der \acs{PSG} abbilden. Die Differenz in Sekunden zwischen den beiden Zeitpunkten entspricht jedoch größtenteils der auf die nächste ganze Zahl abgerundeten Anzahl an Epochen, um welche die Signale in der \acs{EDF}-Datei länger sind. Dies lässt darauf schließen, dass die \acs{PSG} jeweils früher gestartet worden ist als die Aufzeichnung der Ereignisse. Es müssen daher Hypnogramme erstellt werden, welche auf die \acs{EDF}-Dateien anwendbar sind.

\begin{figure}[H]
	\centering
	\includegraphics[width = \textwidth]{img/Startzeitpunkte.png}
	\caption[Startzeitpunkte von Ereignis- und \acs{EDF}-Datei]{Unterschiedliche Startzeitpunkte in Ereignis-Datei (links) und \acs{EDF}-Header (rechts) lassen darauf schließen, dass die Ereignis-Datei zu einem späteren Zeitpunkt als die \acs{PSG}-Aufzeichnung gestartet wurde}
	\label{fig:zeitpunkte}
\end{figure}

\paragraph{Untersuchung der Schlafstadien} 
Da sich die Ergebnisse der \acs{TDS}-Analyse auf die einzelnen Schlafstadien beziehen, bei Insomnie-Patienten jedoch nicht zwangsläufig sämtliche Schlafstadien durchlaufen werden, ist es notwendig, eine Überprüfung der durchlaufenen Schlafstadien durchzuführen und ggf. Datensätze mit fehlenden Schlafstadien von der Untersuchung auszuschließen. Darüber hinaus müssen die jeweiligen Schlafstadien über einen hinreichenden Zeitraum andauern, um stabile Verbindungen zu erkennen. Da als stabile Verbindung mindestens fünf zusammenhängende Epochen \`{a} 30~s gelten (Kapitel \ref{TDS}), sollte jedes Schlafstadium über mindestens 2,5 min andauern. Um diese Voraussetzungen zu überprüfen, müssen die Ereignis-Dateien sämtlicher \acs{PSG}s untersucht werden.

\paragraph{Klassifizierung in Gruppen}
Die Dokumentation der Patienteninformationen in der Excel-Datei zeigt, dass ein Großteil der Patienten Begleiterkrankungen aufweist und/oder Medikamente einnimmt. Diese können die Ergebnisse der \acs{TDS}-Analyse jedoch beeinflussen. Beispielsweise kann eine Herzrhythmusstörung zu einer abweichenden Darstellung der Kopplung zwischen dem Herzen und anderen Systemen führen. Asthma oder Schlafapnoen wirken sich auf die Kopplung zwischen Atmung und anderen Systemen aus. Darüber hinaus können verabreichte Medikamente, wie beispielsweise Schlafmittel, die Symptome der Insomnie verfälschen. Aus diesem Grund müssen die Daten in verschiedene Gruppen klassifiziert werden. Dies ermöglicht einen späteren Vergleich der Gruppenergebnisse und Rückschlüsse auf den Einfluss der Begleiterkrankungen und -medikationen auf die Ergebnisse der \acs{TDS}-Analyse.\\

Es wird basierend auf den Patienteninformationen des zugrunde liegenden Datensatzes sowie den Ereignissen in den Ereignis-Dateien folgende Gruppeneinteilung vorgenommen:

\begin{itemize}
\item \textbf{Gruppe Alle:} gesamter Untersuchungsdatensatz
\item \textbf{Gruppe Herz:} Herzrhythmusstörungen, zu schnelle oder zu langsame Herzfrequenz, zu hoher oder zu niedriger Blutdruck, Carotisstenose\footnote{Carotisstenose: Verengung der hirnversorgenden Halsschlagader}
\item \textbf{Gruppe Atmung:} Asthma, COPD\footnote{COPD: Chronic Obstructive Pulmonary Disease; chronische Erkrankung der Lunge}, Schlafapnoen
\item \textbf{Gruppe Schlafmittel:} Einnahme von schlaffördernden Mittel (Zolpidem, Rohiypnol, Zopiclon, Doxepin, Schlafsterne, Stilnox, Noctamid, Remifemin)
\item \textbf{Gruppe Insomnie:} Patienten sind nicht den obigen Gruppen zuzuordnen, können jedoch andere Begleiterkrankungen und -medikationen aufweisen, welche sich vermutlich nicht maßgeblich auf die Ergebnisse der \acs{TDS}-Analyse auswirken, Beispiele: Antibiotika nach Blutvergiftung, Hormone und Medikamente zur Linderung von Wechseljahresbeschwerden, Hormone bei Schilddrüsenerkrankungen
\end{itemize}

\paragraph{Kriterien Datenvorverarbeitung} Folgende Kriterien (Tab. \ref{tab:Datenvorauswahl}) muss ein Datensatz zur Einbeziehung in die hiesigen Untersuchungen demnach erfüllen:

\begin{table}[H] 
\begin{tabularx}{\textwidth}{cX}
\toprule
\multicolumn{1}{c}{\textbf{~~~~~~Nr.~~~~~~}} & \multicolumn{1}{l}{\textbf{Kriterien}}\\
\midrule 
D1 & Für jeden Patienten muss aus der Ereignis-Datei ein Hypnogramm erstellt werden, welches auf die \acs{EDF}-Datei anwendbar ist.\\
D2 & Alle relevanten Schlafstadien (\acs{NREM}2, \acs{NREM}3 oder \acs{NREM}4, \acs{REM}-Schlaf und Wachzustand) müssen durchlaufen werden.\\
D3 & Jedes relevante Schlafstadium muss mindestens fünf zusammenhängende Epochen beinhalten.\\
D4 & Patienten, deren Ereignis-Datei mehr als fünf Schlafapnoen pro Stunde aufweist, soll die schlafbedingte Erkrankung Schlafapnoe zugeordnet werden.\\
D5 & Die Patienten sollen anhand ihrer Begleiterkrankungen bzw. -medikationen in fünf Gruppen (Alle, Herz, Atmung, Schlafmittel, Insomnie) eingeteilt werden.\\
D6 & Artefakte sollen nicht entfernt werden.\\
\bottomrule
\end{tabularx}
\caption[Datenvorauswahl - Kriterien]{Datenvorauswahl - Kriterien für die Auswahl und Klassifizierung der Daten vor Anwendung der \acs{TDS}-Analyse}
\label{tab:Datenvorauswahl}
\end{table}

\paragraph{Ergebnisse Datenanalyse} Im Ergebnins der Datenanalyse und Datenvorauswahl wurde für jeden Patienten ein Hypnogramm im Format [Hypnogram\_filename.txt] erstellt (Kriterium D1). Diese enthalten in der ersten Zeile die vergangene Zeit in Sekunden zwischen den Startzeitpunkten aus dem \acs{EDF}-Header sowie der Ereignis-Datei, um bei der \acs{TDS}-Analyse Hypnogramm und \acs{PSG} optimal übereinanderlegen zu können. Darauffolgend sind zeilenweise die Schlafstadien aufgelistet.\\

Die Überprüfung der durchlaufenen Schlafstadien (Kriterien D2 und D3) zeigt, dass insgesamt 41 Patienten aus dem Untersuchungsdatensatz das Schlafstadium \acs{NREM}4 nicht durchlaufen haben. Da bei der \acs{TDS}-Analyse jedoch die Schlafstadien \acs{NREM}3 und \acs{NREM}4 als Tiefschlaf zusammengelegt werden (Kapitel \ref{TDS}), kann dieses Ergebnis vernachlässigt werden. Dementsprechend ist bei diesen 41 Patienten das Schlafstafium \acs{NREM}4 zu kurz. Dies betrifft zusätzlich einen weiteren Patienten. Darüber hinaus dauert das Schlafstadium \acs{NREM}1 bei einem Patienten weniger als fünf zusammenhängende Epochen an. Bei einem weiteren Patienten trifft dies auf das Schlafstadium \acs{NREM}3 zu, das Schlafstadium \acs{NREM}4 wird jedoch in ausreichender Länge durchlaufen. Das fehlende Schlafstadium \acs{NREM}1 kann hier vernachlässigt werden, da dieses nicht in die \acs{TDS}-Analyse mit einfließt (Kapitel \ref{TDS}). Für die fehlenden Schlafstadien \acs{NREM}3 und \acs{NREM}4 gilt, dass bei jedem Patienten mindestens eines davon in ausreichender Länge durchlaufen werden muss. Dies trifft bei sämtlichen Patienten zu. Fehlende Schlafstadien \acs{NREM}2, \acs{REM}-Schlaf oder Wachzustand treten nicht auf. Für die \acs{TDS}-Analyse können demnach sämtliche Daten im Untersuchungsdatensatz verwendet werden.\\

Die Untersuchung hinsichtlich der in den Ereignis-Dateien auftretenden Apnoe-Ereignissen (Kriterium D4) ergibt, dass bei insgesamt lediglich 16 Patienten keine Apnoe-Ereignisse auftreten. Bei ebenfalls 16 der verbleibenden Patienten treten Apnoe-Ereignisse häufiger als fünfmal pro Stunde auf. Diese werden eingestuft in die Gruppe "`Atmung"'.\\

Neben den Patienten mit eindeutigen Apnoe-Ereignissen werden ebenfalls weitere, die Atmung betreffende Krankheiten, in die Gruppe "`Atmung"' klassifiziert. Nach Untersuchung der Begleiterkrankungen und -medikationen wird die Einteilung der Gruppen (Kriterium D5) gemäß Tab. \ref{tab:gruppen} vorgenommen. Der Gesamtdatensatz (Gruppe Alle) besteht aus etwa zwei Dritteln (41) weiblichen und einem Drittel (23) männlichen Patienten bei einem Altersdurchschnitt von 51 Jahren. Der Altersdurchschnitt der weiblichen Patienten aus dieser Gruppe liegt bei 52,3 Jahren (minimales Alter: 23 Jahre, maximales Alter: 65~Jahre). Bei den männlichen Patienten liegt der Altersdurchschnitt bei 48,7 Jahren (minimales Alter: 26 Jahre, maximales Alter: 61 Jahre). In den Gruppen Herz und Atmung ähnelt die Geschlechterverteilung der des Gesamtdatensatzes mit etwa zwei Dritteln weiblichen und einem Drittel männlichen Patienten. Der Altersdurchschnitt in diesen beiden Gruppen liegt leicht über dem Durchschnittsalter der Gruppe Alle. Schlafmittel nehmen hauptsächlich die weiblichen Patienten ein (80 \% der Gruppe Schlafmittel), wobei der Altersdurchschnitt hier ebenfalls über dem der Gruppe Alle liegt. Dies kann darauf zurückgeführt werden, dass in dem Gesamtdatensatz die Frauen ein höheres Alter aufweisen als die Männer. Darüber hinaus ist die Einnahme von Schlafmitteln bei den weiblichen Patienten häufig mit Wechseljahresbeschwerden verbunden, was sich in dem Gruppendurchschnittsalter widerspiegelt. In der Gruppe Insomnie ist die Differenz zwischen der Anzahl weiblicher und männlicher Patienten verältnismäßig am geringsten und der Altersdurchschnitt liegt mit 46,7 Jahren deutlich unter dem des Gesamtdatensatzes. Weitere Gruppenverteilungen können dem Anhang (Tab. \ref{tab:Alle} bis \ref{tab:Insomnie}) entnommen werden. Da einige Patienten mehrere Erkrankungen erleiden, treten Überschneidungen bei der Gruppeneinteilung auf, so dass die Summe der Patienten der Gruppen Herz, Atmung, Schlafmittel und Insomnie nicht der Gesamtanzahl der Patienten des Untersuchungsdatensatzes entspricht.\\

\begin{table}[H] 
\centering
\begin{tabularx}{\textwidth}{ccccccccccccccccccc}
\toprule
\multicolumn{3}{c}{\textbf{Alle}} & &  \multicolumn{3}{c}{\textbf{Herz}} & &  \multicolumn{3}{c}{\textbf{Atmung}} & &  \multicolumn{3}{c}{\textbf{Schlafmittel}} & &  \multicolumn{3}{c}{\textbf{Insomnie}}\\  
\midrule
w  & m  & A  &   & w  & m  & A    &   & w  & m & A  &   & w & m & A    &   & w  & m  & A\\
41 & 23 & 51 & ~ & 10 & 15 & 56,8 & ~ & 13 & 8 & 54 & ~ & 8 & 2 & 55,1 & ~ & 17 & 11 & 46,7\\
\bottomrule
\end{tabularx}
\caption[Klassifizierung der Patientengruppen]{Klassifizierung der Patientengruppen; w = Anzahl weiblicher Patienten;\\m = Anzahl männlicher Patienten; A = durchschnittliches Gruppenalter}
\label{tab:gruppen}
\end{table}


\section{Anforderungsanalyse}

Für die Analyse der physiologischen Netzwerke werden priorisierbare Anforderungen formuliert. Im Rahmen dieser Arbeit stellt sich das Problem, dass kein klassisches Software-Projekt umgesetzt werden soll, welches im Ergebnis ein benutzbares System darstellt. Stattdessen sollen durch Anwendung bestimmter Verfahren spezifische Ergebnisse berechnet werden. Aus diesem Grund stellt die Formulierung funktionaler und nicht-funktionaler Anforderungen oder die Erstellung von Use Cases nach klassischen Software-Engineering-Standards keine optimale Herangehensweise dar. Stattdessen werden allgemeine und gleichzeitig priorisierbare Anforderungen formuliert.\\

Hierbei erfolgt die Einteilung in Muss-, Soll-, Kann- sowie Abgrenzungskriterien. Muss-Kriterien sind zwangsläufig zu erfüllen. Sie stellen die Anforderungen mit der höchsten Priorität dar. In dieser Arbeit sind dies solche Anforderungen, die maßgeblich die wesentlichen Aufgaben der gesamten Analyse beschreiben. Soll-Kriterien sind im bestmöglichen Fall ebenfalls vollständig umzusetzen, sind in ihrer Priorität jedoch den Muss-Kriterien nachgestellt. Zusammen mit den Muss-Kriterien stellen sie diejenigen Anforderungen dar, welche maßgeblich zur erfolgreichen Umsetzung des Projekts beitragen und an deren Erfüllung der Erfolg der Umsetzung gemessen wird. Kann-Kriterien stellen optionale Anforderungen dar. Deren Umsetzung wertet die Qualität des Projekts auf, ein Weglassen beeinflusst jedoch nicht die Hauptfunktionalitäten. Im Rahmen dieser Arbeit würden durch die Nicht-Erfüllung von Kann-Kriterien beispielsweise die Ergebnisse der Analyseverfahren nicht beeinflusst werden. Abgrenzungskriterien werden darüber hinaus formuliert, um zu definieren, welche Funktionalitäten oder Eigenschaften nicht erfüllt werden sollen. Für die hiesige Analyse wäre beispielsweise die Abgrenzung zu klassischen Software-Projekten zu formulieren. Aus diesem Grund wird darüber hinaus auf eine Stakeholderanalyse gänzlich verzichtet.\\

Als Ergebnis der Anforderungsanalyse werden die in Tab. \ref{tab:Muss-Kriterien} bis \ref{tab:Abgr-Kriterien} dargestellten Kriterien definiert.\\

\begin{table}[H] 
\begin{tabularx}{\textwidth}{cllX}
\toprule
\multicolumn{3}{c}{\textbf{Muss-Kriterien}} & \\  
\cmidrule{1-3}
\multicolumn{3}{c}{Nr.} & Anforderung\\ 
\midrule 
\multicolumn{3}{c}{A1} &  Grundlage der Untersuchungen müssen \acs{PSG}s und Hypnogramme von Insomnie-Patienten sein.\\
\multicolumn{3}{c}{A2} & Auf Grundlage der \acs{PSG}s und Hypnogramme müssen physiologische Netzwerke mit Hilfe des \acs{TDS}-Verfahrens untersucht werden.\\
\multicolumn{3}{c}{A3} & Die physiologischen Netzwerke müssen im Vergleich mit gesunden Probanden auf insomniespezifische Merkmale untersucht werden.\\ 
\multicolumn{3}{c}{A4} & Die physiologischen Netzwerke müssen auf Altersabhängigkeiten untersucht werden.\\ 
\multicolumn{3}{c}{A5} & Die physiologischen Netzwerke müssen auf Geschlechterabhängigkeiten untersucht werden.\\
\multicolumn{3}{c}{A6} & Die Analyseergebnisse müssen mit den Ergebnissen von Krefting et al. \parencite{krefting_age_2017} vergleichbar sein.\\
\bottomrule
\end{tabularx}
\caption{Anforderungen - Muss-Kriterien}
\label{tab:Muss-Kriterien}
\end{table}


\begin{table}[H] 
\begin{tabularx}{\textwidth}{cllX}
\toprule
\multicolumn{3}{c}{\textbf{Soll-Kriterien}} & \\  
\cmidrule{1-3}
\multicolumn{3}{c}{Nr.} & Anforderung\\ 
\midrule 
\multicolumn{3}{c}{A7} &  Für die Anwendung des \acs{TDS}-Verfahrens soll Matlab verwendet werden.\\
\multicolumn{3}{c}{A8} & Für die Anwendung statistischer Verfahren soll R verwendet werden.\\
\multicolumn{3}{c}{A9} & Die Ergebnisse der Analysen sollen grafisch dargestellt werden.\\ 
\multicolumn{3}{c}{A10} & Die Untersuchungsdatensätze sollen in geeigneter Weise gemäß den Kriterien aus der Datenanalyse aufbereitet werden (Kriterien D1 - D5 in Kapitel \ref{datenanalyse}).\\
\bottomrule
\end{tabularx}
\caption{Anforderungen - Soll-Kriterien}
\label{tab:Soll-Kriterien}
\end{table}


\begin{table}[H] 
\begin{tabularx}{\textwidth}{cllX}
\toprule
\multicolumn{3}{c}{\textbf{Kann-Kriterien}} & \\  
\cmidrule{1-3}
\multicolumn{3}{c}{Nr.} & Anforderung\\ 
\midrule 
\multicolumn{3}{c}{A11} & Es können Datenstrukturen für die Speicherung prägnanter Patienteneigenschaften in Matlab erstellt werden.\\
\multicolumn{3}{c}{A12} & Für die \acs{TDS}-Analyse kann ein Leitfaden für die Datenvorauswahl und die Anwendung des Verfahrens erstellt werden.\\ 
\multicolumn{3}{c}{A13} & Die bereits bestehenden Funktionen der \acs{TDS}-Analyse können für die Anwendung auf Datensätze der Charit\'{e} erweitert werden.\\ 
\bottomrule
\end{tabularx}
\caption{Anforderungen - Kann-Kriterien}
\label{tab:Kann-Kriterien}
\end{table}


\begin{table}[H] 
\begin{tabularx}{\textwidth}{cllX}
\toprule
\multicolumn{3}{c}{\textbf{Abgrenzungskriterien}} & \\  
\cmidrule{1-3}
\multicolumn{3}{c}{Nr.} & Anforderung\\ 
\midrule 
\multicolumn{3}{c}{A14} &  Die Implementierung entspricht nicht der Form eines Paket-Standards, sondern stellt die Anwendung der Analyseverfahren auf den Untersuchungsdatensatz durch atomare Skripte und/oder Funktionen dar.\\
\bottomrule
\end{tabularx}
\caption{Anforderungen - Abgrenzungskriterien}
\label{tab:Abgr-Kriterien}
\end{table}


\newpage
\chapter{Implementierung}\label{Implementierung}

\section{Datenvorauswahl}\label{Datenvorauswahl}

Die in Kapitel \ref{datenanalyse} beschriebenen Kriterien zur Auswahl und Klassifizierung der Daten werden mithilfe von Skripten und Funktionen in Matlab umgesetzt. Im Folgenden wird auf die Umsetzung der einzelnen Kriterien eingegangen. Das Kriterium D6 bleibt dabei unberücksichtigt, da es sich lediglich auf das Weglassen einer Artefaktbereinigung bezieht.

\paragraph{Kriterium D1} Für die Erstellung der Hypnogramme (Skript restructure\_hypnogram.m) müssen zunächst die Startzeitpunkte aus der \acs{EDF}- sowie Ereignis-Datei jedes Patienten extrahiert werden. Nach dem Einlesen der EDF-Datei lässt sich der Startzeitpunkt der Aufnahme als String aus dem Header auslesen. Um den Startzeitpunkt der Ereignis-Datei zu ermitteln, wird die Ereignis-Datei eingelesen. Da in dieser Datei verschiedene Informationen gespeichert sind und die Spaltenanzahl daher erst ab der tabellarischen Auflistung der Schlafstadien, Zeitpunkte und Ereignisse gleich bleibt, wird mit einem String-Vergleich die Startzeile dieser Tabelle gesucht. Hierbei können die Ereignis-Dateien abweichende Spaltenzahlen aufweisen, da manche Dateien zusätzlich die Körperlage dokumentieren. Die Körperlage kann jedoch als ein zusammenhängendes Wort ("`POSITION-LEFT"' oder als mehrere Wörter ("`Körperlage unbekannt"') angegeben sein, so dass die Spaltenanzahl der Tabellenüberschriften sowie des Tabellenkörpers differieren können und ein ausschließlicher Abgleich der Tabellenüberschriften demnach nicht zweckdienlich ist (Abb. \ref{fig:spalten}). Aus diesem Grund erfolgt die Ermittlung des Spaltenindexes durch \texttt{ismember}-Vergleich zwischen der Spaltenüberschrift "`Schlafstadium"' und einem Doppelpunkt, welcher ausschließlich in den Strings der Zeitpunkte enthalten ist (Listing \ref{lst:D1}). Der Spaltenindex der Ereignisse (\texttt{ind\_ev)} entspricht sodann dem auf den Spaltenindex des Zeitpunkts folgenden Index.\\

\begin{figure}[H]
	\centering
	\includegraphics[width = \textwidth]{img/Ereignis_Spalten.png}
	\caption[Tabellenüberschriften in den Ereignis-Dateien]{Unterschiedliche Tabellenüberschriften in den Ereignis-Dateien;\\links (ACA\_69003) sechs Überschriften und fünf Spalten im Tabellenkörper;\\rechts (NKI\_000580) sieben Überschriften und sechs bzw. sieben Spalten im Tabellenkörper}
	\label{fig:spalten}
\end{figure}

\begin{lstlisting}[caption={Implementierung Kriterium D1 in Skript restructure\_hypnogram.m}, label={lst:D1}]
T = readtable(hypno_file);
% ...
for j = 1:row
    % extract content of one row and convert each element into one string
    C = table2cell(T(j, col));
    M = cell2mat(C);
    newrow = strsplit(M);
    % find row of appearing 'Schlafstadium'
    ss = strcmpi('Schlafstadium', newrow);
    nr_ss = find(ss, 1);
    if ~isempty(nr_ss)
        hypnostart = j+1;
        ind_ss = nr_ss;
    end
end
% extract starttime of hypnogram
for k = hypnostart:row
    % extract content of row k and convert each element into one string
    C = table2cell(T(k, col));
    M = cell2mat(C);
    newrow = strsplit(M);
    % find row containing ':' from time specification (like 20:25:10]
    tm = strfind(newrow, ':');
    ind_ev = find(not(cellfun('isempty', tm)))+1;
    if (any(ismember(newrow(ind_ss),sleep_stage_W)) && any(ismember(newrow(ind_ev),sleep_stage_W)))
        hypno_starttime = newrow(ind_ev-1);
        break;
    end
end
% ...
\end{lstlisting}

Anschließend kann der Startzeitpunkt extrahiert und mit dem Startzeitpunkt aus dem \acs{EDF}-Header verglichen werden. Zu diesem Zweck wird in der Funktion subtract\_timestrings.m die Differenz zwischen den beiden Zeitpunkten berechnet und als Anzahl von Sekunden zurückgegeben. Die Zeitdifferenz entspricht jedoch keiner ganzzahligen Anzahl an Epochen. Da Hypnogramme die Schlafstadien lediglich epochenweise, also in 30-Sekunden-Schritten, abbilden, würde ein Auffüllen der Hypnogramme um die fehlende Anzahl an Epochen nicht dazu führen, dass sich beide exakt übereinanderlegen ließen, so dass die Zuordnung der Schlafstadien nicht korrekt wäre. Aus diesem Grund müssen die zu untersuchenden Signale innerhalb der \acs{PSG} auf den Zeitraum der Hypnogramme begrenzt werden. Zu diesem Zweck wird eine Hypnogramm-Datei im TXT-Format angelegt (Hypnogram\_filename.txt) und in die erste Zeile die Zeitdifferenz in Sekunden geschrieben, um daraus bei der späteren \acs{TDS}-Analyse die Begrenzung der Signale zu berechnen. Anschließend werden aus der Ereignis-Datei die Schlafstadien ausgelesen. Hierbei muss ein Abgleich zwischen den Spalten "`Schlafstadium"' und "`Ereignis"' erfolgen (Abb. \ref{fig:zeitpunkte}), um lediglich die klassifizierten Schlafstadien zu erhalten, nicht jedoch dazwischen auftretende Ereignisse. Die ursprünglichen Bezeichnungen der Schlafstadien werden in numerische Werte geändert und ebenfalls in die Hypnogramm-Datei geschrieben (Tab \ref{tab:schlafstadien}). Dieser Vorgang erfolgt in einer Schleife für sämtliche Patienten des Untersuchungsdatensatzes. Mit dem Skript (extract\_num\_epochs.m) kann darüber hinaus für jeden Patienten die Anzahl der Epochen aus dem \acs{EDF}-Header sowie aus der Ereignis-Datei berechnet und in eine Ergebnis-Datei (Epoch\_length.txt) geschrieben werden. Auf diese Weise kann ermittelt werden, ob Unterschiede existieren und wie groß die Abweichungen sind.

\begin{table}[H] 
\centering
\begin{tabularx}{0.785\textwidth}{Xc}
\toprule
\multicolumn{1}{l}{\textbf{Schlafstadien in Ereignis-Datei}} & \multicolumn{1}{c}{\textbf{Geänderte Bezeichnung}}\\
\midrule 
Wach, SLEEP-S0 & 0\\
S1, SLEEP-S1 & 1\\
S2, SLEEP-S2 & 2\\
S3, SLEEP-S3 & 3\\
S4, SLEEP-S4 & 4\\
REM, SLEEP-REM & 5\\
\bottomrule
\end{tabularx}
\caption[Bezeichnung der Schlafstadien]{Bezeichnung der Schlafstadien in den Ereignis-Dateien und zugeordneter numerischer Wert zur Erstellung des Hypnogramms}
\label{tab:schlafstadien}
\end{table}
 
\paragraph{Kriterium D2} Um zu überprüfen, ob alle Patienten sämtliche für die \acs{TDS}-Analyse relevanten Schlafstadien durchlaufen haben, werden mit Hilfe des Skripts check\_sleep\_ stages.m die zuvor erzeugten Hypnogramm-Dateien untersucht. Hierzu werden die Daten aus den Hypnogramm-Dateien ausgelesen und mit einem \texttt{ismember}-Abgleich nach den entsprechenden Schlafstadien gesucht. Sofern nach dieser Überprüfung ein Schlafstadium bei einem Patienten nicht gefunden werden kann, wird der Dateiname (PatientenID) sowie eine Fehlermeldung (z. B. "`missing sleep stage: S4"') in eine Ergebnis-Datei (sleep\_stages.txt) geschrieben. Diese Ergebnis-Datei bildet die Grundlage für den Ausschluss von Daten aus dem Untersuchungsdatensatz.

\paragraph{Kriterium D3} Für die Suche nach zusammenhängenden Epochen innerhalb eines Schlafstadiums wird ebenfalls das Skript check\_sleep\_stages.m verwendet. Innerhalb einer for-Schleife wird ein Zähler hochgesetzt, solange ein Schlafstadium gleich bleibt. Existiert nach Ausführung der for-Schleife ein Schlafstadium, dessen Zähler nicht mindestens einmal den Wert 5 erreicht hat, so wird die untersuchte Datei (Patienten-ID) sowie das fehlende Schlafstadium in eine Ergebnis-Datei geschrieben (consecutive\_stages.txt). Auf Grundlage dieser Ergebnis-Datei können wiederum Daten von den weiteren Untersuchungen ausgeschlossen werden.

\paragraph{Kriterium D4} Zur Erkennung von Apnoe-Ereignissen wird das Skript count\_apnea.m verwendet. Dieses liest die Ereignis-Dateien aller Patienten ein und ermittelt durch einen \texttt{ismember}-Vergleich, ob eine Ereignis-Datei ein Apnoe-Ereignis aufweist. In diesem Fall wird ein Zähler hochgesetzt. Der Vergleich erfolgt auf Basis definierter Cell Strings, welche die Bezeichnungen der Schlafstadien sowie der Apnoe-Ereignisse enthalten (Listing \ref{lst:D4}). Anschließend wird eine Ergebnis-Datei (Apneas.txt) erstellt und der Dateiname (Patienten-ID) sowie die Anzahl der gefundenen Apnoe-Ereignisse darin gespeichert. Diese Ergebnis-Datei dokumentiert demnach die Anzahl aller gefundenen Apoe-Ereignisse für jeden Patienten.\\

\begin{lstlisting}[caption={Implementierung Kriterium D4 in Skript count\_apneas.m}, label={lst:D4}]
apnea = {'APNEA', 'HYPOPNEA', 'APNEA-CENTRAL', 'APNEA-MIXED', 'APNEA-OBSTRUCTIVE'};
sleep_stages = {'Wach', 'SLEEP-S0', 'S1', 'SLEEP-S1', 'S2', 'SLEEP-S2', 'S3', 'SLEEP-S3', 'S4', 'SLEEP-S4', 'REM', 'SLEEP-REM'};
% ...
if (any(ismember(newrow(ind_ss),sleep_stages)) && any(ismember(newrow(ind_ev),apnea)))
    count_apnea = count_apnea+1;
end
% ...
\end{lstlisting}

Um zu bestimmen, wie viele Apnoe-Ereignisse pro Stunde auftreten, müssen wiederum die Ereignis-Dateien untersucht werden. Zu diesem Zweck liest das Skript check\_apneas5.m die Ereignis-Dateien ein und ermittelt den Beginn der Tabelle sowie die Spaltenindizes der Schlafstadien und Ereignisse. In einer for-Schleife wird mit Hilfe von \texttt{ismember}-Vergleichen zeilenweise ein Apnoe-Ereignis gesucht, dessen Zeitpunkt als Cell String aus der Tabelle ausgelesen und der Schleifenindex zwischengespeichert (\texttt{apnearow}). In einer weiteren for-Schleife wird in identischer Weise nach einem weiteren Apnoe-Ereignis gesucht und der Zeitpunkt ebenfalls als Cell String gespeichert. Da Matlab Cell Strings jedoch nicht als Zeitpunkte erkennt, ermöglicht die Funktion subtract\_timestrings.m eine Umwandlung dieser Cell Strings in numerische Werte und berechnet die zeitliche Differenz in Sekunden (Listing \ref{lst:D4time}). Hierbei werden die Zeitpunkte mit Hilfe der Matlab-eigenen Funktion \texttt{duration} in vergangene Sekunden seit dem Zeitpunkt "`00:00:00"' umgerechnet. Darüber hinaus wird abgefragt, welcher Wert größer ist, so dass auch bei Tagänderung (nach 24:00 Uhr) die zeitliche Differenz korrekt errechnet wird. Anschließend wird geprüft, ob diese zeitliche Differenz größer als eine Stunde ist, und ein Flag zurückgegeben.\\

\begin{lstlisting}[caption={Implementierung Kriterium D4 in Funktion subtract\_timestrings.m}, label={lst:D4time}]
time = cell2mat(time);
hh_time = str2double(time(1:2));
mm_time = str2double(time(4:5));
ss_time = str2double(time(7:8));
% ...
% convert time and time2 into duration class variables with format seconds, so that D_time and D_time2 represent the amount of elapsed seconds from 00:00:00 until time and time2
D_time = duration(hh_time, mm_time, ss_time, 'Format', 's');
D_time2 = duration(hh_time2, mm_time2, ss_time2, 'Format', 's');
% amount of seconds of one day
daysecs = 86400;
% time2 ist bigger than time (= time2 is the same day)
if D_time <= D_time2
% convert D_time and D_time2 into double and calculate elapsed seconds
	durationsecs = seconds(D_time2-D_time);
    % check if elapsed time is not longer than one hour
    if durationsecs <= 3600
    	t = 1;
    else
        t = 0;
    end
% time is bigger than time2 (= time2 is the next day)
else
    % calculate the rest of the first day from time and add the elapsed time of time2
    durationsecs = seconds(daysecs - seconds(D_time));
    durationsecs = seconds(durationsecs + D_time2);
    % check if elapsed time is not longer than one hour
    if durationsecs <= 3600
        t = 1;
    else
        t = 0;
    end
end
\end{lstlisting}

Ist die zeitliche Differenz nicht größer als eine Stunde, wird ein Apnoe-Zähler hochgesetzt. Dies bedeutet demnach, dass das erste sowie das zweite erkannte Apnoe-Ereignis innerhalb einer Stunde liegen. Sobald dieser Zähler größer als der Wert 5 ist, erhöht sich ein Signifikanz-Zähler und der Index der äußeren Schleife (\texttt{apnearow}) wird um den Wert 1 erhöht. Anschließend wird die innere Schleife abgebrochen und erneut in die äußere gesprungen, so dass die Überprüfung, ob Apnoe-Ereignisse innerhalb einer Stunde liegen, erneut eine Zeile nach dem ersten erkannten Apnoe-Ereignis startet (Listing \ref{lst:D4apnea5}). Auf diese Weise wird sichergestellt, dass von jedem Zeitpunkt eines Apnoe-Ereignisses aus der relevante Zeitraum von einer Stunde auf weitere Apnoe-Ereignisse untersucht wird. Ist der Signifikanz-Zähler nach Ausführung der Schleifen größer als Null, wurden demnach Apnoe-Ereignisse mit einer Häufigkeit von mehr als fünf pro Stunde erkannt. Anschließend werden solche Dateinamen (Patienten-ID), bei denen Apnoe-Ereignisse mit einer Häufigkeit von mehr als fünf pro Stunde erkannt worden sind, gemeinsam mit dem Wert des Signifikanz-Zählers in eine Ergebnis-Datei (Apnea5perhour.txt) geschrieben.\\

\begin{lstlisting}[caption={Implementierung Kriterium D4 in Skript check\_apneas5.m}, label={lst:D4apnea5}]
% ...
if (any(ismember(Mrow(ind_ss),sleep_stages)) && any(ismember(Mrow(ind_ev),apnea)))
    time2 = Mrow(ind_ev-1);
    % calculate elapsed time between both time stamps
    [t, durationsecs] = subtract_timestrings(time, time2);
    % count apnea if elapsed time is <= one hour
    if t == 1
        countapnea = countapnea+1;
    else
    % count significance if more than five apneas per hour
        if countapnea > 5
            significance = significance + 1;
            countapnea = 0;
        end
        % reset k to next row after first apnea of the investigated hour
        k = apnearow+1;
        % end this loop
        l = row;
    end
end
% ...
\end{lstlisting}

\paragraph{Kriterium D5} Zum Zwecke der Klassifizierung der Daten in die Gruppen "`alle"', "`Herz"', "`Atmung"', "`Schlafmittel"' sowie "`Insomnie"' erfolgt eine visuelle Auswertung der Excel-Datei, welche die Patienteninformationen beinhaltet, sowie der Ergebnis-Dateien sleep\_stages.txt, consecutive\_stages.txt und Apnea5perhour.txt. Den ersten beiden Ergebnis-Dateien können die Patienten-IDs entnommen werden, welche die einzelnen Schlafstadien nicht alle oder nicht in ausreichender Länge durchlaufen haben. Aus der Ergebnis-Datei über die Apnoen ergeben sich die Patienten-IDs, deren Hypnogramme mehr als fünf Apnoe-Ereignisse pro Stunde aufweisen. Nach Recherche der medizinischen Fachbegriffe, welche in den Begleiterkrankungen und -medikationen in der Excel-Datei dokumentiert worden sind, lassen sich in einer zusätzlichen Spalte die klassifizierten Gruppen nach dem Kriterium D5 eintragen. Anschließend erstellt das Skript create\_insomdata\_struct.m ein Struct (\texttt{insomdata}) mit insgesamt 13 Key-Value-Paaren. Die Keys stellen Fields dar, welche sich auf die Inhalte der Excel-Datei sowie einige Zusatzinformationen beziehen. Die dazugehörigen Values werden anschließend für jeden Patienten aus der Excel-Datei ausgelesen und in das Struct geschrieben. Dieses Struct wird zur weiteren Verwendung in Matlab als .mat-Datei (insomdata.mat) gespeichert. Daraus können sodann zu jedem Patienten die Namen der \acs{EDF}-, Ereignis- und Hypnogramm-Dateien, das Geschlecht, Alter und Geburtstdatum, die Größe, das Gewicht, der \acs{BMI}, das Datum der \acs{PSG}-Aufzeichnung, Begleiterkrankungen- und medikationen sowie die klassifizierte Gruppe entnommen werden (Abb. \ref{fig:struct}). 

\begin{figure}[H]
	\centering
	\includegraphics[width = 0.6\textwidth]{img/struct.png}
	\caption[Struct \texttt{insomdata} mit Beispieldaten]{Struct \texttt{insomdata} mit Beispieldaten des Patienten ACA\_69003}
	\label{fig:struct}
\end{figure}

Ein weiteres Skript (extract\_goup\_information.m) ermöglicht das Laden des Structs und das Extrahieren gruppenspezifischer Informationen. Durch Verwendung der Funktion \texttt{arrayfun} in Matlab kann eine darin aufgerufene Funktion auf das gesamte Struct angewendet werden, ohne zeitaufwendige Schleifen zu durchlaufen. Dies ermöglicht beispielsweise den Vergleich aller Elemente (Values) eines Fields mit einem String (z. B. 'heart', um die Indizes der Gruppe Herz zu erhalten) und die Rückgabe der entsprechenden Indizes. Anhand dieser Indizes können sodann weitere Gruppeninformationen, wie Alter oder Geschlecht, aus dem Struct extrahiert und Minima, Maxima und Mittelwerte berechnet werden (Listing \ref{lst:D5}).\\

\begin{lstlisting}[caption={Implementierung Kriterium D5 in Skript extract\_goup\_information.m}, label={lst:D5}]
% ...
load(fullfile(insomstructpath,'insomdata.mat'));
% find indices of group heart
heart = arrayfun(@(x) strfind(x.group, 'heart'), insomdata, 'UniformOutput', 0);
heart_ind = find(~cellfun('isempty', heart)); % faster than find(~cellfun(@isempty, insomgroupheart))
% ...
heart_w = arrayfun(@(x) strcmp(x.sex, 'w'), insomdata(heart_ind));
heart_w_age = arrayfun(@(x) x.age, insomdata(heart_w)); % all ages of w heart
heart_w_meanage = mean(heart_w_age); % mean age of w heart
% ...
\end{lstlisting}

\section{TDS-Paket}\label{tdsPaket}

Für das Verfahren der TDS (Kapitel \ref{TDS}) liegt bereits ein von Prof. Dr. Dagmar Krefting in Matlab implementiertes Funktionspaket vor. Auf dieses wird im Folgenden zum Zwecke einer Übersicht grob eingegangen, um darauf aufbauend die für diese Arbeit notwendigen Anpassungen und Erweiterungen (Kapitel \ref{erweiterungTDS}) der in dem Paket enthaltenen Funktionen zu erläutern.\\

Im Wesentlichen müssen für die Ausführung der \acs{TDS}-Analyse drei Funktionen des Paketes ausgeführt werden, welche der Berechnung der \acs{TDS} sowie zur Erzeugung der Ergebnismatrizen für jedes Schlafstadium dienen (im Folgenden Hauptfunktionen genannt). Darüber hinaus besteht das Paket aus diversen weiteren Funktionen, welche intern innerhalb der Hauptfunktionen zur Ausführung von Zwischenschritten aufgerufen werden.\\

\paragraph{Ermittlung von Stabilitäten} Die erste der drei Hauptfunktionen (sn\_TDS.m) dient der Berechnung der Kreuzkorrelationskoeffizienten und Verschiebungen (Time Delays) sowie der Ermittlung von Stabilitäten. Zu diesem Zweck werden zu einer übergebenen \acs{PSG} im \acs{EDF}-Format gemäß Kapitel \ref{datenvorverarbeitung} die Zeitreihen der relevanten Signale extrahiert. Dies ist bislang zum einen möglich für \acs{PSG}s, welche die gleiche Standardsortierung der Signale aufweisen, wie die Daten aus der SIESTA-Studie. Hierbei werden die 14 in Kapitel \ref{TDS} beschriebenen und von Krefting et al. verwendeten Signale genutzt. Zum anderen ist die Funktion gültig für \acs{PSG}s, welche den Standardbezeichnungen von Signalen im Alice6-Format folgen, wobei insgesamt 17 Signale (zusätzlich Atem- und \acs{EMG}-Signale) verwendet werden. Anschließend werden sämtliche extrahierte Zeitreihen in der Matrix \texttt{biosignals\_tds} zusammengefasst. Die Zeilen dieser Matrix entsprechen dem zeitlichen Verlauf der Signale in Sekundenschritten. Die Spalten bilden die einzelnen Zeitreihen ab. Anhand dieser Matrix wird sodann die Kreuzkorrelation mit einer Fensterlänge von 60 Sekunden sowie einer Verschiebung von 30 Sekunden durchgeführt, wobei die Zeitreihen auf einen Mittelwert von 0 und eine Standardabweichung von 1 normalisiert werden. Die zurückgegebene Matrizen der Kreuzkorrelationskoeffizienten sowie der Verschiebungen entsprechen jeweils in der Zeilenanzahl dem zeitlichen Verlauf der Signale in Sekunden und in der Spaltenanzahl dem Produkt aus dem Vergleich sämtlicher Zeitreihen miteinander, so dass 

\begin{equation}
columns = (Anzahl~der~Signale)^2
	\label{eq:columns}
\end{equation}

gilt. Die Kreuzkorrelationsmatrix \texttt{xcc} enthält sämtliche Kreuzkorrelationskoeffizienten. Die Matrix \texttt{xcl} enthält sämtliche Werte der Verschiebungen (Time Delays). Zur Berechnung der Stabilitäten werden anschließend in Fenstern mit einer Länge von fünf Sekunden bei einer Verschiebung von einer Sekunde Minimum und Maximum der Verschiebungen aus der Matrix \texttt{xcl} innerhalb des Fensters berechnet. Überschreitet die Differenz dieser beiden Werte nicht den Wert 2, so gelten diese Verschiebungen als stabil. Sämtliche stabilen Verbindungen werden in einer Matrix \texttt{tds} mit dem Wert 1 gespeichert, sämtliche nicht stabilen Verbindungen mit dem Wert 0. Im Ergebnis weist die Matrix \texttt{tds} die gleichen Dimensionen wie die beiden vorherigen Matrizen auf und enthält demnach binäre Werte als Merkmal der Stabilität.

\paragraph{Aufteilung in Schlafstadien} Die zweite Hauptfunktion (sn\_getStagesTDS.m) ermittelt auf Grundlage des Hypnogramms die Verteilung auf die einzelnen Schlafstadien. Hierzu werden zunächst die Schlafstadien \acs{NREM}3 und \acs{NREM}4 gemäß Bashan et al. zusammengefasst, indem sämtliche Schlafstadien mit dem Wert 4 im Hypnogramm auf den Wert 3 gesetzt werden. Das Schlafstadium \acs{NREM}1 wird nicht berücksichtigt. Die Variable \texttt{nsis} enthält die Anzahl der Epochen für jedes Schlafstadium. Anhand der identifizierten Schlafstadien wird anschließend eine dreidimensionale Ergebnis-Matrix \texttt{result} erzeugt, welche die prozentualen Anteile der stabilen Verbindungen pro Schlafstadium enthalten. Diese Matrix entspricht in erster und zweiter Dimension den untersuchten Signalen. Die dritte Dimension steht für die vier Schlafstadien Tiefschlaf, Leichtschlaf, \acs{REM}-Schlaf und Wachzustand. 

\paragraph{Anzeigen der Ergebnisse} Mit Hilfe der dritten Hauptfunktion des \acs{TDS}-Paketes (sn\_plotTDSMatrixSiestaAll.m) lassen sich die Ergebnisse der \acs{TDS}-Analyse plotten. Hierzu wird für jedes der vier Schalfstadien durch die Matlab-Funktion \texttt{imagesc} ein Plot erzeugt. Die prozentualen Anteile werden farbig abgebildet. Dabei wird eine standardisierte Colormap verwendet, so dass die Ergebnis-Grafiken miteinander vergleichbar sind. Dunkelblaue Bereiche stellen eine sehr schwache und sehr helle Bereiche eine besonders starke Verbindungsstärke der Systeme dar. Diese Funktion bezieht sich auf den von Krefting et al. untersuchten SIESTA-Datensatz und muss für den hiesigen Untersuchungsdatensatz angepasst werden.


\section{Erweiterungen am TDS-Paket}\label{erweiterungTDS}

Um das \acs{TDS}-Verfahren auf den hiesigen Untersuchungsdatensatz anwenden zu können, müssen einige Erweiterungen des \acs{TDS}-Paketes implementiert werden. Auf diese sowie auf die Algorithmen zur Anwendung wird im Folgenden näher eingegangen.\\

Wie bereits in Kapitel \ref{datenanalyse} beschrieben unterscheiden sich die \acs{PSG}-Aufzeichnungen und die dazugehörigen Hypnogramme des hiesigen Untersuchungsdatensatzes in ihrer Länge. Um die Funktion zur Ermittlung von Stabilitäten in dieser Arbeit anwenden zu können, wird die Funktion sn\_TDS.m um die Option "`Charit\'{e}"' erweitert. Wird diese Option bei Aufruf der Funktion gesetzt, so erfolgt an den entsprechenden Stellen eine von der ursprünglichen Funktion abweichende Funktionsausführung. Diese Stellen beziehen sich zum einen auf die initialen Zuweisungen der Signalkanäle und zum anderen auf sämtliche Aufrufe der einzelnen Signale innerhalb der \acs{PSG} zur Extraktion der entsprechenden Zeitreihen.\\

Zunächst muss die zeitliche Differenz (in Sekunden) der Startzeitpunkte aus der zuvor neu erzeugten Hypnogramm-Datei extrahiert und anschließend durch Multiplikation mit der Abtastfrequenz der einzelnen Signale in Samples umgerechnet werden. Diese Ergebnisse werden in der Variablen \texttt{durationsamples} für alle Signale gespeichert. Als nächstes wird die Länge des Hypnogramms aus der Hypnogramm-Datei ausgelesen, in Sekunden umgerechnet und für jedes einzelne Signal mit dessen Abtastfrequenz multipliziert, um die Sampleanzahl des Hypnogramms für jedes Signal in Abhängigkeit von dessen Abtastfrequenz zu erhalten. Das Ergebnis wird ebenfalls für alle Signale in einer Variablen \texttt{hypnogramduration} gespeichert. Zuletzt wird die Länge jedes Signals ausgelesen und ebenfalls für sämtliche in der \acs{PSG} enthaltenen Signale in der Variablen \texttt{signallengths} gespeichert. Durch Abgleich dieser drei Variablen wird anschließend für jedes Signal die Bedingung geprüft, ob die Hypnogrammlänge addiert mit der Zeitdifferenz zwischen den Startzeitpunkten kleiner ist als die Signallänge. Das Ergebnis wird für jedes überprüfte Signal in die Variable \texttt{lengthflag} geschrieben. Ist die Bedingung erfüllt, so enthält \texttt{lengthflag} für das überprüfte Signal den Wert 1, anderenfalls den Wert 0 (Listing \ref{lst:snTDS}).\\

\begin{lstlisting}[caption={Anpassung für den Charit\'{e}-Datensatz in der Funktion sn\_TDS.m}, label={lst:snTDS}]
% ...
elseif(strcmp(ch_all,'charite'))
    disp('all channels following Charite standard are processed')
    % hypnogram file exists
    if hypno_flag
        % read hypnogram
        T = dlmread(hypno_filename);
        %starttime difference in seconds
        durationseconds = T(1); 
        %starttime difference in samples for each signal
        durationsamples = [signalheader(:).samples_in_record]/...
            header.data_record_duration * durationseconds;
        %hypnogram duration in samples for each signal
        hypnogramduration = length(T(2:end))*30*sfch; 
        %lengths of all signals
        signallengths = cellfun('length',signalcells);
        %boolean array - 1 if hypnogram+starttime difference is smaller
        %than signal length 
        lengthflag = (durationsamples(:)+1+hypnogramduration(:) <= signallengths(:));
    else
        disp('Hypnogram file is missing, please select hypnogram text file.')
    % ...
    end
    %channelnumbers of EEG signals
    ch_eeg = 2; %EEG C4-A1
    ch_eeg2 = 3; %EEG O2-A1
    ch_eeg3 = 5; %EEG C3-A2
    ch_eeg4 = 6; %EEG O1-A2
% ...
\end{lstlisting}

Da im hiesigen Untersuchungsdatensatz keine frontalen EEG-Signale enthalten sind und demnach weniger Signale als im SIESTA-Datensatz existieren, muss die Funktion zur Ermittlung von Stabilitäten für die hier verwendeten Daten angepasst werden. Ist die Option "`Charit\'{e}"' gesetzt, so werden die Kanäle gemäß Tab. \ref{tab:labels} den Signalen zugeordnet (Listing \ref{lst:snTDS}). Das zusätzliche \acs{EMG} des rechten Beins wird hierbei außer Acht gelassen, um die Vergleichbarkeit mit den Ergebnissen von Krefting et al. zu gewährleisten. Zur Erstellung der Zeitreihen wird sodann vor jeder Verwendung eines Signals anhand des \texttt{lengthflag}  die Bedingung geprüft, ob ab dem Startzeitpunkt des Hypnogramms das Hypnogramm kürzer ist als das Signal (Listing \ref{lst:zeitreihen}). Ist die Bedingung erfüllt, bedeutet dies, dass die \acs{PSG}-Aufzeichnung zu einem späteren Zeitpunkt endet als das Hypnogramm, so dass die zu verwendenden Ausschnitte der Signale in dem Bereich des Hypnogramms (Hypnogrammstart bis Hypnogrammende) liegen müssen. Die Signale werden in diesem Fall für die Berechnung der Zeitreihen demnach am Anfang um die Länge der zeitlichen Differenz und am Ende um die Zeitspanne, die die Signale die Länge des Hypnogramms überschreiten, gekürzt. In diesem Fall enthält \texttt{lengthflag} für das überprüfte Signal den Wert 1. Ist die Bedingung nicht erfüllt, bedeutet dies, dass der Endzeitpunkt des Hypnogramms zeitlich hinter dem der \acs{PSG} liegt, so dass das Signal in dem Bereich ab dem Startzeitpunkt des Hypnogramms bis zum Signalende verwendet wird.\\

\begin{lstlisting}[caption={Berechnung der Zeitreihen für den Charit\'{e}-Datensatz in der Funktion sn\_TDS.m}, label={lst:zeitreihen}]
% ...
if strcmp(ch_all, 'charite')
    % hypnogram ends before signal ends - take signal from hypnogram starttime until hypnogram endtime
    if lengthflag(ch_emgchin)
        var_emgchin = sn_getVariance(signalcells{ch_emgchin}...
            (durationsamples(ch_emgchin)+1:durationsamples(ch_emgchin)...
            +hypnogramduration(ch_emgchin)), 'wl', wl_sfe, 'ws',...
            ws_sfe, 'sf',sfch(ch_emgchin));
    % signal ends before hypnogram ends - take signal from hypnogram starttime until signal endtime    
    else
        var_emgchin = sn_getVariance(signalcells{ch_emgchin}...
            (durationsamples(ch_emgchin)+1:end), 'wl', wl_sfe,...
            'ws', ws_sfe, 'sf', sfch(ch_emgchin));
    end
% ...
\end{lstlisting}

Um nach Aufteilung der ermittelten Stabilitäten auf die einzelnen Schlafstadien die Ergebnis-Matrizen plotten zu können, muss ebenfalls eine Anpassung an den Charit\'{e}-Datensatz erfolgen. Hierzu wird die Funktion sn\_plotTDSMatrixCharite.m erstellt, welche sich an der Funktion sn\_plotTDSMatrixSiestaAll.m orientiert und den Plot der \acs{TDS}-Ergebnisse für die Charit\'{e}-Daten durch Anpassung an das hiesige Format ermöglicht.

\section{Anwendung auf den Charit\'{e}-Datensatz}

Zur Anwendung des \acs{TDS}-Paketes wird ein Skript (apply\_TDS.m) erstellt, welches die Ausführung der Funktionen auf den gesamten Untersuchungsdatensatz erlaubt. Zu diesem Zweck werden zunächst sämtliche \acs{EDF}-Dateien und Hypnogramme gelistet und anschließend in einer for-Schleife an die \acs{TDS}-Funktionen übergeben. Bei dem Aufruf der Funktion sn\_TDS.m müssen sodann die \acs{EDF}-Datei sowie die Hypnogramm-Datei übergeben werden. Zusätzlich wird die Option "`Charit\'{e} gesetzt. Die Ergebnis-Matrizen \texttt{tds}, \texttt{xcc}, \texttt{xcl} sowie \texttt{biosignals\_tds} werden in einem Ergebnispfad gespeichert. Anschließend wird die Hypnogramm-Datei ausgelesen und als Array gemeinsam mit der Matrix \texttt{tds} an die Funktion sn\_getStagesTDS.m übergeben. Die Ergebnisse werden ebenfalls im Ergebnispfad als Matrizen gespeichert. Für die Aufteilung der Ergebnisse in die einzelnen Schlafstadien könnten in dieser Funktion Artefakte aus den Berechnungen ausgeschlossen werden. Um das Kriterium D6 der Datenanalyse (Kapitel \ref{datenanalyse}) zu erfüllen, kommt diese Option hier ausdrücklich nicht zum Einsatz. Zuletzt erfolgt das Plotten der \acs{TDS}-Ergebnisse.\\

Um die Ergebnisdaten strukturiert zusammenzufassen und eine Ergebnisanalyse für andere Nutzer zu ermöglichen, wird mit Hilfe des Skripts create\_insomresults\_struct.m ein Struct (\texttt{insomresults}) erzeugt, welches zu jedem Patientendatensatz die \acs{TDS}-Ergebnisse enthält. Dieses Struct enthält demnach für jeden Patienten die Felder Patienten-ID ("`name"'), Hypnogramm ("`hypnogram"'), Zeitreihen ("`biosignals"'), Time Delays ("`xcl"'), Kreuzkorrelationskoeffizienten ("`xcc"'), stabile Verbindungen ("`tds"'), Anzahl der Epochen pro Schlafstadium ("`nsis"') sowie die prozentualen Verbindungsstärken in jedem Schlafstadium ("`result"') (Abb. \ref{fig:resultstruct}).

\begin{figure}[H]
	\centering
	\includegraphics[width = 0.4\textwidth]{img/insomresults_struct.png}
	\caption[Struct \texttt{insomresults} mit Beispieldaten]{Struct \texttt{insomresults} mit Beispieldaten des Patienten ACA\_69003}
	\label{fig:resultstruct}
\end{figure}

Zum Plotten der Ergebnis-Matrizen der einzelnen Gruppen werden gemäß Kapitel \ref{mean} mit Hilfe der Funktion calculate\_mean.m die Mittelwerte berechnet und die gemittelten Ergebnis-Matrizen für jedes Schalfstadium visualisiert.

\section{Berechnung der Mittelwerte}\label{mean}

Zur Berechnung der Mittelwerte \texttt{smean} der \acs{TDS}-Matrizen verschiedener Gruppen dient die Funktion calculate\_mean.m. Dieser Funktion muss lediglich die entsprechende Gruppenbezeichnung übergeben werden. Anschließend wird das Struct \texttt{insomdata} geladen und es werden die Indizes der entsprechenden Gruppe berechnet. Dies ist möglich für die Gruppen gemäß der Klassifizierung innerhalb der Datenanalyse sowie für junge (jünger als 34 Jahre), ältere (älter als 60 Jahre), weibliche und männliche Insomniepatienten. Anschließend werden aus dem Struct \texttt{insomresults} die Anzahl der Epochen pro Schlafstadium \texttt{nsis} sowie die Ergebnis-Matrizen \texttt{result} jedes Patienten der gewählten Gruppe extrahiert. Die Berechnung der Mittelwerte erfolgt mit Hilfe der Matlab-Funktion \texttt{mean}. Für den Fall, dass bei der \acs{TDS}-Analyse NaN-Elemente (Not a Number) auftreten, wird bei der Matlab-Funktion \texttt{mean} zur Berechnung des Mittelwerts die Option "`omitnan"' gesetzt, so dass NaN-Elemente nicht berücksichtigt werden. Im Ergebnis wird ein Vektor, welcher die gemittelte Anzahl der Epochen pro Schlafstadium enthält, sowie eine dreidimensionale Matrix der gemittelten Verbindungsstärken pro Schlafstadium zurückgegeben. Zuletzt werden die gemittelten Verbindungsstärken der vier Schlafstadien geplottet.\\

Zur Berechnung des globalen Mittelwertes \texttt{nmean} dient die Funktion calculate\_nmean\_ age.m. Zum Zwecke der späteren Gegenüberstellung der globalen Verbindungsstärke \texttt{nmean} und des Alters der Insomniepatienten wird in dieser Funktion darüber hinaus das Alter der Patienten extrahiert. Zunächst werden die Structs \texttt{insomdata} und \texttt{insomresults} geladen und die Indizes der entsprechenden Gruppe mit Hilfe der Matlab-Funktion \texttt{arrayfun} und durch Abgleich mit dem Struct \texttt{insomdata} ermittelt. Anschließend wird die Rückgabe-Matrix \texttt{nmean\_age} alloziert, welche später für sämtliche Patienten der Gruppe die globalen Mittelwerte der Verbindungsstärken \texttt{nmean} sowie das Alter enthält. In einer for-Schleife über die gesamte Gruppe werden zunächst die Patienten-ID sowie die Ergebnis-Matrix jedes Patienten aus dem Struct \texttt{insomresults} extrahiert. Durch Abgleich der Patienten-ID aus dem Struct \texttt{insomdata} wird darüber hinaus das jeweilige Alter extrahiert. Für jedes Schlafstadium werden anschließend in zwei for-Schleifen die Zeilen und Spalten der Ergebnis-Matrix in der Form durchlaufen, dass lediglich das untere Dreieck der Matrix und ohne die Diagonale betrachtet wird. Für diese Positionen werden die Verbindungsstärken addiert und gemäß Kapitel \ref{calcmean} durch die Anzahl der addierten Indizes abzüglich 14 dividiert (Listing \ref{lst:nmean}). Dieses Ergebnis sowie das Alter werden für jeden Patienten und für jedes Schlafstadium in die Rückgabe-Matrix \texttt{nmean\_age} geschrieben und diese als .mat-Datei gespeichert.\\

\begin{lstlisting}[caption={Berechnung des globalen Mittelwertes in der Funktion calculate\_nmean\_age.m}, label={lst:nmean}]
% ...
for s = 1:4
    tdssum = 0;
    % loop over result matrix in x and y direction
    for k = 1:length(tdsmat)
        for l = k+1:length(tdsmat)
            % sum link strengths for each sleep stage
            tdssum = tdssum+tdsmat(k, l, s);
        end
    end
    % calculate global mean link strength and age for each
    % patient and sleep stage
    nmean_age(i, 1, s) = tdssum/((length(tdsmat)^2)/2 - length(tdsmat));
    nmean_age(i, 2, s) = patage;
end
% ...
\end{lstlisting}

\section{Gegenüberstellung Alter - Verbindungsstärke}

Zur Untersuchung des Zusammenhangs zwischen dem Alter und den globalen Verbindungsstärken der Patienten, wird die in Matlab erstellte Matrix \texttt{nmean\_age.mat} verwendet. Hierzu dient das R-Skript linreg\_nmean\_age.R. Zunächst wird das Matlab-Paket importiert, wodurch das Laden von Matlab-Variablen in R mit Hilfe der Funktion \texttt{readMat} ermöglicht wird. Die Variable nmean\_age.mat wird in R als eine Liste mit einer Länge von 512 Elementen angelegt, so dass die Elemente in R aneinandergereiht werden. Um auf die einzelnen Elemente dieser Liste (Verbindungsstärke und Alter jedes Patienten pro Schlafstadium) zugreifen zu können, wird die Liste durch die Funktion \texttt{unlist} in ein dreidimensionales Array vom Typ \texttt{double} und einer Länge von jeweils 64 Elementen umgewandelt. Dies entspricht dem Format, wie die Variable in Matlab erzeugt worden ist. Anschließend können die globalen Verbindungsstärken sowie das Alter der Patienten für jedes Schlafstadium als numerische Vektoren vom Typ \texttt{double} extrahiert werden. Aus diesen Vektoren wird für jedes Schlafstadium ein Data Frame erstellt. Aus diesen werden sodann für jedes Schlafstadium und für die weiblichen, für die männlichen sowie für alle Patienten lineare Modelle erstellt. Hierbei werden die globalen Verbindungsstärken als abhängige Variable und das Alter als unabhängige Variable gesetzt. Anschließend wird ein Streudiagramm mit unterschiedlicher Darstellung der weiblichen und männlichen Patienten erzeugt. Mit Hilfe der R-Funktion \texttt{abline} sowie der linearen Modelle können sodann die Regressionsgeraden eingezeichnet werden.\\

\lstset{language=R, otherkeywords={}, keywords={data,frame,lm,points,abline,pch,col,lty}}
\begin{lstlisting}[caption={Erstellung eines Streudiagramms und Einzeichnung der Regressionslinien in dem Skript linreg\_nmean\_age.R}, label={lst:nmean_age}]
# ...
# create data frames
insomdata_all_ds <- data.frame(age_all, nmean_all_ds)
# ...
# create linear models
insomdata_all_ls.linm <- lm(nmean_all_ls~age_all,data=insomdata_all_ls)
# ...
# plot female pairs in light sleep
points(insomdata_f_ls,pch=19)
# plot male pairs in light sleep
points(insomdata_m_ls, pch=21)
# add a line for linear regression for all, female and male 
abline(insomdata_all_ls.linm, col="green",lty=1)
abline(insomdata_f_ls.linm, col="red",lty=2)
abline(insomdata_m_ls.linm, col="blue",lty=3)
# ...
\end{lstlisting}

Mit Hilfe der R-Funktion \texttt{summary} können darüber hinaus die Parameter der linearen Regression berechnet und ausgegeben werden. Die R-Funktion \texttt{shapiro.test} dient der Überprüfung auf Normalität. Zu diesem Zweck wird eine Stichprobe von jeweils 20 Elementen erstellt. Die Funktion liefert die W-Statistik sowie den p-Wert zurück. Die Korrelationsverfahren nach Pearson und nach Spearman werden durch die R-Funktion \texttt{cor.test} mit den Optionen \texttt{method="pearson"} bzw. \texttt{method="spearman"} implementiert. Diese geben die Korrelationkoeffizienten sowie den p-Wert zurück.

%p-Wert nach Pearson entspricht dem p-Wert der linearen Regression




\newpage
\chapter{Ergebnisse}

Hier kommen die Ergebnisse



\newpage
\chapter{Abschluss}

Hier kommt die Einführung zum Abschluss

\section{Zusammenfassung}

Hier kommt die Zusammenfassung

\section{Fazit}

Hier kommt das Fazit

\section{Ausblick}

Hier kommt der Ausblick



% ============= Literaturverzeichnis =============

%\pagestyle{fancy}

\newpage
\pagenumbering{Roman}
\lhead{}
\renewcommand{\headrulewidth}{0pt}

%\pagestyle{empty}
\addcontentsline{toc}{chapter}{Literaturverzeichnis}
\setcounter{page}{7}
\printbibliography

% ============= Anhang =============

%\appendix
\newpage
%\pagestyle{fancy}
\pagenumbering{Roman}
%\lhead{}
%\renewcommand{\headrulewidth}{0pt}
\addcontentsline{toc}{chapter}{Anhang}
\setcounter{page}{11}
\newgeometry{left= 3 cm,right = 2.5 cm, bottom = 2.5 cm, top = 6 cm}

% in Inhaltsverzeichnis aufnehmen
%\addcontentsline{toc}{chapter}{Anhang}
%\sectionmark{Anhang}
\lhead{}


%\chapter{}%alles gut, aber im Anhang steht nur A.
%\chapter{Anhang}%überall steht A....
%\section*{\Huge{Anhang}}
%\Huge{Anhang}%Verzeichnis okay, aber bei Abb fehlt A.
\addchap{Anhang}%macht alles richtig, aber leere Seite vorher
\refstepcounter{chapter}

\begin{table}[H] 
\centering
\begin{tabularx}{\textwidth}{cX}
\toprule
\multicolumn{2}{c}{\textbf{Allgemeine Insomniekriterien}}\\
\midrule 
a) & Eine Beschwerde über Einschlafschwierigkeiten, Durchschlafprobleme, frühmorgendliches Erwachen oder Schlaf von chronisch nicht erholsamer oder schlechter Qualität. Bei Kindern wird die Schlafschwierigkeit zumeist durch die Erziehungsperson bemerkt und kann darin bestehen, dass die Kinder nicht zu Bett gehen wollen oder nicht unabhängig (d. h. im eigenen Bett) von ihren Eltern schlafen können.\\
b) & Die genannte Schlafschwierigkeit tritt auf, obwohl adäquate Möglichkeiten und Umstände dafür vorhanden sind, genügend Schlaf zu bekommen.\\
c) & Zumindest eine der folgenden Formen von Beeinträchtigungen der Tagesbefindlichkeit/Leistung, die auf die nächtliche Schlafschwierigkeit zurückgeführt werden kann,wird vom Patienten berichtet:
\begin{itemize}
\singlespacing
\setlength\itemsep{0em}
\item Müdigkeit (Fatigue) oder Krankheitsgefühl
\item Beeinträchtigung der Aufmerksamkeit, Konzentration oder des Gedächtnisses
\item Soziale oder berufliche Einschränkungen oder schlechte Schulleistungen
\item Irritabilität oder Beeinträchtigungen der Stimmung (z. B. Gereiztheit)
\item Tagesschläfrigkeit
\item Reduktion von Motivation, Energie oder Initiative
\item Erhöhte Anfälligkeit für Fehler, Arbeitsunfälle oder Unfälle beim Führen eines Kraftfahrzeugs
\item Spannungsgefühle, Kopfschmerzen oder gastrointestinale Symptome als Reaktion auf das Schlafdefizit
\item Sorgen um den Schlaf
\end{itemize}\\
\bottomrule
\end{tabularx}
\caption[Allgemeine Insomniekriterien]{Allgemeine Insomniekriterien nach \acs{ICSD-2} \parencite{mayer_s3-leitlinie_2009}}
\label{tab:allgemeine_insomnie}
\end{table}



\newpage
\restoregeometry
\lhead{\slshape \MakeUppercase{Anhang}}
\renewcommand{\headrulewidth}{0.4pt}



\begin{table}[H] 
\centering
\begin{tabularx}{\textwidth}{cX}
\toprule
\multicolumn{2}{c}{\textbf{Kriterien der anpassungsbedingten Insomnie}}\\
\midrule 
a) & Die Symptome des Patienten entsprechen den allgemeinen Insomniekriterien.\\
b) & Die Schlafstörung ist zeitlich assoziiert mit dem identifizierbaren Stressor auf psychologischer, psychosozialer, interpersoneller, umweltbedingter, physikalischer oder medizinischer Ebene.\\
c) & Die Schlafstörung löst sich auf,wenn der akute Stressor nicht mehr vorhanden ist, oder wenn das Individuum sich an den Stressor anpasst.\\
d) & Die Schlafstörung dauert weniger als drei Monate.\\
e) & Die Schlafstörung kann nicht besser erklärt werden durch eine andere gegenwärtige Schlafstörung, eine medizinische, neurologische oder psychiatrische Erkrankung oder die Einnahme von Medikamenten oder Substanzen, die den Schlaf stören können.\\
\midrule
\multicolumn{2}{p{0.97\textwidth}}{Die Einjahresprävalenz bei Erwachsenen liegt bei 15 \% bis 20 \%. Die anpassungsbedingte Insomnie tritt bei Frauen und Männern in jedem Alter auf, wobei Frauen häufiger betroffen sind. Im Alter steigt die Prävalenz.}\\
\bottomrule
\end{tabularx}
\caption[Kriterien der anpassungsbedingten Insomnie]{Diagnostische Kriterien der anpassungsbedingten (akuten) Insomnie nach \acs{ICSD-2} \parencite{mayer_s3-leitlinie_2009, happe_schlafmedizin_2009}}
\label{tab:akute_insomnie}
\end{table}



\begin{table}[H] 
\centering
\begin{tabularx}{\textwidth}{cX}
\toprule
\multicolumn{2}{c}{\textbf{Kriterien der idiopathischen Insomnie}}\\
\midrule 
a) & Die Beschwerden des Patienten entsprechen den allgemeinen Insomniekriterien.\\
b) & Der Verlauf der Erkrankung ist chronisch, belegt durch die folgenden Symptome:
\begin{itemize}
\singlespacing
\setlength\itemsep{0em}
\item Beginn während Säuglingsalter oder Kindheit
\item Kein identifizierbarer vorhergehender Stressor bzw. keine Ursache
\item Persistiert im Verlauf ohne Perioden längerer Remission
\end{itemize}\\
c) & Die Schlafstörung kann nicht besser erklärt werden durch eine andere Schlafstörung, medizinische, neurologische, psychische Erkrankung oder Medikamenten- oder Substanzeinnahme.\\
\midrule
\multicolumn{2}{p{0.97\textwidth}}{Die Prävalenz liegt bei 0,7 \% bis 1 \% bei jungen Erwachsenen. Frauen und Männer sind gleichermaßen betroffen. Eine familiäre Vulnerabilität sowie eine Häufung bei gleichzeitiger ADHS-Erkrankung werden vermutet.}\\
\bottomrule
\end{tabularx}
\caption[Kriterien der idiopathischen Insomnie]{Diagnostische Kriterien der idiopathischen Insomnie nach \acs{ICSD-2} \parencite{mayer_s3-leitlinie_2009, happe_schlafmedizin_2009}}
\label{tab:idiopathische_insomnie}
\end{table}



\newpage


\begin{table}[H] 
\centering
\begin{tabularx}{\textwidth}{cX}
\toprule
\multicolumn{2}{c}{\textbf{Kriterien der psychophysiologischen Insomnie}}\\
\midrule 
a) & Die Symptome des Patienten entsprechen den allgemeinen Insomniekriterien.\\
b) & Die insomnischen Symptome bestehen mindestens einen Monat.\\
c) & Der Betroffene zeigt Anzeichen eines konditionierten Schlafproblems und/oder erhöhten Arousals im Bett durch eines oder mehrere der folgenden Symptome:
\begin{itemize}
\singlespacing
\setlength\itemsep{0em}
\item Exzessives Fokussieren auf und erhöhte Angst um den Schlaf
\item Einschlafschwierigkeiten zur geplanten Bettzeit oder während beabsichtigter Tagschlafepisoden, aber keine Schlafprobleme während monotoner Aktivitäten, wenn Schlaf nicht beabsichtigt ist
\item Besserer Schlaf in anderer als der gewohnten Schlafumgebung
\item Das kognitive Arousal im Bett wird charakterisiert durch intrusive Gedanken oder die wahrgenommene Unfähigkeit, willentlich schlafverhindernde kognitive Aktivität abzustellen
\item Erhöhte körperliche Anspannung im Bett manifestiert sich in wahrgenommener Unfähigkeit, körperlich zu entspannen, um den Schlafbeginn einzuleiten
\end{itemize}\\
d) & Die Schlafbeschwerde kann nicht besser durch eine andere Schlafstörung, eine medizinische, neurologische oder psychische Erkrankung, Medikamenten- oder Substanzeinnahme erklärt werden.\\
\midrule
\multicolumn{2}{p{0.97\textwidth}}{Von der psychophysiologischen Insomnie sind hauptsächlich Frauen im mittleren Alter betroffen. Die Prävalenz steigt im Alter.}\\
\bottomrule
\end{tabularx}
\caption[Kriterien der psychophysiologischen Insomnie]{Diagnostische Kriterien der psychophysiologischen Insomnie nach \acs{ICSD-2} \parencite{mayer_s3-leitlinie_2009, happe_schlafmedizin_2009}}
\label{tab:psycho_insomnie}
\end{table}



\newpage


\begin{table}[H] 
\centering
\begin{tabularx}{\textwidth}{cX}
\toprule
\multicolumn{2}{c}{\textbf{Kriterien der paradoxen Insomnie}}\\
\midrule 
a) & Die Beschwerden des Patienten entsprechen den allgemeinen Insomniekriterien.\\
b) & Die Insomniebeschwerden bestehen mindestens einen Monat.\\
c) & Eines oder mehrere der folgenden Kriterien treffen zu:
\begin{itemize}
\singlespacing
\setlength\itemsep{0em}
\item Die Patienten berichten über ein chronisches Muster von wenig oder gar keinem Schlaf mit seltenen Nächten, während derer relativ normale Mengen an Schlaf auftreten
\item Daten aus dem Schlaftagebuch, über eine oder mehrere Wochen erfasst, zeigen, dass die Betroffenen eine durchschnittliche Schlafzeit haben, die deutlich unter der der altersentsprechenden Normgruppen liegt. Oft wird für mehrere Nächte hintereinander gar kein Schlaf erlebt. Keine Tagschlafepisoden nach solchen Nächten
\item Patienten zeigen konsistent das Missverhältnis zwischen objektiven Befunden aus der Polysomnographie oder Aktigraphie und ihren subjektiven Schlafeinschätzungen
\end{itemize}\\
d) & Mindestens eines der folgenden Symptome tritt auf:
\begin{itemize}
\singlespacing
\setlength\itemsep{0em}
\item Die Patienten berichten über ständige Wahrnehmungen von nächtlichen Stimuli (z. B. Schlagen der Kirchturmuhr)
\item Die Patienten berichten über anhaltendes Vorhandensein von Gedanken oder Grübeleien während der ganzen Nacht
\end{itemize}\\
e) & Die Tagesbeeinträchtigung, die die Patienten berichten, ist konsistent mit dem, was von anderen Schlafgestörten berichtet wird, aber sie ist weniger ausgeprägt im Verhältnis zum berichteten Schlafverlust. Kein Hinweis auf abrupten Tagschlaf. Sekundenschlaf, Desorientierung oder massive Fehler oder Irrtümer als Folge eines Schlafentzugs treten auf.\\
f) & Die berichtete Schlafstörung wird nicht besser durch eine andere Schlafstörung, medizinische, neurologische oder psychische Erkrankung, Medikamenteneinnahme oder Substanzeinnahme erklärt.\\
\midrule
\multicolumn{2}{p{0.97\textwidth}}{Die paradoxe Insomnie tritt zumeist im Alter zwischen 20 und 40 Jahren bei Frauen und Männern gleichverteilt auf.}\\
\bottomrule
\end{tabularx}
\caption[Kriterien der paradoxen Insomnie]{Diagnostische Kriterien der paradoxen Insomnie nach \acs{ICSD-2} \parencite{mayer_s3-leitlinie_2009, happe_schlafmedizin_2009}}
\label{tab:paradoxe_insomnie}
\end{table}



\newpage


\begin{table}[H] 
\centering
\begin{tabularx}{\textwidth}{cX}
\toprule
\multicolumn{2}{c}{\textbf{Kriterien der Insomnie durch psychische Erkrankung}}\\
\midrule 
a) & Die Symptome des Patienten entsprechen den allgemeinen Insomniekriterien.\\
b) & Die Insomnie besteht mindestens einen Monat.\\
c) & Eine psychische Erkrankung wurde nach Standardkriterien diagnostiziert.\\
d) & Die Insomnie ist zeitlich eng verknüpft mit der psychischen Erkrankung, sie kann jedoch in einigen Fällen einige Tage oder Wochen vor dem Beginn der psychischen Erkrankung auftreten.\\
e) & Die Insomnie ist hervorstechender als typischerweise assoziiert mit der psychischen Erkrankung und führt eigenständig zu erhöhtem Stress oder stellt einen unabhängigen Behandlungsfokus dar.\\
f) & Die Schlafstörung wird nicht besser erklärt durch eine andere Schlafstörung, medizinische, neurologische Erkrankung, Medikamenten- oder Substanzeinnahme.\\
\midrule
\multicolumn{2}{p{0.97\textwidth}}{Psychische Erkrankungen stellen die häufigste Ursache von Schlafstörungen dar. Etwa 80 \% bis 90 \% der Menschen, die an einer Depression leiden, erkranken zusätzlich an einer Insomnie.}\\
\bottomrule
\end{tabularx}
\caption[Kriterien der Insomnie durch psychische Erkrankung]{Diagnostische Kriterien der Insomnie durch psychische Erkrankung nach \acs{ICSD-2} \parencite{mayer_s3-leitlinie_2009, happe_schlafmedizin_2009}}
\label{tab:insomnie_psychisch}
\end{table}




\begin{table}[H] 
\centering
\begin{tabularx}{\textwidth}{cX}
\toprule
\multicolumn{2}{c}{\textbf{Kriterien der Insomnie durch körperliche Erkrankung}}\\
\midrule 
a) & Die Symptome des Patienten entsprechen den allgemeinen Insomniekriterien.\\
b) & Die Insomnie besteht ca. einen Monat lang.\\
c) & Der Patient hat eine koexistierende organische oder körperliche Bedingung, die den Schlaf stören kann.\\
d) & Die Insomnie ist zeitlich eng assoziiert mit der körperlichen Erkrankung. Sie begann um den Zeitpunkt des Beginns der körperlichen Erkrankung, und die Progression entspricht der Progression der zugrunde liegenden körperlichen Erkrankung.\\
e) & Die Schlafstörung kann nicht besser erklärt werden durch eine andere gegenwärtige Schlafstörung, eine medizinische, neurologische oder psychische Erkrankung oder die Einnahme von Medikamenten oder Substanzen, die den Schlaf stören können.\\
\midrule
\multicolumn{2}{p{0.97\textwidth}}{Die Prävalenz der Insomnie durch körperliche Erkrankung liegt bei 0,5 \%. Zumeist sind ältere Menschen betroffen.}\\
\bottomrule
\end{tabularx}
\caption[Kriterien der Insomnie durch körperliche Erkrankung]{Diagnostische Kriterien der Insomnie durch körperliche Erkrankung nach \acs{ICSD-2} \parencite{mayer_s3-leitlinie_2009, happe_schlafmedizin_2009}}
\label{tab:korperliche_insomnie}
\end{table}



\newpage


\begin{table}[H] 
\centering
\begin{tabularx}{\textwidth}{cX}
\toprule
\multicolumn{2}{c}{\textbf{Kriterien der Insomnie durch inadäquate Schlafhygiene}}\\
\midrule 
a) & Die Symptome des Patienten entsprechen den Allgemeinen Insomniekriterien.\\
b) & Die Insomniebeschwerden bestehen ca. einen Monat lang.\\
c) & Inadäquate Schlafhygiene ist belegt durch mindestens eines der folgenden Symptome:
\begin{itemize}
\singlespacing
\setlength\itemsep{0em}
\item Irregulärer Schlaf-Wach-Rhythmus mit häufigem Tagschlaf, variable Bettzeiten oder Aufstehzeiten oder auch sehr lange Bettzeiten
\item Gewöhnlicher Gebrauch von Alkohol, Nikotin oder Koffein, speziell vor der Bettzeit
\item Ausführen kognitiv stimulierender oder emotional stimulierender Aktivitäten nahe an der Bettzeit
\item Das Bett wird für andere Aktivitäten als für Schlaf benutzt (Fernsehen, Lesen, Studieren, Essen etc.)
\item Den Betroffenen gelingt es nicht, eine behagliche Schlafumgebung zu schaffen
\end{itemize}\\
d) & Die Schlafstörung wird nicht besser erklärt durch eine andere Schlafstörung, medizinische, neurologische, psychische Erkrankung, Medikamenten- oder Substanzeinnahme.\\
\bottomrule
\end{tabularx}
\caption[Kriterien der Insomnie durch inadäquate Schlafhygiene]{Diagnostische Kriterien der Insomnie durch inadäquate Schlafhygiene nach \acs{ICSD-2} \parencite{mayer_s3-leitlinie_2009}}
\label{tab:schlafhygiene_insomnie}
\end{table}




\begin{figure}[H]
	\centering
	\includegraphics[width = \textwidth]{img/spindles_k-complex.png}
	\caption[Schlafspinden und K-Komplexe]{\acs{PSG} in der Schlafphase \acs{NREM}2 mit K-Komplex (blau markiert) und Schlafspindeln (grün markiert) im \acs{EEG} (unter Verwendung von \parencite{lee-chiong_sleep_2008})}
	\label{fig:spindel_k-komplex}
\end{figure}


\begin{table}[H] 
\centering
\begin{tabularx}{0.855\textwidth}{ccccccccccc}
\toprule
\multicolumn{4}{c}{\textbf{Gruppe Alle}} & & & & & & &\\  
\cmidrule{1-4}
w  & m  & A  &    & wA   & wminA & wmaxA &    & mA   & mminA & mmaxA\\
\midrule
41 & 23 & 51 & ~~ & 52,3 & 23    & 65    & ~~ & 48,7 & 26    & 61\\
\bottomrule
\end{tabularx}
\caption[Gruppe Alle]{Klassifizierung der Patientengruppe Alle; w = Anzahl weiblicher Patienten;\\m = Anzahl männlicher Patienten; A = durchschnittliches Gruppenalter;\\wA = Durchschnittsalter weiblicher Patienten; wminA = minimales Alter weiblicher Patienten;\\wmaxA = maximales Alter weiblicher Patienten; mA = Durchschnittsalter männlicher Patienten;\\mminA = minimales Alter männlicher Patienten; mmaxA = maximales Alter männlicher Patienten}
\label{tab:Alle}
\end{table}


\begin{table}[H] 
\centering
\begin{tabularx}{0.87\textwidth}{ccccccccccc}
\toprule
\multicolumn{4}{c}{\textbf{Gruppe Herz}} & & & & & & &\\  
\cmidrule{1-4}
w  & m & A    &    & wA   & wminA & wmaxA &    & mA   & mminA & mmaxA\\
\midrule
10 & 5 & 56,8 & ~~ & 57,3 & 41    & 65    & ~~ & 55,8 & 44    & 61\\
\bottomrule
\end{tabularx}
\caption[Gruppe Herz]{Klassifizierung der Patientengruppe Herz; w = Anzahl weiblicher Patienten;\\m = Anzahl männlicher Patienten; A = durchschnittliches Gruppenalter;\\wA = Durchschnittsalter weiblicher Patienten; wminA = minimales Alter weiblicher Patienten;\\wmaxA = maximales Alter weiblicher Patienten; mA = Durchschnittsalter männlicher Patienten;\\mminA = minimales Alter männlicher Patienten; mmaxA = maximales Alter männlicher Patienten}
\label{tab:Herz}
\end{table}


\begin{table}[H] 
\centering
\begin{tabularx}{0.9\textwidth}{ccccccccccc}
\toprule
\multicolumn{4}{c}{\textbf{Gruppe Atmung}} & & & & & & &\\  
\cmidrule{1-4}
w  & m & A  &    & wA   & wminA & wmaxA &    & mA   & mminA & mmaxA\\
\midrule
13 & 8 & 54 & ~~ & 57,2 & 47    & 65    & ~~ & 48,9 & 33    & 60\\
\bottomrule
\end{tabularx}
\caption[Gruppe Atmung]{Klassifizierung der Patientengruppe Atmung; w = Anzahl weiblicher Patienten;\\m = Anzahl männlicher Patienten; A = durchschnittliches Gruppenalter;\\wA = Durchschnittsalter weiblicher Patienten; wminA = minimales Alter weiblicher Patienten;\\wmaxA = maximales Alter weiblicher Patienten; mA = Durchschnittsalter männlicher Patienten;\\mminA = minimales Alter männlicher Patienten; mmaxA = maximales Alter männlicher Patienten}
\label{tab:Atmung}
\end{table}


\begin{table}[H] 
\centering
\begin{tabularx}{0.95\textwidth}{ccccccccccc}
\toprule
\multicolumn{4}{c}{\textbf{Gruppe Schlafmittel}} & & & & & & &\\  
\cmidrule{1-4}
w & m & A    &    & wA   & wminA & wmaxA &    & mA  & mminA & mmaxA\\
\midrule
8 & 2 & 55,1 & ~~ & 56,1 & 42    & 65    & ~~ & 51 & 44     & 58\\
\bottomrule
\end{tabularx}
\caption[Gruppe Schlafmittel]{Klassifizierung der Patientengruppe Schlafmittel; w = Anzahl weiblicher Patienten;\\m = Anzahl männlicher Patienten; A = durchschnittliches Gruppenalter;\\wA = Durchschnittsalter weiblicher Patienten; wminA = minimales Alter weiblicher Patienten;\\wmaxA = maximales Alter weiblicher Patienten; mA = Durchschnittsalter männlicher Patienten;\\mminA = minimales Alter männlicher Patienten; mmaxA = maximales Alter männlicher Patienten}
\label{tab:Schlafmittel}
\end{table}


\begin{table}[H] 
\centering
\begin{tabularx}{0.91\textwidth}{ccccccccccc}
\toprule
\multicolumn{4}{c}{\textbf{Gruppe Insomnie}} & & & & & & &\\  
\cmidrule{1-4}
w  & m  & A    &    & wA   & wminA & wmaxA &    & mA   & mminA & mmaxA\\
\midrule
17 & 11 & 46,7 & ~~ & 47,1 & 23    & 61    & ~~ & 46,1 & 26    & 61\\
\bottomrule
\end{tabularx}
\caption[Gruppe Insomnie]{Klassifizierung der Patientengruppe Insomnie; w = Anzahl weiblicher Patienten;\\m = Anzahl männlicher Patienten; A = durchschnittliches Gruppenalter;\\wA = Durchschnittsalter weiblicher Patienten; wminA = minimales Alter weiblicher Patienten;\\wmaxA = maximales Alter weiblicher Patienten; mA = Durchschnittsalter männlicher Patienten;\\mminA = minimales Alter männlicher Patienten; mmaxA = maximales Alter männlicher Patienten}
\label{tab:Insomnie}
\end{table}



% --------------------------------------------------
% gemittelte Gruppen-TDS gemäß Kriterien
% --------------------------------------------------
\textbf{gemittelte Gruppen-TDS gemäß Kriterien}

\begin{figure}[H]
	\centering
	\includegraphics[width = \textwidth]{img/meanTDSheart_oldcolorbar.png}
	\caption[Verbindungsstärken für die Gruppe Herz]{Gemittelte Verbindungsstärken für die Gruppe Herz;\\Verbindungsstärken gemäß Colorbar mit 1 = 100\%;\\links: unteres linkes Dreieck = \acs{REM}-Schlaf, oberes rechtes Dreieck = Tiefschlaf;\\rechts: unteres linkes Dreieck = Wachzustand, oberes rechtes Dreieck = Leichtschlaf}
	\label{fig:meanTDSheart}
\end{figure}

\begin{figure}[H]
	\centering
	\includegraphics[width = \textwidth]{img/meanTDSbreath_oldcolorbar.png}
	\caption[Verbindungsstärken für die Gruppe Atmung]{Gemittelte Verbindungsstärken für die Gruppe Atmung;\\Verbindungsstärken gemäß Colorbar mit 1 = 100\%;\\links: unteres linkes Dreieck = \acs{REM}-Schlaf, oberes rechtes Dreieck = Tiefschlaf;\\rechts: unteres linkes Dreieck = Wachzustand, oberes rechtes Dreieck = Leichtschlaf}
	\label{fig:meanTDSbreath}
\end{figure}

\begin{figure}[H]
	\centering
	\includegraphics[width = \textwidth]{img/meanTDSsleep_oldcolorbar.png}
	\caption[Verbindungsstärken für die Gruppe Schlafmittel]{Gemittelte Verbindungsstärken für die Gruppe Schlafmittel;\\Verbindungsstärken gemäß Colorbar mit 1 = 100\%;\\links: unteres linkes Dreieck = \acs{REM}-Schlaf, oberes rechtes Dreieck = Tiefschlaf;\\rechts: unteres linkes Dreieck = Wachzustand, oberes rechtes Dreieck = Leichtschlaf}
	\label{fig:meanTDSsleep}
\end{figure}

\begin{figure}[H]
	\centering
	\includegraphics[width = \textwidth]{img/meanTDSinsom_oldcolorbar.png}
	\caption[Verbindungsstärken für die Gruppe Insomnie]{Gemittelte Verbindungsstärken für die Gruppe Insomnie;\\Verbindungsstärken gemäß Colorbar mit 1 = 100\%;\\links: unteres linkes Dreieck = \acs{REM}-Schlaf, oberes rechtes Dreieck = Tiefschlaf;\\rechts: unteres linkes Dreieck = Wachzustand, oberes rechtes Dreieck = Leichtschlaf}
	\label{fig:meanTDSinsom}
\end{figure}



% --------------------------------------------------
% Differenz zwischen SIESTA und Gruppe Alle
% --------------------------------------------------
\textbf{Differenz zwischen SIESTA und Gruppe Alle}

\begin{figure}[H]
	\centering
	\includegraphics[width = \textwidth]{img/diffTDSsiesta_all.png}
	\caption[Differenz-Matrix der Verbindungsstärken der SIESTA-Daten und der Gruppe Alle]{Differenz-Matrix der Verbindungsstärken der SIESTA-Daten und der Gruppe Alle;\\links: unteres linkes Dreieck = \acs{REM}-Schlaf, oberes rechtes Dreieck = Tiefschlaf;\\rechts: unteres linkes Dreieck = Wachzustand, oberes rechtes Dreieck = Leichtschlaf;\\Differenzen der Verbindungsstärken gemäß Colorbar mit 1 = 100\%;\\positive Werte = Verbindungen in SIESTA-Daten stärker; negative Werte = Verbindungen in Gruppe Alle stärker}
	\label{fig:diffTDSsiesta_all}
\end{figure}



% --------------------------------------------------
% alle weiblichen gegen alle männlichen Patienten
% --------------------------------------------------
\textbf{alle weiblichen gegen alle männlichen Patienten}

Abb. \ref{fig:dsls_m_f}

Abb. \ref{fig:rw_m_f}

Abb. \ref{fig:diffTDS_f_m}

\begin{figure}[H]
	\centering
	\includegraphics[width = \textwidth]{img/dsls_m_f_oldcolorbar.png}
	\caption[Verbindungsstärken von weiblichen und männlichen Insomniepatienten]{Gemittelte Verbindungsstärken von weiblichen und männlichen Insomniepatienten im Tiefschlaf (links) und Leichtschlaf (rechts); Verbindungsstärken gemäß Colorbar mit 1 = 100\%;\\unteres linkes Dreieck: männliche Insomniepatienten;\\oberes rechtes Dreieck: weibliche Insomniepatienten}
	\label{fig:dsls_m_f}
\end{figure}

\begin{figure}[H]
	\centering
	\includegraphics[width = \textwidth]{img/rw_m_f_oldcolorbar.png}
	\caption[Verbindungsstärken von weiblichen und männlichen Insomniepatienten]{Gemittelte Verbindungsstärken von weiblichen und männlichen Insomniepatienten im REM-Schlaf (links) und Wachzustand (rechts); Verbindungsstärken gemäß Colorbar mit 1 = 100\%;\\unteres linkes Dreieck: männliche Insomniepatienten;\\oberes rechtes Dreieck: weibliche Insomniepatienten}
	\label{fig:rw_m_f}
\end{figure}

\begin{figure}[H]
	\centering
	\includegraphics[width = \textwidth]{img/diffTDS_f_m.png}
	\caption[Differenz-Matrix der Verbindungsstärken der weiblichen und männlichen Insomniepatienten]{Differenz-Matrix der Verbindungsstärken der weiblichen und männlichen Insomniepatienten;\\links: unteres linkes Dreieck = REM-Schlaf; oberes rechtes Dreieck = Tiefschlaf;\\rechts: unteres linkes Dreieck = Wachzustand; oberes rechtes Dreieck = Leichtschlaf;\\Differenzen der Verbindungsstärken gemäß Colorbar mit 1 = 100\%;\\positive Werte = höhere Verbindungsstärken bei den weiblichen Insomniepatienten; negative Werte = höhere Verbindungsstärken bei männlichen Insomniepatienten}
	\label{fig:diffTDS_f_m}
\end{figure}




% --------------------------------------------------
% jüngste gegen älteste Patienten
% --------------------------------------------------
\textbf{jüngste gegen älteste Patienten}

Abb. \ref{fig:dsls_allyoung_allold}

Abb. \ref{fig:rw_allyoung_allold}

Differenz-Matrix der jungen und älteren Insomiepatienten, Anhang Abb. \ref{fig:diffTDSallyoung_allold}

\begin{figure}[H]
	\centering
	\includegraphics[width = \textwidth]{img/dsls_allyoung_allold.png}
	\caption[Verbindungsstärken der jüngsten und ältesten Insomniepatienten im Tief- und Leichtschlaf]{Gemittelte Verbindungsstärken der jüngsten und ältesten Insomniepatienten im Tief- (links) und Leichtschlaf (rechts); Verbindungsstärken gemäß Colorbar mit 1 = 100\%;\\unteres linkes Dreieck: jüngste Insomniepatienten (23 bis 46, Jahre, n=22);\\oberes rechtes Dreieck: älteste Insomniepatienten (59 bis 65 Jahre, n=22)}
	\label{fig:dsls_allyoung_allold}
\end{figure}

\begin{figure}[H]
	\centering
	\includegraphics[width = \textwidth]{img/rw_allyoung_allold.png}
	\caption[Verbindungsstärken der jüngsten und ältesten Insomniepatienten im REM-Schlaf und Wachzustand]{Gemittelte Verbindungsstärken der jüngsten und ältesten Insomniepatienten im REM-Schlaf (links) und Wachzustand (rechts); Verbindungsstärken gemäß Colorbar mit 1 = 100\%;\\unteres linkes Dreieck: jüngste Insomniepatienten (23 bis 46, Jahre, n=22);\\oberes rechtes Dreieck: älteste Insomniepatienten (59 bis 65 Jahre, n=22)}
	\label{fig:rw_allyoung_allold}
\end{figure}

\begin{figure}[H]
	\centering
	\includegraphics[width = \textwidth]{img/diffTDSallyoung_allold.png}
	\caption[Differenz-Matrix der Verbindungsstärken der jungen und älteren Insomniepatienten]{Differenz-Matrix der Verbindungsstärken der jüngsten (23 bis 46 Jahre, n=22) und ältesten (59 bis 65 Jahre, n=22) Insomniepatienten;\\links: unteres linkes Dreieck = REM-Schlaf; oberes rechtes Dreieck = Tiefschlaf;\\rechts: unteres linkes Dreieck = Wachzustand; oberes rechtes Dreieck = Leichtschlaf;\\Differenzen der Verbindungsstärken gemäß Colorbar mit 1 = 100\%;\\positive Werte = höhere Verbindungsstärken bei jungen Insomniepatienten; negative Werte = höhere Verbindungsstärken bei älteren Insomniepatienten}
	\label{fig:diffTDSallyoung_allold}
\end{figure}



% --------------------------------------------------
% junge weibliche gegen junge männliche Patienten
% --------------------------------------------------
\textbf{junge weibliche gegen junge männliche Patienten}

Abb. \ref{fig:dsls_youngf_youngm}

Abb. \ref{fig:rw_youngf_youngm}

Abb. \ref{fig:diffTDSyoungf_youngm}

\begin{figure}[H]
	\centering
	\includegraphics[width = \textwidth]{img/dsls_youngf_youngm.png}
	\caption[Verbindungsstärken der jüngsten weiblichen und männlichen Insomniepatienten im Tief- und Leichtschlaf]{Gemittelte Verbindungsstärken der jüngsten weiblichen und männlichen Insomniepatienten im Tief- (links) und Leichtschlaf (rechts); Verbindungsstärken gemäß Colorbar mit 1 = 100\%;\\unteres linkes Dreieck: jüngste männliche Insomniepatienten (26 bis 45 Jahre, n=11);\\oberes rechtes Dreieck: jüngste weibliche (23 bis 46 Jahre, n=11) Insomniepatienten}
	\label{fig:dsls_youngf_youngm}
\end{figure}

\begin{figure}[H]
	\centering
	\includegraphics[width = \textwidth]{img/rw_youngf_youngm.png}
	\caption[Verbindungsstärken der jüngsten weiblichen und männlichen Insomniepatienten im REM-Schlaf und Wachzustand]{Gemittelte Verbindungsstärken der jüngsten weiblichen und männlichen Insomniepatienten im REM-Schlaf (links) und Wachzustand (rechts); Verbindungsstärken gemäß Colorbar mit 1 = 100\%;\\unteres linkes Dreieck: jüngste männliche Insomniepatienten (26 bis 45 Jahre, n=11);\\oberes rechtes Dreieck: jüngste weibliche (23 bis 46 Jahre, n=11) Insomniepatienten}
	\label{fig:rw_youngf_youngm}
\end{figure}

\begin{figure}[H]
	\centering
	\includegraphics[width = \textwidth]{img/diffTDSyoungf_youngm.png}
	\caption[Differenz-Matrix der Verbindungsstärken der jüngsten weiblichen und männlichen Insomniepatienten]{Differenz-Matrix der Verbindungsstärken der jüngsten weiblichen (23 bis 46 Jahre, n=11) und jüngsten männlichen (26 bis 45 Jahre, n=11) Insomniepatienten;\\links: unteres linkes Dreieck = REM-Schlaf; oberes rechtes Dreieck = Tiefschlaf;\\rechts: unteres linkes Dreieck = Wachzustand; oberes rechtes Dreieck = Leichtschlaf;\\Differenzen der Verbindungsstärken gemäß Colorbar mit 1 = 100\%;\\positive Werte = höhere Verbindungsstärken bei jungen weiblichen Insomniepatienten; negative Werte = höhere Verbindungsstärken bei jungen männlichen Insomniepatienten}
	\label{fig:diffTDSyoungf_youngm}
\end{figure}



% --------------------------------------------------
% alte weibliche gegen alte männliche Patienten
% --------------------------------------------------
\textbf{alte weibliche gegen alte männliche Patienten}

Abb. \ref{fig:dsls_oldf_oldm}

Abb. \ref{fig:rw_oldf_oldm}

Abb. \ref{fig:diffTDSoldf_oldm}

\begin{figure}[H]
	\centering
	\includegraphics[width = \textwidth]{img/dsls_oldf_oldm.png}
	\caption[Verbindungsstärken der ältesten weiblichen und männlichen Insomniepatienten im Tief- und Leichtschlaf]{Gemittelte Verbindungsstärken der ältesten weiblichen und männlichen Insomniepatienten im Tief- (links) und Leichtschlaf (rechts); Verbindungsstärken gemäß Colorbar mit 1 = 100\%;\\unteres linkes Dreieck: älteste männliche Insomniepatienten (59 bis 61 Jahre, n=8);\\oberes rechtes Dreieck: älteste weibliche (59 bis 65 Jahre, n=14) Insomniepatienten}
	\label{fig:dsls_oldf_oldm}
\end{figure}

\begin{figure}[H]
	\centering
	\includegraphics[width = \textwidth]{img/rw_oldf_oldm.png}
	\caption[Verbindungsstärken der ältesten weiblichen und männlichen Insomniepatienten im REM-Schlaf und Wachzustand]{Gemittelte Verbindungsstärken der ältesten weiblichen und männlichen Insomniepatienten im REM-Schlaf (links) und Wachzustand (rechts); Verbindungsstärken gemäß Colorbar mit 1 = 100\%;\\unteres linkes Dreieck: älteste männliche Insomniepatienten (59 bis 61 Jahre, n=8);\\oberes rechtes Dreieck: älteste weibliche (59 bis 65 Jahre, n=14) Insomniepatienten}
	\label{fig:rw_oldf_oldm}
\end{figure}

\begin{figure}[H]
	\centering
	\includegraphics[width = \textwidth]{img/diffTDSoldf_oldm.png}
	\caption[Differenz-Matrix der Verbindungsstärken der ältesten weiblichen und männlichen Insomniepatienten]{Differenz-Matrix der Verbindungsstärken der ältesten weiblichen (59 bis 65 Jahre, n=14) und ältesten männlichen (59 bis 61 Jahre, n=8) Insomniepatienten;\\links: unteres linkes Dreieck = REM-Schlaf; oberes rechtes Dreieck = Tiefschlaf;\\rechts: unteres linkes Dreieck = Wachzustand; oberes rechtes Dreieck = Leichtschlaf;\\Differenzen der Verbindungsstärken gemäß Colorbar mit 1 = 100\%;\\positive Werte = höhere Verbindungsstärken bei älteren weiblichen Insomniepatienten; negative Werte = höhere Verbindungsstärken bei älteren männlichen Insomniepatienten}
	\label{fig:diffTDSoldf_oldm}
\end{figure}




% --------------------------------------------------
% junge weibliche gegen alte weibliche Patienten
% --------------------------------------------------
\textbf{junge weibliche gegen alte weibliche Patienten}

Abb. \ref{fig:dsls_youngf_oldf}

Abb. \ref{fig:rw_youngf_oldf}

Abb. \ref{fig:diffTDSyoungf_oldf}

\begin{figure}[H]
	\centering
	\includegraphics[width = \textwidth]{img/dsls_youngf_oldf.png}
	\caption[Verbindungsstärken der jüngsten und der ältesten weiblichen Insomniepatienten im Tief- und Leichtschlaf]{Gemittelte Verbindungsstärken der jüngsten und der ältesten weiblichen Insomniepatienten im Tief- (links) und Leichtschlaf (rechts); Verbindungsstärken gemäß Colorbar mit 1 = 100\%;\\unteres linkes Dreieck: älteste weibliche Insomniepatienten (59 bis 65 Jahre, n=14);\\oberes rechtes Dreieck: jüngste weibliche (23 bis 46 Jahre, n=11) Insomniepatienten}
	\label{fig:dsls_youngf_oldf}
\end{figure}

\begin{figure}[H]
	\centering
	\includegraphics[width = \textwidth]{img/rw_youngf_oldf.png}
	\caption[Verbindungsstärken der jüngsten und der ältesten weiblichen Insomniepatienten im REM-Schlaf und Wachzustand]{Gemittelte Verbindungsstärken der jüngsten und der ältesten weiblichen Insomniepatienten im REM-Schlaf (links) und Wachzustand (rechts); Verbindungsstärken gemäß Colorbar mit 1 = 100\%;\\unteres linkes Dreieck: älteste weibliche Insomniepatienten (59 bis 65 Jahre, n=14);\\oberes rechtes Dreieck: jüngste weibliche (23 bis 46 Jahre, n=11) Insomniepatienten}
	\label{fig:rw_youngf_oldf}
\end{figure}

\begin{figure}[H]
	\centering
	\includegraphics[width = \textwidth]{img/diffTDSyoungf_oldf.png}
	\caption[Differenz-Matrix der Verbindungsstärken der jüngsten und ältesten weiblichen Insomniepatienten]{Differenz-Matrix der Verbindungsstärken der jüngsten (23 bis 46 Jahre, n=11) und der ältesten (59 bis 65 Jahre, n=14) weiblichen Insomniepatienten;\\links: unteres linkes Dreieck = REM-Schlaf; oberes rechtes Dreieck = Tiefschlaf;\\rechts: unteres linkes Dreieck = Wachzustand; oberes rechtes Dreieck = Leichtschlaf;\\Differenzen der Verbindungsstärken gemäß Colorbar mit 1 = 100\%;\\positive Werte = höhere Verbindungsstärken bei jungen weiblichen Insomniepatienten; negative Werte = höhere Verbindungsstärken bei älteren weiblichen Insomniepatienten}
	\label{fig:diffTDSyoungf_oldf}
\end{figure}



% --------------------------------------------------
% junge männliche gegen alte männliche Patienten
% --------------------------------------------------
\textbf{junge männliche gegen alte männliche Patienten}

Abb. \ref{fig:dsls_youngm_oldm}

Abb. \ref{fig:rw_youngm_oldm}

Abb. \ref{fig:diffTDSyoungm_oldm}

\begin{figure}[H]
	\centering
	\includegraphics[width = \textwidth]{img/dsls_youngm_oldm.png}
	\caption[Verbindungsstärken der jüngsten und der ältesten männlichen Insomniepatienten im Tief- und Leichtschlaf]{Gemittelte Verbindungsstärken der jüngsten und der ältesten männlichen Insomniepatienten im Tief- (links) und Leichtschlaf (rechts); Verbindungsstärken gemäß Colorbar mit 1 = 100\%;\\unteres linkes Dreieck: älteste männliche Insomniepatienten (59 bis 61 Jahre, n=8);\\oberes rechtes Dreieck: jüngste männliche (26 bis 45 Jahre, n=11) Insomniepatienten}
	\label{fig:dsls_youngm_oldm}
\end{figure}

\begin{figure}[H]
	\centering
	\includegraphics[width = \textwidth]{img/rw_youngm_oldm.png}
	\caption[Verbindungsstärken der jüngsten und der ältesten männlichen Insomniepatienten im REM-Schlaf und Wachzustand]{Gemittelte Verbindungsstärken der jüngsten und der ältesten männlichen Insomniepatienten im REM-Schlaf (links) und Wachzustand (rechts); Verbindungsstärken gemäß Colorbar mit 1 = 100\%;\\unteres linkes Dreieck: älteste männliche Insomniepatienten (59 bis 61 Jahre, n=8);\\oberes rechtes Dreieck: jüngste männliche (26 bis 45 Jahre, n=11) Insomniepatienten}
	\label{fig:rw_youngm_oldm}
\end{figure}

\begin{figure}[H]
	\centering
	\includegraphics[width = \textwidth]{img/diffTDSyoungm_oldm.png}
	\caption[Differenz-Matrix der Verbindungsstärken der jüngsten und ältesten männlichen Insomniepatienten]{Differenz-Matrix der Verbindungsstärken der jüngsten (26 bis 45 Jahre, n=11) und der ältesten (59 bis 61 Jahre, n=8) männlichen Insomniepatienten;\\links: unteres linkes Dreieck = REM-Schlaf; oberes rechtes Dreieck = Tiefschlaf;\\rechts: unteres linkes Dreieck = Wachzustand; oberes rechtes Dreieck = Leichtschlaf;\\Differenzen der Verbindungsstärken gemäß Colorbar mit 1 = 100\%;\\positive Werte = höhere Verbindungsstärken bei jungen männlichen Insomniepatienten; negative Werte = höhere Verbindungsstärken bei älteren männlichen Insomniepatienten}
	\label{fig:diffTDSyoungm_oldm}
\end{figure}




%\begin{figure}[H]
%	\centering
%	\includegraphics[width = 0.7\textwidth]{img/nmean_age_siesta.png}
%	\caption[Gegenüberstellung von Verbindungsstärke \textit{nmean} und Alter gesunder Probanden]{Gegenüberstellung von Verbindungsstärke \textit{nmean} und Alter gesunder Probanden der SIESTA-Studie in den unterschiedlichen Schlafstadien gemäß Krefting et al.; die Linien stellen die lineare Regression für alle Probanden (durchgezogene Linie), für weibliche (gestrichelte Linie) und für männliche (gepunktete Linie) Probanden dar}
%	\label{fig:nmean_age_siesta}
%\end{figure}

\restoregeometry

% ============= Eidesstattliche Versicherung =============

\newpage
\addcontentsline{toc}{chapter}{Eidesstattliche Versicherung}
\setcounter{page}{17}
%\newgeometry{left= 3 cm,right = 2.5 cm, bottom = 2.5 cm, top = 6 cm}
%
%% der Befehl section* unterdrückt die Auflistung im Inhaltsverzeichnis
%\section*{\Huge{Eidesstattliche Versicherung}}
%
%Hiermit versichere ich, die vorliegende Arbeit eigenständig geschrieben und keine anderen als die angegebenen Quellen und Hilfsmittel verwendet zu haben. 
%
%
%Ort, Datum
%
%
%
%Stefanie Breuer
%
%
%\restoregeometry
%---------------------------------

\newgeometry{left= 3cm,right = 2.5cm, bottom = 2.5 cm, top = 6 cm}
\pagenumbering{Roman}
\setcounter{page}{22}
\lhead{}
\renewcommand{\headrulewidth}{0pt}


\sectionmark{Eidesstattliche Versicherung}
\section*{\Huge{Eidesstattliche Versicherung}}


\ \\[1.5cm]Hiermit versichere ich, die vorliegende Arbeit eigenständig geschrieben und keine anderen als die angegebenen Quellen und Hilfsmittel verwendet zu haben. 
\\
\\[2cm]


\vspace{0.5 cm} 
\begin{tabular}{p{5cm}p{.4cm}l}
\hline \\ 
Ort, Datum
\end{tabular}
\hfill 
\begin{tabular}{p{8cm}p{.4cm}l}
\hline \\ 
Stefanie Breuer 
\end{tabular}

\restoregeometry

% ============= End Document =============
\end{document}
