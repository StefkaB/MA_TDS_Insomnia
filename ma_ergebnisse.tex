\chapter{Ergebnisse}

\section{TDS-Merkmale bei Insomnie}

Die gemittelten \acs{TDS}-Matrizen der jeweiligen Gruppen (Abb.~\ref{fig:meanTDSall} sowie Anhang Abb.~\ref{fig:meanTDSheart} bis \ref{fig:meanTDSinsom}) zeigen teilweise unterschiedliche Verbindungsstärken, jedoch insgesamt sehr ähnliche Tendenzen. Insbesondere sind bei jeder Gruppe relativ hohe Verbindungsstärken im Wachzustand und Leichtschlaf und stark reduzierte Verbindungsstärken im Tiefschlaf zu beobachten, während die Verbindungsstärken im \acs{REM}-Schlaf dazwischen liegen. Um die Einflüsse der Erkrankungen bzw. Medikamente auf die Ausprägung der \acs{TDS}-Ergebnisse zu prüfen, werden die Gruppen Herz, Atmung und Schlafmittel zunächst mit der Gruppe Insomnie verglichen.\\

Die Gruppe Herz zeigt insbesondere in den Verbindungen zwischen den \acs{EEG}s C4 und O1 sowie zwischen C3 und den beiden okzipitalen \acs{EEG}s sehr schwache Verbindungsstärken, welche teilweise unter dem Signifikanz-Schwellenwert von 7 \% liegen. Gegenüber der Gruppe Insomnie zeigt die Gruppe Herz im Wachzustand und Leichtschlaf eine leicht erhöhte Verbindungsstärke zwischen der Herzfrequenz und allen anderen Systemen, im \acs{REM}-Schlaf und Tiefschlaf hingegen eine leicht reduzierte. Im Leichtschlaf und Tiefschlaf sind darüber hinaus die Verbindungsstärken zwischen den Systemen C3-O2, C4-O1 sowie O1-O2 sehr gering. Die gleiche Tendenz in weniger starker Ausprägung zeigt sich jedoch auch in der Gruppe Insomnie.\\

Die Gruppe Atmung ähnelt im Wachzustand, im Leicht- und Tiefschlaf in den Verbindungsstärken sehr stark der Gruppe Insomnie. Die Augen sind in der Gruppe Atmung im Vergleich zur Gruppe Insomnie verstärkt mit den $\delta$- und $\theta$-Frequenzbändern aller \acs{EEG}s verbunden. Die Systeme C3 und C4 zu den gegenüberliegenden okzipitalen \acs{EEG}s sowie diese beiden untereinander zeigen ähnlich wie die Gruppe Herz sehr schwache Verbindungsstärken. Die Atemsysteme zeigen leicht erhöhte Kopplungen untereinander.\\

Die Gruppe Schlafmittel zeigt im Wachzustand und Leichtschlaf leicht erhöhte Verbindungsstärken innerhalb der Atemsysteme gegenüber der Gruppe Insomnie. Im \acs{REM}-Schlaf und Tiefschlaf hingegen sind leicht reduzierte Verbindungsstärken innerhalb der Atemsysteme erkennbar. Die Kopplung der Herzfrequenz mit den $\alpha$-, $\sigma$- und $\beta$-Frequenzbändern aller \acs{EEG}s erscheint in der Gruppe Schlafmittel leicht erhöht. Die Verbindungen zwischen den \acs{EMG}-Systemen und den $\delta$-Frequenzbänder des Gehirns sind im Leicht- und Tiefschlaf ebenfalls erhöht, in den anderen beiden Schlafstadien ist lediglich das Kinn leicht verstärkt mit den $\alpha$-, $\sigma$- und $\beta$-Frequenzbändern der \acs{EEG}-Systeme verbunden. Die Augen sind in dieser Gruppe im Leicht-, Tief- und \acs{REM}-Schlaf weniger stark mit allen Systemen des Gehirns verbunden. Innerhalb des Gehirns sind nahezu alle Verbindungen gegenüber der Gruppe Insomnie leicht reduziert. Im Tiefschlaf fallen die Verbindungen $\alpha$C3 mit $\delta$O1 und $\theta$O1 sowie $\alpha$O1 mit $\delta$O2 und umgekehrt unter die Signifikanz-Schwelle. Auch die Verbindungen der Systeme O1 und C3 in den $\alpha$- und $\theta$-Frequenzbändern zeigen reduzierte Verbindungsstärken unterhalb der Signifikanz-Schwelle.\\

Insgesamt bilden die gemittelten \acs{TDS}-Matrizen der Gruppenergebnisse für die Gruppen Herz, Atmung und Schlafmittel nicht eindeutig die eingangs vermuteten krankheitsbedingten Beeinträchtigungen der Kopplungen zwischen Herz bzw. Atmung und den anderen Systemen ab. Die Abweichungen sind gering und verändern nicht die Gesamttendenzen. Dies kann daran liegen, dass viele betroffene Patienten entsprechende Medikamente einnehmen, welche die Symptome abschwächen. Auch bei der Einnahme von Schlafmitteln zeigen sich keine gravierenden Unterschiede zur Gruppe Insomnie. Die Abweichungen werden daher als nicht signifikant eingestuft. Darüber hinaus sind die Gruppen nicht vollends miteinander vergleichbar, da die Gruppen größtenteils unterschiedliche Patienten enthalten, so dass individuelle Abweichungen nicht ausgeschlossen werden können und diese demnach die Abweichungen beeinflussen. Eine Beziehung zwischen den Erkrankungen bzw. der Einnahme von Schlafmitteln und den Verbindungsstärken in den \acs{TDS}-Matrizen kann daher nicht eindeutig hergestellt werden. \\

Ein Vergleich der Gruppe Alle mit der Gruppe Insomnie durch Subtraktion der Ergebnis-Matrizen macht ebenfalls deutlich, dass die übrigen Gruppen Herz, Atmung und Schlafmittel lediglich innerhalb der Systeme des Gehirns und geringfügig in den Atemsystemen Änderungen hervorrufen (Abb. \ref{fig:TDSall_insom}). Im Wachzustand liegen diese Änderungen in einer leichten Verstärkung der Atemsysteme sowie innerhalb der okzipitalen Gehirnsysteme. Darüber hinaus sind beide Augen mit den niedrigen Frequenzbändern ($\delta$ und $\theta$) aller Gehirnsysteme leicht verstärkt verbunden. Da der Differenzwert bei maximal 4 \% liegt, können die Änderungen im Wachzustand durch die Gruppen Herz, Atmung und Schlafmittel als nicht signifikant betrachtet werden. Im Leichtschlaf sind die Atemsysteme ähnlich wie im Wachzustand gekoppelt. Die Augen sind um bis zu 7 \% stärker mit den $\delta$- und $\theta$- Frequenzbändern der zentralen Gehirnregionen verbunden. Innerhalb der Gehirnsysteme finden sich nahezu keine Differenzen ($\pm$ 2 \%), wobei eine Reduzierung vorwiegend die Verbindungen mit dem \acs{EEG} O1 betrifft. Im \acs{REM}-Schlaf bewirken die Gruppen Herz, Atmung und Schlafmittel nahezu ausschließlich leicht erhöhte Verbindungen um bis zu 6 \% der hohen Frequenzbänder $\sigma$ und $\beta$ zwischen den Hemisphären. Im Tiefschlaf sind die Augen um 12 \% stärker miteinander verbunden und um bis zu 8 \% mit den $\delta$-Frequenzbändern der zentralen Gehirnregionen. Da dies nicht hinreichend durch die Krankheitsmerkmale der einzelnen Gruppen erklärt werden können und gegebenenfalls auf die Individualität der Patienten zurückzuführen ist, werden diese leicht verstärkten Kopplungen hier nicht weiter berücksichtigt. Mit den okzipitalen Bereichen sind die Augen um maximal 5 \% stärker verbunden. Eine signifikante Beeinflussung der Kopplungen innerhalb der Atemsysteme oder zwischen den Atemsystemen bzw. der Herzfrequenz mit den übrigen Systemen ist weder im \acs{REM}- noch im Tiefschlaf zu erkennen.\\

\begin{figure}[H]
	\centering
	\includegraphics[width = \textwidth]{img/TDS_all_insom.png}
	\caption[Differenzen zwischen Gruppe Alle und Insomnie]{Differenz-Matrizen der Gruppen Alle und Insomnie mit überwiegend nicht signifikanten Differenzen; positive Werte = Verbindungen in Gruppe Alle sind stärker; negative Werte = Verbindungen in Gruppe Insomnie sind stärker\\links: unteres linkes Dreieck = \acs{REM}-Schlaf, oberes rechtes Dreieck = Tiefschlaf;\\rechts: unteres linkes Dreieck = Wachzustand, oberes rechts Dreieck = Leichtschlaf}
	\label{fig:TDSall_insom}
\end{figure}

Da auch bei diesem Vergleich die nicht signifikanten Unterschiede keinen Rückschluss auf Herzerkrankungen, Atemwegserkrankungen oder die Einnahme von Schlafmitteln zulassen, werden die folgenden Untersuchungen anhand des Gesamtdatensatzes (Gruppe Alle) vorgenommen. Darüber hinaus ist die Insomnie ohnehin häufig mit anderen Erkrankungen verbunden (vgl. Arten der Insomnie, Anhang Tab. \ref{tab:allgemeine_insomnie} bis \ref{tab:korperliche_insomnie}). Für die Untersuchung insomniespezifischer \acs{TDS}-Merkmale werden die Ergebnisse von Krefting et al. \parencite{krefting_age_2017} für gesunde Probanden aus der SIESTA-Studie \parencite{klosch_siesta_2001} zunächst reproduziert. Anschließend werden aus der gemittelten Ergebnis-Matrix die frontopolaren Systeme entfernt (Abb. \ref{fig:meanTDSsiesta}). Auf diese Weise erhält die Ergebnis-Matrix aus den SIESTA-Daten die gleichen Dimensionen wie die hiesige Ergebnis-Matrix der Gruppe Alle (Abb. \ref{fig:meanTDSall}), so dass die Vergleichbarkeit der Daten vereinfacht wird.\\

\begin{figure}[H]
	\centering
	\includegraphics[width = \textwidth]{img/meanTDSall.png}
	\caption[\acs{TDS}-Gruppenergebnis für die Gruppe Alle]{Gemittelte \acs{TDS}-Matrizen für die Gruppe Alle;\\links: unteres linkes Dreieck = \acs{REM}-Schlaf, oberes rechtes Dreieck = Tiefschlaf;\\rechts: unteres linkes Dreieck = Wachzustand, oberes rechts Dreieck = Leichtschlaf}
	\label{fig:meanTDSall}
\end{figure}

\begin{figure}[H]
	\centering
	\includegraphics[width = \textwidth]{img/meanTDSsiesta.png}
	\caption[\acs{TDS}-Gruppenergebnis für den SIESTA-Datensatz]{Gemittelte \acs{TDS}-Matrizen für die Kontrollgruppe des SIESTA-Datensatzes nach Entfernung der frontopolaren Systeme;\\links: unteres linkes Dreieck = \acs{REM}-Schlaf, oberes rechtes Dreieck = Tiefschlaf;\\rechts: unteres linkes Dreieck = Wachzustand, oberes rechts Dreieck = Leichtschlaf}
	\label{fig:meanTDSsiesta}
\end{figure}

\newpage

Deutlich erkennbar ist, dass sich die Schlafstadien Wachzustand, Leicht- und \acs{REM}-Schlaf in beiden Gruppen sehr ähneln, wobei hauptsächlich die Verbindungen innerhalb des Gehirns in der Gruppe Alle leicht stärker ausgeprägt erscheinen als bei den SIESTA-Daten. Der \acs{REM}-Schlaf zeigt bei beiden Datensätzen fragmentierte Verbindungen innerhalb des Gehirns, wobei diese bei den Insomnie-Patienten weniger markant sind. Das \acs{TDS}-Merkmal der hohen Verbindungsstärken im Wachzustand und Leichtschlaf sowie der niedrigeren und stark verringerten Verbindungsstärken im \acs{REM}- und Tiefschlaf sind in beiden Gruppen erkennbar. Die bereits optisch ersichtlichen Unterschiede werden zusätzlich durch Subtraktion der Matrizen (vgl. Differenz-Matrix, Anhang Abb. \ref{fig:diffTDSsiesta_all}) im Folgenden genauer betrachtet.

\paragraph{Wachzustand} Die Herzfrequenz ist bei den Insomnie-Patienten mit den $\theta$- und $\alpha$-Frequenzbändern aller Gehirnbereiche sowie mit dem Kinn leicht verstärkt ($\sim$5 \%) verbunden. Die Atemsysteme weisen untereinander eine höhere Verbindungsstärke auf, insbesondere sind die Atemanstrengungen in Brust und Bauch um etwa 10 \% stärker verbunden als bei dem SIESTA-Datensatz. Zu den anderen Systemen weisen die Atemsysteme in beiden Datensätzen keine signifikanten Verbindungen auf. Die Verbindungsstärken zwischen Kinn bzw. Bein und Frequenzbändern des Gehirns liegen bei den Insomnie-Patienten um maximal 5 \% höher. Die Augen sind in beiden Datensätzen hauptsächlich mit den $\delta$-Frequenzbändern des Gehirns verbunden, wobei die Verbindungsstärke der Augen mit der rechten Hemisphäre bei den Insomnie-Patienten um bis zu 15~\% reduziert ist. Die Hemisphären untereinander (links mit links und rechts mit rechts) sind bei den Insomniepatienten maßgeblich stärker verbunden (etwa 10~\% bis 15~\%), wobei die Verbindungen zwischen den $\theta$- und $\beta$-Frequenzbändern am stärksten ausgeprägt sind (in O1 20~\% höher als bei den SIESTA-Daten). Die Verbindungsstärken der gekreuzten Hemisphären (links mit rechts) liegen bei den Insomniepatienten lediglich bis zu 5~\% höher. Die Autokorrelationen der Frequenzbänder dieser Verbindungen sind jedoch um bis zu 15~\% schwächer. Insgesamt bestätigt sich jedoch die leicht erhöhten Verbindungsstärken bei den Insomniepatienten im Wachzustand. 

\paragraph{Leichtschlaf} Es zeigt sich, dass die Herzfrequenz bei den Insomniepatienten minimal schwächer mit den Augen und \acs{EMG}s sowie mit den $\delta$-Frequenzbändern des Gehirns verbunden ist (maximal 5~\%). Die Atemsysteme hingegen sind um 10~\% bis 15~\% stärker untereinander verbunden, am stärksten wie im Wachzustand die Atemanstrengungen in Brust und Bauch. Das Kinn ist minimal schwächer mit den Augen verbunden. Mit dem Gehirn sind Kinn und Bein in beiden Gruppen nahezu in gleicher Stärke verbunden. Die Augen sind bei den Insomniepatienten untereinander sowie mit den $\delta$-Frequenzbändern der \acs{EEG}s deutlich schwächer (größtenteils 10~\% bis 20~\%, Auge2 zu C4 30~\%) verbunden als bei den SIESTA-Daten, wobei die Verbindungen der Augen zu den zentralen Gehirnbereichen in den $\theta$- und $\alpha$-Frequenzbändern ebenfalls schwächer ausfallen (10~\% bis 15~\%). Im Leichtschlaf ist ähnlich wie im Wachzustand erkennbar, dass die gleichen Hemisphären (links mit links und rechts mit rechts) bei den Insomniepatienten um 10~\% bis 15~\% stärker miteinander verbunden sind. Allerdings weisen die Verbindungen der $\delta$-Frequenzbänder der rechten okzipitalen Region zu allen Frequenzbändern der gleichseitigen zentralen Region um bis zu 10~\% schwächere Verbindungsstärken als bei den SIESTA-Daten auf. Dies gilt auch für die Verbindungen $\delta$C4 mit den Frequenzbändern von C3. Darüber hinaus sind die $\delta$-Frequenzbänder von den \acs{EEG}s O1 und O2 sowie von O1 und C4 um etwa 15~\% schwächer. Die Verbindungsstärken der gekreuzten Hemispähren (links mit rechts) liegen hingegen bis zu 10~\% über denen der SIESTA-Daten.

\paragraph{\acs{REM}-Schlaf} Die Verbindungen zwischen der Herzfrequenz und den anderen Systemen zeigen in beiden Gruppen kaum Unterschiede. Die Verbindung zwischen der Atemanstrengung in Brust und Bauch ist bei den Insomniepatienten um etwa 5~\% erhöht. Zu den übrigen Systemen weisen die Atemsysteme jedoch keine signifikanten Änderungen auf. Das Kinn erscheint um bis zu 10~\% schwächer mit den Augen sowie um maximal 5~\ schwächer mit dem Bein verbunden als bei den SIESTA-Daten. Die Verbindungsstärke des Beins zu den $\delta$-Frequenzbändern des Gehirns sind jedoch um 5~\% erhöht. Die Augen sind bei den Insomniepatienten miteinander um etwa 25~\% schwächer verbunden. Die Hemisphären sind untereinander (links mit links und rechts mit rechts) um bis zu 15~\% stärker verbunden, am deutlichsten erkennbar in den $sigma$- und $beta$-Frequenzbändern. Die gekreuzten Verbindungen  zeigen ähnliche Tendenzen, jedoch weniger stark ausgeprägt, so dass die Verbindungsstärken der Insomniepatienten in diesen Bereichen um maximal 10~\% und am deutlichsten in Kopplungen mit den okzipitalen Bereichen erhöhen.

\paragraph{Tiefschlaf} Im Tiefschlaf zeigt sich eine leichte Absenkung der Verbindungsstärke zwischen der Herzfrequenz und den \acs{EMG}s, die Verbindungen zu den übrigen Systemen scheinen jedoch nahezu unverändert. Die Atemsysteme sind bei den Insomniepatienten untereinander deutlich verstärkt verbunden, wobei der Anstieg mit 22~\% bei der Atemanstrengung zwischen Brust und Bauch am größten ist. Das Bein ist neben der schwächeren Verbindung zur Herzfrequenz etwas stärker (5~\%) mit den $\delta$-Frequenzbändern der okzipitalen Gehirnregionen verbunden. Das Auge2 ist um 20~\% stärker mit dem Auge1 verbunden sowie um 18~\% bzw. 23~\% stärker mit den $\delta$-Frequenzbändern von O1 bzw. C4 verbunden. Das Auge1 ist darüber hinaus leicht verstärkt mit den Gehirnregionen C4 und O2 verbunden. Auch im Tiefschlaf ist erkennbar, dass die Verbindungsstärken der \acs{EEG}s innerhalb der gleichen Hemisphäre bei den Insomniepatienten erhöht sind und die Werte im Gegensatz zum SIESTA-Datensatz die Signifikanz-Schwelle von 7~\% übersteigen. Die gekreuzten Kopplungen zwischen den \acs{EEG}s O2 und C3 bzw. O1 sowie zwischen den \acs{EEG}s C4 und O1 zeigen ebenfalls eine erhöhte Verbindungsstärke als bei den SIESTA-Daten, jedoch mit maximal 5~\% bis 10~\% weniger stark ausgeprägt. Zwischen den zentralen \acs{EEG}s C3 und C4 weisen hingegen die Kopplungen der $\sigma$- und $\beta$-Frequenzbänder sowie der $\delta$- und $\sigma$-Frequenzbänder gegenüber den SIESTA-Daten eine leicht schwächere (maximal 5~\%) Verbindungsstärke auf.

\paragraph{Merkmale} Insgesamt lassen sich die optisch erkennbaren Unterschiede durch Auswertung der Subtraktion beider \acs{TDS}-Matrizen bestätigen. Auffällig ist, dass in allen Schlafstadien bei den Insomniepatienten eine erhöhte Verbindungsstärke innerhalb des Gehirns festzustellen ist. Verbindungen innerhalb der gleichen Hemisphäre sind dabei stärker ausgeprägt als Verbindungen der gekreuzten Hemisphären. Zwischen den zentralen \acs{EEG}s lassen sich hauptsächlich im Leicht- und Tiefschlaf verringerte Verbindungsstärken gegenüber den gesunden Probanden des SIESTA-Datensatzes erkennen. Darüber hinaus ist ersichtlich, dass in allen Schlafstadien die Verbindungen innerhalb der Atemsysteme und insbesondere zwischen der Atemanstrengung in Brust und Bauch deutlich erhöht sind. Während dies im \acs{REM}-Schlaf am wenigsten ausgeprägt erscheint (5~\% höher), ähneln sich Wachzustand und Leichtschlaf (maximal 15~\% höher). Der Tiefschlaf zeigt die stärkste Erhöhung der Kopplungen zwischen den Systemen der Atemanstrengung (22~\% höher). Die Augen sind bei den Insomniepatienten in allen Schlafstadien stärker verbunden als bei gesunden Probanden. Insbesondere im Leichtschlaf zeigt sich auch eine erhöhte Verbindungsstärke zwischen den Augen und den $\delta$-Frequenzbändern aller \acs{EEG}s. Zusammenfassend lassen sich folgende Merkmale der Insomnie anhand des \acs{TDS}-Verfahrens festhalten:

\begin{itemize}
\item um 10~\% bis 15~\% erhöhte Verbindungsstärken innerhalb der gleichen Hemisphäre des Gehirns in allen Schlafstadien
\item um 5~\% bis 10~\% erhöhte Verbindungsstärken zwischen den Hemisphären des Gehirns in allen Schlafstadien
\item erhöhte Kopplung der Atemsysteme untereinander, hauptsächlich im \acs{NREM}-Schlaf
\item verringerte Kopplungen der Augen mit den $\delta$-Frequenzbändern der \acs{EEG}s hauptsächlich im Leichtschlaf
\end{itemize}

Das Erscheinungsbild der Atemsysteme im Tiefschlaf sowie die erhöhten Verbindungsstärken der \acs{EEG}s untereinander lassen vermuten, dass sich hierin maßgeblich das Krankheitsbild der Insomnie widerspiegelt. Da der Tiefschlaf als das Schlafstadium mit der stärksten Erholungsfunktion sowie autonom arbeitender Körperfunktionen gilt, kann eine Reduzierung dieser Autonomie insbesondere in Form verstärkter Verbindungen im Gehirn darauf schließen lassen, dass dies maßgeblich die Erholungsfunktion beeinträchtigt. Insbesondere vor dem Hintergrund des nahezu identischen Prozentanteils des Tiefschlafs bei den Insomniepatienten und gesunden Probanden (Abb. \ref{fig:nsis}) verstärkt diese Annahme. Die erhöhte Verbindungsstärke der Atemsysteme im \acs{NREM}-Schlaf könnte ebenfalls mit einer verringerten Autonomie der Systeme erklärt werden. Die erhöhten Verbindungsstärken im Leichtschlaf ähneln denen des Wachzustands bei den gesunden Probanden. Dies könnte ursächlich für ein verringertes oder verkürztes Auftreten von langsamem Augenrollen, welches für den Leichtschlag charakteristisch ist, und damit für die insbesondere im Leichtschlaf maßgeblich verringerte Verbindungsstärke zwischen den Augen und den niedrigen $\delta$-Frequenzbändern des Gehirns sein. Die dauerhaft erhöhte Kopplung der Systeme könnte demnach zu dem Symptom der mangelnden Schlafqualität bei Insomniepatienten führen. Auf Grundlage dieser Vermutungen könnte ebenfalls begründet werden, warum Patienten mit Insomnie, welche sämtliche Schlafstadien in angemessener Länge und ausreichenden Zyklen durchlaufen, dennoch von einer beeinträchtigten Schlafqualität berichten.

\begin{figure}[H]
	\centering
	\includegraphics[width = 0.5\textwidth]{img/nsis.png}
	\caption[Epochen pro Schlafstadium]{Gemittelte Anzahl der Epochen pro Schlafstadium in Prozent;\\grün = Insomniepatienten; blau = gesunde Probanden aus dem SIESTA-Datensatz;\\während sich der Anteil des Tiefschlafs bei beiden Gruppen stark ähnelt, sind die Anteile des Leicht- und \acs{REM}-Schlafs bei den Insomniepatienten deutlich reduziert und der Wachanteil erhöht}
	\label{fig:nsis}
\end{figure}


Verteilung der Schlafstadien\\


Einfluss der Krankheiten - mit Differenz all-insom vergleichen\\

Abgrenzung kleiner Insomnie-Datensatz und mit Krankheiten\\


%Die Gruppe Insomnie umfasst mit 28 Patienten nahezu die Hälfte des Gesamtdatensatzes von 64 Patienten. Zur Ermittlung insomniespezifischer Merkmale in den \acs{TDS}-Ergebnissen wird diese Gruppe daher ebenfalls in die Untersuchungen mit einbezogen.\\

\section{Altersabhängigkeiten}

\section{Geschlechtsabhängigkeiten}