\chapter{Ergebnisse}

\section{TDS-Merkmale bei Insomnie}

Die gemittelten \acs{TDS}-Matrizen der jeweiligen Gruppen (Abb.~\ref{fig:meanTDSall} sowie Anhang Abb.~\ref{fig:meanTDSheart} bis \ref{fig:meanTDSinsom}) zeigen unterschiedliche Verbindungsstärken, jedoch insgesamt sehr ähnliche Tendenzen. Insbesondere ist bei jeder Gruppe eine relativ hohe Verbindungsstärke im Wachzustand und Leichtschlaf und eine stark reduzierte Verbindungsstärke im Tiefschlaf zu beobachten, während die Verbindungsstärken im \acs{REM}-Schlaf dazwischen liegen. Die Gruppe Herz zeigt insbesondere in den Verbindungen zwischen den \acs{EEG}s C4 und O1 sowie zwischen C3 und den beiden okzipitalen \acs{EEG}s sehr schwache Verbindungsstärken, welche teilweise unter dem Signifikanz-Schwellenwert von 7 \% liegen. Gegenüber der Gruppe Insomnie zeigt die Gruppe Herz im Wachzustand und Leichtschlaf eine leicht erhöhte Verbindungsstärke zwischen der Herzfrequenz und allen anderen Systemen, im \acs{REM}-Schlaf und Tiefschlaf hingegen eine leicht reduzierte. Im Leichtschlaf und Tiefschlaf sind darüber hinaus die Verbindungsstärken zwischen den Systemen C3-O2, C4-O1 sowie O1-O2 sehr gering. Die gleiche Tendenz in weniger starker Ausprägung zeigt sich jedoch auch in der Gruppe Insomnie.\\

Die Gruppe Atmung ähnelt im Wachzustand, im Leicht- und Tiefschlaf in den Verbindungsstärken sehr stark der Gruppe Insomnie. Die Augen sind in der Gruppe Atmung im Vergleich zur Gruppe Insomnie verstärkt mit den $\delta$- und $\theta$-Frequenzbändern aller \acs{EEG}s verbunden. Die Systeme C3 und C4 zu den gegenüberliegenden okzipitalen \acs{EEG}s sowie diese beiden untereinander zeigen ähnlich wie die Gruppe Herz eine sehr schwache Verbindungsstärken. Die Atemsysteme zeigen leicht erhöhte Kopplungen untereinander.\\

Die Gruppe Schlafmittel zeigt im Wachzustand und Leichtschlaf leicht erhöhte Verbindungsstärken innerhalb der Atemsysteme gegenüber der Gruppe Insomnie. Im \acs{REM}-Schlaf und Tiefschlaf hingegen sind leicht reduzierte Verbindungsstärken innerhalb der Atemsysteme erkennbar. Die Kopplung der Herzfrequenz mit den $\alpha$-, $\sigma$- und $\beta$-Frequenzbändern aller \acs{EEG}s erscheint in der Gruppe Schlafmittel leicht erhöht. Die Verbindungen zwischen den \acs{EMG}-Systemen und den $\delta$-Frequenzbänder des Gehirns sind im Leicht- und Tiefschlaf ebenfalls erhöht, in den anderen beiden Schlafstadien ist lediglich das Kinn leicht verstärkt mit den $\alpha$-, $\sigma$- und $\beta$-Frequenzbändern der \acs{EEG}-Systeme verbunden. Die Augen sind in dieser Gruppe im Leicht-, Tief- und \acs{REM}-Schlaf weniger stark mit allen Systemen des Gehirns verbunden. Innerhalb des Gehirns sind nahezu alle Verbindungen gegenüber der Gruppe Insomnie leicht reduziert. Im Tiefschlaf fallen die Verbindungen $\alpha$C3 mit $\delta$ und $\theta$O1 sowie $\alpha$O1 mit $\delta$O2 und umgekehrt unter die Signifikanz-Schwelle. Auch die Verbindungen der Systeme O1 und C3 in den $\alpha$- und $\theta$-Frequenzbändern zeigen reduzierte Verbindungsstärken unterhalb der Signifikanz-Schwelle.\\

Insgesamt bilden die gemittelten \acs{TDS}-Matrizen der Gruppenergebnisse nicht eindeutig die krankheitsbedingten Beeinträchtigungen der Kopplungen zwischen Herz bzw. Atmung und den anderen Systemen ab. Die Abweichungen sind gering und verändern nicht die Gesamttendenzen. Dies kann daran liegen, dass viele betroffene Patienten entsprechende Medikamente einnehmen, welche die Symptome abschwächen. Auch bei der Einnahme von Schlafmitteln zeigen sich keine gravierenden Unterschiede zur Gruppe Insomnie. Die Abweichungen werden daher als nicht signifikant eingestuft. Darüber hinaus sind die Gruppen nicht vollends miteinander vergleichbar, da die Gruppen größtenteils unterschiedliche Patienten enthalten, so dass individuelle Abweichungen nicht ausgeschlossen werden können und diese demnach die Abweichungen beeinflussen. Eine Beziehung zwischen den Erkrankungen bzw. der Einnahme von Schlafmitteln und den Verbindungsstärken in den \acs{TDS}-Matrizen kann daher nicht eindeutig hergestellt werden. \\

Ein Vergleich der Gruppe Insomnie mit der Gruppe Alle durch Subtraktion der Ergebnis-Matrizen macht ebenfalls deutlich, dass die übrigen Gruppen Herz, Atmung und Schlafmittel lediglich innerhalb der Systeme des Gehirns und geringfügig in den Atemsystemen Änderungen hervorrufen (Abb. \ref{fig:TDSall_insom}). Im Wachzustand liegen diese Änderungen in einer leichten Verstärkung der Atemsysteme sowie der okzipitalen Gehirnsysteme untereinander. Darüber hinaus sind beide Augen mit den niedrigen Frequenzbändern ($\delta$ und $\theta$) mit allen Gehirnsystemen leicht verstärkt verbunden. Da der Differenzwert bei maximal 4 \% liegt, können die Änderungen im Wachzustand durch die Gruppen Herz, Atmung und Schlafmittel als nicht signifikant betrachtet werden. Im Leichtschlaf sind die Atemsysteme ähnlich wie im Wachzustand gekoppelt. Die Augen sind um bis zu 7 \% stärker mit den $\delta$- und $\theta$- Frequenzbändern der zentralen Gehirnregionen verbunden. Innerhalb der Gehirnsysteme finden sich nahezu keine Differenzen ($\pm$ 2 \%), wobei die Reduzierung vorwiegend die Verbindungen mit dem \acs{EEG} O1 betrifft. Im \acs{REM}-Schlaf bewirken die Gruppen Herz, Atmung und Schlafmittel nahezu ausschließlich leicht erhöhte Verbindungen um bis zu 6 \% der hohen Frequenzbänder $\sigma$ und $\beta$ zwischen den Hemisphären. Im Tiefschlaf sind die Augen um 12 \% stärker miteinander verbunden und um bis zu 8 \% mit den $\delta$-Frequenzbändern der zentralen Gehirnregionen. Mit den okzipitalen Bereichen sind die Augen um maximal 5 \% stärker verbunden. Eine signifikante Beeinflussung der Kopplungen zwischen den Atemsystemen ist weder im \acs{REM}- noch im Tiefschlaf zu erkennen.\\

\begin{figure}[H]
	\centering
	\includegraphics[width = \textwidth]{img/TDS_all_insom.png}
	\caption[Differenzen zwischen Gruppe Alle und Insomnie]{Differenz-Matrizen der Gruppen Alle und Insomnie mit überwiegend nicht signifikanten Differenzen; positive Werte = Verbindungen in Gruppe Alle sind stärker; negative Werte = Verbindungen in Gruppe Insomnie sind stärker\\links: unteres linkes Dreieck = \acs{REM}-Schlaf, oberes rechtes Dreieck = Tiefschlaf;\\rechts: unteres linkes Dreieck = Wachzustand, oberes rechts Dreieck = Leichtschlaf}
	\label{fig:TDSall_insom}
\end{figure}

Da auch bei diesem Vergleich die nicht signifikanten Unterschiede keinen Rückschluss auf Herzerkrankungen, Atemwegserkrankungen oder die Einnahme von Schlafmitteln zulassen, werden die folgenden Untersuchungen anhand des Gesamtdatensatzes (Gruppe Alle) vorgenommen. Darüber hinaus ist die Insomnie ohnehin häufig mit anderen Erkrankungen verbunden (vgl. Arten der Insomnie, Anhang Tab. \ref{tab:allgemeine_insomnie} bis \ref{tab:korperliche_insomnie}). Für die Untersuchung insomniespezifischer \acs{TDS}-Merkmale werden die Ergebnisse von Krefting et al. \parencite{krefting_age_2017} für gesunde Probanden aus der SIESTA-Studie reproduziert und anschließend aus der gemittelten Ergebnis-Matrix die frontopolaren Systeme entfernt (Abb. \ref{fig:meanTDSsiesta}). Auf diese Weise erhält die Ergebnis-Matrix aus den SIESTA-Daten die gleichen Dimensionen wie die hiesige Ergebnis-Matrix der Gruppe Alle, so dass die Vergleichbarkeit der Daten vereinfacht wird.\\

die gemittelten Ergebnis-Matrizen der Gruppen Alle (Abb. \ref{fig:meanTDSall}) mit dem Ergebnis von  verglichen. Nach Reproduktion der \acs{TDS}-Ergebnisse anhand der SIESTA-Daten und Entfernung der frontopolaren Systeme aus den Matrizen, wird hiervon die Ergebnis-Matrix der Gruppe Alle subtrahiert.\\

Gesamttendenz ls/w und ds/r ist ähnlich

(Differenz Siesta-Alle Anhang Abb. \ref{fig:diffTDSsiesta_all})

%Die Gruppe Insomnie umfasst mit 28 Patienten nahezu die Hälfte des Gesamtdatensatzes von 64 Patienten. Zur Ermittlung insomniespezifischer Merkmale in den \acs{TDS}-Ergebnissen wird diese Gruppe daher ebenfalls in die Untersuchungen mit einbezogen.\\

\begin{figure}[H]
	\centering
	\includegraphics[width = \textwidth]{img/meanTDSall.png}
	\caption[\acs{TDS}-Gruppenergebnis für die Gruppe Alle]{Gemittelte \acs{TDS}-Matrizen für die Gruppe Alle;\\links: unteres linkes Dreieck = \acs{REM}-Schlaf, oberes rechtes Dreieck = Tiefschlaf;\\rechts: unteres linkes Dreieck = Wachzustand, oberes rechts Dreieck = Leichtschlaf}
	\label{fig:meanTDSall}
\end{figure}

\begin{figure}[H]
	\centering
	\includegraphics[width = \textwidth]{img/meanTDSsiesta.png}
	\caption[\acs{TDS}-Gruppenergebnis für den SIESTA-Datensatz]{Gemittelte \acs{TDS}-Matrizen für die Kontrollgruppe des SIESTA-Datensatzes nach Entfernung der frontopolaren Systeme;\\links: unteres linkes Dreieck = \acs{REM}-Schlaf, oberes rechtes Dreieck = Tiefschlaf;\\rechts: unteres linkes Dreieck = Wachzustand, oberes rechts Dreieck = Leichtschlaf}
	\label{fig:meanTDSsiesta}
\end{figure}

Verteilung der Schlafstadien
Tiefschlaf mehr verbunden - dadurch vllt weniger erholungsfunktion, mit verteilung der schlafstadiun überprüfen
Augen im Leichtschlaf (langsames Augenrollen)

\section{Altersabhängigkeiten}

\section{Geschlechtsabhängigkeiten}