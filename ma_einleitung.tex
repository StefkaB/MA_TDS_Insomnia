\chapter{Einleitung}

\section{Motivation}

Gegenwärtig existieren diverse Netzwerke, welche das Merkmal von miteinander über Kanten verbundenen Knoten erfüllen, wie beispielsweise Straßennetze, Stromnetze oder das menschliche Gehirn. Der Ursprung der Netzwerkforschung liegt jedoch in der Sozialwissenschaft. Der Soziologe Georg Simmel sowie dessen Schüler Leopold von Wiese beschäftigten sich erstmals mit Beziehungen und Netzwerken anhand von sozialen Gefügen. Simmel bezeichnet im Jahr 1908 die Soziologie als "`Geometrie sozialer Beziehungen"' \parencite{simmel_soziologie.:_2013}. Von Wiese beschreibt im Jahr 1933 die formale Soziologie als "`ein scheinbar undurchdringliches Netz von Linien [...], die von Punkten (Menschen) ausgehen"' \parencite{wiese_system_1966}. Im darauffolgenden Jahr 1934 erklärt der Soziologe Jacob Moreno, dass die Verbindungen zwischen Knoten innerhalb eines Netzwerks nicht gleichmäßig seien \parencite{moreno_grundlagen_2014}, so dass zentrale Knoten mit vielen von ihnen abgehenden Verbindungen sowie zahlreiche Knoten mit wenigen abgehenden Verbindungen existieren. \parencite{lenzen_alles_2016, kneer_handbuch_2009}\\

Unabhängig von diesen Erkenntnissen veröffentlicht der ungarische Schriftsteller Frigyes Karinthy im Jahr 1929 seine Erzählung "`Kettenglieder"' (Chain-Links), in der er als eine Art Gesellschaftsspiel die These aufstellt, dass jeder Mensch auf der Erde mit jedem anderen Menschen über maximal fünf Bekanntschaften verbunden sei \parencite{karinthy_chain-links_1929}. Diese These greift der amerikanische Psychologe Stanley Milgram im Jahr 1967 auf. Mit einem Experiment, bei dem Briefe durch Probanden über deren Bekanntschaften an zufällig ausgewählte Personen großer Unternehmen oder Parteien zugestellt werden sollen, kann er Karinthys These belegen \parencite{milgram_small-world-problem_1967}. Jedoch ergibt das Experiment, dass durchschnittlich nicht maximal fünf, sondern sechs Verbindungen benötigt werden. Milgram prägt damit den Begriff "`Kleine-Welt-Phänomen"'. \parencite{lenzen_alles_2016}

\newpage

In den folgenden Jahren wird das Kleine-Welt-Phänomen als Netzwerk vielfach untersucht und sogar auf das menschliche Gehirn angewendet. Ein elementares Ziel der Netzwerkforschung stellt bis heute die Erforschung des menschlichen Gehirns dar, welches mit rund $10^{14}$ Verbindungen zwischen den etwa $10^{11}$ Neuronen bekannt ist als das komplexeste natürliche Netzwerk \parencite{ertel_grundkurs_2013}. Während beispielsweise die Funktion verschiedener Hirnareale oder die Kommunikation zwischen Neuronen mittels biochemischen Transmittern ergründet sind, stellt das Verständnis über Wissens- und Persönlichkeitsspeicherung im Gehirn jedoch noch eine enorme Herausforderung dar. Mit Hilfe der Netzwerktheorie ist es möglich, hochkomplexe Zusammenhänge zu untersuchen und besser zu verstehen. Daher dient die Netzwerktheorie nicht nur konventionellen Netzwerken, wie Straßen- oder Stromnetzen, sondern auch solchen, deren Wechselwirkungen noch nicht gänzlich ergründet sind. \parencite{lenzen_alles_2016}\\

Grundlage dieser Arbeit soll ein innovativer Netzwerkansatz sein - das Verfahren der \acl{TDS} (\acs{TDS}), welches verschiedene physiologische Systeme des menschlichen Körpers im Schlaf als Netzwerk betrachtet. Auf dieser Grundlage kann die \acs{TDS}-Analyse das Zusammenwirken der einzelnen Systeme im nicht wachen Zustand abbilden. In der Schlafmedizin spielen die Signalaufzeichnungen der physiologischen Systeme eine entscheidende Rolle, um Schlafstadien zu klassifizieren und schlafbezogene Krankheiten zu diagnostizieren. Die exakten Wechselwirkungen zwischen den Systemen und deren Einfluss auf den Schlaf sind jedoch bislang noch weitgehend unerforscht, nicht zuletzt aufgrund der offenen Fragen in der Gehirnforschung. Dieses neue Verfahren der \acs{TDS} eröffnet daher bislang ungenutzte Möglichkeiten der Untersuchung von Zusammenhängen einzelner Körperfunktionen im Schlaf und kann ebenfalls Ansätze für ein profunderes Verständnis des menschlichen Gehirns liefern. Insbesondere in der Schlafmedizin erweitert die \acs{TDS}-Analyse die bisherigen Methoden der Diagnostik, Behandlung und Patientenüberwachung. \parencite{bashan_network_2012, penzel_schlafstorungen_2005}

\section{Aufgabenstellung und Zielsetzung}

Es sollen mit Hilfe der \acs{TDS}-Analyse die Zusammenhänge zwischen Körperfunktionen im Schlaf untersucht werden. Zu diesem Zweck werden multivariate Biosignalaufzeichnungen aus dem Schlaflabor der Charité Berlin verwendet. Nachdem Untersuchungen der Signalaufzeichnungen gesunder Probanden anhand der \acs{TDS}-Analyse bereits eindeutige Ergebnisse erzielten \parencite{bashan_network_2012} und das Verfahren als robust gegenüber Artefakten bezeichnet werden kann \parencite{breuer_netzwerktopologie_2016}, sind nunmehr Untersuchungen in Hinblick auf verschiedene Krankheitsbilder in der Schlafmedizin von Interesse. 

\newpage

Zu diesem Zweck soll der Fokus dieser Arbeit auf die Anwendung des \acs{TDS}-Verfahrens auf Signalaufzeichnungen von Insomniepatienten gelegt werden, welche krankheitsbedingt an Ein- und Durchschlafstörungen leiden \parencite{mayer_s3-leitlinie_2009}. Hierbei werden die Schlafstadien Wachzustand, Leicht- und Tiefschlaf sowie \acl{REM} Schlaf (\acs{REM}) näher betrachtet. Darüber hinaus sollen Untersuchungen zur Geschlechts- und Altersabhängigkeit angestellt werden. Zur Analyse dienen ferner verschiedene statistische Verfahren.\\

Ziel ist es, die Ergebnisse dieser Arbeit mit den bisherigen Erkenntnissen aus den Untersuchungen gesunder Probanden zu vergleichen und ggf. Hinweise für krankheitsbedingte Netzwerkmerkmale zu extrahieren. Darüber hinaus sollen konkrete Aussagen über die Geschlechts- und Altersabhängigkeit der \acs{TDS} bei Insomniepatienten getroffen werden. 

\section{Gliederung der Arbeit}

Nach Klärung der Motivation, Aufgabenstellung und Zielsetzung wird im Folgenden zunächst auf die medizinischen und methodischen Grundlagen eingegangen, welche die Basis für die durchzuführenden Untersuchungen bilden. Anschließend wird die Thematik in den aktuellen Forschungsstand eingebunden. Daran anschließend werden die Daten sowie die verwendeten Untersuchungsumgebungen und -methoden vorgestellt. Des Weiteren wird das Vorgehen der Untersuchungen beschrieben. Nach Ausführung der Ergebnisse bilden Zusammenfassung, Fazit und Ausblick den Abschluss dieser Arbeit. 