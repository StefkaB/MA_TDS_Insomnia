\chapter{Analyse}

\section{Datenanalyse}\label{datenanalyse}

Die Datenanalyse erfolgt auf Grundlage der zur Verfügung stehenden Dateien des Untersuchungsdatensatzes. Auf die Implementierung der entsprechenden Analysealgorithmen wird in Punkt \ref{Datenvorauswahl} eingegangen.

\paragraph{Untersuchungsdatensatz} Für diese Arbeit steht ein Untersuchungsdatensatz von insgesamt 64 Patienten der Charit\'{e} Berlin mit der Diagnose Insomnie zur Verfügung. Welche Art der Insomnie bei den Patienten diagnostiziert worden ist, ist jedoch nicht bekannt. Sämtliche Aufzeichnungen stammen aus den Jahren 2008 und 2009. Zu jedem Patienten liegt eine \acs{PSG} als \acs{EDF}-Datei vor. 

\paragraph{Patienteninformationen} Vorerkrankungen und Begleitmedikationen sind für jeden Patienten in einer Excel-Datei dokumentiert. In dieser Datei sind für jeden Patienten ebenso Geburtstag, Alter, Geschlecht, Größe, Gewicht, \acl{BMI} (\acs{BMI}) sowie Datum der Aufzeichnung aufgeführt. Besonders häufige Vorerkrankungen sind Blinddarm- bzw. Mandelentzündungen, welche bei 17 bzw. 20 Patienten eine Entfernung des Organs zur Folge hatten. Ebenso auffällig sind diagnostizierte Kurz- bzw. Weitsichtigkeit oder Grüner Star bei insgesamt 19 Patienten. Bei acht weiblichen Patienten kommen Hormone zur Milderung von Wechseljahressymptomen zum Einsatz, teilweise in Verbindung mit Schlafmitteln oder Antidepressiva. Bei zwölf weiblichen Patienten liegen darüber hinaus gynäkologische Vorerkrankungen, wie Myome\footnote{Myom: gutartiger Tumor in der Gebärmutter}, Eierstockzysten oder eine Entfernung der Eierstöcke, vor. Fünf Patienten aus dem Untersuchungsdatensatz leiden an Atemwegserkrankungen, wie beispielsweise Asthma, und nehmen atemwegserweiternde Medikamente ein. Bei etwa einem Viertel (15) der Patienten wurden Bluthochdruck, Herzrhythmusstörungen oder verengte Blutgefäße diagnostiziert, so dass zwölf von ihnen blutgefäßerweiternde Medikamente einnehmen. Bei insgesamt acht Patienten sind Schlafmittel als Begleitmedikation dokumentiert, teilweise in Abhängigkeit von anderen Erkrankungen als die Insomnie, wie beispielsweise Wechseljahresbeschwerden.

\paragraph{Ereignisse} Zu jeder \acs{PSG} existiert eine Ereignis-Datei im TXT-Format. Hierin werden Patienten-ID, Datum der Aufzeichnung, mögliche Ereignisse, wie beispielsweise das Abfallen einer Elektrode oder Augenbewegungen, Name des Auswerters und Datum der Auswertung dokumentiert. Darauffolgend werden tabellarisch das jeweils klassifizierte Schlafstadium, Zeitpunkt der Epoche, Ereignis und dessen Dauer sowie in einigen Fällen auch die Körperlage zu jedem Zeitpunkt aufgelistet. Als Ereignis wurden bei fast allen Patienten Schlafapnoen\footnote{Schlafapnoen: Atemstillstände über mindestens zehn Sekunden im Schlaf und häufiger als fünfmal pro Stunde} vermerkt. Diese sind in de Excel-Datei über die Begleiterkrankungen jedoch nicht aufgeführt. Die Ereignis-Dateien müssen aus diesem Grund nach Apnoen durchsucht werden. Treten mehr als fünf Apnoe-Ereignisse pro Stunde bei einem Patienten auf, so ist diesem Patienten das Krankheitsbild von Schlafapnoen zuzuordnen. 
Da die \acs{TDS} Analyse als robust gegenüber Artefakten gilt \parencite{breuer_netzwerktopologie_2016}, kann auf eine Bereinigung der Daten verzichtet werden.

\paragraph{Erstellung von Hypnogrammen} Die Ereignis-Dateien weisen bei allen Patienten einen anderen Startzeitpunkt der Aufzeichnung auf als in dem Header der \acs{EDF}-Datei (Abb. \ref{fig:zeitpunkte}). Dies führt dazu, dass die Anzahl der Epochen in den \acs{EDF}-Dateien größer ist als in den Ereignis-Dateien. Ein aus der Ereignis-Datei extrahiertes Hypnogramm würde demnach nicht den Verlauf der \acs{PSG} abbilden. Die Differenz in Sekunden zwischen den beiden Zeitpunkten entspricht jedoch stets der auf die nächste ganze Zahl abgerundeten Anzahl an Epochen, um welche die Signale in der \acs{EDF}-Datei länger sind. Dies lässt darauf schließen, dass die \acs{PSG} jeweils früher gestartet worden ist als die Aufzeichnung der Ereignisse. Es müssen daher Hypnogramme erstellt werden, welche auf die \acs{EDF}-Dateien anwendbar sind.

\begin{figure}[H]
	\centering
	\includegraphics[width = \textwidth]{img/Startzeitpunkte.png}
	\caption[Startzeitpunkte von Ereignis- und \acs{EDF}-Datei]{Unterschiedliche Startzeitpunkte in Ereignis-Datei (links) und \acs{EDF}-Header (rechts) lassen darauf schließen, dass die Ereignis-Datei zu einem späteren Zeitpunkt als die \acs{PSG}-Aufzeichnung gestartet wurde}
	\label{fig:zeitpunkte}
\end{figure}

\paragraph{Untersuchung der Schlafstadien} 
Da sich die Ergebnisse der \acs{TDS} Analyse auf die einzelnen Schlafstadien beziehen, ist es notwendig, dass sämtliche relevanten Schlafstadien durchlaufen werden. Darüber hinaus müssen die jeweiligen Schlafstadien über einen hinreichenden Zeitraum andauern, um stabile Verbindungen zu erkennen. Da als stabile Verbindung mindestens fünf zusammenhängende Epochen \`{a} 30 Sekunden gelten (Punkt \ref{TDS}), sollte jedes Schlafstadium über mindestens 2,5 Minuten andauern. Um diese Voraussetzungen zu überprüfen, müssen die Ereignis-Dateien sämtlicher \acs{PSG}s untersucht werden.

\paragraph{Klassifizierung in Gruppen}
Die Dokumentation der Patienteninformationen in der Excel-Datei zeigt, dass ein Großteil der Patienten Begleiterkrankungen aufweist und/oder Medikamente einnimmt. Diese können die Ergebnisse der \acs{TDS} Analyse jedoch beeinflussen. Beispielsweise kann eine Herzrhythmusstörung zu einer abweichenden Darstellung der Kopplung zwischen dem Herzen und anderen Systemen führen. Asthma oder Schlafapnoen wirken sich auf die Kopplung zwischen Atmung und anderen Systemen aus. Darüber hinaus können verabreichte Medikamente, wie beispielsweise Schlafmittel, die Symptome der Insomnie verfälschen. Aus diesem Grund müssen die Daten in verschiedene Gruppen klassifiziert werden. Dies ermöglicht einen späteren Vergleich der Gruppenergebnisse und Rückschlüsse auf den Einfluss der Begleiterkrankungen und -medikationen auf die Ergebnisse der \acs{TDS} Analyse.\\

Es wird basierend auf den Patienteninformationen des zugrunde liegenden Datensatzes sowie den Ereignissen in den Ereignis-Dateien folgende Gruppeneinteilung vorgenommen:

\begin{itemize}
\item \textbf{Gruppe Alle:} gesamter Untersuchungsdatensatz
\item \textbf{Gruppe Herz:} Herzrhythmusstörungen, zu schnelle oder zu langsame Herzfrequenz, zu hoher oder zu niedriger Blutdruck, Carotisstenose\footnote{Carotisstenose: Verengung der hirnversorgenden Halsschlagader}
\item \textbf{Gruppe Atmung:} Asthma, COPD\footnote{COPD: chronische Erkrankung der Lunge}, Schlafapnoen
\item \textbf{Gruppe Schlafmittel:} Einnahme von schlaffördernden Mittel, Zolpidem, Rohiypnol, Zopiclon, Doxepin, Schlafsterne, Stilnox, Noctamid, Remifemin
\item \textbf{Gruppe Insomnie:} Patienten sind nicht den obigen Gruppen zuzuordnen, können jedoch andere Begleiterkrankungen und -medikationen aufweisen, welche sich vermutlich nicht maßgeblich auf die Ergebnisse der \acs{TDS} Analyse auswirken, Beispiele: Antibiotika nach Blutvergiftung, Hormone und Medikamente zur Linderung von Wechseljahresbeschwerden, Hormone bei Schilddrüsenerkrankungen
\end{itemize}

\paragraph{Kriterien Datenvorverarbeitung} Folgende Kriterien (Tab. \ref{tab:Datenvorauswahl}) muss ein Datensatz zur Einbeziehung in die hiesigen Untersuchungen demnach erfüllen:

\begin{table}[H] 
\begin{tabularx}{\textwidth}{cX}
\toprule
\multicolumn{1}{c}{\textbf{~~~~~~Nr.~~~~~~}} & \multicolumn{1}{l}{\textbf{Kriterien}}\\
\midrule 
D1 & Für jeden Patienten muss aus der Ereignis-Datei ein Hypnogramm erstellt werden, welches auf die \acs{EDF}-Datei anwendbar ist.\\
D2 & Alle relevanten Schlafstadien (\acs{NREM}2, \acs{NREM}3 oder \acs{NREM}4, \acs{REM}-Schlaf und Wachzustand) müssen durchlaufen werden.\\
D3 & Jedes relevante Schlafstadium muss mindestens fünf zusammenhängende Epochen beinhalten.\\
D4 & Patienten, deren Ereignis-Datei mehr als fünf Schlafapnoen pro Stunde aufweist, soll die schlafbedingte Erkrankung Schlafapnoe zugeordnet werden.\\
D5 & Die Patienten sollen anhand ihrer Begleiterkrankungen bzw. -medikationen in fünf Gruppen (alle, Herz, Atmung, Schlafmittel, Insomnie) eingeteilt werden.\\
D6 & Artefakte sollen nicht entfernt werden.\\
\bottomrule
\end{tabularx}
\caption[Datenvorauswahl - Kriterien]{Datenvorauswahl - Kriterien für die Auswahl und Klassifizierung der Daten vor Anwendung der \acs{TDS} Analyse}
\label{tab:Datenvorauswahl}
\end{table}

\paragraph{Ergebnisse Datenanalyse} Im Ergebnins der Datenanalyse und Datenvorauswahl wurde für jeden Patienten ein Hypnogramm im Format [Hypnogram\_filename.txt] erstellt (Kriterium D1). Diese enthalten in der ersten Zeile die vergangene Zeit in Sekunden zwischen den Startzeitpunkten aus dem \acs{EDF}-Header sowie der Ereignis-Datei, um bei der \acs{TDS} Analyse Hypnogramm und \acs{PSG} optimal übereinanderlegen zu können. Darauffolgend sind zeilenweise die Schlafstadien aufgelistet.\\

Die Überprüfung der durchlaufenen Schlafstadien (Kriterien D2 und D3) zeigt, dass insgesamt 41 Patienten aus dem Untersuchungsdatensatz das Schlafstadium \acs{NREM}4 nicht durchlaufen haben. Da bei der \acs{TDS} Analyse jedoch die Schlafstadien \acs{NREM}3 und \acs{NREM}4 als Tiefschlaf zusammengelegt werden (Kapitel \ref{TDS}), kann dieses Ergebnis vernachlässigt werden. Dementsprechend ist bei diesen 41 Patienten das Schlafstafium \acs{NREM}4 zu kurz. Dies betrifft zusätzlich einen weiteren Patienten. Darüber hinaus dauert das Schlafstadium \acs{NREM}1 bei einem Patienten weniger als fünf zusammenhängende Epochen an. Bei einem weiteren Patienten trifft dies auf das Schlafstadium \acs{NREM}3 zu, das Schlafstadium \acs{NREM}4 wird jedoch in ausreichender Länge durchlaufen. Das fehlende Schlafstadium \acs{NREM}1 kann hier vernachlässigt werden, da dieses nicht in die \acs{TDS} Analyse mit einfließt (Kapitel \ref{TDS}). Für die fehlenden Schlafstadien \acs{NREM}3 und \acs{NREM}4 gilt, dass bei jedem Patienten mindestens eines davon in ausreichender Länge durchlaufen werden muss. Dies trifft bei sämtlichen Patienten zu. Fehlende Schlafstadien \acs{NREM}2, \acs{REM}-Schlaf oder Wachzustand treten nicht auf. Für die \acs{TDS} Analyse können demnach sämtliche Daten im Untersuchungsdatensatz verwendet werden.\\

Die Untersuchung hinsichtlich der in den Ereignis-Dateien auftretenden Apnoe-Ereignissen (Kriterium D4) ergibt, dass bei insgesamt lediglich 16 Patienten keine Apnoe-Ereignisse auftreten. Bei ebenfalls 16 der verbleibenden Patienten treten Apnoe-Ereignisse häufiger als fünfmal pro Stunde auf. Diese werden eingestuft in die Gruppe "`Atmung"'.\\

Neben den Patienten mit eindeutigen Apnoe-Ereignissen werden ebenfalls weitere, die Atmung betreffende Krankheiten, in die Gruppe "`Atmung"' klassifiziert. Nach Untersuchung der Begleiterkrankungen und -medikationen wird die Einteilung der (Kriterium D5) gemäß Tab. \ref{tab:gruppen} vorgenommen. Der Gesamtdatensatz (Gruppe Alle) besteht aus etwa zwei Dritteln (41) weiblichen und einem Drittel (23) männlichen Patienten bei einem Altersdurchschnitt von 51 Jahren. Der Altersdurchschnitt der weiblichen Patienten aus dieser Gruppe liegt bei 52,3 Jahren (minimales Alter: 23 Jahre, maximales Alter: 65 Jahre). Bei den männlichen Patienten liegt der Altersdurchschnitt bei 48,7 Jahren (minimales Alter: 26 Jahre, maximales Alter: 61 Jahre). In den Gruppen Herz und Atmung ähnelt die Geschlechterverteilung der des Gesamtdatensatzes mit etwa zwei Dritteln weiblichen und einem Drittel männlichen Patienten. Der Altersdurchschnitt in diesen beiden Gruppen liegt leicht über dem Durchschnittsalter der Gruppe Alle. Schlafmittel nehmen hauptsächlich die weiblichen Patienten ein (80 \% der Gruppe Schlafmittel), wobei der Altersdurchschnitt hier ebenfalls über dem der Gruppe Alle liegt. Dies kann darauf zurückgeführt werden, dass in dem Gesamtdatensatz die Frauen ein höheres Alter aufweisen als die Männer. Darüber hinaus ist die Einnahme von Schlafmitteln bei den weiblichen Patienten häufig mit Wechseljahresbeschwerden verbunden, was sich in dem Gruppendurchschnittsalter widerspiegelt. In der Gruppe Insomnie ist die Differenz zwischen der Anzahl weiblicher und männlicher Patienten verältnismäßig am geringsten und der Altersdurchschnitt liegt mit 46,7 Jahren deutlich unter dem des Gesamtdatensatzes. Weitere Gruppenverteilungen können dem Anhang entnommen werden. Da einige Patienten mehrere Erkrankungen erleiden, treten Überschneidungen bei der Gruppeneinteilung auf, so dass die Summe der Patienten der Gruppen Herz, Atmung, Schlafmittel und Insomnie nicht der Gesamtanzahl der Patienten des Untersuchungsdatensatzes entspricht.\\

\begin{table}[H] 
\centering
\begin{tabularx}{\textwidth}{ccccccccccccccccccc}
\toprule
\multicolumn{3}{c}{\textbf{Alle}} & &  \multicolumn{3}{c}{\textbf{Herz}} & &  \multicolumn{3}{c}{\textbf{Atmung}} & &  \multicolumn{3}{c}{\textbf{Schlafmittel}} & &  \multicolumn{3}{c}{\textbf{Insomnie}}\\  
\midrule
w  & m  & A  &   & w  & m  & A    &   & w  & m & A  &   & w & m & A    &   & w  & m  & A\\
41 & 23 & 51 & ~ & 10 & 15 & 56,8 & ~ & 13 & 8 & 54 & ~ & 8 & 2 & 55,1 & ~ & 17 & 11 & 46,7\\
\bottomrule
\end{tabularx}
\caption[Klassifizierung der Patientengruppen]{Klassifizierung der Patientengruppen; w = Anzahl weiblicher Patienten;\\m = Anzahl männlicher Patienten; A = durchschnittliches Gruppenalter}
\label{tab:gruppen}
\end{table}


\section{Anforderungsanalyse}

Für die Analyse der physiologischen Netzwerke werden priorisierbare Anforderungen formuliert. Im Rahmen dieser Arbeit stellt sich das Problem, dass kein klassisches Software-Projekt umgesetzt werden soll, welches im Ergebnis ein benutzbares System darstellt. Stattdessen sollen durch Anwendung bestimmter Verfahren spezifische Ergebnisse berechnet werden. Aus diesem Grund stellt die Formulierung funktionaler und nicht-funktionaler Anforderungen oder die Erstellung von Use Cases nach klassischen Standards keine optimale Herangehensweise dar. Stattdessen werden allgemeine und gleichzeitig priorisierbare Anforderungen an... formuliert.\\

Hierbei erfolgt die Einteilung in Muss-, Soll-, Kann- sowie Abgrenzungskriterien. Muss-Kriterien sind zwangsläufig zu erfüllen. Sie stellen die Anforderungen mit der höchsten Priorität dar. In dieser Arbeit sind dies solche Anforderungen, die maßgeblich die wesentlichen Aufgaben der gesamten Analyse beschreiben. Soll-Kriterien sind im bestmöglichen Fall ebenfalls vollständig umzusetzen, sind in ihrer Priorität jedoch den Muss-Kriterien nachgestellt. Zusammen mit den Muss-Kriterien stellen sie die Anforderungen dar, welche maßgeblich zur erfolgreichen Umsetzung des Projekts beitragen und an deren Erfüllung der Erfolg der Umsetzung gemessen wird. Kann-Kriterien stellen optionale Anforderungen dar. Deren Umsetzung wertet die Qualität des Projekts auf, ein Weglassen beeinflusst jedoch nicht die Hauptfunktionalitäten. Im Rahmen dieser Arbeit würden durch die Nicht-Erfüllung von Kann-Kriterien beispielsweise die Ergebnisse der Analyseverfahren nicht beeinflusst werden. Abgrenzungskriterien werden darüber hinaus formuliert, um zu definieren, welche Funktionalitäten oder Eigenschaften nicht erfüllt werden sollen. Für die hiesige Analyse wäre beispielsweise die Abgrenzung zu klassischen Software-Projekten zu formulieren. Aus diesem Grund wird darüber hinaus auf eine Stakeholderanalyse gänzlich verzichtet.\\

Als Ergebnis der Anforderungsanalyse werden die in Tab. \ref{tab:Muss-Kriterien} bis \ref{tab:Abgr-Kriterien} dargestellten Kriterien definiert.\\

\begin{table}[H] 
\begin{tabularx}{\textwidth}{cllX}
\toprule
\multicolumn{3}{c}{\textbf{Muss-Kriterien}} & \\  
\cmidrule{1-3}
\multicolumn{3}{c}{Nr.} & Anforderung\\ 
\midrule 
\multicolumn{3}{c}{A1} &  Grundlage der Untersuchungen müssen \acs{PSG}s und Hypnogramme von Insomnie-Patienten sein.\\
\multicolumn{3}{c}{A2} & Auf Grundlage der \acs{PSG}s und Hypnogramme müssen physiologische\newline Netzwerke mit Hilfe des \acs{TDS} Verfahrens untersucht werden.\\
\multicolumn{3}{c}{A3} & Die physiologischen Netzwerke müssen auf Altersabhängigkeiten untersucht werden.\\ 
\multicolumn{3}{c}{A4} & Die physiologischen Netzwerke müssen auf Geschlechterabhängigkeiten untersucht werden.\\
\multicolumn{3}{c}{A5} & Die Analyseergebnisse müssen mit den Ergebnissen von Krefting et al. \parencite{krefting_age_2017} vergleichbar sein.\\
\bottomrule
\end{tabularx}
\caption{Anforderungen - Muss-Kriterien}
\label{tab:Muss-Kriterien}
\end{table}


\begin{table}[H] 
\begin{tabularx}{\textwidth}{cllX}
\toprule
\multicolumn{3}{c}{\textbf{Soll-Kriterien}} & \\  
\cmidrule{1-3}
\multicolumn{3}{c}{Nr.} & Anforderung\\ 
\midrule 
\multicolumn{3}{c}{A6} &  Für die Anwendung des \acs{TDS} Verfahrens soll Matlab verwendet werden.\\
\multicolumn{3}{c}{A7} & Für die Anwendung statistischer Verfahren soll R verwendet werden.\\
\multicolumn{3}{c}{A8} & Die Ergebnisse der Analysen sollen grafisch darstellbar sein.\\ 
\multicolumn{3}{c}{A9} & Die Untersuchungsdatensätze sollen in geeigneter Weise gemäß den Kriterien aus der Datenanalyse aufbereitet werden (Kriterien D1 - D5 unter Punkt \ref{datenanalyse}).\\
\multicolumn{3}{c}{A10} & Die Ergebnisse sollen sich auf die fünf in der Datenanalyse klassifizierten Gruppen beziehen.\\
\bottomrule
\end{tabularx}
\caption{Anforderungen - Soll-Kriterien}
\label{tab:Soll-Kriterien}
\end{table}


\begin{table}[H] 
\begin{tabularx}{\textwidth}{cllX}
\toprule
\multicolumn{3}{c}{\textbf{Kann-Kriterien}} & \\  
\cmidrule{1-3}
\multicolumn{3}{c}{Nr.} & Anforderung\\ 
\midrule 
\multicolumn{3}{c}{A11} &  Die Implementierung kann für ähnliche Fragestellungen auf andere Datensätze (beispielsweise Apnoe-Patienten) anwendbar und wiederverwendbar sein.\\
\multicolumn{3}{c}{A12} & Es kann eine Datenstruktur für die Speicherung prägnanter Patienteneigenschaften in Matlab erstellt werden.\\
\multicolumn{3}{c}{A13} & Für die \acs{TDS} Analyse kann ein Leitfaden für die Datenvorauswahl und die Anwendung des Verfahrens erstellt werden.\\ 
\multicolumn{3}{c}{A14} & Die bereits bestehenden Funktionen der \acs{TDS} Analyse können für die Anwendung auf Datensätze der Charit\'{e} erweitert werden.\\ 
\bottomrule
\end{tabularx}
\caption{Anforderungen - Kann-Kriterien}
\label{tab:Kann-Kriterien}
\end{table}


\begin{table}[H] 
\begin{tabularx}{\textwidth}{cllX}
\toprule
\multicolumn{3}{c}{\textbf{Abgrenzungskriterien}} & \\  
\cmidrule{1-3}
\multicolumn{3}{c}{Nr.} & Anforderung\\ 
\midrule 
\multicolumn{3}{c}{A15} &  Die Implementierung entspricht nicht der Form eines Paket-Standards, sondern stellt die Anwendung der Analyseverfahren auf den Untersuchungsdatensatz durch atomare Skripte und/oder Funktionen dar.\\
\bottomrule
\end{tabularx}
\caption{Anforderungen - Abgrenzungskriterien}
\label{tab:Abgr-Kriterien}
\end{table}
