\chapter{Implementierung}\label{Implementierung}

\section{Datenvorauswahl}\label{Datenvorauswahl}

Die in Kapitel \ref{datenanalyse} beschriebenen Kriterien zur Auswahl und Klassifizierung der Daten werden mithilfe von Skripten und Funktionen in Matlab umgesetzt. Im Folgenden wird auf die Umsetzung der einzelnen Kriterien eingegangen. Das Kriterium D6 bleibt dabei unberücksichtigt, da es sich lediglich auf das Weglassen einer Artefaktbereinigung bezieht.

\paragraph{Kriterium D1} Für die Erstellung der Hypnogramme (Skript restructure\_hypnogram.m) müssen zunächst die Startzeitpunkte aus der \acs{EDF}- sowie Ereignis-Datei jedes Patienten extrahiert werden. Nach dem Einlesen der EDF-Datei lässt sich der Startzeitpunkt der Aufnahme als String aus dem Header auslesen. Um den Startzeitpunkt der Ereignis-Datei zu ermitteln, wird die Ereignis-Datei eingelesen. Da in dieser Datei verschiedene Informationen gespeichert sind und die Spaltenanzahl daher erst ab der tabellarischen Auflistung der Schlafstadien, Zeitpunkte und Ereignisse gleich bleibt, wird mit einem String-Vergleich die Startzeile dieser Tabelle gesucht. Hierbei können die Ereignis-Dateien abweichende Spaltenzahlen aufweisen, da manche Dateien zusätzlich die Körperlage dokumentieren. Die Körperlage kann jedoch als ein zusammenhängendes Wort ("`POSITION-LEFT"' oder als mehrere Wörter ("`Körperlage unbekannt"') angegeben sein, so dass die Spaltenanzahl der Tabellenüberschriften sowie des Tabellenkörpers differieren können und ein ausschließlicher Abgleich der Tabellenüberschriften demnach nicht zweckdienlich ist (Abb. \ref{fig:spalten}). Aus diesem Grund erfolgt die Ermittlung des Spaltenindexes durch \texttt{ismember}-Vergleich zwischen der Spaltenüberschrift "`Schlafstadium"' und einem Doppelpunkt, welcher ausschließlich in den Strings der Zeitpunkte enthalten ist (Listing \ref{lst:D1}). Der Spaltenindex der Ereignisse (\texttt{ind\_ev)} entspricht sodann dem auf den Spaltenindex des Zeitpunkts folgenden Index.\\

\begin{figure}[H]
	\centering
	\includegraphics[width = \textwidth]{img/Ereignis_Spalten.png}
	\caption[Tabellenüberschriften in den Ereignis-Dateien]{Unterschiedliche Tabellenüberschriften in den Ereignis-Dateien;\\links (ACA\_69003) sechs Überschriften und fünf Spalten im Tabellenkörper;\\rechts (NKI\_000580) sieben Überschriften und sechs bzw. sieben Spalten im Tabellenkörper}
	\label{fig:spalten}
\end{figure}

\begin{lstlisting}[caption={Implementierung Kriterium D1 in Skript restructure\_hypnogram.m}, label={lst:D1}]
T = readtable(hypno_file);
% ...
for j = 1:row
    % extract content of one row and convert each element into one string
    C = table2cell(T(j, col));
    M = cell2mat(C);
    newrow = strsplit(M);
    % find row of appearing 'Schlafstadium'
    ss = strcmpi('Schlafstadium', newrow);
    nr_ss = find(ss, 1);
    if ~isempty(nr_ss)
        hypnostart = j+1;
        ind_ss = nr_ss;
    end
end
% extract starttime of hypnogram
for k = hypnostart:row
    % extract content of row k and convert each element into one string
    C = table2cell(T(k, col));
    M = cell2mat(C);
    newrow = strsplit(M);
    % find row containing ':' from time specification (like 20:25:10]
    tm = strfind(newrow, ':');
    ind_ev = find(not(cellfun('isempty', tm)))+1;
    if (any(ismember(newrow(ind_ss),sleep_stage_W)) && any(ismember(newrow(ind_ev),sleep_stage_W)))
        hypno_starttime = newrow(ind_ev-1);
        break;
    end
end
% ...
\end{lstlisting}

Anschließend kann der Startzeitpunkt extrahiert und mit dem Startzeitpunkt aus dem \acs{EDF}-Header verglichen werden. Zu diesem Zweck wird in der Funktion subtract\_timestrings.m die Differenz zwischen den beiden Zeitpunkten berechnet und als Anzahl von Sekunden zurückgegeben. Die Zeitdifferenz entspricht jedoch keiner ganzzahligen Anzahl an Epochen. Da Hypnogramme die Schlafstadien lediglich epochenweise, also in 30-Sekunden-Schritten, abbilden, würde ein Auffüllen der Hypnogramme um die fehlende Anzahl an Epochen nicht dazu führen, dass sich beide exakt übereinanderlegen ließen, so dass die Zuordnung der Schlafstadien nicht korrekt wäre. Aus diesem Grund müssen die zu untersuchenden Signale innerhalb der \acs{PSG} auf den Zeitraum der Hypnogramme begrenzt werden. Zu diesem Zweck wird eine Hypnogramm-Datei im TXT-Format angelegt (Hypnogram\_filename.txt) und in die erste Zeile die Zeitdifferenz in Sekunden geschrieben, um daraus bei der späteren \acs{TDS}-Analyse die Begrenzung der Signale zu berechnen. Anschließend werden aus der Ereignis-Datei die Schlafstadien ausgelesen. Hierbei muss ein Abgleich zwischen den Spalten "`Schlafstadium"' und "`Ereignis"' erfolgen (Abb. \ref{fig:zeitpunkte}), um lediglich die klassifizierten Schlafstadien zu erhalten, nicht jedoch dazwischen auftretende Ereignisse. Die ursprünglichen Bezeichnungen der Schlafstadien werden in numerische Werte geändert und ebenfalls in die Hypnogramm-Datei geschrieben (Tab \ref{tab:schlafstadien}). Dieser Vorgang erfolgt in einer Schleife für sämtliche Patienten des Untersuchungsdatensatzes. Mit dem Skript (extract\_num\_epochs.m) kann darüber hinaus für jeden Patienten die Anzahl der Epochen aus dem \acs{EDF}-Header sowie aus der Ereignis-Datei berechnet und in eine Ergebnis-Datei (Epoch\_length.txt) geschrieben werden. Auf diese Weise kann ermittelt werden, ob Unterschiede existieren und wie groß die Abweichungen sind.

\begin{table}[H] 
\centering
\begin{tabularx}{0.785\textwidth}{Xc}
\toprule
\multicolumn{1}{l}{\textbf{Schlafstadien in Ereignis-Datei}} & \multicolumn{1}{c}{\textbf{Geänderte Bezeichnung}}\\
\midrule 
Wach, SLEEP-S0 & 0\\
S1, SLEEP-S1 & 1\\
S2, SLEEP-S2 & 2\\
S3, SLEEP-S3 & 3\\
S4, SLEEP-S4 & 4\\
REM, SLEEP-REM & 5\\
\bottomrule
\end{tabularx}
\caption[Bezeichnung der Schlafstadien]{Bezeichnung der Schlafstadien in den Ereignis-Dateien und zugeordneter numerischer Wert zur Erstellung des Hypnogramms}
\label{tab:schlafstadien}
\end{table}
 
\paragraph{Kriterium D2} Um zu überprüfen, ob alle Patienten sämtliche für die \acs{TDS}-Analyse relevanten Schlafstadien durchlaufen haben, werden mit Hilfe des Skripts check\_sleep\_ stages.m die zuvor erzeugten Hypnogramm-Dateien untersucht. Hierzu werden die Daten aus den Hypnogramm-Dateien ausgelesen und mit einem \texttt{ismember}-Abgleich nach den entsprechenden Schlafstadien gesucht. Sofern nach dieser Überprüfung ein Schlafstadium bei einem Patienten nicht gefunden werden kann, wird der Dateiname (PatientenID) sowie eine Fehlermeldung (z. B. "`missing sleep stage: S4"') in eine Ergebnis-Datei (sleep\_stages.txt) geschrieben. Diese Ergebnis-Datei bildet die Grundlage für den Ausschluss von Daten aus dem Untersuchungsdatensatz.

\paragraph{Kriterium D3} Für die Suche nach zusammenhängenden Epochen innerhalb eines Schlafstadiums wird ebenfalls das Skript check\_sleep\_stages.m verwendet. Innerhalb einer for-Schleife wird ein Zähler hochgesetzt, solange ein Schlafstadium gleich bleibt. Existiert nach Ausführung der for-Schleife ein Schlafstadium, dessen Zähler nicht mindestens einmal den Wert 5 erreicht hat, so wird die untersuchte Datei (Patienten-ID) sowie das fehlende Schlafstadium in eine Ergebnis-Datei geschrieben (consecutive\_stages.txt). Auf Grundlage dieser Ergebnis-Datei können wiederum Daten von den weiteren Untersuchungen ausgeschlossen werden.

\paragraph{Kriterium D4} Zur Erkennung von Apnoe-Ereignissen wird das Skript count\_apnea.m verwendet. Dieses liest die Ereignis-Dateien aller Patienten ein und ermittelt durch einen \texttt{ismember}-Vergleich, ob eine Ereignis-Datei ein Apnoe-Ereignis aufweist. In diesem Fall wird ein Zähler hochgesetzt. Der Vergleich erfolgt auf Basis definierter Cell Strings, welche die Bezeichnungen der Schlafstadien sowie der Apnoe-Ereignisse enthalten (Listing \ref{lst:D4}). Anschließend wird eine Ergebnis-Datei (Apneas.txt) erstellt und der Dateiname (Patienten-ID) sowie die Anzahl der gefundenen Apnoe-Ereignisse darin gespeichert. Diese Ergebnis-Datei dokumentiert demnach die Anzahl aller gefundenen Apoe-Ereignisse für jeden Patienten.\\

\begin{lstlisting}[caption={Implementierung Kriterium D4 in Skript count\_apneas.m}, label={lst:D4}]
apnea = {'APNEA', 'HYPOPNEA', 'APNEA-CENTRAL', 'APNEA-MIXED', 'APNEA-OBSTRUCTIVE'};
sleep_stages = {'Wach', 'SLEEP-S0', 'S1', 'SLEEP-S1', 'S2', 'SLEEP-S2', 'S3', 'SLEEP-S3', 'S4', 'SLEEP-S4', 'REM', 'SLEEP-REM'};
% ...
if (any(ismember(newrow(ind_ss),sleep_stages)) && any(ismember(newrow(ind_ev),apnea)))
    count_apnea = count_apnea+1;
end
% ...
\end{lstlisting}

Um zu bestimmen, wie viele Apnoe-Ereignisse pro Stunde auftreten, müssen wiederum die Ereignis-Dateien untersucht werden. Zu diesem Zweck liest das Skript check\_apneas5.m die Ereignis-Dateien ein und ermittelt den Beginn der Tabelle sowie die Spaltenindizes der Schlafstadien und Ereignisse. In einer for-Schleife wird mit Hilfe von \texttt{ismember}-Vergleichen zeilenweise ein Apnoe-Ereignis gesucht, dessen Zeitpunkt als Cell String aus der Tabelle ausgelesen und der Schleifenindex zwischengespeichert (\texttt{apnearow}). In einer weiteren for-Schleife wird in identischer Weise nach einem weiteren Apnoe-Ereignis gesucht und der Zeitpunkt ebenfalls als Cell String gespeichert. Da Matlab Cell Strings jedoch nicht als Zeitpunkte erkennt, ermöglicht die Funktion subtract\_timestrings.m eine Umwandlung dieser Cell Strings in numerische Werte und berechnet die zeitliche Differenz in Sekunden (Listing \ref{lst:D4time}). Hierbei werden die Zeitpunkte mit Hilfe der Matlab-eigenen Funktion \texttt{duration} in vergangene Sekunden seit dem Zeitpunkt "`00:00:00"' umgerechnet. Darüber hinaus wird abgefragt, welcher Wert größer ist, so dass auch bei Tagänderung (nach 24:00 Uhr) die zeitliche Differenz korrekt errechnet wird. Anschließend wird geprüft, ob diese zeitliche Differenz größer als eine Stunde ist, und ein Flag zurückgegeben.\\

\begin{lstlisting}[caption={Implementierung Kriterium D4 in Funktion subtract\_timestrings.m}, label={lst:D4time}]
time = cell2mat(time);
hh_time = str2double(time(1:2));
mm_time = str2double(time(4:5));
ss_time = str2double(time(7:8));
% ...
% convert time and time2 into duration class variables with format seconds, so that D_time and D_time2 represent the amount of elapsed seconds from 00:00:00 until time and time2
D_time = duration(hh_time, mm_time, ss_time, 'Format', 's');
D_time2 = duration(hh_time2, mm_time2, ss_time2, 'Format', 's');
% amount of seconds of one day
daysecs = 86400;
% time2 ist bigger than time (= time2 is the same day)
if D_time <= D_time2
% convert D_time and D_time2 into double and calculate elapsed seconds
	durationsecs = seconds(D_time2-D_time);
    % check if elapsed time is not longer than one hour
    if durationsecs <= 3600
    	t = 1;
    else
        t = 0;
    end
% time is bigger than time2 (= time2 is the next day)
else
    % calculate the rest of the first day from time and add the elapsed time of time2
    durationsecs = seconds(daysecs - seconds(D_time));
    durationsecs = seconds(durationsecs + D_time2);
    % check if elapsed time is not longer than one hour
    if durationsecs <= 3600
        t = 1;
    else
        t = 0;
    end
end
\end{lstlisting}

Ist die zeitliche Differenz nicht größer als eine Stunde, wird ein Apnoe-Zähler hochgesetzt. Dies bedeutet demnach, dass das erste sowie das zweite erkannte Apnoe-Ereignis innerhalb einer Stunde liegen. Sobald dieser Zähler größer als der Wert 5 ist, erhöht sich ein Signifikanz-Zähler und der Index der äußeren Schleife (\texttt{apnearow}) wird um den Wert 1 erhöht. Anschließend wird die innere Schleife abgebrochen und erneut in die äußere gesprungen, so dass die Überprüfung, ob Apnoe-Ereignisse innerhalb einer Stunde liegen, erneut eine Zeile nach dem ersten erkannten Apnoe-Ereignis startet (Listing \ref{lst:D4apnea5}). Auf diese Weise wird sichergestellt, dass von jedem Zeitpunkt eines Apnoe-Ereignisses aus der relevante Zeitraum von einer Stunde auf weitere Apnoe-Ereignisse untersucht wird. Ist der Signifikanz-Zähler nach Ausführung der Schleifen größer als Null, wurden demnach Apnoe-Ereignisse mit einer Häufigkeit von mehr als fünf pro Stunde erkannt. Anschließend werden solche Dateinamen (Patienten-ID), bei denen Apnoe-Ereignisse mit einer Häufigkeit von mehr als fünf pro Stunde erkannt worden sind, gemeinsam mit dem Wert des Signifikanz-Zählers in eine Ergebnis-Datei (Apnea5perhour.txt) geschrieben.\\

\begin{lstlisting}[caption={Implementierung Kriterium D4 in Skript check\_apneas5.m}, label={lst:D4apnea5}]
% ...
if (any(ismember(Mrow(ind_ss),sleep_stages)) && any(ismember(Mrow(ind_ev),apnea)))
    time2 = Mrow(ind_ev-1);
    % calculate elapsed time between both time stamps
    [t, durationsecs] = subtract_timestrings(time, time2);
    % count apnea if elapsed time is <= one hour
    if t == 1
        countapnea = countapnea+1;
    else
    % count significance if more than five apneas per hour
        if countapnea > 5
            significance = significance + 1;
            countapnea = 0;
        end
        % reset k to next row after first apnea of the investigated hour
        k = apnearow+1;
        % end this loop
        l = row;
    end
end
% ...
\end{lstlisting}

\paragraph{Kriterium D5} Zum Zwecke der Klassifizierung der Daten in die Gruppen "`alle"', "`Herz"', "`Atmung"', "`Schlafmittel"' sowie "`Insomnie"' erfolgt eine visuelle Auswertung der Excel-Datei, welche die Patienteninformationen beinhaltet, sowie der Ergebnis-Dateien sleep\_stages.txt, consecutive\_stages.txt und Apnea5perhour.txt. Den ersten beiden Ergebnis-Dateien können die Patienten-IDs entnommen werden, welche die einzelnen Schlafstadien nicht alle oder nicht in ausreichender Länge durchlaufen haben. Aus der Ergebnis-Datei über die Apnoen ergeben sich die Patienten-IDs, deren Hypnogramme mehr als fünf Apnoe-Ereignisse pro Stunde aufweisen. Nach Recherche der medizinischen Fachbegriffe, welche in den Begleiterkrankungen und -medikationen in der Excel-Datei dokumentiert worden sind, lassen sich in einer zusätzlichen Spalte die klassifizierten Gruppen nach dem Kriterium D5 eintragen. Anschließend erstellt das Skript create\_insomdata\_struct.m ein Struct (\texttt{insomdata}) mit insgesamt 13 Key-Value-Paaren. Die Keys stellen Fields dar, welche sich auf die Inhalte der Excel-Datei sowie einige Zusatzinformationen beziehen. Die dazugehörigen Values werden anschließend für jeden Patienten aus der Excel-Datei ausgelesen und in das Struct geschrieben. Dieses Struct wird zur weiteren Verwendung in Matlab als .mat-Datei (insomdata.mat) gespeichert. Daraus können sodann zu jedem Patienten die Namen der \acs{EDF}-, Ereignis- und Hypnogramm-Dateien, das Geschlecht, Alter und Geburtstdatum, die Größe, das Gewicht, der \acs{BMI}, das Datum der \acs{PSG}-Aufzeichnung, Begleiterkrankungen- und medikationen sowie die klassifizierte Gruppe entnommen werden (Abb. \ref{fig:struct}). 

\begin{figure}[H]
	\centering
	\includegraphics[width = 0.6\textwidth]{img/struct.png}
	\caption[Struct \texttt{insomdata} mit Beispieldaten]{Struct \texttt{insomdata} mit Beispieldaten des Patienten ACA\_69003}
	\label{fig:struct}
\end{figure}

Ein weiteres Skript (extract\_goup\_information.m) ermöglicht das Laden des Structs und das Extrahieren gruppenspezifischer Informationen. Durch Verwendung der Funktion \texttt{arrayfun} in Matlab kann eine darin aufgerufene Funktion auf das gesamte Struct angewendet werden, ohne zeitaufwendige Schleifen zu durchlaufen. Dies ermöglicht beispielsweise den Vergleich aller Elemente (Values) eines Fields mit einem String (z. B. 'heart', um die Indizes der Gruppe Herz zu erhalten) und die Rückgabe der entsprechenden Indizes. Anhand dieser Indizes können sodann weitere Gruppeninformationen, wie Alter oder Geschlecht, aus dem Struct extrahiert und Minima, Maxima und Mittelwerte berechnet werden (Listing \ref{lst:D5}).\\

\begin{lstlisting}[caption={Implementierung Kriterium D5 in Skript extract\_goup\_information.m}, label={lst:D5}]
% ...
load(fullfile(insomstructpath,'insomdata.mat'));
% find indices of group heart
heart = arrayfun(@(x) strfind(x.group, 'heart'), insomdata, 'UniformOutput', 0);
heart_ind = find(~cellfun('isempty', heart)); % faster than find(~cellfun(@isempty, insomgroupheart))
% ...
heart_w = arrayfun(@(x) strcmp(x.sex, 'w'), insomdata(heart_ind));
heart_w_age = arrayfun(@(x) x.age, insomdata(heart_w)); % all ages of w heart
heart_w_meanage = mean(heart_w_age); % mean age of w heart
% ...
\end{lstlisting}

\section{TDS-Paket}\label{tdsPaket}

Für das Verfahren der TDS (Kapitel \ref{TDS}) liegt bereits ein von Prof. Dr. Dagmar Krefting in Matlab implementiertes Funktionspaket vor. Auf dieses wird im Folgenden zum Zwecke einer Übersicht grob eingegangen, um darauf aufbauend die für diese Arbeit notwendigen Anpassungen und Erweiterungen (Kapitel \ref{erweiterungTDS}) der in dem Paket enthaltenen Funktionen zu erläutern.\\

Im Wesentlichen müssen für die Ausführung der \acs{TDS}-Analyse drei Funktionen des Paketes ausgeführt werden, welche der Berechnung der \acs{TDS} sowie zur Erzeugung der Ergebnismatrizen für jedes Schlafstadium dienen (im Folgenden Hauptfunktionen genannt). Darüber hinaus besteht das Paket aus diversen weiteren Funktionen, welche intern innerhalb der Hauptfunktionen zur Ausführung von Zwischenschritten aufgerufen werden.\\

\paragraph{Ermittlung von Stabilitäten} Die erste der drei Hauptfunktionen (sn\_TDS.m) dient der Berechnung der Kreuzkorrelationskoeffizienten und Verschiebungen (Time Delays) sowie der Ermittlung von Stabilitäten. Zu diesem Zweck werden zu einer übergebenen \acs{PSG} im \acs{EDF}-Format gemäß Kapitel \ref{datenvorverarbeitung} die Zeitreihen der relevanten Signale extrahiert. Dies ist bislang zum einen möglich für \acs{PSG}s, welche die gleiche Standardsortierung der Signale aufweisen, wie die Daten aus der SIESTA-Studie. Hierbei werden die 14 in Kapitel \ref{TDS} beschriebenen und von Krefting et al. verwendeten Signale genutzt. Zum anderen ist die Funktion gültig für \acs{PSG}s, welche den Standardbezeichnungen von Signalen im Alice6-Format folgen, wobei insgesamt 17 Signale (zusätzlich Atem- und \acs{EMG}-Signale) verwendet werden. Anschließend werden sämtliche extrahierte Zeitreihen in der Matrix \texttt{biosignals\_tds} zusammengefasst. Die Zeilen dieser Matrix entsprechen dem zeitlichen Verlauf der Signale in Sekundenschritten. Die Spalten bilden die einzelnen Zeitreihen ab. Anhand dieser Matrix wird sodann die Kreuzkorrelation mit einer Fensterlänge von 60 Sekunden sowie einer Verschiebung von 30 Sekunden durchgeführt, wobei die Zeitreihen auf einen Mittelwert von 0 und eine Standardabweichung von 1 normalisiert werden. Die zurückgegebene Matrizen der Kreuzkorrelationskoeffizienten sowie der Verschiebungen entsprechen jeweils in der Zeilenanzahl dem zeitlichen Verlauf der Signale in Sekunden und in der Spaltenanzahl dem Produkt aus dem Vergleich sämtlicher Zeitreihen miteinander, so dass 

\begin{equation}
columns = (Anzahl~der~Signale)^2
	\label{eq:columns}
\end{equation}

gilt. Die Kreuzkorrelationsmatrix \texttt{xcc} enthält sämtliche Kreuzkorrelationskoeffizienten. Die Matrix \texttt{xcl} enthält sämtliche Werte der Verschiebungen (Time Delays). Zur Berechnung der Stabilitäten werden anschließend in Fenstern mit einer Länge von fünf Sekunden bei einer Verschiebung von einer Sekunde Minimum und Maximum der Verschiebungen aus der Matrix \texttt{xcl} innerhalb des Fensters berechnet. Überschreitet die Differenz dieser beiden Werte nicht den Wert 2, so gelten diese Verschiebungen als stabil. Sämtliche stabilen Verbindungen werden in einer Matrix \texttt{tds} mit dem Wert 1 gespeichert, sämtliche nicht stabilen Verbindungen mit dem Wert 0. Im Ergebnis weist die Matrix \texttt{tds} die gleichen Dimensionen wie die beiden vorherigen Matrizen auf und enthält demnach binäre Werte als Merkmal der Stabilität.

\paragraph{Aufteilung in Schlafstadien} Die zweite Hauptfunktion (sn\_getStagesTDS.m) ermittelt auf Grundlage des Hypnogramms die Verteilung auf die einzelnen Schlafstadien. Hierzu werden zunächst die Schlafstadien \acs{NREM}3 und \acs{NREM}4 gemäß Bashan et al. zusammengefasst, indem sämtliche Schlafstadien mit dem Wert 4 im Hypnogramm auf den Wert 3 gesetzt werden. Das Schlafstadium \acs{NREM}1 wird nicht berücksichtigt. Die Variable \texttt{nsis} enthält die Anzahl der Epochen für jedes Schlafstadium. Anhand der identifizierten Schlafstadien wird anschließend eine dreidimensionale Ergebnis-Matrix \texttt{result} erzeugt, welche die prozentualen Anteile der stabilen Verbindungen pro Schlafstadium enthalten. Diese Matrix entspricht in erster und zweiter Dimension den untersuchten Signalen. Die dritte Dimension steht für die vier Schlafstadien Tiefschlaf, Leichtschlaf, \acs{REM}-Schlaf und Wachzustand. 

\paragraph{Anzeigen der Ergebnisse} Mit Hilfe der dritten Hauptfunktion des \acs{TDS}-Paketes (sn\_plotTDSMatrixSiestaAll.m) lassen sich die Ergebnisse der \acs{TDS}-Analyse plotten. Hierzu wird für jedes der vier Schalfstadien durch die Matlab-Funktion \texttt{imagesc} ein Plot erzeugt. Die prozentualen Anteile werden farbig abgebildet. Dabei wird eine standardisierte Colormap verwendet, so dass die Ergebnis-Grafiken miteinander vergleichbar sind. Dunkelblaue Bereiche stellen eine sehr schwache und sehr helle Bereiche eine besonders starke Verbindungsstärke der Systeme dar. Diese Funktion bezieht sich auf den von Krefting et al. untersuchten SIESTA-Datensatz und muss für den hiesigen Untersuchungsdatensatz angepasst werden.


\section{Erweiterungen am TDS-Paket}\label{erweiterungTDS}

Um das \acs{TDS}-Verfahren auf den hiesigen Untersuchungsdatensatz anwenden zu können, müssen einige Erweiterungen des \acs{TDS}-Paketes implementiert werden. Auf diese sowie auf die Algorithmen zur Anwendung wird im Folgenden näher eingegangen.\\

Wie bereits in Kapitel \ref{datenanalyse} beschrieben unterscheiden sich die \acs{PSG}-Aufzeichnungen und die dazugehörigen Hypnogramme des hiesigen Untersuchungsdatensatzes in ihrer Länge. Um die Funktion zur Ermittlung von Stabilitäten in dieser Arbeit anwenden zu können, wird die Funktion sn\_TDS.m um die Option "`Charit\'{e}"' erweitert. Wird diese Option bei Aufruf der Funktion gesetzt, so erfolgt an den entsprechenden Stellen eine von der ursprünglichen Funktion abweichende Funktionsausführung. Diese Stellen beziehen sich zum einen auf die initialen Zuweisungen der Signalkanäle und zum anderen auf sämtliche Aufrufe der einzelnen Signale innerhalb der \acs{PSG} zur Extraktion der entsprechenden Zeitreihen.\\

Zunächst muss die zeitliche Differenz (in Sekunden) der Startzeitpunkte aus der zuvor neu erzeugten Hypnogramm-Datei extrahiert und anschließend durch Multiplikation mit der Abtastfrequenz der einzelnen Signale in Samples umgerechnet werden. Diese Ergebnisse werden in der Variablen \texttt{durationsamples} für alle Signale gespeichert. Als nächstes wird die Länge des Hypnogramms aus der Hypnogramm-Datei ausgelesen, in Sekunden umgerechnet und für jedes einzelne Signal mit dessen Abtastfrequenz multipliziert, um die Sampleanzahl des Hypnogramms für jedes Signal in Abhängigkeit von dessen Abtastfrequenz zu erhalten. Das Ergebnis wird ebenfalls für alle Signale in einer Variablen \texttt{hypnogramduration} gespeichert. Zuletzt wird die Länge jedes Signals ausgelesen und ebenfalls für sämtliche in der \acs{PSG} enthaltenen Signale in der Variablen \texttt{signallengths} gespeichert. Durch Abgleich dieser drei Variablen wird anschließend für jedes Signal die Bedingung geprüft, ob die Hypnogrammlänge addiert mit der Zeitdifferenz zwischen den Startzeitpunkten kleiner ist als die Signallänge. Das Ergebnis wird für jedes überprüfte Signal in die Variable \texttt{lengthflag} geschrieben. Ist die Bedingung erfüllt, so enthält \texttt{lengthflag} für das überprüfte Signal den Wert 1, anderenfalls den Wert 0 (Listing \ref{lst:snTDS}).\\

\begin{lstlisting}[caption={Anpassung für den Charit\'{e}-Datensatz in der Funktion sn\_TDS.m}, label={lst:snTDS}]
% ...
elseif(strcmp(ch_all,'charite'))
    disp('all channels following Charite standard are processed')
    % hypnogram file exists
    if hypno_flag
        % read hypnogram
        T = dlmread(hypno_filename);
        %starttime difference in seconds
        durationseconds = T(1); 
        %starttime difference in samples for each signal
        durationsamples = [signalheader(:).samples_in_record]/...
            header.data_record_duration * durationseconds;
        %hypnogram duration in samples for each signal
        hypnogramduration = length(T(2:end))*30*sfch; 
        %lengths of all signals
        signallengths = cellfun('length',signalcells);
        %boolean array - 1 if hypnogram+starttime difference is smaller
        %than signal length 
        lengthflag = (durationsamples(:)+1+hypnogramduration(:) <= signallengths(:));
    else
        disp('Hypnogram file is missing, please select hypnogram text file.')
    % ...
    end
    %channelnumbers of EEG signals
    ch_eeg = 2; %EEG C4-A1
    ch_eeg2 = 3; %EEG O2-A1
    ch_eeg3 = 5; %EEG C3-A2
    ch_eeg4 = 6; %EEG O1-A2
% ...
\end{lstlisting}

Da im hiesigen Untersuchungsdatensatz keine frontalen EEG-Signale enthalten sind und demnach weniger Signale als im SIESTA-Datensatz existieren, muss die Funktion zur Ermittlung von Stabilitäten für die hier verwendeten Daten angepasst werden. Ist die Option "`Charit\'{e}"' gesetzt, so werden die Kanäle gemäß Tab. \ref{tab:labels} den Signalen zugeordnet (Listing \ref{lst:snTDS}). Das zusätzliche \acs{EMG} des rechten Beins wird hierbei außer Acht gelassen, um die Vergleichbarkeit mit den Ergebnissen von Krefting et al. zu gewährleisten. Zur Erstellung der Zeitreihen wird sodann vor jeder Verwendung eines Signals anhand des \texttt{lengthflag}  die Bedingung geprüft, ob ab dem Startzeitpunkt des Hypnogramms das Hypnogramm kürzer ist als das Signal (Listing \ref{lst:zeitreihen}). Ist die Bedingung erfüllt, bedeutet dies, dass die \acs{PSG}-Aufzeichnung zu einem späteren Zeitpunkt endet als das Hypnogramm, so dass die zu verwendenden Ausschnitte der Signale in dem Bereich des Hypnogramms (Hypnogrammstart bis Hypnogrammende) liegen müssen. Die Signale werden in diesem Fall für die Berechnung der Zeitreihen demnach am Anfang um die Länge der zeitlichen Differenz und am Ende um die Zeitspanne, die die Signale die Länge des Hypnogramms überschreiten, gekürzt. In diesem Fall enthält \texttt{lengthflag} für das überprüfte Signal den Wert 1. Ist die Bedingung nicht erfüllt, bedeutet dies, dass der Endzeitpunkt des Hypnogramms zeitlich hinter dem der \acs{PSG} liegt, so dass das Signal in dem Bereich ab dem Startzeitpunkt des Hypnogramms bis zum Signalende verwendet wird.\\

\begin{lstlisting}[caption={Berechnung der Zeitreihen für den Charit\'{e}-Datensatz in der Funktion sn\_TDS.m}, label={lst:zeitreihen}]
% ...
if strcmp(ch_all, 'charite')
    % hypnogram ends before signal ends - take signal from hypnogram starttime until hypnogram endtime
    if lengthflag(ch_emgchin)
        var_emgchin = sn_getVariance(signalcells{ch_emgchin}...
            (durationsamples(ch_emgchin)+1:durationsamples(ch_emgchin)...
            +hypnogramduration(ch_emgchin)), 'wl', wl_sfe, 'ws',...
            ws_sfe, 'sf',sfch(ch_emgchin));
    % signal ends before hypnogram ends - take signal from hypnogram starttime until signal endtime    
    else
        var_emgchin = sn_getVariance(signalcells{ch_emgchin}...
            (durationsamples(ch_emgchin)+1:end), 'wl', wl_sfe,...
            'ws', ws_sfe, 'sf', sfch(ch_emgchin));
    end
% ...
\end{lstlisting}

Um nach Aufteilung der ermittelten Stabilitäten auf die einzelnen Schlafstadien die Ergebnis-Matrizen plotten zu können, muss ebenfalls eine Anpassung an den Charit\'{e}-Datensatz erfolgen. Hierzu wird die Funktion sn\_plotTDSMatrixCharite.m erstellt, welche sich an der Funktion sn\_plotTDSMatrixSiestaAll.m orientiert und den Plot der \acs{TDS}-Ergebnisse für die Charit\'{e}-Daten durch Anpassung an das hiesige Format ermöglicht.

\section{Anwendung auf den Charit\'{e}-Datensatz}

Zur Anwendung des \acs{TDS}-Paketes wird ein Skript (apply\_TDS.m) erstellt, welches die Ausführung der Funktionen auf den gesamten Untersuchungsdatensatz erlaubt. Zu diesem Zweck werden zunächst sämtliche \acs{EDF}-Dateien und Hypnogramme gelistet und anschließend in einer for-Schleife an die \acs{TDS}-Funktionen übergeben. Bei dem Aufruf der Funktion sn\_TDS.m müssen sodann die \acs{EDF}-Datei sowie die Hypnogramm-Datei übergeben werden. Zusätzlich wird die Option "`Charit\'{e} gesetzt. Die Ergebnis-Matrizen \texttt{tds}, \texttt{xcc}, \texttt{xcl} sowie \texttt{biosignals\_tds} werden in einem Ergebnispfad gespeichert. Anschließend wird die Hypnogramm-Datei ausgelesen und als Array gemeinsam mit der Matrix \texttt{tds} an die Funktion sn\_getStagesTDS.m übergeben. Die Ergebnisse werden ebenfalls im Ergebnispfad als Matrizen gespeichert. Für die Aufteilung der Ergebnisse in die einzelnen Schlafstadien könnten in dieser Funktion Artefakte aus den Berechnungen ausgeschlossen werden. Um das Kriterium D6 der Datenanalyse (Kapitel \ref{datenanalyse}) zu erfüllen, kommt diese Option hier ausdrücklich nicht zum Einsatz. Zuletzt erfolgt das Plotten der \acs{TDS}-Ergebnisse.\\

Um die Ergebnisdaten strukturiert zusammenzufassen und eine Ergebnisanalyse für andere Nutzer zu ermöglichen, wird mit Hilfe des Skripts create\_insomresults\_struct.m ein Struct (\texttt{insomresults}) erzeugt, welches zu jedem Patientendatensatz die \acs{TDS}-Ergebnisse enthält. Dieses Struct enthält demnach für jeden Patienten die Felder Patienten-ID ("`name"'), Hypnogramm ("`hypnogram"'), Zeitreihen ("`biosignals"'), Time Delays ("`xcl"'), Kreuzkorrelationskoeffizienten ("`xcc"'), stabile Verbindungen ("`tds"'), Anzahl der Epochen pro Schlafstadium ("`nsis"') sowie die prozentualen Verbindungsstärken in jedem Schlafstadium ("`result"') (Abb. \ref{fig:resultstruct}).

\begin{figure}[H]
	\centering
	\includegraphics[width = 0.4\textwidth]{img/insomresults_struct.png}
	\caption[Struct \texttt{insomresults} mit Beispieldaten]{Struct \texttt{insomresults} mit Beispieldaten des Patienten ACA\_69003}
	\label{fig:resultstruct}
\end{figure}

Zum Plotten der Ergebnis-Matrizen der einzelnen Gruppen werden gemäß Kapitel \ref{mean} mit Hilfe der Funktion calculate\_mean.m die Mittelwerte berechnet und die gemittelten Ergebnis-Matrizen für jedes Schalfstadium visualisiert.

\section{Berechnung der Mittelwerte}\label{mean}

Zur Berechnung der Mittelwerte \texttt{smean} der \acs{TDS}-Matrizen verschiedener Gruppen dient die Funktion calculate\_mean.m. Dieser Funktion muss lediglich die entsprechende Gruppenbezeichnung übergeben werden. Anschließend wird das Struct \texttt{insomdata} geladen und es werden die Indizes der entsprechenden Gruppe berechnet. Dies ist möglich für die Gruppen gemäß der Klassifizierung innerhalb der Datenanalyse sowie für junge (jünger als 34 Jahre), ältere (älter als 60 Jahre), weibliche und männliche Insomniepatienten. Anschließend werden aus dem Struct \texttt{insomresults} die Anzahl der Epochen pro Schlafstadium \texttt{nsis} sowie die Ergebnis-Matrizen \texttt{result} jedes Patienten der gewählten Gruppe extrahiert. Die Berechnung der Mittelwerte erfolgt mit Hilfe der Matlab-Funktion \texttt{mean}. Für den Fall, dass bei der \acs{TDS}-Analyse NaN-Elemente (Not a Number) auftreten, wird bei der Matlab-Funktion \texttt{mean} zur Berechnung des Mittelwerts die Option "`omitnan"' gesetzt, so dass NaN-Elemente nicht berücksichtigt werden. Im Ergebnis wird ein Vektor, welcher die gemittelte Anzahl der Epochen pro Schlafstadium enthält, sowie eine dreidimensionale Matrix der gemittelten Verbindungsstärken pro Schlafstadium zurückgegeben. Zuletzt werden die gemittelten Verbindungsstärken der vier Schlafstadien geplottet.\\

Zur Berechnung des globalen Mittelwertes \texttt{nmean} dient die Funktion calculate\_nmean\_ age.m. Zum Zwecke der späteren Gegenüberstellung der globalen Verbindungsstärke \texttt{nmean} und des Alters der Insomniepatienten wird in dieser Funktion darüber hinaus das Alter der Patienten extrahiert. Zunächst werden die Structs \texttt{insomdata} und \texttt{insomresults} geladen und die Indizes der entsprechenden Gruppe mit Hilfe der Matlab-Funktion \texttt{arrayfun} und durch Abgleich mit dem Struct \texttt{insomdata} ermittelt. Anschließend wird die Rückgabe-Matrix \texttt{nmean\_age} alloziert, welche später für sämtliche Patienten der Gruppe die globalen Mittelwerte der Verbindungsstärken \texttt{nmean} sowie das Alter enthält. In einer for-Schleife über die gesamte Gruppe werden zunächst die Patienten-ID sowie die Ergebnis-Matrix jedes Patienten aus dem Struct \texttt{insomresults} extrahiert. Durch Abgleich der Patienten-ID aus dem Struct \texttt{insomdata} wird darüber hinaus das jeweilige Alter extrahiert. Für jedes Schlafstadium werden anschließend in zwei for-Schleifen die Zeilen und Spalten der Ergebnis-Matrix in der Form durchlaufen, dass lediglich das untere Dreieck der Matrix und ohne die Diagonale betrachtet wird. Für diese Positionen werden die Verbindungsstärken addiert und gemäß Kapitel \ref{calcmean} durch die Anzahl der addierten Indizes abzüglich 14 dividiert (Listing \ref{lst:nmean}). Dieses Ergebnis sowie das Alter werden für jeden Patienten und für jedes Schlafstadium in die Rückgabe-Matrix \texttt{nmean\_age} geschrieben und diese als .mat-Datei gespeichert.\\

\begin{lstlisting}[caption={Berechnung des globalen Mittelwertes in der Funktion calculate\_nmean\_age.m}, label={lst:nmean}]
% ...
for s = 1:4
    tdssum = 0;
    % loop over result matrix in x and y direction
    for k = 1:length(tdsmat)
        for l = k+1:length(tdsmat)
            % sum link strengths for each sleep stage
            tdssum = tdssum+tdsmat(k, l, s);
        end
    end
    % calculate global mean link strength and age for each
    % patient and sleep stage
    nmean_age(i, 1, s) = tdssum/((length(tdsmat)^2)/2 - length(tdsmat));
    nmean_age(i, 2, s) = patage;
end
% ...
\end{lstlisting}

\section{Lineare Regression}

Zur Untersuchung des Zusammenhangs zwischen dem Alter und den globalen Verbindungsstärken der Patienten, wird die in Matlab erstellte Matrix \texttt{nmean\_age.mat} verwendet. Hierzu dient das R-Skript linreg\_nmean\_age.R. Zunächst wird das Matlab-Paket importiert, wodurch das Laden von Matlab-Variablen in R mit Hilfe der Funktion \texttt{readMat} ermöglicht wird. Die Variable nmean\_age.mat wird in R als eine Liste mit einer Länge von 512 Elementen angelegt, so dass die Elemente in R aneinandergereiht werden. Um auf die einzelnen Elemente dieser Liste (Verbindungsstärke und Alter jedes Patienten pro Schlafstadium) zugreifen zu können, wird die Liste durch die Funktion \texttt{unlist} in ein dreidimensionales Array vom Typ \texttt{double} und einer Länge von jeweils 64 Elementen umgewandelt. Dies entspricht dem Format, wie die Variable in Matlab erzeugt worden ist. Anschließend können die globalen Verbindungsstärken sowie das Alter der Patienten für jedes Schlafstadium als numerische Vektoren vom Typ \texttt{double} extrahiert werden. Aus diesen Vektoren wird für jedes Schlafstadium ein Data Frame erstellt. Aus diesen werden sodann für jedes Schlafstadium und für die weiblichen, für die männlichen sowie für alle Patienten lineare Modelle erstellt. Hierbei werden die globalen Verbindungsstärken als abhängige Variable und das Alter als unabhängige Variable gesetzt. Anschließend wird ein Streudiagramm mit unterschiedlicher Darstellung der weiblichen und männlichen Patienten erzeugt. Mit Hilfe der R-Funktion \texttt{abline} sowie der linearen Modelle können sodann die Regressionsgeraden eingezeichnet werden.\\

\lstset{language=R, otherkeywords={}, keywords={data,frame,lm,points,abline,pch,col,lty}}
\begin{lstlisting}[caption={Erstellung eines Streudiagramms und Einzeichnung der Regressionslinien in dem Skript linreg\_nmean\_age.R}, label={lst:nmean_age}]
# ...
# create data frames
insomdata_all_ds <- data.frame(age_all, nmean_all_ds)
# ...
# create linear models
insomdata_all_ls.linm <- lm(nmean_all_ls~age_all,data=insomdata_all_ls)
# ...
# plot female pairs in light sleep
points(insomdata_f_ls,pch=19)
# plot male pairs in light sleep
points(insomdata_m_ls, pch=21)
# add a line for linear regression for all, female and male 
abline(insomdata_all_ls.linm, col="green",lty=1)
abline(insomdata_f_ls.linm, col="red",lty=2)
abline(insomdata_m_ls.linm, col="blue",lty=3)
# ...
\end{lstlisting}

Mit Hilfe der R-Funktion \texttt{summary} können darüber hinaus die Parameter der linearen Regression berechnet und ausgegeben werden. Die R-Funktion \texttt{shapiro.test} dient der Überprüfung auf Normalität. Zu diesem Zweck wird eine Stichprobe von jeweils 20 Elementen erstellt. Die Funktion liefert die W-Statistik sowie den p-Wert zurück. Die Korrelationsverfahren nach Pearson und nach Spearman werden durch die R-Funktion \texttt{cor.test} mit den Optionen \texttt{method="pearson"} bzw. \texttt{method="spearman"} implementiert. Diese geben die Korrelationkoeffizienten sowie den p-Wert zurück.

%p-Wert nach Pearson entspricht dem p-Wert der linearen Regression

\section{Statistisches Matching}

Zum statistischen Matching dienen zum einen die Structs \texttt{insomdata} und \texttt{insomresults} über die Insomniepatienten und zum anderen die ebenfalls in Structs gespeicherten Informationen über die Teilnehmer der SIESTA-Studie. In dem Struct \texttt{siesta} werden allgemeine Informationen, wie Alter und Geschlecht sowie Status (gesund oder Art der Erkrankung) gehalten. In dem Struct \texttt{bothnights} sind die berechneten \acs{TDS}-Matrizen der gesunden Probanden enthalten. Die Empfängergruppe entspricht bereits einer zufällig sortierten Liste und wird gemäß Smith mit aufsteigendem Index durchlaufen. Die Reihenfolge in der Spendergruppe entspricht den teilnehmenden Schlaflaboren. Anders als bei dem Verfahren von Smith wird die Spendergruppe zuvor nicht zufällig sortiert, um eine durch Neusortierung des Structs \texttt{bothnights} mögliche Fehlerquelle auszuschließen. Stattdessen wird bei jedem gefundenen Zwilling die Spendergruppe für den nächsten Empfängerdatensatz nicht erneut von vorn durchlaufen, sondern ab dem auf den zuletzt gefunden Zwilling folgenden Index. Auf diese Weise wird garantiert, dass nicht zuerst sämtliche Spenderdatensätze des ersten Schlaflabors, anschließend des zweiten etc. überprüft werden, sondern die Spendergruppe fortlaufend überprüft wird. Ist die Spendergruppe vollständig durchlaufen, so wird der Suchindex zurück auf 1 gesetzt und die Suche wiederholt sich in gleicher Weise. Da in dem Struct \texttt{bothnights} für die meisten Probanden \acs{PSG}s von zwei Nächten enthalten sind, werden durch Abgleich der Probanden-ID alle zu dem Probanden gehörigen \acs{PSG}s aus der Spendergruppe gestrichen, so dass jeder gesunde Proband maximal einmal gematcht wird (Listing \ref{lst:matching}). Für vier Insomniepatienten konnten auf diese Weise keine statistischen Zwillinge gefunden werden. Für diese wurden vier gesunde Probanden mit gleichen Merkmalen ausgewählt, welche bereits für vier andere Insomniepatienten einen statistischen Zwilling darstellen.\\

\lstset{language=Matlab}
\begin{lstlisting}[caption={Statistisches Matching in Skript ma\_statistical\_matching.m}, label={lst:matching}]
% ...
% check for match of gender and age +-1
if controls(j) ~= 0 && age_siesta>age_insom-2 && ...
    age_siesta<age_insom+2 && strcmp(gender_siesta,gender_insom)
    % check if controls is still available and no match yet
    if controls(j) == 1 && matches(i) == 0
        matches(i) = j;
        % delete bothnights index of match from controls
        controls(j) = 0;
        % check for other psg of this control patient
        for l = 1:leng_bothnights
            pat2delete = bothnights(l).edffilename;
            % delete all psgs from this control patient
            if strcmp(pat2delete(1:end-6), patid_siesta_bothnights(1:end-6))
                controls(l) = 0;
%...
\end{lstlisting}



