\newgeometry{left= 3 cm,right = 2.5 cm, bottom = 2.5 cm, top = 6 cm}

% in Inhaltsverzeichnis aufnehmen
%\addcontentsline{toc}{chapter}{Anhang}
%\sectionmark{Anhang}
\lhead{}


%\chapter{}%alles gut, aber im Anhang steht nur A.
%\chapter{Anhang}%überall steht A....
%\section*{\Huge{Anhang}}
%\Huge{Anhang}%Verzeichnis okay, aber bei Abb fehlt A.
\addchap{Anhang}%macht alles richtig, aber leere Seite vorher
\refstepcounter{chapter}

\begin{table}[H] 
\centering
\begin{tabularx}{\textwidth}{cX}
\toprule
\multicolumn{2}{c}{\textbf{Allgemeine Insomniekriterien}}\\
\midrule 
a) & Eine Beschwerde über Einschlafschwierigkeiten, Durchschlafprobleme, frühmorgendliches Erwachen oder Schlaf von chronisch nicht erholsamer oder schlechter Qualität. Bei Kindern wird die Schlafschwierigkeit zumeist durch die Erziehungsperson bemerkt und kann darin bestehen, dass die Kinder nicht zu Bett gehen wollen oder nicht unabhängig (d. h. im eigenen Bett) von ihren Eltern schlafen können.\\
b) & Die genannte Schlafschwierigkeit tritt auf, obwohl adäquate Möglichkeiten und Umstände dafür vorhanden sind, genügend Schlaf zu bekommen.\\
c) & Zumindest eine der folgenden Formen von Beeinträchtigungen der Tagesbefindlichkeit/Leistung, die auf die nächtliche Schlafschwierigkeit zurückgeführt werden kann,wird vom Patienten berichtet:
\begin{itemize}
\singlespacing
\setlength\itemsep{0em}
\item Müdigkeit (Fatigue) oder Krankheitsgefühl
\item Beeinträchtigung der Aufmerksamkeit, Konzentration oder des Gedächtnisses
\item Soziale oder berufliche Einschränkungen oder schlechte Schulleistungen
\item Irritabilität oder Beeinträchtigungen der Stimmung (z. B. Gereiztheit)
\item Tagesschläfrigkeit
\item Reduktion von Motivation, Energie oder Initiative
\item Erhöhte Anfälligkeit für Fehler, Arbeitsunfälle oder Unfälle beim Führen eines Kraftfahrzeugs
\item Spannungsgefühle, Kopfschmerzen oder gastrointestinale Symptome als Reaktion auf das Schlafdefizit
\item Sorgen um den Schlaf
\end{itemize}\\
\bottomrule
\end{tabularx}
\caption[Allgemeine Insomniekriterien]{Allgemeine Insomniekriterien nach \acs{ICSD-2} \parencite{mayer_s3-leitlinie_2009}}
\label{tab:allgemeine_insomnie}
\end{table}



\newpage
\restoregeometry
\lhead{\slshape \MakeUppercase{Anhang}}
\renewcommand{\headrulewidth}{0.4pt}



\begin{table}[H] 
\centering
\begin{tabularx}{\textwidth}{cX}
\toprule
\multicolumn{2}{c}{\textbf{Kriterien der anpassungsbedingten Insomnie}}\\
\midrule 
a) & Die Symptome des Patienten entsprechen den allgemeinen Insomniekriterien.\\
b) & Die Schlafstörung ist zeitlich assoziiert mit dem identifizierbaren Stressor auf psychologischer, psychosozialer, interpersoneller, umweltbedingter, physikalischer oder medizinischer Ebene.\\
c) & Die Schlafstörung löst sich auf,wenn der akute Stressor nicht mehr vorhanden ist, oder wenn das Individuum sich an den Stressor anpasst.\\
d) & Die Schlafstörung dauert weniger als drei Monate.\\
e) & Die Schlafstörung kann nicht besser erklärt werden durch eine andere gegenwärtige Schlafstörung, eine medizinische, neurologische oder psychiatrische Erkrankung oder die Einnahme von Medikamenten oder Substanzen, die den Schlaf stören können.\\
\midrule
\multicolumn{2}{p{0.97\textwidth}}{Die Einjahresprävalenz bei Erwachsenen liegt bei 15 \% bis 20 \%. Die anpassungsbedingte Insomnie tritt bei Frauen und Männern in jedem Alter auf, wobei Frauen häufiger betroffen sind. Im Alter steigt die Prävalenz.}\\
\bottomrule
\end{tabularx}
\caption[Kriterien der anpassungsbedingten Insomnie]{Diagnostische Kriterien der anpassungsbedingten (akuten) Insomnie nach \acs{ICSD-2} \parencite{mayer_s3-leitlinie_2009, happe_schlafmedizin_2009}}
\label{tab:akute_insomnie}
\end{table}



\begin{table}[H] 
\centering
\begin{tabularx}{\textwidth}{cX}
\toprule
\multicolumn{2}{c}{\textbf{Kriterien der idiopathischen Insomnie}}\\
\midrule 
a) & Die Beschwerden des Patienten entsprechen den allgemeinen Insomniekriterien.\\
b) & Der Verlauf der Erkrankung ist chronisch, belegt durch die folgenden Symptome:
\begin{itemize}
\singlespacing
\setlength\itemsep{0em}
\item Beginn während Säuglingsalter oder Kindheit
\item Kein identifizierbarer vorhergehender Stressor bzw. keine Ursache
\item Persistiert im Verlauf ohne Perioden längerer Remission
\end{itemize}\\
c) & Die Schlafstörung kann nicht besser erklärt werden durch eine andere Schlafstörung, medizinische, neurologische, psychische Erkrankung oder Medikamenten- oder Substanzeinnahme.\\
\midrule
\multicolumn{2}{p{0.97\textwidth}}{Die Prävalenz liegt bei 0,7 \% bis 1 \% bei jungen Erwachsenen. Frauen und Männer sind gleichermaßen betroffen. Eine familiäre Vulnerabilität sowie eine Häufung bei gleichzeitiger ADHS-Erkrankung werden vermutet.}\\
\bottomrule
\end{tabularx}
\caption[Kriterien der idiopathischen Insomnie]{Diagnostische Kriterien der idiopathischen Insomnie nach \acs{ICSD-2} \parencite{mayer_s3-leitlinie_2009, happe_schlafmedizin_2009}}
\label{tab:idiopathische_insomnie}
\end{table}



\newpage


\begin{table}[H] 
\centering
\begin{tabularx}{\textwidth}{cX}
\toprule
\multicolumn{2}{c}{\textbf{Kriterien der psychophysiologischen Insomnie}}\\
\midrule 
a) & Die Symptome des Patienten entsprechen den allgemeinen Insomniekriterien.\\
b) & Die insomnischen Symptome bestehen mindestens einen Monat.\\
c) & Der Betroffene zeigt Anzeichen eines konditionierten Schlafproblems und/oder erhöhten Arousals im Bett durch eines oder mehrere der folgenden Symptome:
\begin{itemize}
\singlespacing
\setlength\itemsep{0em}
\item Exzessives Fokussieren auf und erhöhte Angst um den Schlaf
\item Einschlafschwierigkeiten zur geplanten Bettzeit oder während beabsichtigter Tagschlafepisoden, aber keine Schlafprobleme während monotoner Aktivitäten, wenn Schlaf nicht beabsichtigt ist
\item Besserer Schlaf in anderer als der gewohnten Schlafumgebung
\item Das kognitive Arousal im Bett wird charakterisiert durch intrusive Gedanken oder die wahrgenommene Unfähigkeit, willentlich schlafverhindernde kognitive Aktivität abzustellen
\item Erhöhte körperliche Anspannung im Bett manifestiert sich in wahrgenommener Unfähigkeit, körperlich zu entspannen, um den Schlafbeginn einzuleiten
\end{itemize}\\
d) & Die Schlafbeschwerde kann nicht besser durch eine andere Schlafstörung, eine medizinische, neurologische oder psychische Erkrankung, Medikamenten- oder Substanzeinnahme erklärt werden.\\
\midrule
\multicolumn{2}{p{0.97\textwidth}}{Von der psychophysiologischen Insomnie sind hauptsächlich Frauen im mittleren Alter betroffen. Die Prävalenz steigt im Alter.}\\
\bottomrule
\end{tabularx}
\caption[Kriterien der psychophysiologischen Insomnie]{Diagnostische Kriterien der psychophysiologischen Insomnie nach \acs{ICSD-2} \parencite{mayer_s3-leitlinie_2009, happe_schlafmedizin_2009}}
\label{tab:psycho_insomnie}
\end{table}



\newpage


\begin{table}[H] 
\centering
\begin{tabularx}{\textwidth}{cX}
\toprule
\multicolumn{2}{c}{\textbf{Kriterien der paradoxen Insomnie}}\\
\midrule 
a) & Die Beschwerden des Patienten entsprechen den allgemeinen Insomniekriterien.\\
b) & Die Insomniebeschwerden bestehen mindestens einen Monat.\\
c) & Eines oder mehrere der folgenden Kriterien treffen zu:
\begin{itemize}
\singlespacing
\setlength\itemsep{0em}
\item Die Patienten berichten über ein chronisches Muster von wenig oder gar keinem Schlaf mit seltenen Nächten, während derer relativ normale Mengen an Schlaf auftreten
\item Daten aus dem Schlaftagebuch, über eine oder mehrere Wochen erfasst, zeigen, dass die Betroffenen eine durchschnittliche Schlafzeit haben, die deutlich unter der der altersentsprechenden Normgruppen liegt. Oft wird für mehrere Nächte hintereinander gar kein Schlaf erlebt. Keine Tagschlafepisoden nach solchen Nächten
\item Patienten zeigen konsistent das Missverhältnis zwischen objektiven Befunden aus der Polysomnographie oder Aktigraphie und ihren subjektiven Schlafeinschätzungen
\end{itemize}\\
d) & Mindestens eines der folgenden Symptome tritt auf:
\begin{itemize}
\singlespacing
\setlength\itemsep{0em}
\item Die Patienten berichten über ständige Wahrnehmungen von nächtlichen Stimuli (z. B. Schlagen der Kirchturmuhr)
\item Die Patienten berichten über anhaltendes Vorhandensein von Gedanken oder Grübeleien während der ganzen Nacht
\end{itemize}\\
e) & Die Tagesbeeinträchtigung, die die Patienten berichten, ist konsistent mit dem, was von anderen Schlafgestörten berichtet wird, aber sie ist weniger ausgeprägt im Verhältnis zum berichteten Schlafverlust. Kein Hinweis auf abrupten Tagschlaf. Sekundenschlaf, Desorientierung oder massive Fehler oder Irrtümer als Folge eines Schlafentzugs treten auf.\\
f) & Die berichtete Schlafstörung wird nicht besser durch eine andere Schlafstörung, medizinische, neurologische oder psychische Erkrankung, Medikamenteneinnahme oder Substanzeinnahme erklärt.\\
\midrule
\multicolumn{2}{p{0.97\textwidth}}{Die paradoxe Insomnie tritt zumeist im Alter zwischen 20 und 40 Jahren bei Frauen und Männern gleichverteilt auf.}\\
\bottomrule
\end{tabularx}
\caption[Kriterien der paradoxen Insomnie]{Diagnostische Kriterien der paradoxen Insomnie nach \acs{ICSD-2} \parencite{mayer_s3-leitlinie_2009, happe_schlafmedizin_2009}}
\label{tab:paradoxe_insomnie}
\end{table}



\newpage


\begin{table}[H] 
\centering
\begin{tabularx}{\textwidth}{cX}
\toprule
\multicolumn{2}{c}{\textbf{Kriterien der Insomnie durch psychische Erkrankung}}\\
\midrule 
a) & Die Symptome des Patienten entsprechen den allgemeinen Insomniekriterien.\\
b) & Die Insomnie besteht mindestens einen Monat.\\
c) & Eine psychische Erkrankung wurde nach Standardkriterien diagnostiziert.\\
d) & Die Insomnie ist zeitlich eng verknüpft mit der psychischen Erkrankung, sie kann jedoch in einigen Fällen einige Tage oder Wochen vor dem Beginn der psychischen Erkrankung auftreten.\\
e) & Die Insomnie ist hervorstechender als typischerweise assoziiert mit der psychischen Erkrankung und führt eigenständig zu erhöhtem Stress oder stellt einen unabhängigen Behandlungsfokus dar.\\
f) & Die Schlafstörung wird nicht besser erklärt durch eine andere Schlafstörung, medizinische, neurologische Erkrankung, Medikamenten- oder Substanzeinnahme.\\
\midrule
\multicolumn{2}{p{0.97\textwidth}}{Psychische Erkrankungen stellen die häufigste Ursache von Schlafstörungen dar. Etwa 80 \% bis 90 \% der Menschen, die an einer Depression leiden, erkranken zusätzlich an einer Insomnie.}\\
\bottomrule
\end{tabularx}
\caption[Kriterien der Insomnie durch psychische Erkrankung]{Diagnostische Kriterien der Insomnie durch psychische Erkrankung nach \acs{ICSD-2} \parencite{mayer_s3-leitlinie_2009, happe_schlafmedizin_2009}}
\label{tab:insomnie_psychisch}
\end{table}




\begin{table}[H] 
\centering
\begin{tabularx}{\textwidth}{cX}
\toprule
\multicolumn{2}{c}{\textbf{Kriterien der Insomnie durch körperliche Erkrankung}}\\
\midrule 
a) & Die Symptome des Patienten entsprechen den allgemeinen Insomniekriterien.\\
b) & Die Insomnie besteht ca. einen Monat lang.\\
c) & Der Patient hat eine koexistierende organische oder körperliche Bedingung, die den Schlaf stören kann.\\
d) & Die Insomnie ist zeitlich eng assoziiert mit der körperlichen Erkrankung. Sie begann um den Zeitpunkt des Beginns der körperlichen Erkrankung, und die Progression entspricht der Progression der zugrunde liegenden körperlichen Erkrankung.\\
e) & Die Schlafstörung kann nicht besser erklärt werden durch eine andere gegenwärtige Schlafstörung, eine medizinische, neurologische oder psychische Erkrankung oder die Einnahme von Medikamenten oder Substanzen, die den Schlaf stören können.\\
\midrule
\multicolumn{2}{p{0.97\textwidth}}{Die Prävalenz der Insomnie durch körperliche Erkrankung liegt bei 0,5 \%. Zumeist sind ältere Menschen betroffen.}\\
\bottomrule
\end{tabularx}
\caption[Kriterien der Insomnie durch körperliche Erkrankung]{Diagnostische Kriterien der Insomnie durch körperliche Erkrankung nach \acs{ICSD-2} \parencite{mayer_s3-leitlinie_2009, happe_schlafmedizin_2009}}
\label{tab:korperliche_insomnie}
\end{table}



\newpage


\begin{table}[H] 
\centering
\begin{tabularx}{\textwidth}{cX}
\toprule
\multicolumn{2}{c}{\textbf{Kriterien der Insomnie durch inadäquate Schlafhygiene}}\\
\midrule 
a) & Die Symptome des Patienten entsprechen den Allgemeinen Insomniekriterien.\\
b) & Die Insomniebeschwerden bestehen ca. einen Monat lang.\\
c) & Inadäquate Schlafhygiene ist belegt durch mindestens eines der folgenden Symptome:
\begin{itemize}
\singlespacing
\setlength\itemsep{0em}
\item Irregulärer Schlaf-Wach-Rhythmus mit häufigem Tagschlaf, variable Bettzeiten oder Aufstehzeiten oder auch sehr lange Bettzeiten
\item Gewöhnlicher Gebrauch von Alkohol, Nikotin oder Koffein, speziell vor der Bettzeit
\item Ausführen kognitiv stimulierender oder emotional stimulierender Aktivitäten nahe an der Bettzeit
\item Das Bett wird für andere Aktivitäten als für Schlaf benutzt (Fernsehen, Lesen, Studieren, Essen etc.)
\item Den Betroffenen gelingt es nicht, eine behagliche Schlafumgebung zu schaffen
\end{itemize}\\
d) & Die Schlafstörung wird nicht besser erklärt durch eine andere Schlafstörung, medizinische, neurologische, psychische Erkrankung, Medikamenten- oder Substanzeinnahme.\\
\bottomrule
\end{tabularx}
\caption[Kriterien der Insomnie durch inadäquate Schlafhygiene]{Diagnostische Kriterien der Insomnie durch inadäquate Schlafhygiene nach \acs{ICSD-2} \parencite{mayer_s3-leitlinie_2009}}
\label{tab:schlafhygiene_insomnie}
\end{table}




\begin{figure}[H]
	\centering
	\includegraphics[width = \textwidth]{img/spindles_k-complex.png}
	\caption[Schlafspinden und K-Komplexe]{\acs{PSG} in der Schlafphase \acs{NREM}2 mit K-Komplex (blau markiert) und Schlafspindeln (grün markiert) im \acs{EEG} (unter Verwendung von \parencite{lee-chiong_sleep_2008})}
	\label{fig:spindel_k-komplex}
\end{figure}


\begin{table}[H] 
\centering
\begin{tabularx}{0.855\textwidth}{ccccccccccc}
\toprule
\multicolumn{4}{c}{\textbf{Gruppe Alle}} & & & & & & &\\  
\cmidrule{1-4}
w  & m  & A  &    & wA   & wminA & wmaxA &    & mA   & mminA & mmaxA\\
\midrule
41 & 23 & 51 & ~~ & 52,3 & 23    & 65    & ~~ & 48,7 & 26    & 61\\
\bottomrule
\end{tabularx}
\caption[Gruppe Alle]{Klassifizierung der Patientengruppe Alle; w = Anzahl weiblicher Patienten;\\m = Anzahl männlicher Patienten; A = durchschnittliches Gruppenalter;\\wA = Durchschnittsalter weiblicher Patienten; wminA = minimales Alter weiblicher Patienten;\\wmaxA = maximales Alter weiblicher Patienten; mA = Durchschnittsalter männlicher Patienten;\\mminA = minimales Alter männlicher Patienten; mmaxA = maximales Alter männlicher Patienten}
\label{tab:Alle}
\end{table}


\begin{table}[H] 
\centering
\begin{tabularx}{0.87\textwidth}{ccccccccccc}
\toprule
\multicolumn{4}{c}{\textbf{Gruppe Herz}} & & & & & & &\\  
\cmidrule{1-4}
w  & m & A    &    & wA   & wminA & wmaxA &    & mA   & mminA & mmaxA\\
\midrule
10 & 5 & 56,8 & ~~ & 57,3 & 41    & 65    & ~~ & 55,8 & 44    & 61\\
\bottomrule
\end{tabularx}
\caption[Gruppe Herz]{Klassifizierung der Patientengruppe Herz; w = Anzahl weiblicher Patienten;\\m = Anzahl männlicher Patienten; A = durchschnittliches Gruppenalter;\\wA = Durchschnittsalter weiblicher Patienten; wminA = minimales Alter weiblicher Patienten;\\wmaxA = maximales Alter weiblicher Patienten; mA = Durchschnittsalter männlicher Patienten;\\mminA = minimales Alter männlicher Patienten; mmaxA = maximales Alter männlicher Patienten}
\label{tab:Herz}
\end{table}


\begin{table}[H] 
\centering
\begin{tabularx}{0.9\textwidth}{ccccccccccc}
\toprule
\multicolumn{4}{c}{\textbf{Gruppe Atmung}} & & & & & & &\\  
\cmidrule{1-4}
w  & m & A  &    & wA   & wminA & wmaxA &    & mA   & mminA & mmaxA\\
\midrule
13 & 8 & 54 & ~~ & 57,2 & 47    & 65    & ~~ & 48,9 & 33    & 60\\
\bottomrule
\end{tabularx}
\caption[Gruppe Atmung]{Klassifizierung der Patientengruppe Atmung; w = Anzahl weiblicher Patienten;\\m = Anzahl männlicher Patienten; A = durchschnittliches Gruppenalter;\\wA = Durchschnittsalter weiblicher Patienten; wminA = minimales Alter weiblicher Patienten;\\wmaxA = maximales Alter weiblicher Patienten; mA = Durchschnittsalter männlicher Patienten;\\mminA = minimales Alter männlicher Patienten; mmaxA = maximales Alter männlicher Patienten}
\label{tab:Atmung}
\end{table}


\begin{table}[H] 
\centering
\begin{tabularx}{0.95\textwidth}{ccccccccccc}
\toprule
\multicolumn{4}{c}{\textbf{Gruppe Schlafmittel}} & & & & & & &\\  
\cmidrule{1-4}
w & m & A    &    & wA   & wminA & wmaxA &    & mA  & mminA & mmaxA\\
\midrule
8 & 2 & 55,1 & ~~ & 56,1 & 42    & 65    & ~~ & 51 & 44     & 58\\
\bottomrule
\end{tabularx}
\caption[Gruppe Schlafmittel]{Klassifizierung der Patientengruppe Schlafmittel; w = Anzahl weiblicher Patienten;\\m = Anzahl männlicher Patienten; A = durchschnittliches Gruppenalter;\\wA = Durchschnittsalter weiblicher Patienten; wminA = minimales Alter weiblicher Patienten;\\wmaxA = maximales Alter weiblicher Patienten; mA = Durchschnittsalter männlicher Patienten;\\mminA = minimales Alter männlicher Patienten; mmaxA = maximales Alter männlicher Patienten}
\label{tab:Schlafmittel}
\end{table}


\begin{table}[H] 
\centering
\begin{tabularx}{0.91\textwidth}{ccccccccccc}
\toprule
\multicolumn{4}{c}{\textbf{Gruppe Insomnie}} & & & & & & &\\  
\cmidrule{1-4}
w  & m  & A    &    & wA   & wminA & wmaxA &    & mA   & mminA & mmaxA\\
\midrule
17 & 11 & 46,7 & ~~ & 47,1 & 23    & 61    & ~~ & 46,1 & 26    & 61\\
\bottomrule
\end{tabularx}
\caption[Gruppe Insomnie]{Klassifizierung der Patientengruppe Insomnie; w = Anzahl weiblicher Patienten;\\m = Anzahl männlicher Patienten; A = durchschnittliches Gruppenalter;\\wA = Durchschnittsalter weiblicher Patienten; wminA = minimales Alter weiblicher Patienten;\\wmaxA = maximales Alter weiblicher Patienten; mA = Durchschnittsalter männlicher Patienten;\\mminA = minimales Alter männlicher Patienten; mmaxA = maximales Alter männlicher Patienten}
\label{tab:Insomnie}
\end{table}



% --------------------------------------------------
% gemittelte Gruppen-TDS gemäß Kriterien
% --------------------------------------------------
\textbf{gemittelte Gruppen-TDS gemäß Kriterien}

\begin{figure}[H]
	\centering
	\includegraphics[width = \textwidth]{img/meanTDSheart_oldcolorbar.png}
	\caption[Verbindungsstärken für die Gruppe Herz]{Gemittelte Verbindungsstärken für die Gruppe Herz;\\Verbindungsstärken gemäß Colorbar mit 1 = 100\%;\\links: unteres linkes Dreieck = \acs{REM}-Schlaf, oberes rechtes Dreieck = Tiefschlaf;\\rechts: unteres linkes Dreieck = Wachzustand, oberes rechtes Dreieck = Leichtschlaf}
	\label{fig:meanTDSheart}
\end{figure}

\begin{figure}[H]
	\centering
	\includegraphics[width = \textwidth]{img/meanTDSbreath_oldcolorbar.png}
	\caption[Verbindungsstärken für die Gruppe Atmung]{Gemittelte Verbindungsstärken für die Gruppe Atmung;\\Verbindungsstärken gemäß Colorbar mit 1 = 100\%;\\links: unteres linkes Dreieck = \acs{REM}-Schlaf, oberes rechtes Dreieck = Tiefschlaf;\\rechts: unteres linkes Dreieck = Wachzustand, oberes rechtes Dreieck = Leichtschlaf}
	\label{fig:meanTDSbreath}
\end{figure}

\begin{figure}[H]
	\centering
	\includegraphics[width = \textwidth]{img/meanTDSsleep_oldcolorbar.png}
	\caption[Verbindungsstärken für die Gruppe Schlafmittel]{Gemittelte Verbindungsstärken für die Gruppe Schlafmittel;\\Verbindungsstärken gemäß Colorbar mit 1 = 100\%;\\links: unteres linkes Dreieck = \acs{REM}-Schlaf, oberes rechtes Dreieck = Tiefschlaf;\\rechts: unteres linkes Dreieck = Wachzustand, oberes rechtes Dreieck = Leichtschlaf}
	\label{fig:meanTDSsleep}
\end{figure}

\begin{figure}[H]
	\centering
	\includegraphics[width = \textwidth]{img/meanTDSinsom_oldcolorbar.png}
	\caption[Verbindungsstärken für die Gruppe Insomnie]{Gemittelte Verbindungsstärken für die Gruppe Insomnie;\\Verbindungsstärken gemäß Colorbar mit 1 = 100\%;\\links: unteres linkes Dreieck = \acs{REM}-Schlaf, oberes rechtes Dreieck = Tiefschlaf;\\rechts: unteres linkes Dreieck = Wachzustand, oberes rechtes Dreieck = Leichtschlaf}
	\label{fig:meanTDSinsom}
\end{figure}



% --------------------------------------------------
% Differenz zwischen SIESTA und Gruppe Alle
% --------------------------------------------------
\textbf{Differenz zwischen SIESTA und Gruppe Alle}

\begin{figure}[H]
	\centering
	\includegraphics[width = \textwidth]{img/diffTDSsiesta_all.png}
	\caption[Differenz-Matrix der Verbindungsstärken der SIESTA-Daten und der Gruppe Alle]{Differenz-Matrix der Verbindungsstärken der SIESTA-Daten und der Gruppe Alle;\\links: unteres linkes Dreieck = \acs{REM}-Schlaf, oberes rechtes Dreieck = Tiefschlaf;\\rechts: unteres linkes Dreieck = Wachzustand, oberes rechtes Dreieck = Leichtschlaf;\\Differenzen der Verbindungsstärken gemäß Colorbar mit 1 = 100\%;\\positive Werte = Verbindungen in SIESTA-Daten stärker; negative Werte = Verbindungen in Gruppe Alle stärker}
	\label{fig:diffTDSsiesta_all}
\end{figure}



% --------------------------------------------------
% alle weiblichen gegen alle männlichen Patienten
% --------------------------------------------------
\textbf{alle weiblichen gegen alle männlichen Patienten}

Abb. \ref{fig:dsls_m_f}

Abb. \ref{fig:rw_m_f}

Abb. \ref{fig:diffTDS_f_m}

\begin{figure}[H]
	\centering
	\includegraphics[width = \textwidth]{img/dsls_m_f_oldcolorbar.png}
	\caption[Verbindungsstärken von weiblichen und männlichen Insomniepatienten]{Gemittelte Verbindungsstärken von weiblichen und männlichen Insomniepatienten im Tiefschlaf (links) und Leichtschlaf (rechts); Verbindungsstärken gemäß Colorbar mit 1 = 100\%;\\unteres linkes Dreieck: männliche Insomniepatienten;\\oberes rechtes Dreieck: weibliche Insomniepatienten}
	\label{fig:dsls_m_f}
\end{figure}

\begin{figure}[H]
	\centering
	\includegraphics[width = \textwidth]{img/rw_m_f_oldcolorbar.png}
	\caption[Verbindungsstärken von weiblichen und männlichen Insomniepatienten]{Gemittelte Verbindungsstärken von weiblichen und männlichen Insomniepatienten im REM-Schlaf (links) und Wachzustand (rechts); Verbindungsstärken gemäß Colorbar mit 1 = 100\%;\\unteres linkes Dreieck: männliche Insomniepatienten;\\oberes rechtes Dreieck: weibliche Insomniepatienten}
	\label{fig:rw_m_f}
\end{figure}

\begin{figure}[H]
	\centering
	\includegraphics[width = \textwidth]{img/diffTDS_f_m.png}
	\caption[Differenz-Matrix der Verbindungsstärken der weiblichen und männlichen Insomniepatienten]{Differenz-Matrix der Verbindungsstärken der weiblichen und männlichen Insomniepatienten;\\links: unteres linkes Dreieck = REM-Schlaf; oberes rechtes Dreieck = Tiefschlaf;\\rechts: unteres linkes Dreieck = Wachzustand; oberes rechtes Dreieck = Leichtschlaf;\\Differenzen der Verbindungsstärken gemäß Colorbar mit 1 = 100\%;\\positive Werte = höhere Verbindungsstärken bei den weiblichen Insomniepatienten; negative Werte = höhere Verbindungsstärken bei männlichen Insomniepatienten}
	\label{fig:diffTDS_f_m}
\end{figure}




% --------------------------------------------------
% jüngste gegen älteste Patienten
% --------------------------------------------------
\textbf{jüngste gegen älteste Patienten}

Abb. \ref{fig:dsls_allyoung_allold}

Abb. \ref{fig:rw_allyoung_allold}

Differenz-Matrix der jungen und älteren Insomiepatienten, Anhang Abb. \ref{fig:diffTDSallyoung_allold}

\begin{figure}[H]
	\centering
	\includegraphics[width = \textwidth]{img/dsls_allyoung_allold.png}
	\caption[Verbindungsstärken der jüngsten und ältesten Insomniepatienten im Tief- und Leichtschlaf]{Gemittelte Verbindungsstärken der jüngsten und ältesten Insomniepatienten im Tief- (links) und Leichtschlaf (rechts); Verbindungsstärken gemäß Colorbar mit 1 = 100\%;\\unteres linkes Dreieck: jüngste Insomniepatienten (23 bis 46, Jahre, n=22);\\oberes rechtes Dreieck: älteste Insomniepatienten (59 bis 65 Jahre, n=22)}
	\label{fig:dsls_allyoung_allold}
\end{figure}

\begin{figure}[H]
	\centering
	\includegraphics[width = \textwidth]{img/rw_allyoung_allold.png}
	\caption[Verbindungsstärken der jüngsten und ältesten Insomniepatienten im REM-Schlaf und Wachzustand]{Gemittelte Verbindungsstärken der jüngsten und ältesten Insomniepatienten im REM-Schlaf (links) und Wachzustand (rechts); Verbindungsstärken gemäß Colorbar mit 1 = 100\%;\\unteres linkes Dreieck: jüngste Insomniepatienten (23 bis 46, Jahre, n=22);\\oberes rechtes Dreieck: älteste Insomniepatienten (59 bis 65 Jahre, n=22)}
	\label{fig:rw_allyoung_allold}
\end{figure}

\begin{figure}[H]
	\centering
	\includegraphics[width = \textwidth]{img/diffTDSallyoung_allold.png}
	\caption[Differenz-Matrix der Verbindungsstärken der jungen und älteren Insomniepatienten]{Differenz-Matrix der Verbindungsstärken der jüngsten (23 bis 46 Jahre, n=22) und ältesten (59 bis 65 Jahre, n=22) Insomniepatienten;\\links: unteres linkes Dreieck = REM-Schlaf; oberes rechtes Dreieck = Tiefschlaf;\\rechts: unteres linkes Dreieck = Wachzustand; oberes rechtes Dreieck = Leichtschlaf;\\Differenzen der Verbindungsstärken gemäß Colorbar mit 1 = 100\%;\\positive Werte = höhere Verbindungsstärken bei jungen Insomniepatienten; negative Werte = höhere Verbindungsstärken bei älteren Insomniepatienten}
	\label{fig:diffTDSallyoung_allold}
\end{figure}



% --------------------------------------------------
% junge weibliche gegen junge männliche Patienten
% --------------------------------------------------
\textbf{junge weibliche gegen junge männliche Patienten}

Abb. \ref{fig:dsls_youngf_youngm}

Abb. \ref{fig:rw_youngf_youngm}

Abb. \ref{fig:diffTDSyoungf_youngm}

\begin{figure}[H]
	\centering
	\includegraphics[width = \textwidth]{img/dsls_youngf_youngm.png}
	\caption[Verbindungsstärken der jüngsten weiblichen und männlichen Insomniepatienten im Tief- und Leichtschlaf]{Gemittelte Verbindungsstärken der jüngsten weiblichen und männlichen Insomniepatienten im Tief- (links) und Leichtschlaf (rechts); Verbindungsstärken gemäß Colorbar mit 1 = 100\%;\\unteres linkes Dreieck: jüngste männliche Insomniepatienten (26 bis 45 Jahre, n=11);\\oberes rechtes Dreieck: jüngste weibliche (23 bis 46 Jahre, n=11) Insomniepatienten}
	\label{fig:dsls_youngf_youngm}
\end{figure}

\begin{figure}[H]
	\centering
	\includegraphics[width = \textwidth]{img/rw_youngf_youngm.png}
	\caption[Verbindungsstärken der jüngsten weiblichen und männlichen Insomniepatienten im REM-Schlaf und Wachzustand]{Gemittelte Verbindungsstärken der jüngsten weiblichen und männlichen Insomniepatienten im REM-Schlaf (links) und Wachzustand (rechts); Verbindungsstärken gemäß Colorbar mit 1 = 100\%;\\unteres linkes Dreieck: jüngste männliche Insomniepatienten (26 bis 45 Jahre, n=11);\\oberes rechtes Dreieck: jüngste weibliche (23 bis 46 Jahre, n=11) Insomniepatienten}
	\label{fig:rw_youngf_youngm}
\end{figure}

\begin{figure}[H]
	\centering
	\includegraphics[width = \textwidth]{img/diffTDSyoungf_youngm.png}
	\caption[Differenz-Matrix der Verbindungsstärken der jüngsten weiblichen und männlichen Insomniepatienten]{Differenz-Matrix der Verbindungsstärken der jüngsten weiblichen (23 bis 46 Jahre, n=11) und jüngsten männlichen (26 bis 45 Jahre, n=11) Insomniepatienten;\\links: unteres linkes Dreieck = REM-Schlaf; oberes rechtes Dreieck = Tiefschlaf;\\rechts: unteres linkes Dreieck = Wachzustand; oberes rechtes Dreieck = Leichtschlaf;\\Differenzen der Verbindungsstärken gemäß Colorbar mit 1 = 100\%;\\positive Werte = höhere Verbindungsstärken bei jungen weiblichen Insomniepatienten; negative Werte = höhere Verbindungsstärken bei jungen männlichen Insomniepatienten}
	\label{fig:diffTDSyoungf_youngm}
\end{figure}



% --------------------------------------------------
% alte weibliche gegen alte männliche Patienten
% --------------------------------------------------
\textbf{alte weibliche gegen alte männliche Patienten}

Abb. \ref{fig:dsls_oldf_oldm}

Abb. \ref{fig:rw_oldf_oldm}

Abb. \ref{fig:diffTDSoldf_oldm}

\begin{figure}[H]
	\centering
	\includegraphics[width = \textwidth]{img/dsls_oldf_oldm.png}
	\caption[Verbindungsstärken der ältesten weiblichen und männlichen Insomniepatienten im Tief- und Leichtschlaf]{Gemittelte Verbindungsstärken der ältesten weiblichen und männlichen Insomniepatienten im Tief- (links) und Leichtschlaf (rechts); Verbindungsstärken gemäß Colorbar mit 1 = 100\%;\\unteres linkes Dreieck: älteste männliche Insomniepatienten (59 bis 61 Jahre, n=8);\\oberes rechtes Dreieck: älteste weibliche (59 bis 65 Jahre, n=14) Insomniepatienten}
	\label{fig:dsls_oldf_oldm}
\end{figure}

\begin{figure}[H]
	\centering
	\includegraphics[width = \textwidth]{img/rw_oldf_oldm.png}
	\caption[Verbindungsstärken der ältesten weiblichen und männlichen Insomniepatienten im REM-Schlaf und Wachzustand]{Gemittelte Verbindungsstärken der ältesten weiblichen und männlichen Insomniepatienten im REM-Schlaf (links) und Wachzustand (rechts); Verbindungsstärken gemäß Colorbar mit 1 = 100\%;\\unteres linkes Dreieck: älteste männliche Insomniepatienten (59 bis 61 Jahre, n=8);\\oberes rechtes Dreieck: älteste weibliche (59 bis 65 Jahre, n=14) Insomniepatienten}
	\label{fig:rw_oldf_oldm}
\end{figure}

\begin{figure}[H]
	\centering
	\includegraphics[width = \textwidth]{img/diffTDSoldf_oldm.png}
	\caption[Differenz-Matrix der Verbindungsstärken der ältesten weiblichen und männlichen Insomniepatienten]{Differenz-Matrix der Verbindungsstärken der ältesten weiblichen (59 bis 65 Jahre, n=14) und ältesten männlichen (59 bis 61 Jahre, n=8) Insomniepatienten;\\links: unteres linkes Dreieck = REM-Schlaf; oberes rechtes Dreieck = Tiefschlaf;\\rechts: unteres linkes Dreieck = Wachzustand; oberes rechtes Dreieck = Leichtschlaf;\\Differenzen der Verbindungsstärken gemäß Colorbar mit 1 = 100\%;\\positive Werte = höhere Verbindungsstärken bei älteren weiblichen Insomniepatienten; negative Werte = höhere Verbindungsstärken bei älteren männlichen Insomniepatienten}
	\label{fig:diffTDSoldf_oldm}
\end{figure}




% --------------------------------------------------
% junge weibliche gegen alte weibliche Patienten
% --------------------------------------------------
\textbf{junge weibliche gegen alte weibliche Patienten}

Abb. \ref{fig:dsls_youngf_oldf}

Abb. \ref{fig:rw_youngf_oldf}

Abb. \ref{fig:diffTDSyoungf_oldf}

\begin{figure}[H]
	\centering
	\includegraphics[width = \textwidth]{img/dsls_youngf_oldf.png}
	\caption[Verbindungsstärken der jüngsten und der ältesten weiblichen Insomniepatienten im Tief- und Leichtschlaf]{Gemittelte Verbindungsstärken der jüngsten und der ältesten weiblichen Insomniepatienten im Tief- (links) und Leichtschlaf (rechts); Verbindungsstärken gemäß Colorbar mit 1 = 100\%;\\unteres linkes Dreieck: älteste weibliche Insomniepatienten (59 bis 65 Jahre, n=14);\\oberes rechtes Dreieck: jüngste weibliche (23 bis 46 Jahre, n=11) Insomniepatienten}
	\label{fig:dsls_youngf_oldf}
\end{figure}

\begin{figure}[H]
	\centering
	\includegraphics[width = \textwidth]{img/rw_youngf_oldf.png}
	\caption[Verbindungsstärken der jüngsten und der ältesten weiblichen Insomniepatienten im REM-Schlaf und Wachzustand]{Gemittelte Verbindungsstärken der jüngsten und der ältesten weiblichen Insomniepatienten im REM-Schlaf (links) und Wachzustand (rechts); Verbindungsstärken gemäß Colorbar mit 1 = 100\%;\\unteres linkes Dreieck: älteste weibliche Insomniepatienten (59 bis 65 Jahre, n=14);\\oberes rechtes Dreieck: jüngste weibliche (23 bis 46 Jahre, n=11) Insomniepatienten}
	\label{fig:rw_youngf_oldf}
\end{figure}

\begin{figure}[H]
	\centering
	\includegraphics[width = \textwidth]{img/diffTDSyoungf_oldf.png}
	\caption[Differenz-Matrix der Verbindungsstärken der jüngsten und ältesten weiblichen Insomniepatienten]{Differenz-Matrix der Verbindungsstärken der jüngsten (23 bis 46 Jahre, n=11) und der ältesten (59 bis 65 Jahre, n=14) weiblichen Insomniepatienten;\\links: unteres linkes Dreieck = REM-Schlaf; oberes rechtes Dreieck = Tiefschlaf;\\rechts: unteres linkes Dreieck = Wachzustand; oberes rechtes Dreieck = Leichtschlaf;\\Differenzen der Verbindungsstärken gemäß Colorbar mit 1 = 100\%;\\positive Werte = höhere Verbindungsstärken bei jungen weiblichen Insomniepatienten; negative Werte = höhere Verbindungsstärken bei älteren weiblichen Insomniepatienten}
	\label{fig:diffTDSyoungf_oldf}
\end{figure}



% --------------------------------------------------
% junge männliche gegen alte männliche Patienten
% --------------------------------------------------
\textbf{junge männliche gegen alte männliche Patienten}

Abb. \ref{fig:dsls_youngm_oldm}

Abb. \ref{fig:rw_youngm_oldm}

Abb. \ref{fig:diffTDSyoungm_oldm}

\begin{figure}[H]
	\centering
	\includegraphics[width = \textwidth]{img/dsls_youngm_oldm.png}
	\caption[Verbindungsstärken der jüngsten und der ältesten männlichen Insomniepatienten im Tief- und Leichtschlaf]{Gemittelte Verbindungsstärken der jüngsten und der ältesten männlichen Insomniepatienten im Tief- (links) und Leichtschlaf (rechts); Verbindungsstärken gemäß Colorbar mit 1 = 100\%;\\unteres linkes Dreieck: älteste männliche Insomniepatienten (59 bis 61 Jahre, n=8);\\oberes rechtes Dreieck: jüngste männliche (26 bis 45 Jahre, n=11) Insomniepatienten}
	\label{fig:dsls_youngm_oldm}
\end{figure}

\begin{figure}[H]
	\centering
	\includegraphics[width = \textwidth]{img/rw_youngm_oldm.png}
	\caption[Verbindungsstärken der jüngsten und der ältesten männlichen Insomniepatienten im REM-Schlaf und Wachzustand]{Gemittelte Verbindungsstärken der jüngsten und der ältesten männlichen Insomniepatienten im REM-Schlaf (links) und Wachzustand (rechts); Verbindungsstärken gemäß Colorbar mit 1 = 100\%;\\unteres linkes Dreieck: älteste männliche Insomniepatienten (59 bis 61 Jahre, n=8);\\oberes rechtes Dreieck: jüngste männliche (26 bis 45 Jahre, n=11) Insomniepatienten}
	\label{fig:rw_youngm_oldm}
\end{figure}

\begin{figure}[H]
	\centering
	\includegraphics[width = \textwidth]{img/diffTDSyoungm_oldm.png}
	\caption[Differenz-Matrix der Verbindungsstärken der jüngsten und ältesten männlichen Insomniepatienten]{Differenz-Matrix der Verbindungsstärken der jüngsten (26 bis 45 Jahre, n=11) und der ältesten (59 bis 61 Jahre, n=8) männlichen Insomniepatienten;\\links: unteres linkes Dreieck = REM-Schlaf; oberes rechtes Dreieck = Tiefschlaf;\\rechts: unteres linkes Dreieck = Wachzustand; oberes rechtes Dreieck = Leichtschlaf;\\Differenzen der Verbindungsstärken gemäß Colorbar mit 1 = 100\%;\\positive Werte = höhere Verbindungsstärken bei jungen männlichen Insomniepatienten; negative Werte = höhere Verbindungsstärken bei älteren männlichen Insomniepatienten}
	\label{fig:diffTDSyoungm_oldm}
\end{figure}




%\begin{figure}[H]
%	\centering
%	\includegraphics[width = 0.7\textwidth]{img/nmean_age_siesta.png}
%	\caption[Gegenüberstellung von Verbindungsstärke \textit{nmean} und Alter gesunder Probanden]{Gegenüberstellung von Verbindungsstärke \textit{nmean} und Alter gesunder Probanden der SIESTA-Studie in den unterschiedlichen Schlafstadien gemäß Krefting et al.; die Linien stellen die lineare Regression für alle Probanden (durchgezogene Linie), für weibliche (gestrichelte Linie) und für männliche (gepunktete Linie) Probanden dar}
%	\label{fig:nmean_age_siesta}
%\end{figure}

\restoregeometry