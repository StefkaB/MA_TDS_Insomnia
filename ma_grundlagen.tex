\chapter{Grundlagen}


\section{Medizinische Grundlagen}\label{medgrundlagen} 


\subsection{Schlafmedizin}\label{schlafmedizin} 

Früher wurde Schlaf als ein Zustand definiert, welcher durch die Abwesenheit von Wachheit geprägt ist. Aufgrund der durch die Elektrophysiologie errungenen Erkenntnisse, welche auf den elektrochemischen Prozessen des Zentralnervensystems beruhen, konnten dem Schlaf mit der Zeit jedoch spezifische Merkmale zugewiesen werden. Vor allem das durch Hans Berger im Jahr 1929 entwickelte \acl{EEG} (\acs{EEG}) zur Messung von Gehirnströmen bildete die Grundlage der frühen Schlafmedizin, so dass beispielsweise die Schlafregulation, verschiedene Schlafphasen oder auch damit verbundene Veränderungen im menschlichen Organismus definiert werden konnten. \parencite{ebner_eeg_2006, penzel_schlafstorungen_2005}\\

Der Schlaf wird demnach zum einen durch die vom Hypothalamus gesteuerte "`Innere Uhr"' des Menschen reguliert und zum anderen durch den "`Hell-Dunkel-Rhythmus"', der die Innere Uhr direkt beeinflusst. Während bei Tageslicht der wachheitsfördernde Neurotransmitter Serotonin freigesetzt wird, wird aus ihm bei zunehmender Dunkelheit das schlaffördernde Hormon Melatonin gebildet \parencite{steinberg_schlafmedizin_2010}. Darüber hinaus ist bekannt, dass Veränderungen im Muskeltonus, in Atmung, Kreislauf und Verdauung direkt an den Schlaf-Wach-Rhythmus sowie an die einzelnen Schlafphasen gekoppelt sind. Andere biologische Prozesse, wie die Körperkerntemperatur, sind fest an die Innere Uhr und den 24-Stunden-Rhythmus gebunden, unabhängig von Schlaf oder Wachzustand. \parencite{penzel_schlafstorungen_2005, rasche_update_2003}\\

Obwohl Methoden zur Ableitung physiologischer Funktionen bereits in der ersten Hälfte des 20. Jahrhunderts entwickelt worden sind, wurde die Schlafmedizin erst nach dem 2. Weltkrieg und hauptsächlich ab den 1980er Jahren verstärkt untersucht. Heutzutage existieren in Deutschland mehrere Hundert akkreditierte Schlaflabore, die Relevanz der Schlafmedizin muss jedoch weiterhin durch Forschung, Lehre und Entwicklung verstärkt vertieft werden.\parencite{penzel_schlafstorungen_2005}\\

Der Stellenwert der Schlafmedizin zeigt sich zum einen darin, dass die Erholungsfunktion des Schlafs bislang noch nicht hinreichend erklärt werden kann. Zum anderen leiden Untersuchungen zufolge etwa 15 \% bis 35 \% der deutschen Bevölkerung an Ein- und Durchschlafstörungen. Frauen sind hiervon stärker betroffen als Männer und die Verbreitung steigt mit dem Alter an (Abb. \ref{fig:ein-durchschlafstörung}). Eine Schlafstörung wird oft erst durch eine Beeinträchtigung der Tagesbefindlichkeit deutlich. Schwerwiegende Schlafstörungen, die nicht auf mangelnde Schlafhygiene\footnote{mangelnde Schlafhygiene: beispielsweise zu ausgedehnter Mittagsschlaf oder unregelmäßige Aufsteh- und Zubettgehzeiten} zurückzuführen sind, beeinflussen die Erholungsfunktion des Schlafes und können sich maßgeblich auf die Lebensqualität auswirken, z. B. durch Befindensstörungen oder Leistungseinschränkungen. So sind beispielsweise 30 \% der Autounfälle in Deutschland auf Müdigkeit am Steuer zurückzuführen. \parencite{mayer_s3-leitlinie_2009, happe_schlafmedizin_2009, penzel_schlafstorungen_2005}

\begin{figure}[H]
	\centering
	\includegraphics[width = \textwidth]{img/Ein-Durchschlafstorung.png}
	\caption[Prävalenz von Ein- und Durchschlafstörungen]{Prävalenz von Ein- und Durchschlafstörungen:\\a) Prävalenz von Einschlafstörungen 2013 nach Alter und Geschlecht (unter Verwendung von \parencite{schlack_haufigkeit_2013});\\b) Prävalenz von Durchschlafstörungen 2013 nach Alter und Geschlecht(unter Verwendung von  \parencite{schlack_haufigkeit_2013})}
	\label{fig:ein-durchschlafstörung}
\end{figure}

\subsection{Polysomnographie}\label{psg} 

Um Schlafstörungen und den Einfluss einzelner physioligischer Systeme auf den Schlaf zu untersuchen, werden \acl{PSG}n (\acs{PSG}) erstellt. Diese stellen Langzeitbiosignalaufzeichnungen dar, im Rahmen derer verschiedene Körperaktivitäten elektrophysiologisch über Nacht abgeleitet und digital aufgezeichnet werden (Abb. \ref{fig:beispiel-psg}). Die Aufnahmen erfolgen zumeist über zwei Nächte im Schlaflabor. Dies hat den Hintergrund, dass der Patient in der ersten Nacht aufgrund der ungewohnten Umgebung sowie der Verkabelung oft unruhiger schläft und dadurch die Aufzeichnungen falsch gedeutet werden können. Dieses Phänomen ist auch als "`Erste-Nacht-Effekt"' bekannt und konnte in aktuellen Untersuchungen auf eine erhöhte Alarmbereitschaft der linken Hemisphäre während des Tiefschlafs zurückgeführt werden \parencite{tamaki_night_2016}. \\

\begin{figure}[H]
	\centering
	\includegraphics[width = \textwidth]{img/Beispiel-PSG.png}
	\caption[Beispielhafte \acs{PSG}]{Beispielhafte Aufzeichnung von Körperfunktionen in einer \acs{PSG} und Andeutung der Elektroden- und Sensorenpositionen am Körper \parencite{penzel_schlafstorungen_2005}}
	\label{fig:beispiel-psg}
\end{figure}

Die \acl{AASM} (\acs{AASM}) regt die Verwendung standardisierter Regeln für die Auswertung von Schlaf an. Hierzu zählen Empfehlungen über die zu verwendenden Ableitungen, die entsprechenden Ableitungsmethoden sowie Abtastfrequenzen \parencite{iber_aasm_2007}. Im Folgenden werden diese abzuleitenden Signale vorgestellt. Die Ausführungen beziehen sich auf den Schlaf eines jungen, normal schlafenden Erwachsenen.\\

%---------------
%------EEG------
%---------------
%\textbf{\acs{EEG}}
\paragraph{\acs{EEG}}
Das \acs{EEG} misst Potenzialschwankungn in der Großhirnrinde. Gemäß den Kriterien der \acs{AASM} werden die Elektroden nach dem 10-20-System frontal, zentral und okzipital mit einem Abstand von 10 \% oder 20 \% des Kopfes auf beiden Hemisphären angebracht (Abb. \ref{fig:10-20jasper}) \parencite{iber_aasm_2007}. Die Signale der linken Hemispähre werden über die rechte Mastoid\footnote{Mastoid: Warzenfortsatz hinter der Gehörgangswand}-Elektrode und umgekehrt abgeleitet (A1 und A2 in Abb. \ref{fig:10-20jasper}). In der Praxis werden anstelle der frontalen Elektroden F3 und F4 häufig auch die vorgelagerten frontopolaren Elektrodenpositionen Fp1 und Fp2 verwendet, welche sich direkt an der Stirn befinden. Die Signaldynamik zeigt jedoch keine signifikanten Unterschiede zu den frontalen Positionen \parencite{dorffner_effects_2015}. \acs{EEG}-Signale weisen eine stetige Wellenform und je nach Ableitungsposition und Schlafphase Frequenzen zwischen 0.5 Hz und 35 Hz sowie Amplituden im Bereich von 5 $\upmu$V bis 200 $\upmu$V auf \parencite{lee-chiong_sleep_2008}.

\begin{figure}[H]
	\centering
	\includegraphics[scale = 0.7]{img/10-20jasper.png}
	\caption[Elektrodenpositionen des \acs{EEG}]{Elektrodenpositionen für die Ableitung des \acs{EEG} nach dem 10-20-System nach Jasper (1958); die frontalen, zentralen und okzipitalen Positionen werden von der \acs{AASM} empfohlen \parencite{kemp_edf+:_????}}
	\label{fig:10-20jasper}
\end{figure}

Darüber hinaus wird das \acs{EEG} in fünf verschiedene Frequenzbänder eingeteilt. Das Frequenzband der Delta-Wellen liegt im Bereich 0.5 Hz bis 3 Hz. Theta-Wellen weisen eine Frequenz von 4 Hz bis 7 Hz auf. Eine etwas höhere Frequenz ist mit 8 Hz bis 13~Hz bei den Alpha-Wellen zu verzeichnen. Sigma-Wellen liegen im Frequenzbereich von 14 Hz bis 16 Hz. Beta-Wellen weisen dagegen die höchste Frequenz von 17 Hz bis 35 Hz auf. \parencite{lee-chiong_sleep_2008, steinberg_schlafmedizin_2010}\\

%---------------
%------EOG------
%---------------
%\textbf{\acs{EOG}}
\paragraph{\acs{EOG}}
Mit Hilfe des \acl{EOG}s (\acs{EOG}) werden die Potenzialschwankungen der Augen zwischen Hornhaut und Retina aufgezeichnet, um Augenrollen, Blinzeln oder Rapid-Eye-Movements im Schlaf zu erkennen. Die Signale werden wie das \acs{EEG} gegen eine Referenzelektrode am Ohr abgeleitet. Die \acs{AASM} gibt die Anbringung der Elektroden an den äußeren Augenwinkeln gemäß Abb. \ref{fig:eog} vor. Eine Augenbewegung in Richtung einer Elektrode verursacht einen positiven Ausschlag, in entgegengesetzter Richtung einen negativen. Bei synchroner binokularer Augenbewegung zeichnet die eine Elektrode demnach einen positiven und die anderen einen negativen Ausschlag auf, so dass sich die Signale ohne Einfluss von Störfaktoren gegenseitig annähernd aufheben (Abb. \ref{fig:beispiel-psg}). Dieses als normal geltende Verhalten wird als "`außer Phase"' bezeichnet. Artefakte, die beispielsweise durch hohe \acs{EEG}-Ausschläge auftreten können, verursachen Ausschläge in nur einem \acs{EOG}-Kanal. Dieses Verhalten wird als "`in Phase"' bezeichnet. \parencite{iber_aasm_2007, lee-chiong_sleep_2008}

\begin{figure}[H]
	\centering
	\includegraphics[scale = 0.6]{img/eog.png}
	\caption[Elektrodenpositionen des \acs{EOG}]{Ableitung des \acs{EOG} an den äußeren Augenwinkeln (links parallel und rechts diagonal) gemäß \acs{AASM} \parencite{iber_aasm_2007}}
	\label{fig:eog}
\end{figure}

%---------------
%------EMG------
%---------------
%\textbf{\acs{EMG}}
\paragraph{\acs{EMG}}
Mittels \acl{EMG} (\acs{EMG}) werden Muskelaktivitäten oberhalb des Unterkiefers im Kinn (Abb. \ref{fig:emgchin}) sowie in den vorderen Schienbeinen gemessen. Das \acs{EMG} dient neben dem \acs{EEG} sowie dem \acs{EOG} der Einteilung der Schlafphasen sowie der Diagnose bewegungsbezogener Schlafkrankheiten. Minimale Muskelaktivität von bis zu 15 $\upmu$V ist während eines Großteils des Schlafs messbar, die Amplitudenwerte eindeutiger Bewegungen liegen jedoch mindestens 8 $\upmu$V höher als in Phasen geringer Muskelaktivität. Der Frequenzbereich von Muskelkontraktionen liegt zwischen 0.5 Hz und 3 Hz. \parencite{iber_aasm_2007, leroux_handbuch_2009, lee-chiong_sleep_2008}

\begin{figure}[H]
	\centering
	\includegraphics[scale = 0.6]{img/EMGchin.jpg}
	\caption[Elektrodenpositionen des \acs{EMG}]{Ableitung des \acs{EMG} am Kinn gemäß \acs{AASM} \parencite{leroux_handbuch_2009}}
	\label{fig:emgchin}
\end{figure}

%---------------
%------EKG------
%---------------
%\textbf{\acs{EKG}}
\paragraph{\acs{EKG}}
Mittels \acl{EKG} (\acs{EKG}) werden die Spannungspotenziale der Herzaktivität gemessen. Die Extremitätenmessung erfolgt entweder an beiden Armen oder an einem Arm und einem Bein. Die \acs{AASM} empfiehlt die Platzierung der Elektroden am rechten Arm und linken Bein und zusätzlich eine Brustwandableitung (Abb. \ref{fig:ekg}). Je nach Elektrodenposition äußert sich die Aufzeichnung des Herzschlags jedoch unterschiedlich (Abb. \ref{fig:qrs_extr_brustw}b)/ \ref{fig:qrs_extr_brustw}c)). \parencite{bamberger_1x1_2015, iber_aasm_2007}

\begin{figure}[H]
	\centering
	\includegraphics[scale = 0.5]{img/ekg.png}
	\caption[Elektrodenpositionen des \acs{EKG}]{Brustwandableitung (links) und Extremitätenableitung (rechts) gemäß \acs{AASM} \parencite{leroux_handbuch_2009}}
	\label{fig:ekg}
\end{figure}

Das Herz eines gesunden, jungen Erwachsenen schlägt etwa 60- bis 80-mal pro Minute. Bei einem langsameren Herzschlag liegt eine Bradykardie vor, bei einer Frequenz von mehr als 100 Schlägen pro Minute eine Tachykardie. Die Frequenz eines gesunden Herzschlags liegt demnach zwischen 1 Hz und 1,6 Hz. Das \acs{EKG} weist eine typische Zackenform auf, bestehend aus einer P- und T-Welle sowie dem QRS-Komplex, welche die Ausbreitung einer Herzerregung darstellen (Abb. \ref{fig:beispiel-psg} und \ref{fig:qrs_extr_brustw}a)). Die P-Welle entspricht der Erregungsausbreitung in den Herzvorhöfen, der QRS-Komplex stellt hingegen die Erregungsausbreitung in den Herzkammern dar. Die Erregungsrückbildung in den Herzkammern wird durch die T-Welle abgebildet. \parencite{bamberger_1x1_2015, oresick_kania_ekg_2016}\\

\begin{figure}[H]
	\centering
	\includegraphics[width = \textwidth]{img/qrs_extr_brustw.png}
	\caption[Ableitung des \acs{EKG}]{Ableitung des \acs{EKG}:\\a) P-Welle, QRS-Komplex und T-Welle als typische Ausprägung des Herzschlags im \acs{EKG} \parencite{oresick_kania_ekg_2016};\\b) Extremitätenableitung mit unterschiedlichen Elektrodenpositionen;\\c) Brustwandableitung mit unterschiedlichen Elektrodenpositionen;\\ein Kästchen in b) und c) entspricht 500 $\upmu$V \parencite{bamberger_1x1_2015}}
	\label{fig:qrs_extr_brustw}
\end{figure}

%---------------
%----Atmung-----
%---------------
%\textbf{Atmung}
\paragraph{Atmung}
Gemäß den Kriterien der \acs{AASM} wird die Aufzeichnung von Atemfluss sowie Atemanstrengung empfohlen. Die Messung des Atemflusses soll die Unterschiede zwischen In- und Exspiration aufzeigen, um festzustellen, ob im Schlaf atmungsbezogene Störungen auftreten. Hierzu werden Temperatur- oder Drucksensoren empfohlen, welche die nasale oder oronasale\footnote{oronasal: Mund und Nase betreffend} Atmung aufzeichnen. Die Messung der Atemanstrengung verfolgt das gleiche Ziel, erfolgt gemäß den Empfehlungen der \acs{AASM} jedoch entweder mittels Ösophagusmanometrie\footnote{Ösophagusmanometrie: Druckmessung in der Speiseröhre} in der Speiseröhre oder äußerlich mittels um Brust und Bauch gelegte Gurte. Bei beiden Methoden werden die durch die Atmung hervorgerufenen Druckunterschiede gemessen. Eine gesunde Atmung erzeugt im Signal eine gleichmäßige Wellenform (Abb. \ref{fig:beispiel-psg}), wobei die Atemfrequenz eines ruhenden Erwachsenen mit etwa zwölf bis 18 Atemzügen pro Minute bei 0,2 Hz bis 0,3 Hz liegt. \parencite{iber_aasm_2007, lee-chiong_sleep_2008, leroux_handbuch_2009, steffel_lunge_2014}\\

%---------------
%----Weitere----
%---------------
%\textbf{Weitere Signale}
\paragraph{Weitere Signale}
Die \acs{AASM} empfiehlt darüber hinaus die Ableitung weiterer Signale. Hierzu zählen die Sauerstoffsättigung, Körperlage sowie Schnarchgeräusche. Da die in dieser Arbeit verwendete \acs{TDS} Analyse lediglich auf den zuvor beschriebenen Signalen beruht, wird auf die weiteren Signale nicht näher eingegangen. In der Praxis werden darüber hinaus je nach Krankheitsverdacht und Untersuchung häufig noch andere Signale abgeleitet, beispielsweise weitere \acs{EEG}s, so dass \acs{PSG}s zumeist 15 bis 40 Signale enthalten. \\

Die Qualität der Signale innerhalb einer \acs{PSG} ist nicht in jedem Schlaflabor und bei jedem Patienten identisch, sondern hängt vielmehr von den Aufnahmegeräten, Ableitungsmethoden (Abb. \ref{fig:qrs_extr_brustw}b) und \ref{fig:qrs_extr_brustw}c)) und Störfaktoren ab. Um die digitale Analyse der Daten zu vereinfachen, wurde das \acl{EDF} (\acs{EDF}) im Jahr 1987 speziell für die Schlafmedizin entwickelt und im Jahr 2001 durch das EDF+ erweitert \parencite{kemp_european_????}. Dieses Datenformat definiert Standards zur Speicherung und zum Datenaustausch von Biosignalen. Es regelt die Aufteilung der aufgezeichneten Daten in ein Header Record sowie mehrere Data Records entsprechend der Anzahl der in der \acs{PSG} enthaltenen Signale. Im Header Record werden in 256 Bytes allgemeine Informationen über den Patienten sowie technische Aufnahmeeigenschaften gespeichert (Tab.~\ref{tab:edf_header}). Hierzu zählen insbesondere die Patienten-ID, Startzeitpunkt der Aufnahme, Dauer und Anzahl der Data Records sowie die Anzahl der enthaltenen Signale. In den Data Records werden jeweils in den ersten 256 Bytes allgemeine Informationen über jedes aufgezeichnete Signal gespeichert, wie beispielsweise die Signalbezeichnung (Label), die physikalische und digitale Auflösung sowie die Anzahl der aufgenommenen Signalwerte (Samples) (Tab. \ref{tab:edf_data}). Die einzelnen Signale selbst werden anschließend zum zugehörigen Data Record gespeichert. Darüber hinaus werden im Rahmen dieses Format-Standards bestimmte Labels für die Signalbezeichnungen vorgegeben. \parencite{kemp_european_2003}

\begin{table}[H]
\begin{small}
\centering
\begin{tabular}{|l|c|}
\hline
\multicolumn{1}{|c|}{\textbf{Header Record}}     & \textbf{Beispieldaten}                                                                                   \\ \hline
8 ascii : version of this data format (0)        & 0                                                                                                        \\ \hline
80 ascii : local patient identification          & \begin{tabular}[c]{@{}c@{}}MCH-0234567 F 02-MAY-1951\\ Haagse\_Harry\end{tabular}                        \\ \hline
80 ascii : local recording identification.       & \begin{tabular}[c]{@{}c@{}}Startdate 02-MAR-2002 EMG561 BK/\\ JOP Sony. MNC R Median Nerve.\end{tabular} \\ \hline
8 ascii : startdate of recording (dd.mm.yy)      & 17.04.01                                                                                                 \\ \hline
8 ascii : starttime of recording (hh.mm.ss).     & 11.25.00                                                                                                 \\ \hline
8 ascii : number of bytes in header record       & 768                                                                                                      \\ \hline
44 ascii : reserved                              & EDF+D                                                                                                    \\ \hline
8 ascii : number of data records (-1 if unknown) & 1200                                                                                                        \\ \hline
8 ascii : duration of a data record, in seconds  & 1.00                                                                                                    \\ \hline
4 ascii : number of signals (ns) in data record  & 2                                                                                                        \\ \hline
\end{tabular}
\caption[Header Record einer EDF-Datei]{Aufteilung des Header Records mit Beispiel-Headerdaten (unter Verwendung von \parencite{kemp_edf+_????})}
\label{tab:edf_header}
\end{small}
\end{table}

\begin{table}[H]
\begin{small}
\centering
\begin{tabular}{|l|c|c|}
\hline
\multicolumn{1}{|c|}{\textbf{Data Record}}                                                              & \textbf{Signal 1} & \textbf{Signal 2} \\ \hline
ns * 16 ascii : ns * label                                                                              & EEG C4-A1         & ECG               \\ \hline
\begin{tabular}[c]{@{}l@{}}ns * 80 ascii : ns * transducer type \\ (e.g. AgAgCl electrode)\end{tabular} & AgAgCl electrodes &                   \\ \hline
ns * 8 ascii : ns * physical dimension (e.g. uV)                                                        & uV                & uV                \\ \hline
ns * 8 ascii : ns * physical minimum (e.g. -500 or 34)                                                  & -300              & -1000             \\ \hline
ns * 8 ascii : ns * physical maximum (e.g. 500 or 40)                                                   & 300               & 1000              \\ \hline
ns * 8 ascii : ns * digital minimum (e.g. -2048)                                                        & -32768            & -32768            \\ \hline
ns * 8 ascii : ns * digital maximum (e.g. 2047)                                                         & 32767             & 32767             \\ \hline
\begin{tabular}[c]{@{}l@{}}ns * 80 ascii : ns * prefiltering \\ (e.g. HP:0.1Hz LP:75Hz)\end{tabular}    & HP:0.5Hz LP:50Hz  &                   \\ \hline
ns * 8 ascii : ns * nr of samples in each data record                                                   & 256               & 256               \\ \hline
ns * 32 ascii : ns * reserved                                                                           &                   &                   \\ \hline
\end{tabular}
\caption[Data Record einer EDF-Datei]{Aufteilung des Data Records mit Beispiel-Signaldaten (unter Verwendung von \parencite{kemp_edf+_????, kemp_edf+:_????})}
\label{tab:edf_data}
\end{small}
\end{table}

Dieser Standard ist zwar allgemeingültig, jedoch wird er nicht in jedem Schlaflabor verwendet. Viele Aufnahmegeräte unterstützen lediglich ein proprietäres Datenformat, wodurch die Vergleichbarkeit von \acs{PSG}s zunächst beeinträchtigt wird. Darüber hinaus werden die Labels manuell gesetzt. Um umfangreiche Studien mit Aufzeichnungen aus verschiedenen Schlaflaboren durchführen zu können, ist meist eine Konvertierung in \acs{EDF} und eine Standardisierung der Labels erforderlich. 

\subsection{Insomnie}\label{insomnie} 

Zur Klassifikation verschiedener schlafbezogener Krankheiten dient der internationale Katalog über Schlafstörungen "`\acl{ICSD-2}"' (\acs{ICSD-2}), welcher die bislang etwa 80 bekannten primären Schlafstörungen nach empirischen und pragmatischen Merkmalen in acht Gruppen einteilt. Diese gliedern sich in Insomnien, schlafbezogene Atmungsstörungen, Hypersomnien\footnote{Hypersomnie: Tagesschläfrigkeit}, Schlaf-Wach-Rhythmus-Störungen, Parasomnien\footnote{Parasomnie: Störung beim Erwachen (Arousal) oder bei Schlafstadienwechsel}, schlafbezonene Bewegungsstörungen sowie andere Schlafstörungen. In dieser Arbeit relevant sind die Insomnien, welche im Folgenden näher beschrieben werden.\parencite{happe_schlafmedizin_2009, american_academy_of_sleep_medicine_international_2005}\\

Gemäß \acs{ICSD-2} werden als Symptome einer Insomnie Ein- und Durchschlafstörungen oder nicht erholsamer Schlaf über mindestens einen Monat und mindestens dreimal pro Woche bezeichnet. Symptomatisch sind hierbei eine reduzierte Gesamtschlafzeit, eine verlängerte Einschlaflatenz und eine erhöhte Anzahl an Aufwachmomenten (Arousals). Zusätzlich muss sich die Schlafeffizienz negativ auf die Tagesbefindlichkeit oder tägliche Atktivitäten auswirken, beispielsweise in Form von Konzentrations- oder Motivationsschwierigkeiten, Tagesschläfrigkeit, sozialen oder beruflichen Einschränkungen oder auch Befindensbeeinträchtigungen. Gemäß \acs{ICSD-2} erfolgt eine Unterteilung in zehn unterschiedliche, nach Ursachen und Symptomen klassifizierte, Insomnietypen (primäre\footnote{primäre Insomnien: idiopathische, psychophysiologische oder paradoxe Insomnie (vgl. Anhang Tab. \ref{tab:idiopathische_insomnie}, \ref{tab:psycho_insomnie}, \ref{tab:paradoxe_insomnie}}, infolge äußerer Einflüsse\footnote{Insomnien infolge äußerer Einflüsse: hervorgerufen durch z. B. Hitze, Kälte, Lärm, Vibration, Gebrauch von Genussmitteln und Pharmaka sowie andere verhaltensabhängige Faktoren (vgl. Anhang Tab. \ref{tab:akute_insomnie}, \ref{tab:schlafhygiene_insomnie})} oder symptomatische Insomnien\footnote{symptomatische Insomnien: bei vorbestehenden körperlichen oder psychiatrischen Erkrankungen (vgl. Anhang Tab. \ref{tab:insomnie_psychisch}, \ref{tab:korperliche_insomnie})}, vgl. Anhang Tab. \ref{tab:allgemeine_insomnie} bis \ref{tab:schlafhygiene_insomnie}). In der Regel wird die Insomnie anhand einer umfangreichen, auf Fragebögen und Schlaftagebüchern basierenden Anamnese diagnostiziert. In komplizierteren oder chronischen Fällen, bei therapierefraktärer\footnote{therapierefraktär: nicht auf die Therapie anschlagend} Insomnie oder zur Abklärung gegenüber anderen Schlafstörungen kann eine \acs{PSG} jedoch indiziert sein. \parencite{happe_schlafmedizin_2009, american_academy_of_sleep_medicine_international_2005, mayer_s3-leitlinie_2009}\\

Eine Studie zur Gesundheit Erwachsener in Deutschland aus dem Jahr 2013 ergab, dass 5,7 \% der Befragten (7,7 \% der Frauen und 3,8 \% der Männer) an eindeutigen Symptomen einer Insomnie leiden (Abb. \ref{fig:praevalenz}). Hierbei ist zu beobachten, dass das Auftreten einer Insomnie im Alter von 40 bis 59 Jahren bei Frauen und Männern am höchsten ist und Frauen stärker betroffen sind (Abb. \ref{fig:praevalenz}). Die Prävalenz der Insomnie in Deutschland von 5,7 \% ist vergleichbar mit anderen Ländern.\footnote{In Spanien wurde eine Prävalenzrate von 6,4 \% ermittelt, in Italien von 7 \%. Deutlich höhere Prävalenzraten zeigen jedoch beispielsweise Kanada mit 9,5 \% und Großbritannien mit 22 \%.} Eine epidemiologische Studie zeigt hingegen eine weltweite Prävalenz der Insomnie von durchschnittlich 10 \% bis 30 \%\parencite{ohayon_epidemiological_2011}. Eine Untersuchung im Schlaflabor ist bei etwa 1 \% der Gesamtbevölkerung erforderlich \parencite{penzel_schlafstorungen_2005}.
\parencite{schlack_haufigkeit_2013, robert_koch_institut_gesundheit_2015}

\begin{figure}[H]
	\centering
	\includegraphics[scale = 0.5]{img/Praevalenz.png}
	\caption[Prävalenz von Insomnie]{Prävalenz von Insomnie in Deutschland im Jahr 2011 (unter Verwendung von \parencite{schlack_haufigkeit_2013})}
	\label{fig:praevalenz}
\end{figure}

Eine weitere Untersuchung aus dem Jahr 2009 an 1476 Erwerbstätigen im Alter zwischen 35 und 65 Jahren ergab, dass wiederholtes Aufwachen (37,5 \%) und zu kurzer Nachtschlaf (32,6 \%) als häufigste subjektiv wahrgenommenen Symptome der Insomnie berichtet werden (Abb.~\ref{fig:symptome}). Nicht erholsamer Schlaf, Nicht-Durchschlafen-Können, sehr frühes Erwachen oder spätes Einschlafen werden von durchschnittlich rund 26~\% der Studienteilnehmer angegeben. Am seltensten (7,5 \%) wurde die Angst, nicht einschlafen zu können, als Symptom genannt. \parencite{dak_forschung_gesundheitsreport_2010}

\begin{figure}[H]
	\centering
	\includegraphics[scale = 0.5]{img/Symptome.png}
	\caption[Symptome der Insomnie]{Symptome einer Insomnie bei Erwerbstätigen in Deutschland im Jahr 2009 (unter Verwendung von \parencite{dak_forschung_gesundheitsreport_2010})}
	\label{fig:symptome}
\end{figure}

\subsection{Schlafphasen}\label{schlafphasen} 

Der Schlaf lässt sich anhand seiner Merkmale in den Biosignalen in unterschiedliche Schlafphasen einteilen. Nach Rechtschaffen und Kales erfolgt die Klassifizierung in insgesamt sechs Stadien (Wachzustand, \acs{REM}-Schlaf sowie vier \acl{NREM} (\acs{NREM}) Schlafphasen\footnote{\acs{NREM}1 und \acs{NREM}2 gelten als Leichtschlaf, \acs{NREM}3 und \acs{NREM}4 als Tiefschlaf.}) \parencite{rechtschaffen_manual_1968}. Seit dem Jahr 2007 empfiehlt die \acs{AASM} jedoch eine Einteilung in Wachzustand, \acs{REM}, \acs{NREM}1, \acs{NREM}2 sowie \acs{NREM}3, wobei \acs{NREM}3 die zwei Tiefschlafphasen nach Rechtsschaffen und Kales zusammenfasst. Die Schlafphasen werden während der Nacht durchschnittlich fünfmal zyklisch durchlaufen, wobei ein Zyklus etwa 90 Minuten andauert und typischerweise eine Sequenz den Übergang von Leichtschlaf zum Tiefschlaf und anschließendem REM-Schlaf darstellt. Sämtliche durchlaufenen Zyklen und Schlafphasen werden durch das Hypnogramm erfasst (Abb. \ref{fig:hypnogram}). Dieses wird entweder automatisch vom Aufnahmesystem generiert oder manuell eingeteilt. Computerbasierte Einteilungen der Schlafphasen erzielen hierbei eine Übereinstimmung mit der menschlichen Klassifizierung von etwa 80 \% \parencite{rasche_update_2003, penzel_computer_2000}. \parencite{happe_schlafmedizin_2009, iber_aasm_2007}

\begin{figure}[H]
	\centering
	\includegraphics[width = \textwidth]{img/hypnogram.png}
	\caption[Hypnogram von gesundem Schlaf]{Hypnogramm eines 32-jährigen gesunden Schläfers \parencite{happe_schlafmedizin_2009}:\\"`Awake"' entspricht dem Wachzustand;\\"`Stage REM"' entspricht dem \acs{REM}-Schlaf;\\"`Stage 1"' bis "`Stage 4"' entsprechen den \acs{NREM}-Schlafphasen gemäß \parencite{rechtschaffen_manual_1968}}
	\label{fig:hypnogram}
\end{figure}

Die Schlafphasen werden anhand der Ausprägungen und Frequenzen in den \acs{EEG}-, \acs{EOG}- sowie \acs{EMG}-Signalen klassifiziert. Im Wachzustand (Tab. \ref{tab:wach}) herrschen im \acs{EEG} Sigma- und Beta-Wellen vor, bei geschlossenen Augen vorwiegend Alpha-Wellen. Im \acs{EOG} und \acs{EMG} können Bewegungen und erhöhte Frequenzen beobachtet werden. Der Leichtschlaf (\acs{NREM}1 und \acs{NREM}2; Tab. \ref{tab:leichtschlaf}) wird klassifiziert durch niedrigamplitudige, gemischte Aktivität sowie Schlafspindeln\footnote{Schlafspindeln: Stoß von schnellen Wellen (12 Hz bis 14 Hz) mit einer Dauer von über 0.5 Sekunden (vgl. Anhang Abb. \ref{fig:spindel_k-komplex}) \parencite{lee-chiong_sleep_2008}} und K-Komplexe\footnote{K-Komplex: spitzer negativer Ausschlag mit anschließender positiver Welle mit einer Dauer von über 0.5 Sekunden (vgl. Anhang Abb. \ref{fig:spindel_k-komplex}) \parencite{lee-chiong_sleep_2008}} im \acs{EEG}. Im Stadium \acs{NREM}1 treten langsame, rollende Augenbewegungen auf. Geringe Muskelaktivität ist im Stadium \acs{NREM}2 messbar. Der Tiefschlaf (Tab. \ref{tab:tiefschlaf}) ist geprägt durch vorwiegend langsame, hochamplitudige Delta-Wellen im \acs{EEG}, keine Augenbewegungen sowie stark herabgesetzte Muskelaktivität im Kinn. Der \acs{REM}-Schlaf (Tab. \ref{tab:rem}) hingegen ist charakterisiert durch niederamplitudige, gemischte \acs{EEG}-Aktivität, schnelle Augenbewegungen sowie keinen oder lediglich sehr geringen Muskeltonus. Da im Rahmen des \acs{TDS} Verfahrens die Herz- und Atemaktivität ebenfalls als Netzwerksysteme betrachtet werden, sind diese in den Tab. \ref{tab:wach} bis \ref{tab:rem} mit erfasst. Diese beiden Systeme zeigen eine Frequenzreduzierung von Wachzustand zum Tiefschlaf. \parencite{lee-chiong_sleep_2008, steinberg_schlafmedizin_2010, rasche_update_2003, ebner_eeg_2006}

\begin{table}[H]
\begin{small}
\centering
\begin{tabular}{|l|p{13cm}|}
\hline
\textbf{Signal} & \textbf{Klassifizierungsmerkmale für den Wachzustand} \\
\hline
\acs{EEG} & bei geöffneten Augen Sigma- und Beta-Wellen; bei geschlossenen Augen in Entspannung Alpha-Wellen (am stärksten im okzipitalen Bereich)\\
\hline
\acs{EOG} & Augenbewegungen, Zwinkern und erhöhte Frequenzen möglich\\
\hline
\acs{EMG} & Muskelaktivitäten und erhöhte Frequenzen möglich\\
\hline
\acs{EKG} & Herzfrequenz höher als in allen anderen Schlafphasen\\
\hline
Atmung & unregelmäßige Atemfrequenz möglich\\
\hline
\end{tabular}
\caption[Klassifizierungsmerkmale im Wachzustand]{Klassifizierungsmerkmale in den Biosignalen im Wachzustand (unter Verwendung von \parencite{lee-chiong_sleep_2008, steinberg_schlafmedizin_2010, rasche_update_2003, ebner_eeg_2006})}
\label{tab:wach}
\end{small}
\end{table}

\begin{table}[h]
\begin{small}
\centering
\begin{tabular}{|l|p{13cm}|}
\hline
\textbf{Signal} & \textbf{Klassifizierungsmerkmale für den Leichtschlaf} \\
\hline
\acs{EEG} & hauptsächlich Theta-Wellen und bis zu 20 \% Delta-Wellen; Schlafspindeln möglich (12 Hz bis 14 Hz, zentral am stärksten); K-Komplexe möglich (zentral und zentral-parietal am stärksten)\\
\hline
\acs{EOG} & langsame, rollende Augenbewegungen im \acs{NREM}1 keine Augenbewegungen im \acs{NREM}2\\
\hline
\acs{EMG} & geringe Muskelaktivität im Kinn möglich\\
\hline
\acs{EKG} & Herzfrequenz niedriger als im Wachzustand\\
\hline
Atmung & regelmäßige Atemfrequenz; geringer als im Wachzustand oder gleichbleibend\\
\hline
\end{tabular}
\caption[Klassifizierungsmerkmale im Leichtschlaf]{Klassifizierungsmerkmale in den Biosignalen im Leichtschlaf (unter Verwendung von \parencite{lee-chiong_sleep_2008, steinberg_schlafmedizin_2010, rasche_update_2003, ebner_eeg_2006});}
\label{tab:leichtschlaf}
\end{small}
\end{table}

\begin{table}[h]
\begin{small}
\centering
\begin{tabular}{|l|p{13cm}|}
\hline
\textbf{Signal} & \textbf{Klassifizierungsmerkmale für den Tiefschlaf} \\
\hline
\acs{EEG} & hauptsächlich Delta-Wellen; Schlafspindeln möglich\\
\hline
\acs{EOG} & keine Augenbewegungen\\
\hline
\acs{EMG} & sehr geringe Muskelaktivität im Kinn möglich\\
\hline
\acs{EKG} & Herzfrequenz niedriger als im Wachzustand\\
\hline
Atmung & regelmäßige Atemfrequenz; geringer als im Wachzustand oder gleichbleibend\\
\hline
\end{tabular}
\caption[Klassifizierungsmerkmale im Tiefschlaf]{Klassifizierungsmerkmale in den Biosignalen im Tiefschlaf (unter Verwendung von \parencite{lee-chiong_sleep_2008, steinberg_schlafmedizin_2010, rasche_update_2003, ebner_eeg_2006})}
\label{tab:tiefschlaf}
\end{small}
\end{table}

\begin{table}[H]
\begin{small}
\centering
\begin{tabular}{|l|p{13cm}|}
\hline
\textbf{Signal} & \textbf{Klassifizierungsmerkmale für den \acs{REM}-Schlaf} \\
\hline
\acs{EEG} & hauptsächlich Theta-Wellen (zentral und temporal am stärksten)\\
\hline
\acs{EOG} & Stöße von schnellen Außer-Phase-Bewegungen der Augen (1 Hz bis 4 Hz)\\
\hline
\acs{EMG} & keine Muskelaktivität; Stöße sehr geringer Muskelaktivität im Kinn möglich\\
\hline
\acs{EKG} & Herzfrequenz niedriger als im Wachzustand und höher als in den \acs{NREM}-Schlafphasen\\
\hline
Atmung & Anstieg oder Schwankungen in der Atemfrequenz möglich\\
\hline
\end{tabular}
\caption[Klassifizierungsmerkmale im \acs{REM}-Schlaf]{Klassifizierungsmerkmale in den Biosignalen im \acs{REM}-Schlaf (unter Verwendung von \parencite{lee-chiong_sleep_2008, steinberg_schlafmedizin_2010, rasche_update_2003, ebner_eeg_2006})}
\label{tab:rem}
\end{small}
\end{table}

Der Anteil des Wachzustands am Gesamtschlaf bei jungen, gesunden Menschen liegt bei maximal 5 \%. Mit zunehmendem Alter können auch die Wachanteile ansteigen, so dass bei Menschen in hohem Alter ein Anteil des Wachzustandes von 20 \% ebenfalls als erwartungskonform gilt. Der Leichtschlaf liegt in jedem Alter bei etwa 45 \% bis 55~\%. Der Tiefschlaf macht einen Anteil von 5 \% bis 20 \% aus, wobei mit zunehmendem Alter eine fallende Tendenz beobachtbar ist. Auch der prozentuale Anteil des \acs{REM}-Schlafs reduziert sich mit dem Alter auf durchschnittlich etwa 15 \%, wobei junge, gesunde Menschen ihren Schlaf zu 20 \% bis 25 \% im \acs{REM}-Schlaf verbringen (Tab. \ref{tab:anteile_schlafphasen}). In der Regel erfolgt der Großteil des Tiefschlafs in der ersten Nachthälfte. In der zweiten Nachthälfte hingegen überwiegen bei gesunden Schläfern Leichtschlaf (\acs{NREM}1 und \acs{NREM}2) und \acs{REM}-Schlaf (Abb. \ref{fig:hypnogram}). \parencite{lee-chiong_sleep_2008, steinberg_schlafmedizin_2010, danker-hopfe_percentile_2005}

\begin{table}[H]
\begin{small}
\centering
\begin{tabular}{|l|c|}
\hline
\textbf{Schlafphase} & \textbf{Prozentualer Anteil} \\
\hline
Wachzustand & 1 \% bis 20 \% \\
\hline
Leichtschlaf (\acs{NREM}1 und \acs{NREM}2) & 45 \% bis 55 \%\\
\hline
Tiefschlaf (\acs{NREM}3) & 5 \% bis 20 \%\\
\hline
\acs{REM}-Schlaf & 15 \% bis 25 \%\\
\hline
\end{tabular}
\caption[Prozentuale Schlafphasenverteilung]{Prozentualer Anteil der Schlafphasen am Gesamtschlaf bei gesunden Menschen aller Altersklassen \parencite{lee-chiong_sleep_2008, steinberg_schlafmedizin_2010, danker-hopfe_percentile_2005}}
\label{tab:anteile_schlafphasen}
\end{small}
\end{table}

Die Schlafphasenverteilung kann bei Insomniepatienten aufgrund reduzierter Gesamtschlafzeit, verlängerter Einschlaflatenz oder vermehrter Aufwachmomente von der Norm abweichen. Tiefschlaf und/oder \acs{REM}-Schlaf können beispielsweise reduziert sein oder vollständig ausbleiben, während Wachanteile und Leichtschlaf deutlich erhöht sein können. Auch die Schlafzyklen können aperiodisch oder gänzlich gestört sein, wie das beispielhafte Hypnogramm in Abb. \ref{fig:hypnogram_insomnia} zeigt. \parencite{happe_schlafmedizin_2009}

\begin{figure}[H]
	\centering
	\includegraphics[width = \textwidth]{img/hypnogram_insomnia.png}
	\caption[Hypnogramm von gestörtem Schlaf]{Hypnogramm einer 47-jahrigen Patientin mit Insomnie durch körperliche Erkrankung: deutlich fragmentiertes Schlafprofil mit verzögertem Einschlafen, Fehlen von Tiefschlaf und reduziertem Gesamtschlaf  \parencite{happe_schlafmedizin_2009}}
	\label{fig:hypnogram_insomnia}
\end{figure}

\section{Stand der Forschung}\label{stand} 

Das vegetative (autonome) Nervensystem ist maßgeblich an der Schlafregulation beteiligt und regelt vor allem in den \acs{NREM}-Schlafphasen unter anderem Herzschlag, Blutdruck sowie die Atmung \parencite{steinberg_schlafmedizin_2010}. Wie die einzelnen Systeme interagieren und welchen Einflüssen sie unterliegen, ist ein wesentlicher Forschungsschwerpunkt der Schlafmedizin. Es ist bereits bekannt, dass einzelne Systeme in den unterschiedlichen Schlafphasen spezifische Merkmale aufweisen. Beispielsweise sinken Herz- und Atemfrequenz vom Wachzustand zum Tiefschlaf kontinuierlich, während sie im REM-Schlaf wieder ansteigen (vgl. Tab. \ref{tab:wach} bis \ref{tab:rem} unter Punkt \ref{schlafphasen}). \parencite{lee-chiong_sleep_2008, rasche_update_2003, penzel_schlafstorungen_2005}\\

Univariate Untersuchungen können verschiedene signalspezifische und schlafphasenabhängige Eigenschaften einzelner Systeme spezifizieren. Mit Hilfe der Trendbereinigenden Fluktuationsanalyse (Detrended Fluctuation Analysis, DFA) zur Quantifizierung von Langzeitkorrelationen in Zeitreihen kann beispielsweise gezeigt werden, dass die Herzfrequenz im \acs{REM}-Schlaf eine ähnlich starke Langzeitkorrelation aufweist wie im Wachzustand. Die Fluktuationen in der Herzfrequenz sind in diesen beiden Schlafphasen demnach gering. In den \acs{NREM}-Schlafphasen hingegen ist die Korrelation schwach und die Fluktuation demnach erhöht. Dies wird bislang darauf zurückgeführt, dass die durch das vegetative Nervensystem gesteuerten Systeme im Leicht- und Tiefschlaf weitgehend autonom arbeiten. Auch für die Atmung konnte eine starke Langzeitkorrelation während des \acs{REM}-Schlafs herausgestellt werden, während in den \acs{NREM}-Schlafphasen keine signifikante Korrelation existiert. \parencite{penzel_cardiovascular_2007}\\

Bivariate Untersuchungen hingegen werden angestellt, um Korrelationen zwischen zwei Signalen zu erkennen. Die kardiorespiratorischen Zusammenhänge (Korrelationen zwischen Herzschlag und Atmung) im Schlaf sind bereits intensiv untersucht. Auf Grundlage der Phasensynchronisation sowie der Erstellung von Synchrogrammen kann beispielsweise gezeigt werden, dass Herzschlag und Atmung im \acs{REM}-Schlaf kaum und in den \acs{NREM}-Schlafphasen stark synchronisiert sind. Auch dies wird bislang auf das vegetative Nervensystem sowie die unterschiedliche Hirnaktivität in den einzelnen Schlafphasen zurückgeführt. Dieser Zusammenhang kann unabhängig von Geschlecht und Alter beobachtet werden. Da insbesondere der Tiefschlaf durch niedrigen Energieverbrauch und eine Erholungsfunktion des Körpers charakterisiert ist, wird angenommen, dass eine Synchronisierung zwischen Herzschlag und Atmung im \acs{NREM}-Schlaf energiesparsam ist und die Erholungsfunktion fördert, so dass auch hier eine Ähnlichkeit zwischen Leicht- und Tiefschlaf sowie zwischen Wachzustand und \acs{REM}-Schlaf manifestiert worden ist. \parencite{penzel_cardiovascular_2007, hamann_automated_2009, bartsch_experimental_2007}\\

Auf Basis uni- oder bivariater Signaluntersuchungen einzelner Systeme können jedoch weder Interaktionen mit anderen physiologischen Systemen noch mögliche störende Einflüsse auf den Schlafprozess oder die Schlafregulation hinreichend analysiert werden. An diesem Punkt setzt unter anderem das erstmals im Jahr 2012 von Bashan et al. vorgestellte Verfahren der \acs{TDS} an, mit dem Untersuchungen multidimensionaler Biosignalaufzeichnungen angestellt werden können. Hierbei werden die einzelnen physiologischen Systeme, deren Biosignale durch die \acs{PSG} erfasst werden, als Knoten eines zusammenhängenden Netzwerks betrachtet und hinsichtlich Netzwerktopologie und Vernetzungsstärke untersucht (vgl. Punkt \ref{TDS}). Die grundlegenden Errungenschaften dieser Methode liegen in der Möglichkeit, Zusammenhänge zwischen den Systemen darzustellen. Auf Basis eines relativ kleinen Datensatzes aus der SIESTA-Studie (36 \acs{PSG}s) \parencite{klosch_siesta_2001} und unter Verwendung von \acs{EKG}, nasaler Atmung, \acs{EOG}, fünf requenzbänder eines \acs{EEG}, Kinn- und Bein-\acs{EMG} konnte gezeigt werden, dass sich Netztopologie und Verbindungsstärke in Abhängigkeit von der jeweiligen Schlafphase ändern. Insbesondere wird deutlich, dass sich die \acs{NREM}-Schlafphasen unter den Gesichtspunkten der \acs{TDS} stark voneinander unterscheiden. Dies widerspricht massiv den bisherigen Erkenntnissen uni- und bivariater Biosignaluntersuchungen. Der Leichtschlaf ähnelt vielmehr dem Wachzustand, wobei beide von einem hohen Verbindungsgrad sowie einer großen Verbindungsstärke geprägt sind. Der Tiefschlaf hingegen weist stark reduzierte Verbindungsstärken und einen schwachen Verbindugsgrad auf, während im \acs{REM}-Schlaf Ergebnisse zwischen denen des Leicht- und Tiefschlafs zu beobachten sind (Abb. \ref{fig:TDS}). \parencite{bashan_network_2012}

\begin{figure}[H]
	\centering
	\includegraphics[width = \textwidth]{img/TDS.png}
	\caption[Vernetzungsstärke und Netztopologie im \acs{TDS} Verfahren]{Darstellung der Vernetzungsstärke (oben) und der Netztopologie (unten) im Wachzustand, Leicht-, Tief- und \acs{REM}-Schlaf nach Anwendung der \acs{TDS} und gemittelt über den gesamten Untersuchungsdatensatz (36 junge, gesunde Probanden) \parencite{bashan_network_2012}:\\HR = Herzschlag; Resp = Atmung; Chin = Kinn-\acs{EMG}; Leg = Bein-\acs{EMG}; $\delta$, $\theta$, $\alpha$, $\sigma$, $\beta$ = Frequenzbänder eines \acs{EEG}; diese Systeme sind in Zeilen und Spalten der Matrizen in gleicher Reihenfolge abgetragen;\\Wachzustand und Leichtschlaf weisen hohe Vernetzungsstärke und nahezu vollständig verknüpftes Netz auf; \acs{REM}-Schlaf und Tiefschlaf zeigen mäßig bis stark reduzierte Vernetzungsstärke und Netztopologie\\}
	\label{fig:TDS}
\end{figure}

Auch unter Verwendung des gesamten Kontrolldatensatzes der SIESTA-Studie (385 \acs{PSG}s) sowie unter Einbeziehung weiterer Systeme (\acs{EKG}, nasale Atmung, Atemanstrengung in Brust und Bauch, Kinn- und Bein-\acs{EMG}, zwei \acs{EOG}s sowie jeweils fünf Frequenzbänder von insgesamt sechs \acs{EEG}s) konnten diese Ergebnisse reproduziert werden. Die Untersuchungen zeigen insbesondere, dass verschiedene Systeme des Gehirns (frontopolare, zentrale und okzipitale Ableitungen) in sämtlichen Schlafphasen miteinander verbunden sind. Eine Studie über die Robustheit der \acs{TDS} Analyse zeigt darüber hinaus, dass Artefakte im \acs{NREM}- und \acs{REM}-Schlaf keine signifikanten und im Wachzustand lediglich geringe Auswirkungen auf die Ergebnisse haben, was hauptsächlich auf Bewegungen im Wachzustand zurückgeführt werden kann. \parencite{bartsch_network_2015, breuer_netzwerktopologie_2016}\\

Abhängigkeiten zwischen der Vernetzungsstärke und dem Alter gesunder Probanden können insbesondere in den \acs{NREM}-Schlafphasen beobachtet werden. Der Vergleich zwischen den Altersgruppen 20 bis 25 Jahre und 78 bis 95 Jahre zeigt im Tiefschlaf massive Unterschiede. Insbesondere im Gehirn findet eine Reduzierung der Vernetzungsstärke statt. Der Leichtschlaf zeigt eine ähnliche Ausprägung, während diese im Wachzustand weniger signifikant ist. Im \acs{REM}-Schlaf existieren keine signifikanten Veränderungen in Abhängigkeit vom Alter. Obwohl eine Abhängigkeit zwischen der Vernetzungsstärke und dem Geschlecht weniger stark ausgeprägt scheint, können Unterschiede zwischen Männern und Frauen im Leichtschlaf (N2) und \acs{REM}-Schlaf beobachtet werden. Hierbei sind bei den Frauen im Gehirn stärkere Verbindungen unabhängig vom Alter erkennbar. Da gemäß Krefting et al. Alter und Geschlecht mögliche Störfaktoren des physiologischen Netzwerks bei schlafgestörten Patienten darstellen können, sollen diese Eigenschaften im Rahmen dieser Arbeit bei Insomniepatienten untersucht werden. \parencite{krefting_altersabhangigkeit_2016, krefting_age_2017}

%zeitgleich
%mono-/univariat: Spektralanalysen, DFA (für Korrelationen und Fluktiationen), phase rectified signal averaging
%bivariat: Kreuzkorrelation,  transfer function analysis, phase synchronization analysis, bivariate phase rectified signal averaging (phasengleichgerichtete Signalmittelung)
%
%SCT:
%established methods: cross correlation, mutual information, and cross recurrence analysis
%"method of symbolic coupling traces SCT is used to analyze and quantify time-delayed coupling"
%"these methods (cross correlation, mutual information, and cross recurrence analysis) are more sensitive to nonstationarities, nonlinearities,
%and noise."
%
%bivariat: Synchronisation zwischen Resp und EKG existiert im DS bei Probanden; bei Apnoe nicht erkennbar (mit Synchrogram)
%basiert alles auf Korrelation ohne Time Delay